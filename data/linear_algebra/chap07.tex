\chapter{矩阵的特征值理论}

\section{知识点解析}

\begin{Def}
设$A$为$n$阶方阵,若存在数$\lambda$以及$n$维非零(列)向量$\vec{x}$,使得
$$
A\vec{x}=\lambda\vec{x}
$$
则称$\lambda$是$A$的特征值,$\vec{x}$是$A$的属于特征值$\lambda$的特征向量。
\end{Def}

\begin{Def}
下述关于$\lambda$的$n$次多项式称为$n$次方阵$A$的特征多项式
\begin{equation*}
f_A(\lambda)=|\lambda I-A|
=\begin{vmatrix}\lambda-a_{11}&-a_{12}&\cdots&-a_{n1}\\
-a_{21}&\lambda-a_{22}&\cdots&-a_{2n}\\
\cdots&\cdots&\cdots&\cdots\\
-a_{n1}&-a_{n2}&\cdots&\lambda-a_{nn}\end{vmatrix}
\end{equation*}
\end{Def}

\begin{thm}
设$A=(a_{ij})_{n\times n}$为$n$阶方阵,又设$\lambda_1,\lambda_2,\cdots,\lambda_n$为$A$的全部(复)特征值,则
\begin{enumerate}
  \item $\sum_{i=1}^n\lambda_i=\sum_{i=1}^na_{ii}=trA$;
  \item $|A|=\lambda_1\cdots\lambda_n=\Pi_{i=1}^n\lambda_i$
\end{enumerate}
\end{thm}

\begin{cor}
$n$阶方阵$A$不可逆$\Leftrightarrow$$|A|\neq 0$$\Leftrightarrow$$A$的特征值全不为零;
\end{cor}

\begin{cor}
设$\lambda_0$为$A$的特征值,$\vec{x}_0$为对应的特征向量,则
\begin{enumerate}
  \item $k\lambda_0$为$kA$的特征值,$\vec{x}_0$为对应的特征向量;
  \item $\lambda_0^k$为$A^k$的特征值,$\vec{x}_0$为对应的特征向量;
  \item $g(\lambda_0)$为$g(A)$的特征值,$\vec{x}_0$为对应的特征向量;
  \item 进一步,若$A$可逆,$\lambda^{-1}_0$为$A^{-1}$的特征值,$\vec{x}$为对应的特征向量,$\lambda_0|A|$为$A^{*}$的特征值,$\vec{x}$为对应的特征向量。
\end{enumerate}
\end{cor}

\begin{Def}
设$A$为$n$阶方阵,$\lambda$是$A$的一个特征值,则称
$$V_{\lambda}:=\{\vec{x}\in\mathbb{R}^n|A\vec{x}=\lambda\vec{x}\}$$
为$A$的关于特征值$\lambda$的特征子空间,其维数为特征值$\lambda$的几何重数。
\end{Def}

\begin{thm}
属于$A$的不同特征值的特征向量是线性无关的。
\end{thm}

\begin{Def}
设$A,B$是两个$n$阶方阵,如果存在一个$n$阶可逆矩阵$P$,使得$P^{-1}AP=B$,
则称方阵$B$相似于方阵$A$,记作$A\sim B$.
\end{Def}

\begin{cor}
相似矩阵的基本性质:
\begin{enumerate}
  \item 若$A\sim B$,$A_1\sim B_1$,则$kA\sim kB$,$A^m\sim B^m$,$A+A_1\sim B+B_1$,进而有$g(A)\sim g(B)$,对任意多项式$g(x)$成立。
  \item 相似矩阵有相同的秩,即若$A\sim B$,则$r(A)=r(B)$;进而,相似的矩阵有
  相同的可逆性,且若$A\sim B$,则$A^{-1}\sim B^{-1}$;
  \item 相似矩阵有相同的特征多项式;
  \item 相似矩阵具有相同的特征值;
  \item 相似矩阵具有相同的行列式与迹;
  \item 设$P^{-1}AP=B$,且$\lambda_0$为$A$的特征值,$\vec{x}$为相对应的特征向量,则$P^{-1}\vec{x}$为$B$的属于特征值$\lambda_0$的特征向量。
\end{enumerate}
\end{cor}

\begin{thm}
$n$阶方阵$A$的可对角化的充分必要条件是$A$有$n$个线性无关的特征向量。
\end{thm}

\begin{cor}
若$n$阶方阵$A$有$n$个互异的特征值,则$A$可对角化。
\end{cor}

\begin{cor}
若$n$阶方阵$A$可对角化,即存在可逆阵$P$与对角阵$\Lambda$,满足$P^{-1}AP=\Lambda$,则$\Lambda$对角线上的$n$个元即为$A$的$n$个特征值$\lambda_i(1\leq i\leq n)$;而可逆阵$P$的$n$个列向量$\vec{x}_i$即为属于
$\lambda_i(1\leq i\leq n)$的特征向量。
\end{cor}

\begin{Def}
设复数$z=a+bi,(a,b\in\mathbb{R}^n,i=\sqrt{-1})$,则$z$的复共轭运算记为$\bar{z}$,定义为:$\bar{z}:=a-bi$
\end{Def}

\begin{Def}
若$A=(a_{ij})_{m\times n}$,其中$a_{ij}\in\mathbb{C}$,则称$A$为$m\times n$ 型的复矩阵,复矩阵的共轭定义为:$\bar{A}=(\bar{a}_{ij})_{m\times n}$.
\end{Def}

\begin{thm}
设$A$是$n$阶实对称矩阵,则$A$的特征值都是实数。
\end{thm}

\begin{thm}
实对称矩阵$A$的属于不同特征值的特征向量是正交的。
\end{thm}

\begin{cor}
设$\lambda_i,\lambda_j$为实对称矩阵$A$的两个互异的特征值,则$V_{\lambda_i}\bot V_{\lambda_j}$。
\end{cor}

\begin{thm}
对$n$阶实对称阵$A$,总存在正交阵$Q$,使得$Q^{-1}AQ=Q^TAQ$为对角阵。
\end{thm}

\begin{cor}
$n$阶实对称阵必有$n$个线性无关的特征向量,且对$A$的全体互异的特征值
$\lambda_1,\lambda_2,\cdots,\lambda_s$,
有$\sum_{i=1}^s m_i=\sum_{i=1}^s dimV_{\lambda_i}=n$
\end{cor}

\begin{cor}
$n$阶实对称阵有$n$个相互正交的单位特征向量。
\end{cor}

%%%%%%%%%%%%%%%%%%%%%%%%%%%%%%%%%%%%%%%%%%%%%%%%%%%%%%%%%%%%%%%%%%%%%%%%%%%%%%%%%%%

\section{例题讲解}
\begin{eg}
求$A=\begin{bmatrix}1&-2\\1&4\end{bmatrix}$的所有特征值和特征向量。
\end{eg}
解:设$A$的特征值是$\lambda$,属于特征值$\lambda$的特征向量为$\vec{x}$,则$A\vec{x}=\lambda\vec{x}$,即
\begin{equation*}
\begin{bmatrix}1&-2\\1&4\end{bmatrix}\begin{bmatrix}x_1\\x_2\end{bmatrix}=
\lambda\begin{bmatrix}x_1\\x_2\end{bmatrix}
\end{equation*}
亦即
\begin{equation*}
\begin{cases}
(\lambda-1)x_1+2x_2=0\\
-x_1+(\lambda-4)x_2=0
\end{cases}
\end{equation*}
这是两个未知数两个方程的其次线性方程组,按定义,特征向量是非零向量,于是要求
上述齐次线性方程组有非零解,这等价于
\begin{equation*}
\begin{vmatrix}
\lambda-1&2\\-1&\lambda-4
\end{vmatrix}
=(\lambda-1)(\lambda-4)+2=(\lambda-2)(\lambda-3)=0
\end{equation*}
解得$\lambda_1=2,\lambda_2=3$,这就是$A$的两个特征值。\\
下面,分别求解属于特征值$\lambda_1=2$与$\lambda_2=3$的特征向量。\\
对于特征值2,代入得到齐次线性方程组:
$\begin{cases}x_1+2x_2=0\\-x_1-2x_2=0\end{cases}$
解得属于2的特征向量是$\begin{bmatrix}x_1\\x_2\end{bmatrix}=k\begin{bmatrix}2\\-1\end{bmatrix}$,
其中$k$是任意非零常数。\\
对于特征值3,代入得到齐次线性方程组:
$\begin{cases}2x_1+2x_2=0\\-x_1-x_2=0\end{cases}$
解得属于3的特征向量是$\begin{bmatrix}x_1\\x_2\end{bmatrix}=k\begin{bmatrix}1\\-1\end{bmatrix}$,
其中$k$是任意非零常数。

\begin{eg}
考虑对角阵$A=\begin{bmatrix}a_1&&&\\&a_2&&\\&&\ddots&\\&&&a_n\end{bmatrix}$,
其中$a_1,a_2,\cdots,a_n$为互不相同的实数,求$A$的特征值与特征向量。
\end{eg}
解:计算$A$的特征多项式:$|\lambda I-A|=(\lambda-a_1)(\lambda-a_2)\cdots(\lambda-a_n)$,显然得到全部特征值为
\begin{equation*}
\lambda_1=a_1,\lambda_2=a_2,\cdots,\lambda_n=a_n.
\end{equation*}
对每一个特征值$\lambda_i=a_i$,计算属于它的特征向量满足的线性方程组
$(a_iI-A)\vec{x}=\vec{0}$,即
\begin{equation*}
\begin{bmatrix}
a_i-a_1&&&&\\
&\ddots&&&\\
&&0&&\\
&&&\ddots&\\
&&&&a_i-a_n
\end{bmatrix}
\vec{x}=\vec{0}
\end{equation*}
由于$a_i\neq a_j(i\neq j)$,故上述方程组的稀疏矩阵主对角线上只有一个位置为0,
故它的基础解系为$\vec{x}_i=\vec{e}_i$
故属于特征值$\lambda=a_i$的特征向量为$k_i\vec{e}_i(k_i\neq 0)$,对所有的$i=1,2,\cdots,n$.


\begin{eg}
设$A$是三阶方阵,它的特征值为1,2,-1,$B=A^3-5A^2$,求$|B|$.
\end{eg}
解:设$A\vec{x}=\lambda\vec{x},\vec{x}\neq\vec{0}$,则$$B\vec{x}=A^3\vec{x}-5A^2\vec{x}=(\lambda^3-5\lambda^2)\vec{x}$$
取$\lambda=1,2,-1$,则$\lambda^3-5\lambda^2=-4,-12,-6$.
所以$B$的特征值为-4,-12,-6.\\
故$|B|=(-4)(-12)(-6)=-288$.

\begin{eg}
已知$n$阶方阵$A$的$n$个特征值为$\lambda_1,\cdots,\lambda_n$,求$2I-A$的特征值
以及$det(2I-A)$。
\end{eg}
解:由题设知$|\lambda I-A|=\Pi_{i=1}^n(\lambda-\lambda_i)$。下考虑$2I-A$的特征多项式
\begin{equation*}
|\mu I-(2I-A)|=|(\mu -2)I + A|=\Pi_{i=1}^n(\mu-(2-\lambda_i))
\end{equation*}
所以$2I-A$的$n$个特征值为$2-\lambda_i,i=1,2,\cdots,n$,从而
$$det(2I-A)=\Pi_{i=1}^n(2-\lambda_i)$$.

\begin{eg}
已知$A^2=A$,求矩阵$A$的特征值。
\end{eg}
解:设$\lambda$是$A$任意一个特征值,$\vec{x}$是$\lambda$所属的特征向量,则$A\vec{x}=\lambda\vec{x}$,所以$A^2\vec{x}=A(\lambda\vec{x})=\lambda A\vec{x}=\lambda^2\vec{x}$,
利用已知条件$A^2=A$与上式,有
$$\lambda\vec{x}=A\vec{x}=A^2\vec{x}=\lambda^2\vec{x}$$,
所以$(\lambda^2-\lambda)\vec{x}=\vec{0}$,因为$\vec{x}$为特征向量,所以$\vec{x}\neq\vec{0}$,故
$$\lambda^2=\lambda,\Rightarrow\lambda =0 \text{或}\lambda=1$$.

\begin{eg}
设$A=\begin{bmatrix}0.95&0.15\\0.05&0.85\end{bmatrix},
\Lambda=\begin{bmatrix}1&0\\0&0.8\end{bmatrix},
P=\begin{bmatrix}3&-1\\1&1\end{bmatrix}.$
验证$P^{-1}AP=\Lambda$,并且求$A^k$($k$为正整数)。
\end{eg}
解:由于
\begin{equation*}
AP=\begin{bmatrix}0.95&0.15\\0.05&0.85\end{bmatrix}
\begin{bmatrix}3&-1\\1&1\end{bmatrix}=
\begin{bmatrix}3&-0.8\\1&0.8\end{bmatrix}=
\begin{bmatrix}3&-1\\1&1\end{bmatrix}
\begin{bmatrix}1&0\\0&0.8\end{bmatrix}
=P\Lambda
\end{equation*}
\begin{equation*}
A^k=(P\Lambda P^{-1})^k=P\Lambda^kP^{-1}=\frac{1}{4}
\begin{bmatrix}3+0.8^k&3-3\cdot0.8^k\\1-0.8^k&1+3\cdot0.8^k\end{bmatrix}
\end{equation*}

\begin{eg}
判断下列矩阵能否对角化,如能对角化,求出可逆阵$P$与对角阵$\Lambda$,使得
$P^{-1}AP=\Lambda$.\\
(1)$A=\begin{bmatrix}2&1&1\\1&2&1\\1&1&2\end{bmatrix}$;
(2)$A=\begin{bmatrix}2&2&-1\\-1&0&1\\-1&0&2\end{bmatrix}$。
\end{eg}
解:(1)先求$A$的特征值,特征多项式为
\begin{equation*}
f_A(\lambda)=
\begin{vmatrix}\lambda-2&-1&-1\\-1&\lambda-2&-1\\-1&-1&\lambda-2\end{vmatrix}
=(\lambda-1)^2(\lambda-1)
\end{equation*}
所以得到$A$的特征值为:$\lambda-1=1$(重数为2);$\lambda-2=4$。
再求$A$的特征向量。\\
对于$\lambda_1=1$,解齐次线性方程组$(\lambda_1 I-A)\vec{x}=\vec{0}$,
得到属于特征值$\lambda_1=1$的两个线性无关的特征向量为$\vec{x}_{11}=(-1,1,0)^T,\vec{x}_{12}=(-1,0,1)^T$.\\
对于$\lambda_2=4$,解齐次线性方程组$(\lambda_2 I-A)\vec{x}=\vec{0}$,
得到属于特征值$\lambda_2=4$的一个线性无关的特征向量为$\vec{x}_{21}=(1,1,1)^T$.\\
因此,$A$存在3个线性无关的特征向量$\vec{x}_{11},\vec{x}_{12},\vec{x}_{21}$,
故$A$可对角化,且
\begin{equation*}
P=(\vec{x}_{11},\vec{x}_{12},\vec{x}_{21})=\begin{bmatrix}-1&-1&1\\1&0&1\\0&1&1
\end{bmatrix}
\end{equation*}
\begin{equation*}
\Lambda=P^{-1}AP=\begin{bmatrix}1&0&0\\0&1&0\\0&0&4\end{bmatrix}
\end{equation*}
(2)先求$A$的特征值,特征多项式为
\begin{equation*}
f_A(\lambda)=
\begin{vmatrix}\lambda-2&-2&1\\1&\lambda&-1\\1&0&\lambda-2\end{vmatrix}
=(\lambda-1)^2(\lambda-2)
\end{equation*}
所以得到$A$的特征值为:$\lambda_1=1$(重数为2);$\lambda_2=2$。
再求$A$的特征向量。\\
对于$\lambda_1=1$,解齐次线性方程组$(\lambda_1 I-A)\vec{x}=\vec{0}$,
得到属于特征值$\lambda_1=1$的一个线性无关的特征向量为$\vec{x}_{11}=(1,0,1)^T$.\\
对于$\lambda_2=2$,解齐次线性方程组$(\lambda_2 I-A)\vec{x}=\vec{0}$,
得到属于特征值$\lambda_2=2$的一个线性无关的特征向量为$\vec{x}_{21}=(0,1,2)^T$.\\
因此,$A$只存在2个线性无关的特征向量$\vec{x}_{11},\vec{x}_{21}$,因此$A$不可对角化。

\begin{eg}
(1)若$A$为$n$阶方阵,且$A$可对角化,则$A^k$($k$为正整数),$g(A)$($g$为实系数多项式)均可以对角化。\\
(2)若$A$可逆,则$A^{-1},A^{*}$也可以对角化。
\end{eg}
证明:(1)由于$A$可对角化,则存在可逆阵$P$与对角阵$\Lambda$,使得
$$P^{-1}AP=\Lambda=diag(\lambda_1,\lambda_2,\cdots,\lambda_n)$$,
故$A^k=Pdiag(\lambda_1^k,\lambda_2^k,\cdots,\lambda_n^k)P^{-1}$ $\Rightarrow$ $A^k$可对角化。
再由矩阵乘法的分配律有:
$$g(A) = P diag(g(\lambda_1),g(\lambda_2),\cdots,g(\lambda_n)) P^{-1}$$
$\Rightarrow$ $g(A)$可对角化。\\
(2)若$A$可逆,则$\lambda_i$均不为0,两边求逆可得:
$$P^{-1}A^{-1}P=diag(\lambda_1^{-1},\lambda_2^{-1},\cdots,\lambda_n^{-1})$$,
故$A^{-1}$可对角化。\\
由$A^*=|A|A^{-1}$得:
$$P^{-1}A^{*}P=|A|diag(\lambda_1^{-1},\lambda_2^{-1},\cdots,\lambda_n^{-1})$$,
故$A^*$可对角化。

\begin{eg}
设$A$为$n$阶方阵,满足$A^2=I$,证明:$A$可对角化。
\end{eg}
证明:设$\lambda$是$A$的任一特征值,$\vec{x}$是$\lambda$所属的特征向量,则
$$A^2\vec{x}=\lambda^2\vec{x},\Rightarrow(\lambda^2-1)\vec{x}=\vec{0}
(\vec{x}\neq\vec{0}),\Rightarrow\lambda=\pm 1$$
设$\lambda_1=1,\lambda_2=-1$,且$m_1,m_2$为对应的几何重数,则
$$m_1+m_2=(n-r(A-I))+(n-r(A+I))=2n-[r(I-A)+r(I+A)]$$,
由矩阵秩的相关结论知
$$n=r(2I)=r[(I-A)+(I+A)]\leq r(I-A)+r(I+A)\leq n$$
其中第二个小于等于号是因为$(I-A)(I+A)=O$。故
$$r(A-I)+r(A+I)=n,\Rightarrow m_1+m_2=n$$
从而$A$可对角化。

\begin{eg}
设$A=\begin{bmatrix}0&1&1&-1\\1&0&-1&1\\1&-1&0&1\\-1&1&1&0\end{bmatrix}$,将其对角化。
\end{eg}
解:首先求特征值与代数重数
\begin{equation*}
|\lambda I-A|=\begin{vmatrix}\lambda&-1&-1&1\\-1&\lambda&1&-1\\
-1&1&\lambda&-1\\1&-1&-1&\lambda\end{vmatrix}=(\lambda-1)^2(\lambda+3)
\end{equation*}
得到$\lambda_1=1,n_1=3,\lambda_2=-3,n_2=1$.\\
其次,对特征值$\lambda_1=1$,解齐次线性方程组$(\lambda_1 I-A)\vec{x}=\vec{0}$,
求特征向量$\vec{x}_{11}=(1,1,0,0)^T,\vec{x}_{12}=(1,0,1,0)^T,\vec{x}_{13}=(-1,0,0,1)^T$;\\
再次,对特征值$\lambda_2=-3$,解齐次线性方程组$(\lambda_2 I-A)\vec{x}=\vec{0}$,
求特征向量$\vec{x}_{21}=(1,-1,-1,1)^T$;\\
于是,令$P=\begin{bmatrix}1&1&-1&1\\1&0&0&-1\\0&1&0&-1\\0&0&1&1\end{bmatrix}$
从而$\Lambda=P^{-1}AP=
\begin{bmatrix}1&0&0&0\\0&1&0&0\\0&0&1&0\\0&0&0&-3\end{bmatrix}$

\begin{eg}
已知$A$为3阶实对称矩阵,其特征值为1,1,2,且属于2的特征向量是$(1,0,1)^T$,求$A$.
\end{eg}
解:$A$是3阶实对称矩阵,正交相似于对角阵$\Lambda=diag(1,1,2)$,属于特征值1的特征向量与属于特征值2的特征向量$(1,0,1)^T$正交,这等价于求解齐次线性方程组:
$$x_1+x_3=0$$
求解得到属于1的特征向量为:$(0,1,0)^T,(1,0,-1)^T$,且彼此正交,故得到相应的
正交矩阵为
$$Q=\begin{bmatrix}0&\frac{1}{\sqrt{2}}&\frac{1}{\sqrt{2}}\\1&0&0\\
0&-\frac{1}{\sqrt{2}}&\frac{1}{\sqrt{2}}\end{bmatrix}$$
故
$$A=Q\Lambda Q^T=\begin{bmatrix}\frac{3}{2}&0&\frac{1}{2}\\0&1&0\\
\frac{1}{2}&0&\frac{3}{2}\end{bmatrix}$$

\begin{eg}
经过统计,某地区猫头鹰和森林鼠的数量有如下的规律:每个月只有一半的猫头鹰可以存活;森林鼠的数量每个月会增加$10\%$。如果森林鼠充足(数量为$Q$),则下个月猫头鹰的数量将会增加$0.4Q$。平均每个月每只猫头鹰的捕食会导致104森林鼠死亡,试确定该系统的长期演变情况。
\end{eg}
解:设猫头鹰和森林鼠在时刻$k$的数量为$\vec{x}=\begin{bmatrix}P_k\\Q_k\end{bmatrix}$,其中$k$是以月份为单位的时间,$P_k$为研究区域中猫头鹰的数量(只),$Q_k$为研究区域中森林鼠的数量(千只),则
\begin{equation*}
\begin{cases}
P_{k+1}=0.5P_{k}+0.4Q_k\\
Q_{k+1}=-0.104P_k+1.1Q_k
\end{cases}
\Rightarrow
\begin{bmatrix}P_{k+1}\\Q_{k+1}\end{bmatrix}=
\begin{bmatrix}0.5&0.4\\-0.104&1.1\end{bmatrix}
\begin{bmatrix}P_k\\Q_k\end{bmatrix}
\end{equation*}
令$A=\begin{bmatrix}0.5&0.4\\-0.104&1.1\end{bmatrix}$,经过计算得,矩阵$A$的两个特征值为$\lambda_1=1.02,\lambda_2=0.58$,对应的特征向量分别为:
$\vec{x}_1=(10,13)^T,\vec{x}_2=(5,1)^T$,设初始向量$\vec{x}_0$可以表示为:
$\vec{x}_0=c_1\vec{x}_1+c_2\vec{x}_2$,于是对于$k\geq0$,有:
\begin{align*}
\vec{x}_k=&A^k(c_1\vec{x}_1+c_2\vec{x}_2)=c_1\lambda_1^k\vec{x}_1+c_2\lambda_2^k\vec{x}_2\\
=&c_1(1.02)^k\begin{bmatrix}10\\13\end{bmatrix}+c_2(0.58)^k\begin{bmatrix}5\\1\end{bmatrix}
\end{align*}

%%%%%%%%%%%%%%%%%%%%%%%%%%%%%%%%%%%%%%%%%%%%%%%%%%%%%%%%%%%%%%%%%%%%%%%%%%%%%%%%%%

\section{课后习题}

\begin{ex}\label{7.1}
求下列矩阵的特征值与特征向量。\\
(1)$\begin{bmatrix}1&2\\0&3\end{bmatrix}$;
(2)$\begin{bmatrix}1&3\\2&2\end{bmatrix}$;
(3)$\begin{bmatrix}1&1&1\\0&2&1\\0&0&3\end{bmatrix}$;\\
(4)$\begin{bmatrix}1&2&3\\2&1&3\\3&3&6\end{bmatrix}$;
(5)$\begin{bmatrix}1&-\sqrt{3}\\ \sqrt{3}&1\end{bmatrix}$;
\end{ex}

\begin{ex}\label{7.2}
已知$A=\begin{bmatrix}a&1&b\\0&c&0\\-4&c&1-a\end{bmatrix}$ 有一个特征值是1. 对应的特征向量是$\vec{\alpha}=(1,1,1)^T$,求$a,b,c$的值。
\end{ex}

\begin{ex}\label{7.3}
已知$A=\begin{bmatrix}1&1&2\\5&a&0\\1&b&2\end{bmatrix}$ 有一个特征向量是$\vec{\alpha}=(1,1,-1)^T$,求$a,b$的值以及$\vec{\alpha}$对应的特征值。
\end{ex}

\begin{ex}\label{7.4}
设三阶方阵$A$的特征值分别为$\lambda_1=-1,\lambda_2=1,\lambda_3=3$,对应的特征向量依次是:
\begin{equation*}
\vec{\alpha}_1=\begin{bmatrix}1\\0\\-1\end{bmatrix},
\vec{\alpha}_2=\begin{bmatrix}1\\2\\1\end{bmatrix},
\vec{\alpha}_3=\begin{bmatrix}0\\1\\-1\end{bmatrix},
\end{equation*}
求矩阵$A$。
\end{ex}

\begin{ex}\label{7.5}
求$\begin{bmatrix}2&1&1\\0&3&1\\0&0&5\end{bmatrix}^{n}$。
\end{ex}

\begin{ex}\label{7.6}
假设$\lambda_1$和$\lambda_2$是矩阵$A$ 的特征值,且$\lambda_1\neq\lambda_2$,
$\vec{\xi}_1$和$\vec{\xi}_2$分别是$\lambda_1$ 以及$\lambda_2$所对应的特征向量。\\
(1)证明$\vec{\xi}_1,\vec{\xi}_2$ 线性无关。\\
(2)证明$\vec{\xi}_1-\vec{\xi}_2$ 不是$A$ 的特征向量。
\end{ex}

\begin{ex}\label{7.7}
设$f(x)=a_0+a_1x+\cdots+a_nx^n$,$A$是$n$ 阶方阵且$f(A)=0$,若$A$没有零特征值,
证明$a_0\neq0$。
\end{ex}

\begin{ex}\label{7.8}
设$E$是$n$阶单位矩阵,如果$n$ 阶矩阵$A$ 满足$|A+E|=|A-2E|=|4E-A|=0$,求$|A^3-5A^2|$ 的值。
\end{ex}

\begin{ex}\label{7.9}
已知3阶矩阵$A$的特征值为$1,2,-1$,求$|A^*+3A|$的值。
\end{ex}

\begin{ex}\label{7.10}
已知$A$是三阶实矩阵,$tr(A)=5$且$|A|=4$,设$A$的特征值为$\lambda_1,\lambda_2,\lambda_2$,
且$\lambda_1=2$,求$\lambda_2,\lambda_2$ 的值。
\end{ex}

\begin{ex}\label{7.11}
已知$A$是$n$阶方阵,假设$\vec{\alpha}_1,\vec{\alpha}_2,\vec{\alpha}_3$ 是$A$ 的三个属于
不同特征值$\lambda_1,\lambda_2,\lambda_3$ 的特征向量,
令$\vec{\beta}=\vec{\alpha}_1+\vec{\alpha}_2+\vec{\alpha}_3$,证明$\vec{\beta},A\vec{\beta},A^2\vec{\beta}$ 线性无关。
\end{ex}

\begin{ex}\label{7.12}
已知矩阵$A=\begin{bmatrix}2&0&0\\0&1&-1\\0&-1&1\end{bmatrix}$,求$A$ 的每个特征子空间的一组基。
\end{ex}

\begin{ex}\label{7.13}
判断$A=\begin{bmatrix}1&1&0\\0&2&1\\0&0&5\end{bmatrix}$
与$B=\begin{bmatrix}1&0&0\\0&2&0\\0&0&5\end{bmatrix}$ 是否相似,并说明理由。
\end{ex}

\begin{ex}\label{7.14}
设$A\sim B$,其中
\begin{equation*}
A=\begin{bmatrix}1&-1&1\\x&4&-2\\-3&-3&5\end{bmatrix},
B=\begin{bmatrix}3&\sqrt{2}&1\\ \sqrt{2}&4&\sqrt{2}\\1&\sqrt{2}&y\end{bmatrix},
\end{equation*}
求$x$和$y$的值。
\end{ex}

\begin{ex}\label{7.15}
设$A$和$B$都是$n$阶实矩阵,且$B$可逆,证明$AB$和$BA$是相似的。
\end{ex}

\begin{ex}\label{7.16}
设
\begin{equation*}
  A=\begin{bmatrix}1&2&-2\\0&0&0\\0&0&0\end{bmatrix}
\end{equation*}
试问$A$能否对角化?如果可以对角化,求出使$A$ 相似于对角矩阵的相似变换矩阵$P$,同时写出这个对角阵。
\end{ex}

\begin{ex}\label{7.17}
设$\vec{\alpha}=(a_1,a_2,\cdots,a_n)^T$,$\vec{\beta}=(b_1,b_2,\cdots,b_n)^T$且$\vec{\alpha}^T\vec{\beta}=1$,
$E$是$n$阶单位矩阵,令$A=E+\vec{\alpha}\vec{\beta}^T$。\\
(1)求$A$的特征值和特征向量。\\
(2)讨论$A$是否可以对角化;若可以,求可逆阵$P$ 以及对角阵$\Lambda$,使$P^{-1}AP=\Lambda$。若不行,说明理由。
\end{ex}

\begin{ex}\label{7.18}
设$A=\begin{bmatrix}3&-1&1\\0&x&y\\0&z&w\end{bmatrix}$,已知$A$有3线性无关的特征向量,
$\lambda=1$是$A$的二重特征值。\\
(1)求$x,y,z,w$的值。\\
(2)将$A$对角化,求可逆阵$P$以及对角阵$\Lambda$,使$P^{-1}AP=\Lambda$。
\end{ex}

\begin{ex}\label{7.19}
已知二阶实矩阵$A=\begin{bmatrix}a&c\\b&d\end{bmatrix}$,若$bc>0$,证明$A$可以对角化。
\end{ex}

\begin{ex}\label{7.20}
判断$A=\begin{bmatrix}1&-3&3\\3&-5&3\\6&-6&4\end{bmatrix}$ 与
是否可以对角化,若可以,请求出可逆阵$P$ 以及对角阵$\Lambda$,若不可以,请说明理由。
\end{ex}

\begin{ex}\label{7.21}
已知$A$是实对称矩阵,满足$A^2+3A=O$ 且$r(A)=2$,其中$O$表示零矩阵,求$A$的全部特征值。
\end{ex}

\begin{ex}\label{7.22}
设$A$是3阶实对称矩阵,其特征值为$\lambda_1=1,\lambda_2=2,\lambda_3=-3$,$\vec{\alpha}_1=(1,1,1)^T$是
$A$属于$\lambda_2$的一个特征向量,记$B=A^2+2A$,求$A$的所有特征值以及特征向量。
\end{ex}

\begin{ex}\label{7.23}
已知实对称矩阵$A=\begin{bmatrix}1&-2&0\\-2&2&-2\\0&-2&3\end{bmatrix}$,将其正交对角化。
\end{ex}

\begin{ex}\label{7.24}
已知实对称矩阵$A=\begin{bmatrix}1&-2&2\\-2&-2&4\\2&4&-2\end{bmatrix}$,将其正交对角化。
\end{ex}

\begin{ex}\label{7.25}
设$A$是3阶实矩阵,且$A$有3个正交的特征向量,证明$A$是对称矩阵。
\end{ex}

\begin{ex}\label{7.26}
设$A$是$n$阶实对称矩阵,$\lambda_1,\lambda_2,\cdots,\lambda_n$ 是$A$ 的特征值,
相应的标准正交特征向量为$\vec{\alpha}_1,\vec{\alpha}_2,\cdots,\vec{\alpha}_n$,证明
\begin{equation*}
  A = \lambda_1\vec{\alpha}_1\vec{\alpha}_1^T+\cdots+\lambda_n\vec{\alpha}_n\vec{\alpha}_n^T
\end{equation*}
\end{ex}

\begin{ex}\label{7.27}
设$A$是$n$阶实对称矩阵,且存在正整数$k$,使得$A^{k}=O$,其中$O$是$n$ 阶零矩阵,则$A=O$。
\end{ex}

\begin{ex}\label{7.28}
在某省,每年有比例$p$的农村居民移居城镇,有比例为$q$的城镇居民移居到农村,假设该省的人口总数不变,
且上述人口迁移的规律不变,将$n$ 年后农村人口和城镇人口占总人口的比例分别记为$x_n$ 和$y_n$ ($x_n+y_n=1$)。\\
(1)求关系式$\begin{bmatrix}x_{n+1}\\y_{n+1}\end{bmatrix}=A\begin{bmatrix}x_n\\y_n\end{bmatrix}$ 中的矩阵$A$;\\
(2)设目前农村人口和城镇人口相等,即$\begin{bmatrix}x_0\\y_0\end{bmatrix}=\begin{bmatrix}0.5\\0.5\end{bmatrix}$,
求$\begin{bmatrix}x_{n}\\y_{n}\end{bmatrix}$。
\end{ex}

%%%%%%%%%%%%%%%%%%%%%%%%%%%%%%%%%%%%%%%%%%%%%%%%%%%%%%%%%%%%%%%%%%%%%%%%%%%%%%%%%%%%%

\section{习题答案}

\textbf{习题 \ref{7.1} 解答:}\\
(1)考虑$A$的特征多项式:
\begin{align*}
  |A-\lambda E|&=\begin{vmatrix}1-\lambda&2\\0&3-\lambda\end{vmatrix}\\
   & =(1-\lambda)(3-\lambda)
\end{align*}
因此$A$的特征值为$\lambda_1=1,\lambda_2=3$。 对于特征值$\lambda_1=1$,解方程组$(A-E)\vec{x}_1=\vec{0}$,
得到特征向量$\vec{x}_1=k(1,0)^T$。同理,对于特征值$\lambda_2=3$,解方程组$(A-3E)\vec{x}_2=\vec{0}$,
得到特征向量$\vec{x}_2=k(1,-1)^T$。\\
(2)特征值为:$\lambda_1=-1,\lambda_2=4$;\\
对应的特征向量为:$\vec{x}_1=k(-\frac{3}{2},1)^T,\vec{x}_2=k(1,1)^T$。\\
(3)特征值为:$\lambda_1=1,\lambda_2=2,\lambda_3=3$;\\
对应的特征向量为:$\vec{x}_1=k(1,0,0)^T,\vec{x}_2=k(1,1,0)^T,\vec{x}_3=k(1,1,1)^T$。\\
(4)特征值为:$\lambda_1=-1,\lambda_2=0,\lambda_3=9$;\\
对应的特征向量为:$\vec{x}_1=k(-1,1,0)^T,\vec{x}_2=k(-1,-1,0)^T,\vec{x}_3=k(1,1,2)^T$。\\
(5)特征值为:$\lambda_1=1+\sqrt{3}i,\lambda_2=1-\sqrt{3}i$;\\
对应的特征向量为:$\vec{x}_1=k(i,1)^T,\vec{x}_2=k(-i,1)^T$。\\
\textbf{习题 \ref{7.2} 解答:}\\
根据题目,有
\begin{equation*}
  \begin{bmatrix}a&1&b\\0&c&0\\-4&c&1-a\end{bmatrix}
  \begin{bmatrix}1\\1\\1\end{bmatrix}=
  \begin{bmatrix}1\\1\\1\end{bmatrix}
\end{equation*}
解得:
\begin{equation*}
  \begin{cases}
  a = -3\\
  b = 4\\
  c = 1
  \end{cases}
\end{equation*}
\textbf{习题 \ref{7.3} 解答:}\\
设$\vec{\alpha}$所对应的特征值为$\lambda$,则$A\vec{\alpha}=\lambda\vec{\alpha}$。
得到:
\begin{equation*}
  \begin{cases}
  0 = \lambda\\
  5+a = \lambda\\
  1+b-2= -\lambda
  \end{cases}
\end{equation*}
解得
\begin{equation*}
  \begin{cases}
  \lambda=0\\
  a = -5\\
  b = 1
  \end{cases}
\end{equation*}
\textbf{习题 \ref{7.4} 解答:}\\
由于$A\vec{\alpha}_1=\lambda_1\vec{\alpha}_1$,$A\vec{\alpha}_2=\lambda_2\vec{\alpha}_2$,
$A\vec{\alpha}_3=\lambda_3\vec{\alpha}_3$ 可得
\begin{equation*}
A(\vec{\alpha}_1,\vec{\alpha}_2,\vec{\alpha}_3)
=(A\vec{\alpha}_1,A\vec{\alpha}_2,A\vec{\alpha}_3)
=(\lambda_1\vec{\alpha}_1,\lambda_2\vec{\alpha}_2,\lambda_3\vec{\alpha}_3)
\end{equation*}
令$P=(\vec{\alpha}_1,\vec{\alpha}_2,\vec{\alpha}_3)$,则有
\begin{equation*}
AP=P\begin{bmatrix}\lambda_1&0&0\\0&\lambda_2&0\\0&0&\lambda_3\end{bmatrix}
\end{equation*}
由于$\vec{\alpha}_1,\vec{\alpha}_2,\vec{\alpha}_3$ 线性无关,$P$可逆,所以
\begin{align*}
A=&P\begin{bmatrix}\lambda_1&0&0\\0&\lambda_2&0\\0&0&\lambda_3\end{bmatrix}P^{-1}\\
 =&\begin{bmatrix}1&1&0\\0&2&1\\-1&1&-1\end{bmatrix}
   \begin{bmatrix}-1&0&0\\0&1&0\\0&0&3\end{bmatrix}
   \begin{bmatrix}1&1&0\\0&2&1\\-1&1&-1\end{bmatrix}^{-1}\\
  =&\begin{bmatrix}-\frac{1}{2}&\frac{1}{2}&\frac{1}{2}\\
     -1&2&-1\\ \frac{5}{2}&-\frac{3}{2}&\frac{3}{2}\end{bmatrix}
\end{align*}
\textbf{习题 \ref{7.5} 解答:}\\
设$A=\begin{bmatrix}2&-1&2\\0&3&1\\0&0&5\end{bmatrix}$,
则$A$的特征值为$2$,$3$,$5$,对应的特征向量为$k(1,0,0)^T$,
$k(-1,1,0)^T$,$k(1,1,2)^T$,
那么令$P=\begin{bmatrix}1&-1&1\\0&1&1\\0&0&2\end{bmatrix}$,则有
\begin{equation*}
A=P\begin{bmatrix}2&0&0\\0&3&0\\0&0&5\end{bmatrix}P^{-1}
\end{equation*}
因此
\begin{equation*}
A^{n}=P\begin{bmatrix}2^{n}&0&0\\0&3^{n}&0\\0&0&5^{n}\end{bmatrix}P^{-1}
=\begin{bmatrix}2^{n}&2^n-3^n&\frac{1}{2}(3^n+5^n-2^{n+1})\\0&3^n&\frac{1}{2}(5^n-3^n)\\0&0&5^n\end{bmatrix}
\end{equation*}
\textbf{习题 \ref{7.6} 解答:}\\
(1)假设$\xi_1$和$\xi_2$是线性相关的,即存在不为零的常数$k$,使得$\xi_1=k\xi_2$。
由于$A\xi_2=\lambda_2\xi$,那么
\begin{equation*}
  A\xi_1=A(k\xi_2)=kA\xi_2=k\lambda_2\xi_2
\end{equation*}
又由于$A\xi_1=\lambda_1\xi_1$,那么$k\lambda_2\xi_2=\lambda_1\xi_1=k\lambda_1\xi_2$,
又$k\neq0$,故$\lambda_2=\lambda_1$。 这与$\lambda_1\neq\lambda_2$矛盾,
故$\xi_1$和$\xi_2$线性无关。\\
(2)假设$\xi_1-\xi_2$是$A$的特征值,即存在$\lambda_0$,使得:
\begin{equation*}
  A(\xi_1-\xi_2)=\lambda_0(\xi_1-\xi_2)
\end{equation*}
由于$A\xi_1=\lambda_1\xi_1$以及$A\xi_2=\lambda_2\xi$,因此:
\begin{equation*}
  \lambda_1\xi_1-\lambda_2\xi_2=\lambda_0(\xi_1-\xi_2)
\end{equation*}
即$(\lambda_1-\lambda_0)\xi_1-(\lambda_2-\lambda_0)\xi_2=0$。 由(1)可知,
$\xi_1$和$\xi_2$是线性无关的,故
$\lambda_1-\lambda_0=\lambda_2-\lambda_0=0$
这与$\lambda_1\neq\lambda_2$矛盾。
故$\xi_1-\xi_2$不是$A$的特征值。\\
\textbf{习题 \ref{7.7} 解答:}\\
设$\lambda$是$A$的特征值,由于$f(A)=0$,那么$f(\lambda)=0$,即$\lambda$ 是$f(x)=0$ 的根,
 又由于$\lambda\neq0$,那么$a_0\neq0$。\\
\textbf{习题 \ref{7.8} 解答:}\\
通过题目可知,$A$的特征值为$-1,2,4$;因此
$A^3-5A^2$的特征值为$-6,-12,-16$,故$|A^3-5A^2|=-1152$。\\
\textbf{习题 \ref{7.9} 解答:}\\
由于$|A|=1\times2\times2=-2$,因此$A$ 可逆。故$|A^*|=|A|A^{-1}=-2A^{-1}$,即
\begin{equation*}
  A^*+3A=-2A^{-1}+3A
\end{equation*}
令$\varphi(x)=-\frac{2}{x}+3x$,则$\varphi(1)=1$,$\varphi(2)=5$,$\varphi(-1)=-1$。故
\begin{equation*}
  |A^*+3A|=|-2A^{-1}+3A|=\varphi(1)\times\varphi(2)\times\varphi(-1)=-5
\end{equation*}
\textbf{习题 \ref{7.10} 解答:}\\
由题意可知:
\begin{equation*}
  \begin{cases}
  \lambda_2+\lambda_3=5-\lambda_1=3\\
  \lambda_2\lambda_3=\frac{4}{\lambda_1}=2\\
  \end{cases}
\end{equation*}
解得
\begin{equation*}
  \begin{cases}
  \lambda_2=1\\
  \lambda_3=2
  \end{cases},
  \text{或者}
    \begin{cases}
  \lambda_2=2\\
  \lambda_3=1
  \end{cases},
\end{equation*}
\textbf{习题 \ref{7.11} 解答:}\\
假设存在常数$k_1,k_2,k_3$,使得
\begin{equation*}
  k_1\vec{\beta}+k_2A\vec{\beta}+k_3A^2\vec{\beta}=0
\end{equation*}
由于$\vec{\beta}=\vec{\alpha}_1+\vec{\alpha}_2+\vec{\alpha}_3$,即
\begin{equation*}
  k_1(\vec{\alpha}_1+\vec{\alpha}_2+\vec{\alpha}_3)+
  k_2A(\vec{\alpha}_1+\vec{\alpha}_2+\vec{\alpha}_3)+
  k_3A^2(\vec{\alpha}_1+\vec{\alpha}_2+\vec{\alpha}_3)=0
\end{equation*}
又由于$A\vec{\alpha}_1=\lambda_1\vec{\alpha}_1,A\vec{\alpha}_2=\lambda_2\vec{\alpha}_2,A\vec{\alpha}_3=\lambda_3\vec{\alpha}_3$,化简得到
\begin{equation*}
  (k_1+k_2\lambda_1+k_3\lambda_1^2)\vec{\alpha}_1+
  (k_1+k_2\lambda_2+k_3\lambda_2^2)\vec{\alpha}_2+
  (k_1+k_2\lambda_3+k_3\lambda_3^2)\vec{\alpha}_3)=0
\end{equation*}
由于$\vec{\alpha}_1,\vec{\alpha}_2,\vec{\alpha}_3$ 线性无关,故
\begin{equation}\label{aa}
\begin{cases}
  k_1+k_2\lambda_1+k_3\lambda_1^2=0\\
  k_1+k_2\lambda_2+k_3\lambda_2^2=0\\
  k_1+k_2\lambda_3+k_3\lambda_3^2=0
\end{cases}
\end{equation}
将\eqref{aa}可以看做是关于$k_1,k_2,k_3$ 的方程组,则系数矩阵的行列式是一个三阶的范德蒙行列式,
且因为$\lambda_1,\lambda_2,\lambda_3$ 是互不相等,从而
\begin{equation*}
\begin{vmatrix}
  1&\lambda_1&\lambda_1^2\\
  1&\lambda_2&\lambda_2^2\\
  1&\lambda_3&\lambda_3^2
\end{vmatrix}=(\lambda_1-\lambda_2)(\lambda_2-\lambda_3)(\lambda_3-\lambda_1)\neq0
\end{equation*}
因此$k_1=k_2=k_3=0$,故$\vec{\beta},A\vec{\beta},A^2\vec{\beta}$ 线性无关。\\
\textbf{习题 \ref{7.12} 解答:}\\
\begin{align*}
  |A-\lambda E|&=\begin{vmatrix}2-\lambda&0&0\\0&1-\lambda&-1\\0&-1&1-\lambda\end{vmatrix}\\
   & =(2-\lambda)[(1-\lambda)^2-1]=(2-\lambda)^2\lambda
\end{align*}
因此$A$的特征值为$\lambda_1=0,\lambda_2=2(\text{二重})$。 首先,考虑$A\vec{x}=\vec{0}$,
得到$\lambda_1$的特征子空间的基为$(0,1,1)^T$;考虑$(A-2E)\vec{x}=\vec{0}$,得到$\lambda_2$ 的
特征子空间的基为$(1,0,0)^T,(0,1,-1)^T$。\\
\textbf{习题 \ref{7.13} 解答:}\\
因为$|A-\lambda E|=(1-\lambda)(2-\lambda)(5-\lambda)$,故$A$ 有3个不同的特征值,从而有3 个不同的特征向量,故$A$
可以对角化,即存在可逆矩阵$P$,使得:
\begin{equation*}
P^{-1}AP=\begin{bmatrix}
  1&0&0\\
  0&2&0\\
  0&0&5
\end{bmatrix}=B
\end{equation*}
因此$A\sim B$。\\
\textbf{习题 \ref{7.14} 解答:}\\
由于$A\sim B$,故$tr(A)=tr(B)$,即$1+4+5=3+4+y$,因此$y=3$。
于是
\begin{equation*}
|B-\lambda E|=\begin{vmatrix}3-\lambda&\sqrt{2}&1\\ \sqrt{2}&4-\lambda&\sqrt{2}\\1&\sqrt{2}&3-\lambda\end{vmatrix}
=(6-\lambda)(2-\lambda)^2.
\end{equation*}
因此,$B$的特征值为$\lambda_1=2(\text{二重})$ 以及$\lambda_2=6$。\\
由于$A\sim B$,故$A$的特征值为$\lambda_1=2(\text{二重})$以及$\lambda_2=6$。
\begin{equation*}
  |A-6E|=\begin{vmatrix}-5&-1&1\\x&-2&-2\\-3&-3&-1\end{vmatrix}=8-4x
\end{equation*}
故$x=2$。\\
\textbf{习题 \ref{7.15} 解答:}\\
取$P=B$,则$P^{-1}BAP=AB$,因此$AB$与$BA$是相似的。\\
\textbf{习题 \ref{7.16} 解答:}\\
由$|A-\lambda E|=0$解得特征值$\lambda_1=\lambda_2=0$,$\lambda_3=1$。下面求对应于特征值的特征向量:
当$\lambda_1=\lambda_2=0$时,由方程组$A-\lambda E)\vec{x}=\vec{0}$得到对应的两个线性无关的特征向量:
\begin{equation}
  \vec{\alpha}_1=(0,1,1)^T;\vec{\alpha}_2=(-2,1,0)^T.
\end{equation}
当$\lambda_3=1$时,由方程组$A-\lambda E)\vec{x}=\vec{0}$得到对应的两个线性无关的特征向量:
\begin{equation}
  \vec{\alpha}_3=(1,0,0)^T.
\end{equation}
显然$\vec{\alpha}_1,\vec{\alpha}_2,\vec{\alpha}_3$ 线性无关,故$A$可对角化。令矩阵:
\begin{equation*}
 P=(\vec{\alpha}_1,\vec{\alpha}_2,\vec{\alpha}_3)=\begin{bmatrix}0&-2&1\\1&1&0\\1&0&0\end{bmatrix}
\end{equation*}
则$P$为所求相似变换矩阵,且
\begin{equation*}
 P^{-1}AP=\begin{bmatrix}0&0&0\\0&0&0\\0&0&1\end{bmatrix}
\end{equation*}
\textbf{习题 \ref{7.17} 解答:}\\
设$\lambda$是$A$的特征值,$\vec{\xi}\neq0$ 是相应的特征向量,则
\begin{equation*}
A\xi=(E+\vec{\alpha}\vec{\beta}^T)\vec{\xi}=\lambda\vec{\xi}
\end{equation*}
左乘以$\vec{\beta}^T$,得
\begin{equation*}
(1+\vec{\beta}^T\vec{\alpha})\vec{\beta}^T\vec{\xi}=\lambda\vec{\beta}^T\vec{\xi}
\end{equation*}
若$\vec{\beta}^T\vec{\xi}\neq0$,则$\lambda=1+\vec{\beta}^T\vec{\alpha}=2$;
若$\vec{\beta}^T\vec{\xi}=0$,则$A\vec{\xi}=\lambda\vec{\xi}$,故$\lambda=1$;
即$A$的特征值为$\lambda_1=1$,$\lambda_2=3$。
对于$\lambda_1=1$,即$\vec{\beta}^T\vec{\xi}=0$,故可取
\begin{eqnarray*}
  \vec{\xi}_1&=& (-b_2,b_1,0,\cdots,0)^T \\
  \vec{\xi}_2&=& (-b_3,0,b_1,\cdots,0)^T\\
  \cdots \\
  \vec{\xi}_{n-1} &=& (-b_n,0,0,\cdots,b_1)^T
\end{eqnarray*}
对于$\lambda_2=2$,可取$\vec{\xi}_n=\vec{\alpha}$。\\
(2)由于$A$有$n$个线性无关的特征向量,故可对角化,且
\begin{equation*}
P=(\vec{\xi}_1,\vec{\xi}_2,\cdots,\vec{\xi}_n),\Lambda=diag{1,1,\cdots,1,3}.
\end{equation*}
\textbf{习题 \ref{7.18} 解答:}\\
由于$A$有3个线性无关的特征向量且$\lambda=1$ 是$A$的2重特征值,故$A$的属于
特征值$\lambda=1$的线性无关的特征向量有2 个,从而$r(A-E)=3-2=1$。由于
\begin{equation*}
  A-E=\begin{bmatrix}2&-1&1\\0&x-1&y\\0&z&w-1\end{bmatrix}
\end{equation*}
故为使$r(A-E)=1$,则
\begin{equation*}
 \begin{cases}
 x-1=0\\
 y=0\\
 z=0\\
 w-1=0
 \end{cases}
\end{equation*}
解得
\begin{equation*}
 \begin{cases}
 x=1\\
 y=0\\
 z=0\\
 w=1
 \end{cases}
\end{equation*}
因此,$A=\begin{bmatrix}3&-1&1\\0&1&0\\0&0&1\end{bmatrix}$。\\
(2)从而确定$A$的另一个特征值为$\lambda_1=3$,其对应的特征向量$(1,0,0)^T$;
对于$\lambda_2=\lambda_3=1$,其对应的线性无关特征向量为$(1,2,0)^T,(1,0,-2)^T$。
因此,令$P=\begin{bmatrix}1&1&1\\0&2&0\\0&0&-2\end{bmatrix}$,则:
\begin{equation*}
 P^{-1}AP=\begin{bmatrix}2&0&0\\0&1&0\\0&0&1\end{bmatrix}
\end{equation*}
\textbf{习题 \ref{7.19} 解答:}\\
\begin{equation*}
|A-\lambda E|=\begin{vmatrix}a-\lambda&c\\b&d-\lambda\end{vmatrix}=\lambda^2-(a+d)\lambda+ad-bc
\end{equation*}
又由于$bc>0$,则$(a+d)^2-4(ad-bc)=(a-d)^2+4bc>0$,因此$A$有两个不同的特征值,即$A$ 可以对角化。\\
\textbf{习题 \ref{7.20} 解答:}\\
可以对角化。
\begin{equation*}
  P=\begin{bmatrix}1&1&1\\1&0&-1\\0&-1&2\end{bmatrix},
  P^{-1}AP=\begin{bmatrix}-2&0&0\\0&-2&0\\0&0&4\end{bmatrix}
\end{equation*}
\textbf{习题 \ref{7.21} 解答:}\\
设$\lambda$是$A$的特征值,由题意知,$\lambda^2+3\lambda=0$,
因此$A$的特征值只可能是$-3$和$0$ 是$A$ 的特征值。因为$A$是对称矩阵,所以$A$可以经过正交变换化为
对角矩阵$\Lambda$,由于$r(A)=2$,故$r(\Lambda)=2$,由此可知$A$的特征值为:
\begin{equation*}
  \lambda_1=\lambda_2=-3,\lambda_3=0
\end{equation*}
\textbf{习题 \ref{7.22} 解答:}\\
易知$A^n\vec{\alpha}_1=\lambda_1^n\vec{\alpha}_1(n=1,2,...)$,于是
\begin{equation*}
  B\vec{\alpha}_1=A^2\vec{\alpha}_1+2A\vec{\alpha}_1=(\lambda_1^2+2\lambda)\vec{\alpha}_1=3\vec{\alpha}_1
\end{equation*}
因此$\vec{\alpha}_1$是$B$的特征向量,$B$ 的特征值可以由$A$的特征值以及$A$ 和$B$ 的关系得到,
即$B$的全部特征值为$3,8,3$。由于$A$是实对称的矩阵,那么$B$也是实对称的,设$B$ 属于$\lambda_2=3$的特征值对应
的特征向量为$(x_1,x_2,x_3)^T$,则有$x_1+x_2+x_3=0$,求得$B$属于$\lambda_2=3$的特征值对应的特征向量为:
\begin{equation*}
  \vec{\alpha}_2=(1,0,-1)^T,\vec{\alpha}_3=(1,-1,0)^T
\end{equation*}
\textbf{习题 \ref{7.23} 解答:}\\
\begin{equation*}
 |A-\lambda E| = \begin{vmatrix}1-\lambda&-2&0\\-2&2-\lambda&-2\\0&-2&3-\lambda\end{vmatrix}
               =(\lambda+1)(\lambda-2)(\lambda-5)
\end{equation*}
因此$A$的特征值为$\lambda_1=-1,\lambda_2=2,\lambda_3=5$。
对于$\lambda_1$,考虑方程组$(A+E)\vec{x}=\vec{0}$,
解得其对应的单位特征向量为$\vec{\alpha}_1=(\frac{2}{3},\frac{2}{3},\frac{1}{3})^T$;\\
对于$\lambda_2$,考虑方程组$(A+E)\vec{x}=\vec{0}$,
解得其对应的单位特征向量为$\vec{\alpha}_2=(-\frac{2}{3},\frac{1}{3},\frac{2}{3})^T$;\\
对于$\lambda_3$,考虑方程组$(A+E)\vec{x}=\vec{0}$,
解得其对应的单位特征向量为$\vec{\alpha}_3=(\frac{1}{3},-\frac{2}{3},\frac{2}{3})^T$。
故令$Q=\begin{bmatrix}\frac{2}{3}&\frac{2}{3}&\frac{1}{3}\\-\frac{2}{3}&\frac{1}{3}&\frac{2}{3}\\
\frac{1}{3}&-\frac{2}{3}&\frac{2}{3}\end{bmatrix}$,那么
\begin{equation*}
  Q^TAQ=\begin{bmatrix}-1&0&0\\0&2&0\\0&0&5\end{bmatrix}
\end{equation*}
\textbf{习题 \ref{7.24} 解答:}\\
\begin{equation*}
  Q=\begin{bmatrix}\frac{1}{3}&\frac{2}{\sqrt{5}}&\frac{2}{\sqrt{5}}
       \\ \frac{2}{3}&-\frac{1}{\sqrt{5}}&0\\-\frac{2}{3}&0&\frac{1}{\sqrt{5}}\end{bmatrix},
  Q^TAQ=\begin{bmatrix}-7&0&0\\0&2&0\\0&0&2\end{bmatrix}
\end{equation*}
\textbf{习题 \ref{7.25} 解答:}\\
设$A$的3个正交特征向量是$\vec{\xi}_1,\vec{\xi}_2,\vec{\xi}_3$,
单位化后是$\vec{\eta}_1,\vec{\eta}_2,\vec{\eta}_3$ 仍是$A$的特征向量,
令$Q=(\vec{\eta}_1,\vec{\eta}_2,\vec{\eta}_3)$,则$Q$是正交矩阵,$Q^{-1}=Q^T$,
由于$A$是3阶实矩阵且有3个正交的特征向量,从而$A$可以对角化,即
\begin{equation*}
  Q^{-1}AQ=\Lambda
\end{equation*}
其中$\Lambda$为对角阵,因此$A=Q\Lambda Q^T$,故
\begin{equation*}
  A^T=Q\Lambda^TQ^T=Q\Lambda Q^T=A
\end{equation*}
\textbf{习题 \ref{7.26} 解答:}\\
令$Q=(\vec{\alpha}_1,\vec{\alpha}_2,\cdots,\vec{\alpha}_n)$,则
\begin{equation*}
  Q^{-1}AQ=\Lambda=diag\{\lambda_1,\lambda_2,\cdots,\lambda_n\}.
\end{equation*}
于是$A=Q\Lambda Q^T=\lambda_1\vec{\alpha}_1\vec{\alpha}_1^T+\cdots+\lambda_n\vec{\alpha}_n\vec{\alpha}_n^T$。\\
\textbf{习题 \ref{7.27} 解答:}\\
由于$A$是$n$阶实对称矩阵,故存在正交阵$Q$以及对角矩阵$\Lambda$,使得
\begin{equation*}
  A=Q\Lambda Q^T
\end{equation*}
故$A^{k}=Q\Lambda^kQ^T$,即$\Lambda^k=O$,所以$\Lambda=O$。 从而$A=O$。\\
\textbf{习题 \ref{7.28} 解答:}\\
由题意知:
\begin{eqnarray*}
  x_{n+1} &=& x_n+qy_n-px_n=(1-p)x_n+qy_n \\
  y_{n+1} &=& y_n+px_n-qy_n=px_n+(1-q)y_n
\end{eqnarray*}
可用矩阵表示为:
\begin{equation*}
\begin{bmatrix}x_{n+1}\\y_{n+1}\end{bmatrix}=
\begin{bmatrix}1-p&q\\p&1-q\end{bmatrix}
\begin{bmatrix}x_n\\y_n\end{bmatrix}
\end{equation*}
即
\begin{equation*}
A=\begin{bmatrix}1-p&q\\p&1-q\end{bmatrix}
\end{equation*}
(2)由于$\begin{bmatrix}x_{n+1}\\y_{n+1}\end{bmatrix}=A\begin{bmatrix}x_n\\y_n\end{bmatrix}$,
可知$\begin{bmatrix}x_{n}\\y_{n}\end{bmatrix}=A^n\begin{bmatrix}x_0\\y_0\end{bmatrix}$。
由
\begin{equation*}
  |A-\lambda E|=\begin{vmatrix}1-p-\lambda&q\\p&1-q-\lambda\end{vmatrix}=(\lambda-1)(\lambda-1+p+q)
\end{equation*}
得到$A$的特征值为$\lambda_1=1,\lambda_2=r$,其中$r=1-p-q$。
得到相应的特征向量为$\vec{\alpha}_1=(q,p)^T,\vec{\alpha}_2=(-1,1)^T$。
令$P=\begin{bmatrix}q&-1\\p&1\end{bmatrix}$,$\Lambda=\begin{bmatrix}1&0\\0&r\end{bmatrix}$,故
\begin{equation*}
  A^n=P\Lambda^nP^{-1}=\frac{1}{p+q}\begin{bmatrix}q+pr^n&q-qr^n\\p-pr^n&p+qr^n\end{bmatrix}
\end{equation*}
故
\begin{equation*}
 \begin{bmatrix}x_{n}\\y_{n}\end{bmatrix}=\frac{1}{2(p+q)}\begin{bmatrix}2q+(p-q)r^n\\2p+(q-p)r^n\end{bmatrix}
\end{equation*}

%%%%%%%%%%%%%%%%%%%%%%%%%%%%%%%%%%%%%%%%%%%%%%%%%%%%%%%%%%%%%%%%%%%%%%%%%%%%%%%%%%%%
%%%%%%%%%%%%%%%%%%%%%%%%%%%%%%%%%%%%%%%%%%%%%%%%%%%%%%%%%%%%%%%%%%%%%%%%%%%%%%%%%%%%
