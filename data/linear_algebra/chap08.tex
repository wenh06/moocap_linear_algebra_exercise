\chapter{矩阵与变换}

\section{知识点解析}
\begin{Def}
给定矩阵$A\in M_{m\times n}$, 则如下定义由$A$确定的矩阵映射,
\begin{displaymath}\begin{aligned}
\varphi_A: & \mathbb{R}^n\ \rightarrow\ \mathbb{R}^m\\
& \vec{\alpha}\ \mapsto\ A\vec{\alpha}.\end{aligned}\end{displaymath}
特别地,当$m=n$时, $\varphi_A$称为矩阵变换.
\end{Def}

\begin{thm}
任意$\mathbb{R}^n$到$ \mathbb{R}^m$的线性映射$\sigma$均为矩阵映射, 且其表示矩阵的第 $j$ 列就是第 $j$ 个自然基$\vec{e}$在$\sigma$作用下的像.

\end{thm}

\begin{thm}
在$\mathbb{R}^3$中, 以单位向量$\vec{l} =(a,b,c)^T$为旋转轴, 旋转角度为$\theta$ 的旋转变换的变换矩阵为:
\begin{displaymath}
T_{\vec{l},\theta}=\begin{bmatrix} a^2+(1-a^2)\cos\theta & ab(1-\cos\theta)-c\sin\theta & ac(1-\cos\theta)+b\sin\theta\\
ab(1-\cos\theta)+c\sin\theta & b^2+(1-b^2)\cos\theta & bc(1-\cos\theta)-a\sin\theta  \\
ac(1-\cos\theta)-b\sin\theta& bc(1-\cos\theta)+a\sin\theta & c^2+(1-c^2)\cos\theta\end{bmatrix}\end{displaymath}
\end{thm}


\begin{Def}
设$\varphi_B:\mathbb{R}^l \rightarrow \mathbb{R}^n, \varphi_A:\mathbb{R}^n \rightarrow \mathbb{R}^m$为矩阵映射, 则$\varphi_A$与$\varphi_B$的复合映射$\varphi_A\circ \varphi_B$如下定义:
\begin{displaymath}
 \varphi_A\circ \varphi_B(\vec{x}):=\varphi_A( \varphi_B(\vec{x}))\ \ \ (\forall \vec{x}\in \mathbb{R}^l).\end{displaymath}
\end{Def}

\begin{Def}
设$A$为$n$阶方阵, 若有子空间$W\in\mathbb{R}^n$, 满足$\forall \vec{x}\in W$, 均有$A\vec{x}\in W$ , 则称$W$ 是矩阵变换$A$的不变子空间.
\end{Def}

\begin{Def}
在$\mathbb{R}^n$, 设$(\vec{\alpha},\vec{\beta})$表示标准内积, $\varphi_A$ 为一个矩阵变换, 若$\forall \vec{\alpha},\vec{\beta}$, 均有
$$(\varphi_A(\vec{\alpha}),\varphi_A(\vec{\beta}))=(A\vec{\alpha}  ,A\vec{\beta})=(\alpha,\beta).$$
则称$\varphi_A$为正交变换.
\end{Def}

\begin{thm}
在$\mathbb{R}^n$中, $\varphi_A$为矩阵变换, 则下述条件等价
\begin{enumerate}
\item $\varphi_A$为正交变换(保内积);
\item $|A\vec{\alpha}|=|\vec{\alpha}|(\forall \vec{\alpha}\in \mathbb{R}^n)$(保长度);
\item $A$为正交矩阵.
\end{enumerate}
\end{thm}

\begin{thm}
在$\mathbb{R}^2$与$\mathbb{R}^3$中, 第一类正交变换必为旋转变换. 第二类正交变换必为反射变换, 或某个反射变换与某个旋转变换的复合.
\end{thm}

%%%%%%%%%%%%%%%%%%%%%%%%%%%%%%%%%%%%%%%%%%%%%%%%%%%%%%%%%%%%%%%%%%%%%%%%%%%%%%%%%%%%

\section{例题讲解}

\begin{eg}
令$A=\begin{bmatrix}0&1\\1&0\end{bmatrix}$, 求下列列向量在矩阵变换$A$作用下的像,
\begin{displaymath}
\vec{e}_1=\begin{bmatrix}1\\0\end{bmatrix},\ \vec{e}_2=\begin{bmatrix}0\\1\end{bmatrix},\ \vec{\alpha}=\begin{bmatrix}-1\\-2\end{bmatrix}.\end{displaymath}
\end{eg}
解: \begin{displaymath}\begin{aligned}
&A\vec{e}_1=\begin{bmatrix}0&\ 1\\1&\ 0\end{bmatrix}\ \begin{bmatrix}1\\0\end{bmatrix}
=\begin{bmatrix}0\\1\end{bmatrix};\\
&A\vec{e}_2=\begin{bmatrix}0&\ 1\\1&0\end{bmatrix}\ \begin{bmatrix}0\\1\end{bmatrix}
=\begin{bmatrix}1\\0\end{bmatrix};\\
&A\vec{\alpha}=\begin{bmatrix}0&\ 1\\1&0\end{bmatrix}\ \begin{bmatrix}-1\\-2\end{bmatrix}
=\begin{bmatrix}-2\\-1\end{bmatrix}.
\end{aligned}\end{displaymath}

\begin{eg}
(平移变换)设变换$\varphi_:  \mathbb{R}^2\ \rightarrow\ \mathbb{R}^2$ 将所有的点向右移动$a_1$个单位, 同时向上移动$a_2$个单位, 那么可以用向量表示为: \begin{displaymath}
\varphi(\vec{x})=\varphi\begin{bmatrix}x_1\\x_2\end{bmatrix}=\begin{bmatrix} x_1+a_1\\x_2+a_2\end{bmatrix}=\begin{bmatrix}x_1\\x_2\end{bmatrix}+\begin{bmatrix}
a_1\\a_2\end{bmatrix}=\vec{x}+\vec{\alpha},\end{displaymath}
其中, $\vec{\alpha}=\begin{bmatrix}a_1\\a_2\end{bmatrix}$. 由矩阵变换的定义, 可以看出平移变换一般并不是矩阵变换. 而这一点, 从下一小节矩阵映射(变换)的性质也可看出.
\end{eg}

\begin{eg}
在$\mathbb{R}^3$中, 矩阵变换把线段变成什么图形?
\end{eg}
解: 设线段的端点分别为$X$和$Y$, 则以线段上点为终点的向量可表示为
\begin{displaymath}
\lambda\vec{OX}+(1-\lambda)\vec{OY}\ \ \ (\forall \lambda\in[0,1])
\end{displaymath}
如果记$\vec{OX}=\vec{\alpha}, \vec{OY}=\vec{\beta}$, 则在矩阵变换$A$的作用下, 可得
\begin{displaymath}
A(\lambda\vec{\alpha}+(1-\lambda)\vec{\beta})=\lambda A\vec{\alpha}+(1-\lambda)A \vec{\beta}\ \ \ (\forall  \lambda\in[0,1])
\end{displaymath}
这依然是一条线段, 两个端点分别为向量$A\vec{\alpha}$与$A \vec{\beta}$的终点.

因此矩阵变换将线段变成线段, 或者变成点 (线段的退化情形).

\begin{eg}
设$A=\begin{bmatrix}1&2\\2&1\end{bmatrix}$, 在平面$\mathbb{R}^2$中三点:
$$E=\begin{bmatrix}1\\1\end{bmatrix}, \ \ F=\begin{bmatrix}-1\\0\end{bmatrix}, \ \
G=\begin{bmatrix}1\\-1\end{bmatrix}.$$
构成的$\triangle EFG$, 在$A$作用下变换成什么图形?
\end{eg}
解: 由于矩阵变换将线段变为线段,故$A$作用将$\triangle EFG$映为$\triangle E^{'}F^{'}G^{'}$
$$E^{'}=A\begin{bmatrix}1\\1\end{bmatrix}=\begin{bmatrix}3\\3\end{bmatrix},\ \ F^{'}=A\begin{bmatrix}-1\\0\end{bmatrix}=\begin{bmatrix}-1\\-2\end{bmatrix},\ \ G^{'}=A\begin{bmatrix}1\\-1\end{bmatrix}=\begin{bmatrix}-1\\1\end{bmatrix}.$$


\begin{eg}
对$\mathbb{R}^3$中的线投影变换$P_{\vec{l}}=\vec{l}\vec{l}^T$, 其中$\vec{l}$为单位向量.
\begin{displaymath}P_{\vec{l}}\vec{l}=(\vec{l}\vec{l}^T)\vec{l}
=\vec{l}(\vec{l}^T\vec{l})=\vec{l}\ \Rightarrow\ \lambda=1.\end{displaymath}

特征子空间$L(\vec{l})$维数为1.

若$\vec{v}$与$\vec{l}$正交, 有

$$P_{\vec{l}}\vec{v}=(\vec{l}\vec{l}^T)\vec{v}
=\vec{l}(\vec{l}^T\vec{v})=\vec{l}\dot 0=\vec{0}\ \Rightarrow \ \lambda=0.$$

特征子空间为$L(\vec{l})^{\bot}$, 维数为$n-1$.

\end{eg}

\begin{eg}
对$\mathbb{R}^3$中的面投影变换$P_{\pi}=I-\vec{n}\vec{n}^T$, 其中$\vec{n}$为投影平面$\pi$的单位法向量.
$$P_{\pi}\vec{n}=\vec{n}-\vec{n}(\vec{n}^T\vec{n})=\vec{0}\ \Rightarrow \ \lambda_1=0.$$

特征子空间为$L(\vec{n})$, 1维.

若$\vec{v}$与$\vec{n}$正交, 则$$P_{\pi}\vec{v}=\vec{v}-\vec{n}(\vec{n}^T\vec{v})=\vec{v}-\vec{0}=\vec{v}. \Rightarrow \ \lambda_2=1.$$

特征子空间为$L(\vec{n})^{\bot}$, 2维.

\end{eg}

\begin{eg}
$\mathbb{R}^3$中的线反射变换: $R_{\vec{l}}=-I+2\vec{l}\vec{l}^T$. 其中$\vec{l}$为反射直线的单位向量.
$$R_{\vec{l}}\vec{l}=-\vec{l}+2\vec{l}\vec{l}^T\vec{l}=\vec{l}\ \Rightarrow \  \lambda_1=1$$

特征子空间为$L(\vec{l})$, 1维.

若$\vec{v}$与$\vec{l}$正交, 有$R_{\vec{l}}\vec{v}=-\vec{v}$. 所以$\lambda_2=-1$, 特征子空间为$L(\vec{l})^{\bot}$, 2维.
\end{eg}

\begin{eg}
对$\mathbb{R}^3$中的面反射变换$R_{\pi}=I-2\vec{n}\vec{n}^T$, 其中$\vec{n}$为反射平面$\pi$的单位法向量.

$$R_{\pi}\vec{n}=\vec{n}-2\vec{n}\vec{n}^T\vec{n}=-\vec{n}\ \Rightarrow \ \lambda_1=-1.$$

特征子空间为$L(\vec{n})$, 1维.

若$\vec{v}$与$\vec{n}$正交, 则$$R_{\pi}\vec{v}=\vec{v}-\vec{0}=\vec{v}\ \Rightarrow \ \lambda_2=1$$

特征子空间为$L(\vec{n})^{\bot}$, 2维.

\end{eg}

\begin{eg}
对$\mathbb{R}^2$中的旋转变换$T_{\theta}=\begin{bmatrix} \cos\theta&-\sin\theta\\ \sin\theta&\cos\theta\end{bmatrix}.$

特征多项式为
$$|\lambda I-T_{\theta}|=\lambda^2-(2\cos\theta)\lambda +1.$$
特征值$\lambda_{1,2}=\cos\theta\pm\sqrt{1-\cos^2\theta}i.$

当$\theta\not=0,\pi$时, $T_{\theta}$无实特征值, 从而没有实特征向量. 这说明, 此时$T_{\theta}$在$\mathbb{R}^2$中, 除$\vec{0}$之外, 没有向量是保持方向不变或方向相反的.
\end{eg}

\begin{eg}
用坐标系替换的方式,重新推导$\mathbb{R}^2$内线投影变换的矩阵公式.

设$\vec{l}$为投影直线的单位向量, 选取$\{\vec{l},\vec{l}^{\bot}\}$为新的坐标系, 其中
$$\vec{l}=\begin{bmatrix}l_1\\l_2\end{bmatrix},\ \vec{l}=\begin{bmatrix}-l_2\\l_1\end{bmatrix}$$
故过渡矩阵为
$$P=(\vec{l},\vec{l}^{\bot})=\begin{bmatrix} l_1&-l_2\\l_2&l_1\end{bmatrix}$$
设该变换在旧基$\{\vec{e}_1,\vec{e}_2\}$与新基$\{\vec{l},\vec{l}^{\bot}\}$下的矩阵表示分别为$A$和$B$. 于是
有$B=P^{-1}AP$. 易知$B=\begin{bmatrix}1&0\\0&0\end{bmatrix}$, 从而有
$$A=PBP^{-1}=\begin{bmatrix}l_1&-l_2\\l_2&l_1\end{bmatrix}\begin{bmatrix}1&0\\
0&0\end{bmatrix}\begin{bmatrix}l_1&l_2\\-l_2&l_1\end{bmatrix}=\begin{bmatrix}
l_1^2&l_1l_2\\l_1l_2&l_2^2\end{bmatrix}=\vec{l}\vec{l}^T.$$
\end{eg}

\begin{eg}
设平行光束的方向为$\vec{\beta}=\begin{bmatrix}b_1\\b_2\end{bmatrix}$, 投影直线的方向向量为$\vec{\alpha}=\begin{bmatrix}a_1\\a_2\end{bmatrix}$. 设$\vec{\alpha}$与$\vec{\alpha}$不共线(也不一定正交). 那么, 沿$\vec{\beta}$方向在$\vec{\alpha}$上的投影变换所对应的矩阵是什么?

用坐标系替换的方法来解决. 以$\{\vec{\alpha},\vec{\beta}\}$为新坐标系, 于是过渡矩阵为
$$P=(\vec{\alpha},\vec{\beta})=\begin{bmatrix}a_1&b_1\\a_2&b_2\end{bmatrix}.$$
易知在新坐标系下投影变换的矩阵为$\begin{bmatrix}1&0\\0&0\end{bmatrix}$, 故在原基(自然基)下的矩阵为
$$A=P\begin{bmatrix}1&0\\0&0\end{bmatrix}P^{-1}=(a_1b_2-a_2b_1)^{-1}\begin{bmatrix}
a_1b_1&-a_1b_1\\a_2b_2&-a_2b_1\end{bmatrix}.$$
\end{eg}

\begin{eg}
对$\mathbb{R}^n$中的旋转变换,
$$R_{\theta}=\begin{bmatrix}\cos\theta&-\sin\theta\\ \sin\theta &\cos\theta\end{bmatrix},$$
易知, $\forall \theta\in [0, 2\pi), \vec{\alpha}_1=\begin{bmatrix} \cos\theta\\ \sin\theta\end{bmatrix}, \vec{\alpha}_2=\begin{bmatrix} -\sin\theta\\ \cos\theta\end{bmatrix}.$ 总满足
$$(\vec{\alpha}_i, \vec{\alpha}_j)=\delta_{ij}(1\leq i, j\leq 2).$$
故$R_{\theta}$为正交阵, 且$\det R_{\theta}=1$, 所以旋转变换为正交变换.
\end{eg}

\begin{eg}
对$\mathbb{R}^n$中的线反射变换$R_{\vec{l}}=-I+2\vec{l}\vec{l}^T$(简记为$R$), 其中$\vec{l}$为反射轴方向的单位向量. 由于
\begin{displaymath}\begin{aligned}
RR^T&=(-I+2\vec{l}\vec{l}^T)(-I+2\vec{l}\vec{l}^T)^T\\
&=I^2-4\vec{l}\vec{l}^T+4(\vec{l}\vec{l}^T)(\vec{l}\vec{l}^T)\\
&=I-4\vec{l}\vec{l}^T+4\vec{l}(\vec{l}^T\vec{l})\vec{l}^T\\
&=I-4\vec{l}\vec{l}^T+4\vec{l}\vec{l}^T=I\end{aligned}\end{displaymath}
其中$\vec{l}^T\vec{l}=(\vec{l},\vec{l})=1$. 故$R_{\vec{l}}$为正交阵. 线反射变换为正交变换.
\end{eg}

\begin{eg}
对$\mathbb{R}^n$中的面反射变换, $R_{\pi}=I-2\vec{n}\vec{n}^T$, 其中$\vec{n}$为反射平面$\pi$的单位法向量. 由于$R_{\pi}=-R_{\vec{n}}$. 由上例可知, $R_{\pi}$也为正交变换, 且将$\vec{n}$扩充为$\mathbb{R}^3$的一组标准正交基$\{\vec{n}=\vec{n}_1,\vec{n}_2,\vec{n}_3\}$后, $R_{\pi}$可正交对角化. 即
$$Q^TR_{\pi}Q=\begin{bmatrix} -1&&\\&1&\\&&1\end{bmatrix}.$$
其中$Q=(\vec{n}_1,\vec{n}_2,\vec{n}_3)$为正交阵. 从而, 面反射变换$R_{\pi}$ 为正交变换, 且$\det R_{\pi}=-1$.
\end{eg}

%%%%%%%%%%%%%%%%%%%%%%%%%%%%%%%%%%%%%%%%%%%%%%%%%%%%%%%%%%%%%%%%%%%%%%%%%%%%%%%%%%%

\section{课后习题}
\begin{ex}\label{8.1}
1. 以下哪些变换不是矩阵变换?\\
A. 关于点$(1,1)$的旋转变换.\\		
B. 关于直线$y=kx$的对称变换.\\
C. 关于向量$(3,4)$的平移变换.\\		
D. 在$y$轴上的投影变换.
\end{ex}

\begin{ex}\label{8.2}
恒等变换是(    ).\\
A. 保持所有向量不变的矩阵变换.\\
B. 保持所有向量方向不变的矩阵变换.\\
C. 保持所有向量长度不变的矩阵变换.\\
D. 保持所有向量夹角不变的矩阵变换.
\end{ex}

\begin{ex}\label{8.3}
矩阵变换$\begin{bmatrix}1&0\\0&-2\end{bmatrix}$ 的作用是在$x_1$方向上(~~~~),(~~~~).
\end{ex}

\begin{ex}\label{8.4}
设$\vec{\alpha}_0=\vec{OO^{'}}$, 则向量$\vec{x}$作关于$O^{'}$的中心对称是(~~~~).(用$\vec{\alpha}_0,\vec{x}$填写)
\end{ex}

\begin{ex}\label{8.5}
在$\mathbb{R}^2$中,求解满足如下矩阵变换的矩阵.\\
(1) 关于$y$轴轴对称变换.\\
(2) 关于原点逆时针旋转90度.\\
(3) 为关于直线$x_1=-x_2$的对称变换.\\
(4) 在$x$轴上的投影.\\
(5) 关于原点顺时针旋转90度.\\
(6) 关于原点顺时针旋转45度.
\end{ex}

\begin{ex}\label{8.6}
设过原点的直线L的单位方向向量为$\vec{l}_0$ (列向量),则向量$\vec{x}$ 作关于$L$ 的轴对称后变为(     ).\\
A. $(I-2\vec{l}_0 \vec{l}_0^T )\vec{x}$  \\
B. $(I+2\vec{l}_0 \vec{l}_0^T )\vec{x}$\\
C. $(-I-2\vec{l}_0 \vec{l}_0^T )\vec{x}$  \\
D. $(-I+2\vec{l}_0 \vec{l}_0^T )\vec{x}$
\end{ex}

\begin{ex}\label{8.7}
在$\mathbb{R}^2$中,给定列向量$\vec{l}$ ,设$L$为过原点,方向为$\vec{l}$的直线。求$L$的线反射变换矩阵$R_{\vec{l}}$.\\
(1) $\vec{l}=\begin{bmatrix}0\\1\end{bmatrix}$\\
(2) $\vec{l}=\begin{bmatrix}1\\0\end{bmatrix}$\\
(3) $\vec{l}=\begin{bmatrix}1\\1\end{bmatrix}$\\
(4) $\vec{l}=\begin{bmatrix}-3\\4\end{bmatrix}$\\
(5) $\vec{l}=\begin{bmatrix}12\\-5\end{bmatrix}$
\end{ex}

\begin{ex}\label{8.8}
(1) 设过原点的平面$\pi$的单位法向量为$\vec{n}_0$, 则向量$\vec{x}$作关于$\pi$ 的镜面对称后变为(     ).\\
A. $(-I-2\vec{n}_0\vec{n}_0^T )\vec{x}$\\
B. $(-I+2\vec{n}_0\vec{n}_0^T )\vec{x}$\\
C. $(I-2\vec{n}_0\vec{n}_0^T )\vec{x}$\\
D. $(I+2\vec{n}_0\vec{n}_0^T )\vec{x}$\\
(2). 设$\vec{l}_0$为单位向量, 则向$\vec{l}_0$ 方向投影对应的线投影矩阵为(     ).\\
A. $-\vec{l}_0 \vec{l}_0^T$\\
B. $\vec{l}_0 \vec{l}_0^T$\\
C. $I-\vec{l}_0 \vec{l}_0^T$\\
D. $I-\vec{l}_0 \vec{l}_0^T$\\
(3). 设过原点的平面$\pi$的单位法向量为$\vec{n}_0$, 则关于$\pi$的投影对应的面投影矩阵为(      ).\\
A. $I-\vec{n}_0 \vec{n}_0^T$\\
B. $I+\vec{n}_0 \vec{n}_0^T$\\
C. $-\vec{n}_0 \vec{n}_0^T$\\
D. $-\vec{n}_0 \vec{n}_0^T$
\end{ex}

\begin{ex}\label{8.9}
平面上绕原点逆时针旋转$\theta$角对应的旋转矩阵为(      ).\\
A.  $\begin{bmatrix}-\cos\theta&\sin\theta \\ \sin\theta&\cos\theta\end{bmatrix}$\\
B.  $\begin{bmatrix}\cos\theta&-\sin\theta \\ \sin\theta&\cos\theta\end{bmatrix}$\\
C. $\begin{bmatrix}\cos\theta&\sin\theta \\ -\sin\theta&\cos\theta\end{bmatrix}$\\
D. $\begin{bmatrix}\cos\theta&\sin\theta \\ \sin\theta&-\cos\theta\end{bmatrix}$
\end{ex}

\begin{ex}\label{8.10}
$\phi_A,\phi_B$分别为对应于2 阶矩阵$A$,$B$ 的映射,记$\phi_C=\phi_A\circ\phi_B$. 任取向量$\vec{x}=(x_1,x_2 )^T$,
$\phi_B\begin{bmatrix}x_1\\x_2\end{bmatrix}=\begin{bmatrix}x_1+2x_2\\2x_1+3x_2\end{bmatrix}$,
而$\phi_C\begin{bmatrix}x_1\\x_2\end{bmatrix}=\begin{bmatrix}2x_1+3x_2\\5x_1+8x_2\end{bmatrix}$,则矩阵$A$为(     ).\\
A.  $\begin{bmatrix}0&1\\-1&2\end{bmatrix}$\\
B.  $\begin{bmatrix}0&1\\1&2\end{bmatrix}$\\
C.  $\begin{bmatrix}12&7\\31&18\end{bmatrix}$\\
D.  $\begin{bmatrix}4&7\\-1&-2\end{bmatrix}$
\end{ex}

\begin{ex}\label{8.11}
11. (1) 计算$A^{10}\cdot\vec{b}$ 其中$A=\begin{bmatrix}0&1\\-2&3\end{bmatrix},\vec{b}=\begin{bmatrix}3\\4\end{bmatrix}$.\\
(2) 计算$A^4\cdot\vec{b}$,其中$A=\begin{bmatrix}1&2\\2&1\end{bmatrix},\vec{b}=\begin{bmatrix}6\\-4\end{bmatrix}$.
\end{ex}

\begin{ex}\label{8.12}
设$\vec{l}=\begin{bmatrix}1\\1\end{bmatrix}$,$P_{\vec{l}}$是$\mathbb{R}^2$ 中对$\vec{l}$ 的线投影变换矩阵,求$P_{\vec{l}}^{2016}\begin{bmatrix}4\\2\end{bmatrix}$.
\end{ex}

\begin{ex}\label{8.13}
设$\vec{l}=\begin{bmatrix}3\\4\\5\end{bmatrix}$,$P_{\vec{l}}$是$\mathbb{R}^3$ 中对$\vec{l}$的线投影变换矩阵,求$P_{\vec{l}}^{314}\begin{bmatrix}50\\50\\50\end{bmatrix}$.
\end{ex}

\begin{ex}\label{8.14}
设$\vec{n}=\begin{bmatrix}1\\2\\2\end{bmatrix}$,$R_{\pi}$是$\mathbb{R}^3$ 中对$\pi$的面反射变换矩阵,其中$\vec{n}$是$\pi$ 的法向量.
求$R_{\pi}^{233}\begin{bmatrix}-9\\-9\\-9\end{bmatrix}$.
\end{ex}

\begin{ex}\label{8.15}
设$\vec{n}=\begin{bmatrix}2*\\*2\\1**\end{bmatrix}$,$R_\pi$是$\mathbb{R}^3$ 中对$\pi$的面反射变换矩阵,
其中$\vec{n}$是$\pi$的法向量. 但是法向量$\vec{n}$已经模糊辨认不清. 求$R_{\pi}^{666}\begin{bmatrix}1\\2\\3\end{bmatrix}$.
\end{ex}

\begin{ex}\label{8.16}
判断下列命题真假.\\
(1)从一组基到另一组基的过渡矩阵可能是奇异阵.\\
(2)从一组基到任意一组基有且仅有一个过渡矩阵.\\
(3)同一个线性变换在不同基下的矩阵相似.
\end{ex}

\begin{ex}\label{8.17}
判断下列命题真假.\\
(1)刚体变换(在变换前后能保持任意两点距离,两直线夹角不变的变换)一定是正交变换.\\
(2)正交变换的逆变换也为正交变换.\\
(3)两个正交变换的加法也为正交变换.\\
(4)两个正交变换的乘法也为正交变换.
\end{ex}

%%%%%%%%%%%%%%%%%%%%%%%%%%%%%%%%%%%%%%%%%%%%%%%%%%%%%%%%%%%%%%%%%%%%%%%%%%%%%%%%%%%%%

\section{习题答案}

\textbf{习题 \ref{8.1} 解答:}\\
A, C.(矩阵变换必将零向量映为零向量,只要观察变换对零向量的作用即可)\\
\textbf{习题 \ref{8.2} 解答:}\\
A. (显然,由定义所得.)\\
\textbf{习题 \ref{8.3} 解答:}\\
保持不变, 在$x_2$反方向上伸长为原来的2 倍.\\
\textbf{习题 \ref{8.4} 解答:}\\
$2\vec{\alpha}_0-\vec{x}$. (由中点公式,$\vec{x}+\vec{y}(\text{设为答案})=2\vec{\alpha}_0$,得证.)\\
\textbf{习题 \ref{8.5} 解答:}\\
(1)设该变换为$\eta$,将$(x,y)^T$ 变为$(-x,y)^T$. 可以看做
\begin{eqnarray*}
    \eta(x) &=& -1\cdot x+0\cdot y \\
    \eta(y) &=& 0\cdot x+1\cdot y
\end{eqnarray*}
写成矩阵形式为
\begin{equation*}
\eta\begin{bmatrix}x\\y\end{bmatrix}=
\begin{bmatrix}-1&0\\0&1)\end{bmatrix}
\begin{bmatrix}x\\y\end{bmatrix}
\end{equation*}
所以η的变换矩阵为$\begin{bmatrix}-1&0\\0&1\end{bmatrix}$.\\
(2) 我们利用复数工具,将$(x,y)^T$视为$x+yi$,
则平面上的点绕原点逆时针旋转90度即等效于对应的复数乘以复数$i$,结果为$-y+xi$.
由(1)的做法,该矩阵变换为$\begin{bmatrix}0&-1\\1&0\end{bmatrix}$.\\
(3) 方法同上,根据
\begin{eqnarray*}
    \eta(x) &=& - y \\
    \eta(y) &=& - x
\end{eqnarray*}
得变换矩阵为$\begin{bmatrix}0&-1\\-1&0\end{bmatrix}$.\\
(4) 方法同上,根据
\begin{eqnarray*}
    \eta(x) &=& x \\
    \eta(y) &=& 0
\end{eqnarray*}
得变换矩阵为$\begin{bmatrix}1&0\\0&0\end{bmatrix}$.\\
(5)  方法同上,根据$(x+yi)(-i)=y-xi$
\begin{eqnarray*}
    \eta(x) &=& y \\
    \eta(y) &=& -x
\end{eqnarray*}
得变换矩阵为$\begin{bmatrix}0&1\\-1&0\end{bmatrix}$.\\
(6)方法同上,
根据$(x+yi)(\cos{\frac{-\pi}{4}}?+i\sin{\frac{-\pi}{4}})=\frac{\sqrt{2}}{2}(x+y)+\frac{\sqrt{2}}{2}(y-x)i$,得
\begin{eqnarray*}
    \eta(x) &=& \frac{\sqrt{2}}{2}x+\frac{\sqrt{2}}{2}y\\
    \eta(y) &=& -\frac{\sqrt{2}}{2}x+\frac{\sqrt{2}}{2}y
\end{eqnarray*}
得变换矩阵为$\begin{bmatrix}\frac{\sqrt{2}}{2}&\frac{\sqrt{2}}{2}\\
-\frac{\sqrt{2}}{2}&\frac{\sqrt{2}}{2}\end{bmatrix}$.\\
\textbf{习题 \ref{8.6} 解答:}\\
分析:设直线$L$的单位法方向向量为$\vec{n}_0$,
注意有$(\vec{l}_0,\vec{l}_0)=\vec{l}_0^T \vec{l}_0=1,(\vec{l}_0,\vec{n}_0)=(\vec{l}_0^T \vec{n}_0=\vec{n}_0^T\vec{l}_0=0$.
这里,$(\cdot,\cdot)$ 表示向量的内积,即数量积.
注意到选项里的四个变换可统一写为$aI+b\vec{l}_0 \vec{l}_0^T$,
我们只需要在一组线基上观察作用,即可以获得其完整信息.
显然关于$L$的轴对称变换$\eta$,应有$\eta\vec{l}_0=\vec{l}_0$,$\eta\vec{n}_0=-\vec{n}_0$,
则$\vec{l}_0=\eta\vec{l}_0=(aI+b\vec{l}_0 \vec{l}_0^T ) \vec{l}_0=aI\vec{l}_0+b\vec{l}_0 (\vec{l}_0^T \vec{l}_0 )=a\vec{l}_0+b\vec{l}_0\Rightarrow a+b=1$;
$-\vec{n}_0=\eta\vec{n}_0=(aI+b\vec{l}_0 \vec{l}_0^T ) \vec{n}_0=a\vec{n}_0 \Rightarrow a=-1,b=-2$,故选D.\\
\textbf{习题 \ref{8.7} 解答:}\\
根据公式$R_{\vec{l}}=-I+2\vec{l}_0 \vec{l}_0^T$.\\
(1)$R_{\vec{l}} =-I_2+2\begin{bmatrix}0\\1\end{bmatrix}\begin{bmatrix}0&1\end{bmatrix}
     =-I_2+2\begin{bmatrix}0&0\\0&1\end{bmatrix}=\begin{bmatrix}-1&0\\0&1\end{bmatrix}$\\
(2)$R_{\vec{l}} =-I_2+2\begin{bmatrix}1\\0\end{bmatrix}\begin{bmatrix}1&0\end{bmatrix}=-I_2+2\begin{bmatrix}1&0\\0&0\end{bmatrix}=
      \begin{bmatrix}1&0\\0&-1\end{bmatrix}.$\\
(3) 要将$\vec{l}$单位化,\\
      $R_{\vec{l}} =-I_2+2\begin{bmatrix}\frac{1}{\sqrt{2}}\\ \frac{1}{\sqrt{2}}\end{bmatrix}
      \begin{bmatrix}\frac{1}{\sqrt{2}}&\frac{1}{\sqrt{2}}\end{bmatrix}=-I_2+2\begin{bmatrix}1&1\\1&1\end{bmatrix}=
      \begin{bmatrix}0&1\\1&0\end{bmatrix}.$\\
(4) 要将$\vec{l}$单位化,\\
      $R_{\vec{l}} =-I_2+2\begin{bmatrix}\frac{-3}{5}\\ \frac{4}{5}\end{bmatrix}
      \begin{bmatrix}\frac{-3}{5}&\frac{4}{5}\end{bmatrix}=-I_2+\frac{2}{25}\begin{bmatrix}9&-12\\-12&16\end{bmatrix}=
      \begin{bmatrix}\frac{-7}{25}&\frac{-24}{25}\\ \frac{-24}{25}&\frac{7}{25}\end{bmatrix}.$\\
(5) 要将$\vec{l}$单位化,\\
      $R_{\vec{l}} =-I_2+2\begin{bmatrix}\frac{12}{13}\\ \frac{-5}{13}\end{bmatrix}
      \begin{bmatrix}\frac{12}{13}&\frac{-5}{13}\end{bmatrix}=-I_2+\frac{2}{169}\begin{bmatrix}144&-60\\-60&25\end{bmatrix}=
      \begin{bmatrix}\frac{119}{169}&\frac{-120}{169}\\ \frac{-120}{169}&\frac{-119}{169}\end{bmatrix}.$\\
\textbf{习题 \ref{8.8} 解答:}\\
(1) 分析:设平面$\pi$上的一个单位方向向量为$\vec{l}_0$,
则有$(\vec{n}_0,\vec{n}_0)=\vec{n}_0^T\vec{n}_0=1$,$(\vec{n}_0,\vec{l}_0)=\vec{n}_0^T\vec{l}_0=0$.
设关于$\pi$的镜面对称变换为$\eta=aI+b\vec{n}_0\vec{n}_0^T$,应有$\eta\vec{l}_0=\vec{l}_0,\eta\vec{n}_0=-\vec{n}_0$,
则$-\vec{n}_0=\eta\vec{n}_0=(aI+b\vec{n}_0\vec{n}_0^T)\vec{n}_0=aI\vec{n}_0+b\vec{n}_0 (\vec{n}_0^T\vec{n}_0)=a\vec{n}_0
+b\vec{n}_0a+b=-1$; $\vec{l}_0=\eta\vec{l}_0=(aI+b\vec{n}_0\vec{n}_0^T )\vec{l}_0=a\vec{l}_0 \Rightarrow a=1$,故选C.\\
(2)  方法同上,设线投影变换$\eta=aI+b\vec{l}_0\vec{l}_0^T$ 根据$\eta\vec{l}_0=\vec{l}_0$,$\eta\vec{n}_0=0$,得
\begin{equation*}\begin{cases}a+b=1\\a=0\end{cases}\end{equation*}
故选B.\\
(3)  方法同上,设面投影变换$\eta=aI+b\vec{n}_0 \vec{n}_0^T$ 根据$\vec{l}_0=\vec{l}_0$,$\eta\vec{n}_0=0$,
\begin{equation*}\begin{cases}a=1\\a+b=0\end{cases}\end{equation*}
故选A.\\
\textbf{习题 \ref{8.9} 解答:}\\
B.  (视为乘以复数$\cos\theta+i\sin\theta?$)\\
\textbf{习题 \ref{8.10} 解答:}\\
B.  (可以得到矩阵$B=\begin{bmatrix}1&2\\2&3\end{bmatrix}$,而$C=\begin{bmatrix}2&3\\5&8\end{bmatrix}$,$AB=C$)\\
\textbf{习题 \ref{8.11} 解答:}\\
(1) 由特征值理论,$A$的特征多项式为$\lambda^2-3\lambda+2$,两个特征值为1,2,
其中属于1的特征向量取$\vec{x}_1=\begin{bmatrix}1\\1\end{bmatrix}$,
属于2的特征向量取$\vec{x}_1=\begin{bmatrix}1\\2\end{bmatrix}$,
将$\vec{b}$在$A$取定的特征向量上进行分解,有$\vec{b}=2\vec{x}_1 +\vec{x}_2$.
则由性质$A^{10}\cdot\vec{b}=A^{10}(2\vec{x}_1+\vec{x}_2)=2A^{10}\cdot\vec{x}_1+A^{10}\cdot\vec{x}_2=2\vec{x}_1+2^{10}\vec{x}_2=
\begin{bmatrix}2\\2\end{bmatrix}+ \begin{bmatrix}1024\\2048\end{bmatrix}=\begin{bmatrix}1026\\2050\end{bmatrix}$\\
(2)  同上,由特征值理论,$A$的特征多项式为$\lambda^2-2\lambda-3$,
两个特征值为-1,3,其中属于-1的特征向量取$\vec{x}_1=\begin{bmatrix}1\\-1\end{bmatrix}$,
属于3的特征向量取$\vec{x}_2=\begin{bmatrix}1\\1\end{bmatrix}$,
将$\vec{b}$在$A$取定的特征向量上进行分解,有$\vec{b}=5\vec{x}_1 +\vec{x}_2$.
则由性质$A^4\cdot\vec{b}=A^4(5\vec{x}_1+\vec{x}_2)=5A^4\cdot\vec{x}_1+A^4\cdot\vec{x}_2=5(-1)^4\vec{x}_1+3^4\vec{x}_2=
\begin{bmatrix}5\\-5\end{bmatrix}+ \begin{bmatrix}81\\81\end{bmatrix}=\begin{bmatrix}86\\76\end{bmatrix}$\\
\textbf{习题 \ref{8.12} 解答:}\\
$(3,3)^T$.(由于投影变换是幂等变换,即$\eta^2=\eta$,故$P_{\vec{l}}^{2016}=P_{\vec{l}}$)\\
\textbf{习题 \ref{8.13} 解答:}\\
$(72,96,120)^T$. (由于投影变换是幂等变换,故$P_{\vec{l}}^{314}=P_{\vec{l}}$ ,
由公式$P_{\vec{l}}=\vec{l}_0 \vec{l}_0^T$,得
\begin{equation*}
P_{\vec{l}}=\begin{bmatrix}\frac{3}{5}\\ \frac{4}{5}\\1\end{bmatrix}
\begin{bmatrix}\frac{3}{5}&\frac{4}{5}&1\end{bmatrix}=\frac{1}{25}\begin{bmatrix}9&12&15\\12&16&20\\15&20&25\end{bmatrix}.
\end{equation*}
\begin{equation*}
P_{\vec{l}}^{314}\begin{bmatrix}50\\50\\50\end{bmatrix}=
\begin{bmatrix}9&12&15\\12&16&20\\15&20&25\end{bmatrix}
\begin{bmatrix}2\\2\\2\end{bmatrix}=
\begin{bmatrix}72\\96\\120\end{bmatrix}.
\end{equation*}
\textbf{习题 \ref{8.14} 解答:}\\
$(1,11,11)^T$.  (由于反射变换是对合变换$\eta^2=1$,故$R_{\pi}^{233}=R_{\pi}$)\\
\textbf{习题 \ref{8.15} 解答:}\\
$(1,2,3)^T$.  ($R_{\pi}^{666}=1$,即$\mathbb{R}^3$上的恒等变换,与$\vec{n}$ 无关)\\
\textbf{习题 \ref{8.16} 解答:}\\
(1)假.  (一定是可逆阵。由于两组基可以互相表示,所以过渡矩阵是可逆的.)\\
(2)真. (因为一组新的基(作为向量看)在原来的基上的线性表示是唯一的.)\\
(3)真. (由过渡矩阵的复合知.)\\
\textbf{习题 \ref{8.17} 解答:}\\
(1)假.(平移变换是刚体变换,但不是正交变换,因为正交变换必将零向量映为零向量. )\\
(2)真. (若$AA^T=A^TA=I$,则$A^(-1) (A^T )^(-1)=(A^T A)^(-1)=I$,$(A^T )^(-1) A^(-1)=(AA^T )^(-1)=I$)\\
(3)假. (反例:$A=I$,$B=-I$)\\
(4)真. (若$AA^T=A^T A=I$,$BB^T=B^T B=I$,则$AB(AB)^T=ABB^T A^T=AA^T=I$, $(AB)^T AB=B^T A^T AB=B^T B=I$).

