\chapter{矩阵}

\section{知识点解析}

\begin{Def}
由$mn$个实数排成行列的矩形数表, 用圆(或方)括号括起来,即
$$A = \begin{bmatrix}
a_{11} & a_{12} & \cdots & a_{1n} \\ a_{21} & a_{22} & \cdots & a_{2n} \\ \vdots & \vdots & \vdots & \vdots \\ a_{m1} & a_{m2} & \cdots & a_{mn}
\end{bmatrix}$$
称为$m\times n$型的矩阵,简记为$A = (a_{ij})_{m\times n}$,其中横排称为矩阵的行,竖排称为矩阵的列。$a_{ij}$称为矩阵的元素,其第一下标表示所在的行数,第二下标表示其所在的列数。全体$m\times n$型的矩阵组成的集合,记为$M_{m\times n}(\mathbb{R})$。
\end{Def}

\begin{rmk}
几类特殊矩阵:
\enum
\item[(1)] 方阵:行数与列数都等于$n$的矩阵称为$n$阶方阵。全体$n$阶方阵组成的集合,记为$M_n(\mathbb{R})$。
\item[(2)] 行矩阵,列矩阵:

行数$m=1$时,$(a_1,a_2,\cdots,a_n)$称为行矩阵(或行向量)。

列数$n=1$时,$\begin{bmatrix} a_1 \\ a_2 \\ \vdots \\ a_n \end{bmatrix}$称为列矩阵(或列向量)。
\item[(3)] 零矩阵:元素全为零的矩阵称为零矩阵,记为$O$或$0_{m\times n}$。
\item[(4)] 负矩阵:$-A = -(a_{ij})_{m\times n} = (-a_{ij})_{m\times n}$
\item[(5)] 三角形矩阵:

上三角矩阵:主对角线下方的元素全为零的方阵称为上三角形矩阵。

下三角矩阵:主对角线上方的元素全为零的方阵称为下三角形矩阵。
\item[(6)] 对角矩阵:除主对角线上元素外,全为零的方阵。
$$\begin{bmatrix} a_{11} & & \\ & \ddots & \\ & & a_{nn} \end{bmatrix} =: diag(a_{11},\cdots,a_{nn})$$
特别地有:
\begin{eqnarray*}
\text{纯量矩阵} & : & \begin{bmatrix} c & & \\ & \ddots & \\ & & c \end{bmatrix} = diag(c,\cdots,c) \\
\text{单位矩阵} & : & \begin{bmatrix} 1 & & \\ & \ddots & \\ & & 1 \end{bmatrix} = diag(1,\cdots,1) =: I_n
\end{eqnarray*}
\item[(7)] 阶梯形矩阵(零行在最下方,非零行左端的零严格增加)。

简化阶梯形矩阵(阶梯形矩阵,主元素$=1$,主元所在列的其它元均为$0$)。
\end{list}
\end{rmk}

\begin{Def}\

\enum
\item[(1)] 如果两个矩阵行数相同且列数相同,则称这两个矩阵是同型的;
\item[(2)] 两个矩阵同型且对应元素相等,则称这两个矩阵是相等的。
\end{list}
\end{Def}

\begin{Def}[矩阵的线性运算]\

\enum
\item[(1)] 加法:同型矩阵对应分量相加。
$$A = (a_{ij})_{m\times n}, B = (b_{ij})_{m\times n} \Rightarrow A + B := (a_{ij} + b_{ij})_{m\times n}$$
\item[(2)] 减法:同型矩阵对应分量相减。
$$A - B := A + (-B) = (a_{ij} - b_{ij})_{m\times n}$$
\item[(3)] 数量乘法(或数乘):矩阵的每个分量同乘一个数。
$$\text{设 }k\in\mathbb{R}, A = (a_{ij})_{m\times n}, \text{则 } kA = k(a_{ij})_{m\times n} := (ka_{ij})_{m\times n}$$
\end{list}
\end{Def}

\begin{rmk}
由数的乘法可交换,知$kA = (ka_{ij})_{m\times n} = (a_{ij}k)_{m\times n} = Ak$。一般统一都把数$k$乘在矩阵$A$的左边。
\end{rmk}

\begin{prop}[矩阵线性运算(加法与数乘)的运算律]\

加法运算律:
\enum
\item[(1)] 交换律:$A+B = B+A$。
\item[(2)] 结合律:$A+(B+C) = (A+B)+C$。
\item[(3)] 零矩阵:$A+0 = A$。
\item[(4)] 负矩阵:$A+(-A) = 0$。
\end{list}

数乘运算律
\enum
\item[(5)] 单位:$1\cdot A = A$。
\item[(6)] 结合律:$k(lA) = (kl)A$。
\item[(7)] 分配律1:$k(A+B) = kA+kB$。
\item[(8)] 分配律2:$(k+l)A = kA+lA$。
\end{list}
\end{prop}

\begin{Def}[矩阵的乘法]
设$A = (a_{ij})_{m\times s}, B = (b_{ij})_{t\times n}$。如果矩阵$A$的列数等于矩阵$B$的行数,即$s=t$,则$A$与$B$ 可以相乘,记为$AB$。$AB$为一个$m\times n$阶的矩阵,其$(i,j)$位置上的元素是$A$ 的第$i$行的元素与$B$的第$j$列对应位置上的元素乘积之和,即如果记$AB = (c_{ij})_{m\times n}$,那么
$$c_{ij} = a_{i1}b_{1j} + a_{i2}b_{2j} + \cdots + a_{is}b_{sj} = \sum\limits_{k=1}^s a_{ik}b_{kj}, \quad (i=1,\cdots,m; j=1,\cdots,n)$$
\end{Def}

\begin{prop}[矩阵乘法的一些``反常''性质]\

\enum
\item[(1)] 不满足交换律:$AB\neq BA$。
\item[(2)] 存在零因子:
\begin{eqnarray*}
A \neq 0, B \neq 0 & \not\Rightarrow & AB \neq 0; \\
AB = 0 & \not\Rightarrow & A = 0 \text{ 或 } B = 0.
\end{eqnarray*}
\item[(3)] 不满足消去律:$AB = AC \not\Rightarrow B = C$。
\end{list}
\end{prop}

\begin{prop}[矩阵乘法的运算律]\
设$A\in M_{m\times n}(\mathbb{R})$为一个$m\times n$阶矩阵设,且取$B, C$使得下列运算可行,则
\enum
\item[(1)] 零矩阵:$0_{k\times m}A = 0_{k\times n}, A0_{n\times l} = 0_{m\times l}$。
\item[(2)] 单位阵:$I_mA = A, AI_n = A$。
\item[(3)] 数乘:$(kA)B = A(kB) = k(AB)$。
\item[(4)] 左右分配律:$A(B+C) = AB + AC, (B+C)A = BA+CA$。
\item[(5)] 结合律:$A(BC) = (AB)C$。
\end{list}
\end{prop}

\begin{Def}
对于$n$阶方阵$A$,定义$A$的方幂为:$A^k = \underbrace{A\cdots A}_{k \text{ 个}}, k\in \mathbb{Z}$。

另对$A^0$,我们规定$A^0 = I_n$。
\end{Def}

\begin{prop}[矩阵方幂的性质]
设$A$为$n$阶方阵,$k,l$为非负整数,则
\enum
\item[$\bullet$] $A^k A^l = A^{k+l} = A^l A^k$。
\item[$\bullet$] $(A^k)^l = A^{kl}$。
\end{list}
\end{prop}

\begin{rmk}
对同一个$n$阶方阵$A$反复进行线性运算与乘法(方幂)运算,经过合并化简后,可得
$$a_mA^m + a_{m-1}A^{m-1} + \cdots + a_1A + a_0I_n$$
把上式记为$f(A)$,称为关于矩阵$A$的$m$次多项式。

由于$A^k$与$A^l$乘法可交换,对任意多项式$f(x)$与$g(x)$,有矩阵多项式$f(A)$与$g(A)$乘法可交换,即
$$f(A)g(A) = g(A)f(A).$$
\end{rmk}

\begin{Def}
设矩阵$A=(a_{ij})_{m\times n}$,将$A$的行与列互换得到的矩阵称为$A$的转置,记为$A^T$。
\end{Def}

\begin{rmk}
$A^T$为$n\times m$阶矩阵,若记$A^T = (a_{ij}')_{n\times m}$,则有$a_{ij}' = a_{ji}$。
\end{rmk}

\begin{prop}[矩阵转置的运算律]
设矩阵$A,B$和实数$\lambda$使得下述相关运算有定义,则
\enum
\item[(1)] 两次还原:$(A^T)^T = A$。
\item[(2)] 加法相容:$(A+B)^T = A^T + B^T$。
\item[(3)] 数乘相容:$(\lambda A)^T = \lambda A^T$。
\item[(4)] 乘法反序:$(AB)^T = B^T A^T$。
\end{list}
\end{prop}

\begin{Def}
设$A$为方阵。若$A^T = A$,则称$A$为对称阵;若$A^T = -A$,则称$A$为反对称阵。
\end{Def}

\begin{prop}[对称阵与反对称阵的运算律]\

线性运算保持矩阵的(反)对称性,即
\enum
\item[(1)] 如果$A, B$是同阶对称矩阵,则$A+B, kA$也是对称矩阵。
\item[(2)] 如果$A, B$是同阶反对称矩阵,则$A+B, kA$也是反对称矩阵。
\end{list}
而乘法运算不一定保持(反)对称性。
\end{prop}

\begin{thm}
设$A,B\in M_n(\mathbb{R})$,则$\det(AB) = \det(A)\cdot\det(B)$。
\end{thm}

\begin{cor}
设$A_1,\cdots,A_s$为$s$个$n$阶方阵,则
$$\det(A_1\cdots A_s) = \det A_1\cdots\det A_s.$$
\end{cor}

\begin{prop}
方阵的行列式运算与其他运算有如下关系(设$A$为$n$阶方阵):
\enum
\item[$\bullet$] $\det A^T = \det A$。
\item[$\bullet$] $\det (kA) = k^n \det A$。
\end{list}
\end{prop}

\begin{Def}
对一个$m\times n$的矩阵$A$,用若干横线和竖线把$A$的行分成$p$个部分,列分成$q$个部分,整个矩阵$A$分成$pq$个小矩阵,即
$$A = \begin{bmatrix}
A_{11} & A_{12} & \cdots & A_{1q} \\
A_{21} & A_{22} & \cdots & A_{2q} \\
\vdots & \vdots & \ddots & \vdots \\
A_{p1} & A_{p2} & \cdots & A_{pq}
\end{bmatrix}$$
其中每个小矩阵$A_{ij}(1 \leqslant i \leqslant p, 1 \leqslant j \leqslant q)$称为矩阵$A$的子块,$A$也可视为由子块$A_{ij}$构成的$p\times q$阶矩阵,称为分块矩阵。
\end{Def}

\begin{rmk}
矩阵分块的三个原则:
\enum
\item[(1)] 体现原矩阵特点,按需划分。
\item[(2)] 能够把子块看作元素进行运算(除乘法的次序外)。
\item[(3)] 保持原有运算性质。
\end{list}
\end{rmk}

\begin{prop}[分块矩阵的运算]\

\enum
\item[(1)] 分块矩阵的加法:设分块矩阵$A$与$B$的行列数均相同(同型矩阵),且采用同样的分块方法,即
$$A = \begin{bmatrix}
A_{11} & \cdots & A_{1q} \\
\vdots & \ddots & \vdots \\
A_{p1} & \cdots & A_{pq}
\end{bmatrix},
B = \begin{bmatrix}
B_{11} & \cdots & B_{1q} \\
\vdots & \ddots & \vdots \\
B_{p1} & \cdots & B_{pq}
\end{bmatrix},
$$
其中$A_{ij}$与$B_{ij}$的行数和列数均相同(同型矩阵),则
$$A + B = \begin{bmatrix}
A_{11} + B_{11} & \cdots & A_{1q} + B_{1q}\\
\vdots & \ddots & \vdots \\
A_{p1} + B_{p1} & \cdots & A_{pq} + B_{pq}
\end{bmatrix}.
$$
\item[(2)] 分块矩阵的数乘:

设$A = \begin{bmatrix}
A_{11} & \cdots & A_{1q} \\
\vdots & \ddots & \vdots \\
A_{p1} & \cdots & A_{pq}
\end{bmatrix}$(可任意分块),$\lambda$是数,则
$$\lambda A = \begin{bmatrix}
\lambda A_{11} & \cdots & \lambda A_{1q} \\
\vdots & \ddots & \vdots \\
\lambda A_{p1} & \cdots & \lambda A_{pq}
\end{bmatrix}$$
\item[(3)] 分块矩阵的转置:

设$$A = \begin{bmatrix}
A_{11} & \cdots & A_{1q} \\
\vdots & \ddots & \vdots \\
A_{p1} & \cdots & A_{pq}
\end{bmatrix}$$,则分块矩阵A的转置为
$$A^T = \begin{bmatrix}
A_{11}^T & \cdots & A_{p1}^T \\
\vdots & \ddots & \vdots \\
A_{1q}^T & \cdots & A_{pq}^T
\end{bmatrix},$$
即,每一块子矩阵均做转置,且行列下标互换。
\item[(4)] 分块矩阵的乘法:

设$A_{m\times n} B_{n\times l}=A_{m\times l}$,要求
\enum
\item[$\bullet$] $A$的列数$= B$的行数$= n$。
\item[$\bullet$] $A$的列的分法$= B$的行的分法($n$的加法有序分拆)。
\end{list}

$$A = \begin{bmatrix}
A_{11} & \cdots & A_{1s} \\
\vdots & \ddots & \vdots \\
A_{r1} & \cdots & A_{rs}
\end{bmatrix},
B = \begin{bmatrix}
B_{11} & \cdots & B_{1t} \\
\vdots & \ddots & \vdots \\
B_{s1} & \cdots & B_{st}
\end{bmatrix},
C = \begin{bmatrix}
C_{11} & \cdots & C_{1t} \\
\vdots & \ddots & \vdots \\
C_{r1} & \cdots & C_{rt}
\end{bmatrix}.$$
对乘积矩阵$C$,有
\enum
\item[$\bullet$] $C$的行数及行分块法由$A$的行数及行分块法决定;$C$的列数及列分块法由$B$的列数及列分块法决定。
\item[$\bullet$] $C$的每个子块
$$C_{ij} = A_{i1}B_{1j} + A_{i2}B_{2j} +\cdots +  A_{is}B_{sj} = \sum\limits_{k=1}^s  A_{ik}B_{kj}, \quad (1 \leqslant i \leqslant r, 1 \leqslant j \leqslant t)$$
\end{list}
\end{list}
\end{prop}

\begin{rmk}
特殊的分块矩阵:
\enum
\item[$\bullet$] 准对角矩阵:

设$A$为方阵,若
$$A = \begin{bmatrix} A_{11} & & \\ & \ddots & \\ & & A_{nn} \end{bmatrix} = diag(A_{11},\cdots,A_{nn}),$$
其中$A_{11},\cdots,A_{nn}$都是小方阵,则称$A$为准对角矩阵。

准对角矩阵可作为对角矩阵的推广情形,是最简单的一类分块矩阵。

\item[$\bullet$] 准上三角矩阵:

设矩阵$A$的行与列均分为个$n$子块,且
$$A = \begin{bmatrix} A_{11} & A_{12} & \cdots & A_{1n} \\ 0 & A_{22} & \cdots & A_{2n} \\ \vdots & \vdots & \ddots & \vdots \\ 0 & 0 & \cdots & A_{nn} \end{bmatrix},$$
则称$A$为准上三角矩阵。

\item[$\bullet$] 准下三角矩阵:

设矩阵$A$的行与列均分为个$n$子块,且
$$A = \begin{bmatrix} A_{11} & 0 & \cdots & 0 \\ A_{21} & A_{22} & \cdots & 0 \\ \vdots & \vdots & \ddots & \vdots \\ A_{n1} & A_{n2} & \cdots & A_{nn} \end{bmatrix},$$
则称$A$为准下三角矩阵。

一般地,准三角形矩阵不一定是方阵。
\end{list}
\end{rmk}

\begin{thm}
在可运算的条件下,准上(下)三角形矩阵的加法,数乘与乘法仍是准上(下)三角形矩阵。
\end{thm}

\begin{cor}
在可运算的条件下,上(下)三角形矩阵的加法,数乘与乘法仍是上(下)三角形矩阵。
\end{cor}

\begin{Def}
单位矩阵$I$经过一次初等变换所得到的矩阵称为初等矩阵。

初等矩阵有以下三类:
\enum
\item[(1)] 单位矩阵$I$的第$i$行乘以非零数$k$(倍乘行变换)所得矩阵,记为$E_i(k)$,称为倍乘矩阵。它同时也是$I$的第$i$列乘以非零数$k$(倍乘行变换)所得矩阵。
$$I_n \xrightarrow[(kc_i)]{kr_i} \begin{bmatrix} 1 & & & & & & \\ & \ddots & & & & & \\ & & 1 & & & & \\ & & & k & & & \\ & & & & 1 & &  \\ & & & & & \ddots & \\ & & & & & & 1 \end{bmatrix} =: E_i(k)$$
\item[(2)] 将单位矩阵$I$第$i$行的$k$倍加到第$j$行,所得矩阵记为$E_{i,j}(k)$,称为倍加矩阵。它也是将$I$第$j$列的$k$倍加到第$i$列,所得的矩阵。
$$\begin{bmatrix} 1 & & & & & & \\ & \ddots & & & & & \\ & & 1 & & & & \\ & & \vdots & \ddots & & & \\ & & k & \cdots & 1 & &  \\ & & & & & \ddots & \\ & & & & & & 1 \end{bmatrix} =: E_{i,j}(k)$$
\item[(3)] 交换单位矩阵$I$的第$i$行与第$j$行,所得矩阵记为$E_{i,j}$,称为对换矩阵。它也是交换$I$的第$i$列与第$j$列,所得的矩阵。
%$$
%\begin{bmatrix} 1 & & & & & & & \\ & \ddots & & & & & & \\ & & 0 & & 1 & & \\ & & & \ddots & & & & \\ & & 1 & & 0 & & \\ & & & & & & \ddots & \\ & & & & & & & 1\end{bmatrix} =: E_{ij}$$
$$\left[\begin{array}{*{11}c}
1 & & & & & & & & &  & \\ & \ddots & & & & & & & & & \\ & & 1 & & & & & & & & \\ & & & 0 & \cdots & \cdots & \cdots & 1 & & & \\ & & & \vdots & 1 & & & \vdots & & & \\ & & & \vdots & & \ddots & & \vdots & & & \\ & & & \vdots & & & 1 & \vdots & & & \\ & & & 1 & \cdots & \cdots & \cdots & 0 & & & \\ & & & & & & & & 1 & & \\ & & & & & & & & & \ddots & \\ & & & & & & & & & & 1 \end{array}\right] =: E_{ij}$$
\end{list}
\end{Def}

\begin{prop}[初等矩阵的性质]\

\enum
\item[(1)] 初等矩阵的转置:
$$E_i(k)^T = E_i(k), \quad E_{ij}^T = \left[\begin{array}{*{11}c}
1 & & & & & & & & &  & \\ & \ddots & & & & & & & & & \\ & & 1 & & & & & & & & \\ & & & 0 & \cdots & \cdots & \cdots & 1 & & & \\ & & & \vdots & 1 & & & \vdots & & & \\ & & & \vdots & & \ddots & & \vdots & & & \\ & & & \vdots & & & 1 & \vdots & & & \\ & & & 1 & \cdots & \cdots & \cdots & 0 & & & \\ & & & & & & & & 1 & & \\ & & & & & & & & & \ddots & \\ & & & & & & & & & & 1 \end{array}\right] = E_{ij},$$
故倍乘矩阵、对换矩阵转置不变(对称阵),转置后仍为初等矩阵。

设$i < j$,则
$$E_{i,j}(k) = \begin{bmatrix} 1 & & & & & & \\ & \ddots & & & & & \\ & & 1 & & & & \\ & & \vdots & \ddots & & & \\ & & k & \cdots & 1 & &  \\ & & & & & \ddots & \\ & & & & & & 1 \end{bmatrix} = E_{j,i}(k)^T,$$
故倍加矩阵转置仍为倍加矩阵,但下标顺序交换。
\item[(2)] 初等矩阵的行列式:易知$|E_i(k)|=k\neq 0; |E_{i,j}(k)|=1$,而
\begin{eqnarray*}
|E_{ij}| & = & \left|\begin{array}{*{11}c}
1 & & & & & & & & &  & \\ & \ddots & & & & & & & & & \\ & & 1 & & & & & & & & \\ & & & 0 & \cdots & \cdots & \cdots & 1 & & & \\ & & & \vdots & 1 & & & \vdots & & & \\ & & & \vdots & & \ddots & & \vdots & & & \\ & & & \vdots & & & 1 & \vdots & & & \\ & & & 1 & \cdots & \cdots & \cdots & 0 & & & \\ & & & & & & & & 1 & & \\ & & & & & & & & & \ddots & \\ & & & & & & & & & & 1 \end{array}\right| \\
& = & (-1)^{\tau(1\cdots j\cdots i\cdots n)}a_{11}\cdots a_{i-1,i-1}a_{ij}a_{i+1,i+1}\cdots a_{j-1,j-1}a_{ji}a_{j+1,j+1}\cdots a_{nn} \\
& = & (-1)\times 1\times\cdots\times 1 = -1
\end{eqnarray*}
\item[(3)] 初等矩阵的分块表示:

设$e_i = (0,\cdots,0,1,0,\cdots,0)^T$ ($1$在第$i$位)列矩阵(列向量),则
$$I_n = (e_1,\cdots,e_i,\cdots,e_j,\cdots,e_n), \quad \text{ 或 }\ I_n = I_n^T = \begin{bmatrix} e_1^T \\ \vdots \\ e_i^T \\ \vdots \\ e_n^T \end{bmatrix},$$
从而,由初等矩阵的定义,有
\begin{eqnarray*}
E_{ij} & = & E_{ij}^T = (e_1,\cdots,e_j,\cdots,e_i,\cdots,e_n), \\
E_i(k) & = & E_i(k)^T = (e_1,\cdots,ke_i,\cdots,e_n), \quad (k\neq 0), \\
E_{ij}(k) & = & \begin{bmatrix} e_1^T \\ \vdots \\ e_i^T \\ \vdots \\ ke_i^T + e_j^T \\ \cdots \\ e_n^T \end{bmatrix} = (e_1,\cdots,ke_j + e_i,\cdots,e_j,\cdots,e_n)
\end{eqnarray*}
\end{list}
\end{prop}

\begin{prop}[初等矩阵与初等变换的关系]\

\enum
\item[(1)] 左乘倍乘矩阵$E_i(k)$,相当于对$A$的第$i$行做$k$倍的倍乘变换;右乘倍乘矩阵$E_i(k)$,相当于对$A$的第$i$列做$k$倍的倍乘变换。
\item[(2)] 左乘对换矩阵$E_{ij}$,相当于对$A$的第$i, j$行做对换变换;右乘对换矩阵$E_{ij}$,相当于对$A$的第$i, j$列做对换变换。
\item[(3)] 左乘倍加矩阵$E_{ij}(k)$,相当于对$A$做第$i$行的$k$倍加到第$j$行上的倍加变换;右乘倍加矩阵$E_{ij}(k)$,相当于对$A$做第$j$列的$k$倍加到第$i$列上的倍加变换。
\end{list}
\end{prop}

\begin{thm}
用初等矩阵左乘矩阵$A$,相当于对$A$进行一次相应的初等行变换。用初等矩阵右乘矩阵$A$,相当于对$A$进行一次相应的初等列变换。
\end{thm}

\begin{Def}
下面三种针对分块矩阵$M$的变形,统称为分块矩阵的初等变换:
\enum
\item[(1)] 倍乘:用特定矩阵$P$左(右)乘$M$的某一``行(列)'';
\item[(2)] 倍加:用矩阵$Q$乘$M$的某``行(列)''加到另外一``行(列)''。
\item[(3)] 对换:交换$M$的两``行''或``列''。
\end{list}
\end{Def}

\begin{Def}
将单位矩阵分块成准对角形矩阵$I = diag(I_s, I_t)$,对其进行一次初等变换,得到的分块矩阵称为分块初等矩阵:
\enum
\item[(1)] 分块倍乘矩阵:$\begin{bmatrix} P & 0 \\ 0 & I_t \end{bmatrix}, \begin{bmatrix} I_s & 0 \\ 0 & P \end{bmatrix}$(其中$P$为可逆方阵);
\item[(2)] 分块倍加矩阵:$\begin{bmatrix} I_s & 0 \\ Q & I_t \end{bmatrix}, \begin{bmatrix} I_s & Q \\ 0 & I_t \end{bmatrix}$;
\item[(3)] 分块对换矩阵:$\begin{bmatrix} 0 & I_t \\ I_s & 0 \end{bmatrix}, \begin{bmatrix} 0 & I_s \\ I_t & 0 \end{bmatrix}$。
\end{list}
\end{Def}

\begin{thm}
对分块矩阵进行一次初等行(列)变换,相当于给它左(右)乘以一个相应的分块初等矩阵。
\end{thm}

\begin{Def}
设$A$为$n$阶方阵,若存在$n$阶方阵$B$,使得 $AB = BA = I_n$,则称$A$为可逆矩阵或非奇异矩阵,而称$B$为$A$的逆矩阵,并记为$A^{-1}$。
\end{Def}

\begin{rmk}\

\enum
\item[(1)] 定义中矩阵$A$与$B$的地位相同,因而若$A$可逆,且$B$是$A$的逆,则$B$也可逆,$A$即是$B$的逆,并且$A$与$A^{-1}$乘法可交换。
\item[(2)] 对上述定义式两边取行列式知:若$A$为可逆矩阵,则
$$|A|\neq 0, |A^{-1}| = |A|^{-1} = 1 / |A|$$
\item[(3)] 若$A$有逆矩阵,则其逆是唯一的。
\end{list}
\end{rmk}

\begin{prop}
设$A, B, A_i$为$n$阶可逆矩阵,实数$k\neq 0$,则
\enum
\item[(1)] $A^{-1}$也可逆,且$(A^{-1})^{-1} = A$。
\item[(2)] $AB$也可逆,且$(AB)^{-1} = B^{-1}A^{-1}$。进一步有,$(A_1A_2\cdots A_s)^{-1} = A_s^{-1}\cdots A_1^{-1}$。
\item[(3)] $kA$也可逆,且$(kA)^{-1} = k^{-1}A^{-1}$。
\item[(4)] $A^T$也可逆,且$(A^T)^{-1} = (A^{-1})^T$。
\end{list}
\end{prop}

\begin{Def}
用$n$阶方阵$A$的元素的代数余子式$A_{ij}$组成的矩阵的转置
$$(A_{ij})^T_{\tiny \substack{1 \leqslant i \leqslant n \\ 1 \leqslant j \leqslant n}\normalfont} = \begin{bmatrix}
A_{11} & A_{21} & \cdots & A_{n1} \\
A_{12} & A_{22} & \cdots & A_{n2} \\
\vdots & \vdots & \ddots & \vdots \\
A_{1n} & A_{2n} & \cdots & A_{nn}
\end{bmatrix} =: A^{\ast}$$
称为矩阵$A$的伴随矩阵。
\end{Def}

\begin{rmk}\

\enum
\item[(1)] 对方阵$A, AA^{\ast} = A^{\ast}A = |A|I$总是成立的(含$|A| = 0$时)。
\item[(2)] 若$|A| \neq 0$,则$A|A|^{-1}A^{\ast} = |A|^{-1}A^{\ast}A = I$,故$A$和$A^{\ast}$可逆,且
$$A^{-1} = |A|^{-1}A^{\ast}, \quad (A^{\ast})^{-1} = |A|^{-1}A.$$
\end{list}
\end{rmk}

\begin{thm}
$n$阶方阵$A$可逆的充要条件是$|A| \neq 0$。
\end{thm}

\begin{cor}
设$A$为$n$阶方阵,若存在$n$阶方阵$B$,使得$AB = I_n$(或 $BA = I_n$),则$A$可逆,且$B = A^{-1}$。
\end{cor}

\begin{thm}
$n$阶方阵$A$可逆 $\Longleftrightarrow$ $A$为若干个初等矩阵的乘积。
\end{thm}

\begin{cor}
$n$阶方阵$A$可逆 $\Longleftrightarrow$ 齐次线性方程组$AX=0$只有零解。
\end{cor}

\begin{prop}[求逆矩阵的初等变换法]\

\enum
\item[$\bullet$] 行变换法:构造一个$n\times(2n)$阶分块矩阵
$$[A,I] \xrightarrow{\text{若干初等行变换}} [I,A^{-1}]:$$
$$P_s\cdots P_1[A,I] = [P_s\cdots P_1A, P_s\cdots P_1I] = [A^{-1}A,A^{-1}I] = [I,A^{-1}].$$
\item[$\bullet$] 列变换法:构造一个$(2n)\times n$阶分块矩阵
$$\begin{bmatrix} A \\ I \end{bmatrix} \xrightarrow{\text{若干初等列变换}} \begin{bmatrix} I \\ A^{-1} \end{bmatrix}:$$
$$\begin{bmatrix} A \\ I \end{bmatrix}P_s\cdots P_1 = \begin{bmatrix} AP_s\cdots P_1 \\ IP_s\cdots P_1 \end{bmatrix} = \begin{bmatrix} AA^{-1} \\ IA^{-1} \end{bmatrix} = \begin{bmatrix} I \\ A^{-1} \end{bmatrix}.$$
\end{list}
\end{prop}

\begin{prop}
对准上三角分块矩阵有
$$A = \begin{bmatrix} A_{11} & A_{12} & \cdots & A_{1n} \\ 0 & A_{22} & \cdots & A_{1n} \\ \vdots & \vdots & \ddots & \vdots \\ 0 & 0 & \cdots & A_{nn} \end{bmatrix} \text{ 可逆 }
\Longleftrightarrow  A_{11},\cdots,A_{nn}\text{ 均可逆,}$$
此时,
$$A^{-1} = \begin{bmatrix} A_{11}^{-1} & \ast & \cdots & \ast \\ 0 & A_{22}^{-1} & \cdots & \ast \\ \vdots & \vdots & \ddots & \vdots \\ 0 & 0 & \cdots & A_{nn}^{-1} \end{bmatrix}.$$

特别地,若$A = diag(A_{11},\cdots,A_{nn})$,则$A^{-1} = diag(A_{11}^{-1},\cdots,A_{nn}^{-1})$。
\end{prop}

%%%%%%%%%%%%%%%%%%%%%%%%%%%%%%%%%%%%%%%%%%%%%%%%%%%%%%%%%%%%%%%%%%%%%%%%%%%%%%%%%%%%%%%%%%%%

\section{例题讲解}

\begin{eg}
设有如下两组变量替换
\begin{eqnarray*}
\begin{cases}
x_1 = a_{11}y_1 + a_{12}y_2 + a_{13}y_3 \\ x_2 = a_{21}y_1 + a_{22}y_2 + a_{23}y_3
\end{cases}
& \longrightarrow &
\begin{bmatrix}
a_{11} & a_{12} & a_{13} \\ a_{21} & a_{22} & a_{23}
\end{bmatrix} \\
\begin{cases}
y_1 = b_{11}z_1 + b_{12}z_2 \\ y_2 = b_{21}z_1 + b_{22}z_2 \\ y_3 = b_{31}z_1 + b_{32}z_2
\end{cases}
& \longrightarrow &
\begin{bmatrix}
b_{11} & b_{12} \\ b_{21} & b_{22} \\ b_{31} & b_{32}
\end{bmatrix}
\end{eqnarray*}
将第二组变量替换代入到第一组中,即可将$x_1, x_2$表示为$z_1, z_2$的形式:
$$
\begin{cases}
x_1 = (a_{11}b_{11} + a_{12}b_{21} + a_{13}b_{31})z_1 + (a_{11}b_{12} + a_{12}b_{22} + a_{13}b_{32})z_2 \\ x_2 = (a_{21}b_{11} + a_{22}b_{21} + a_{23}b_{31})z_1 + (a_{21}b_{12} + a_{22}b_{22} + a_{23}b_{32})z_2
\end{cases}
\longrightarrow
\begin{bmatrix}
c_{11} & c_{12} \\ c_{21} & c_{22}
\end{bmatrix}
$$
且$c_{ij} = a_{i1}b_{1j} + a_{i2}b_{2j} + a_{i3}b_{3j} = \sum\limits_{k=1}^3 a_{ik}b_{kj}, \ (1 \leqslant i,j \leqslant 2).$
\end{eg}

\begin{eg}
设$A = \begin{bmatrix} 2 & 3 & 1 \\ 1 & 2 & -1 \\ 0 & 3 & 1\end{bmatrix}, B = \begin{bmatrix} 1 \\ 0 \\ 2 \end{bmatrix}$,求$AB$。
\end{eg}

\begin{solution}
$\begin{bmatrix} 2 & 3 & 1 \\ 1 & 2 & -1 \\ 0 & 3 & 1\end{bmatrix} \begin{bmatrix} 1 \\ 0 \\ 2 \end{bmatrix} = \begin{bmatrix} 2\times 1 + 3\times 0 + 1\times 2 \\ 1\times 1 + 2\times 0 + (-1)\times 2 \\ 0\times 1 + 3\times 0 + 1\times 2\end{bmatrix} = \begin{bmatrix} 4 \\ -1 \\ 2 \end{bmatrix}$
\end{solution}

\begin{eg}
考虑如下线性方程组
$$\left\{ \begin{array}{rcl} a_{11}x_1 + a_{12}x_2 + \cdots + a_{1n}x_n & = & b_1 \\ a_{21}x_1 + a_{22}x_2 + \cdots + a_{2n}x_n & = & b_2 \\ \hdotsfor{3} \\ a_{m1}x_1 + a_{m2}x_2 + \cdots + a_{mn}x_n & = & b_m \end{array}\right.$$
记$A = (a_{ij})_{m\times n} = \begin{bmatrix} a_{11} & \cdots & a_{1n} \\ \vdots & & \vdots \\ a_{m1} & \cdots & a_{mn} \end{bmatrix}, X_{n\times 1} = \begin{bmatrix} x_1 \\ \vdots \\ x_n \end{bmatrix} B_{m\times 1} = \begin{bmatrix} b_1 \\ \vdots \\ b_m \end{bmatrix}$。则由矩阵的乘法定义可知,线性方程组可以写为:$AX = B$,其中$A$为系数矩阵,$\left( A \middle| B \right)_{m\times(n+1)}$为增广系数矩阵。
\end{eg}

\begin{eg}
设$A = \begin{bmatrix} a_1 \\ a_2 \end{bmatrix}, B = \begin{bmatrix} b_1 & b_2 \end{bmatrix}$,则$AB = \begin{bmatrix} a_1b_1 & a_1b_2 \\ a_2b_1 & a_2b_2 \end{bmatrix}, BA = (b_1a_1 + b_2a_2)$。
\end{eg}

\begin{eg}
设$A = \begin{bmatrix} 1 & 1 \\ -1 & -1 \end{bmatrix}, B = \begin{bmatrix} 1 & -1 \\ -1 & 1 \end{bmatrix}$,则$AB = \begin{bmatrix} 0 & 0 \\0 & 0 \end{bmatrix}, BA = \begin{bmatrix} 2 & 2 \\ -2 & -2 \end{bmatrix}$。
\end{eg}

\begin{eg}
  设$A = \begin{bmatrix} 1 & 2 \\ 2 & 4 \end{bmatrix}, B = \begin{bmatrix} -1 & 3 \\ -2 & 1 \end{bmatrix}, C = \begin{bmatrix} -7 & 1 \\ 1 & 2 \end{bmatrix}$,求$AB$以及$AC$。
\end{eg}

\begin{solution}
$$\left. \begin{array}{c}
AB = \begin{bmatrix} 1 & 2 \\ 2 & 4 \end{bmatrix} \begin{bmatrix} -1 & 3 \\ -2 & 1 \end{bmatrix} = \begin{bmatrix} -5 & 5 \\ -10 & 10 \end{bmatrix} \\
AC = \begin{bmatrix} 1 & 2 \\ 2 & 4 \end{bmatrix} \begin{bmatrix} -7 & 1 \\ 1 & 2 \end{bmatrix} = \begin{bmatrix} -5 & 5 \\ -10 & 10 \end{bmatrix}
\end{array} \right\} \Rightarrow AB = AC, B \neq C.$$
\end{solution}

\begin{eg}
考虑对角阵与矩阵的乘积:
\begin{eqnarray*}
\begin{bmatrix}
k_1 & & & \\ & k_2 & & \\ & & \ddots & \\ & & & k_m
\end{bmatrix}_{m\times m}
\begin{bmatrix}
a_{11} & a_{12} & \cdots & a_{1n} \\ a_{21} & a_{22} & \cdots & a_{2n} \\ \vdots & \vdots & \vdots & \vdots \\ a_{m1} & a_{m2} & \cdots & a_{mn}
\end{bmatrix}_{m\times n}
& = & \begin{bmatrix}
k_1 a_{11} & k_1 a_{12} & \cdots & k_1 a_{1n} \\ k_2 a_{21} & k_2 a_{22} & \cdots & k_2 a_{2n} \\ \vdots & \vdots & \vdots & \vdots \\ k_m a_{m1} & k_m a_{m2} & \cdots & k_m a_{mn}
\end{bmatrix}_{m\times n} \\
\begin{bmatrix}
a_{11} & a_{12} & \cdots & a_{1n} \\ a_{21} & a_{22} & \cdots & a_{2n} \\ \vdots & \vdots & \vdots & \vdots \\ a_{m1} & a_{m2} & \cdots & a_{mn}
\end{bmatrix}_{m\times n}
\begin{bmatrix}
k_1 & & & \\ & k_2 & & \\ & & \ddots & \\ & & & k_n
\end{bmatrix}_{n\times n}
& = & \begin{bmatrix}
k_1 a_{11} & k_2 a_{12} & \cdots & k_n a_{1n} \\ k_1 a_{21} & k_2 a_{22} & \cdots & k_n a_{2n} \\ \vdots & \vdots & \vdots & \vdots \\ k_1 a_{m1} & k_2 a_{m2} & \cdots & k_n a_{mn}
\end{bmatrix}_{m\times n}
\end{eqnarray*}
\end{eg}

\begin{eg}
设$X = \begin{bmatrix} x_1 \\ x_2 \\ x_3 \end{bmatrix}, Y = \begin{bmatrix} y_1 & y_2 & y_3 \end{bmatrix}$,求$(XY)^{100}$。
\end{eg}

\begin{solution}
\begin{eqnarray*}
(XY)^{100} & = & (XY)(XY)\cdots (XY) \\
& = & X(YX)(YX)\cdots (YX)Y \\
& = & X(YX)^{99}Y \\
& = & (YX)^{99}XY \\
& = & (x_1y_1 + x_2y_2 + x_3y_3)^{99} \begin{bmatrix} x_1y_1 & x_1y_2 & x_1y_3 \\ x_2y_1 & x_2y_2 & x_2y_3 \\ x_3y_1 & x_3y_2 & x_3y_3 \end{bmatrix}.
\end{eqnarray*}
\end{solution}

\begin{eg}
已知$A = \begin{bmatrix} 2 & 0 & -1 \\ 1 & 3 & 2 \end{bmatrix}, B = \begin{bmatrix} 1 & 7 & -1 \\ 4 & 2 & 3 \\ 2 & 0 & 1 \end{bmatrix}$,求$(AB)^T$。
\end{eg}

\begin{solution}[解法一]
\begin{eqnarray*}
& AB & = \begin{bmatrix} 2 & 0 & -1 \\ 1 & 3 & 2 \end{bmatrix} \begin{bmatrix} 1 & 7 & -1 \\ 4 & 2 & 3 \\ 2 & 0 & 1 \end{bmatrix} = \begin{bmatrix} 0 & 14 & -3 \\ 17 & 13 & 10 \end{bmatrix}\\
\Longrightarrow & (AB)^T & = \begin{bmatrix} 0 & 17 \\ 14 & 13 \\ -3 & 10 \end{bmatrix}
\end{eqnarray*}
\end{solution}

\begin{solution}[解法二]
$$(AB)^T = B^T A^T =  \begin{bmatrix} 1 & 4 & 2 \\ 7 & 2 & 0 \\ -1 & 3 & 1 \end{bmatrix} \begin{bmatrix} 2 & 1 \\ 0 & 3 \\ -1 & 2 \end{bmatrix} = \begin{bmatrix} 0 & 17 \\ 14 & 13 \\ -3 & 10 \end{bmatrix}.$$
\end{solution}

\begin{eg}
试证:奇数阶反对称矩阵的行列式一定为$0$。
\end{eg}

\begin{proof}[证明]
设$n$为奇数,$A$为$n$阶反对称矩阵,则
\begin{eqnarray*}
& |A| & = |A^T| = |-A| = (-1)^n|A| = -|A| \\
\Longrightarrow & 2|A| & = 0 \Longrightarrow |A| = 0.
\end{eqnarray*}
\end{proof}

\begin{eg}
矩阵的分块运算:
\[
  \setlength{\dashlinegap}{2pt}
  A = \left[ \begin{array}{c:cc}
    2 & 0 & 0 \\
    1 & 0 & 0 \\
    \hdashline
    3 & 1 & 0 \\
    2 & 0 & 1
  \end{array} \right],
  B = \left[ \begin{array}{cc:ccc}
    1 & 2 & 0 & 0 & 0\\
    \hdashline
    2 & 3 & 1 & 3 & 1 \\
    0 & 2 & 2 & 0 & 1
  \end{array} \right], \text{ 那么 }
  AB = \left[ \begin{array}{cc:ccc}
    2 & 4 & 0 & 0 & 0\\
    1 & 2 & 0 & 0 & 0\\
    \hdashline
    5 & 9 & 1 & 3 & 1 \\
    2 & 6 & 2 & 0 & 1
  \end{array} \right]
\]
\end{eg}

\begin{eg}
矩阵的分块运算:
\[
  \setlength{\dashlinegap}{2pt}
  A = \left[ \begin{array}{c:cc}
    2 & 0 & 0 \\
    1 & 0 & 0 \\
    \hdashline
    3 & 1 & 0 \\
    2 & 0 & 1
  \end{array} \right],
  A^T = \left[ \begin{array}{cc:cc}
    2 & 1 & 0 & 0 \\
    \hdashline
    0 & 0 & 1 & 0 \\
    0 & 0 & 0 & 1
  \end{array} \right],
\]
将$A$与$A^T$如下分块,有
\begin{eqnarray*}
A = \begin{bmatrix}
A_{11} & A_{12} \\ A_{21} & A_{22}
\end{bmatrix},
& \text{ 其中 } &
A_{11} = \begin{bmatrix} 2 \\ 1 \end{bmatrix}, A_{12} = 0_{2\times 2}, A_{21} = \begin{bmatrix} 3 \\ 2 \end{bmatrix}, A_{22} = I_2, \\
A^T = \begin{bmatrix}
B_{11} & B_{12} \\ B_{21} & B_{22}
\end{bmatrix},
& \text{ 其中 } &
B_{11} = \begin{bmatrix} 2 & 1 \end{bmatrix}, B_{12} = \begin{bmatrix} 3 & 2 \end{bmatrix}, B_{21} = 0_{2\times 2}, B_{22} = I_2.
\end{eqnarray*}
\begin{eqnarray*}
& \Longrightarrow & \qquad A_{11} = B_{11}^T, A_{12} = B_{21}^T = 0_{2\times 2}, A_{21} = B_{12}^T, A_{22} = B_{22}^T = I_2 \\
& \Longrightarrow & \qquad A_{ij} = B_{ji}^T, (1 \leqslant i,j \leqslant 2).
\end{eqnarray*}
\end{eg}

\begin{eg}
用初等矩阵与初等变换的关系,再次验证行列式的性质
\end{eg}

\begin{solution}[验证]\

\enum
\item[(1)] 逐行保数乘:
\begin{eqnarray*}
A \xrightarrow{kr_i} B & \Longleftrightarrow & B = E_i(k)A; \\
& \Longrightarrow & |B| = |E_i(k)|\cdot|A| = k|A|;
\end{eqnarray*}
\item[(2)] 交错性:
\begin{eqnarray*}
A \xrightarrow{r_i\leftrightarrow r_j} B & \Longleftrightarrow & B = E_{ij}A; \\
& \Longrightarrow & |B| = |E_{ij}|\cdot|A| = (-1)|A| = -|A|;
\end{eqnarray*}
\item[(3)] 倍加不变性:
\begin{eqnarray*}
A \xrightarrow{kr_i + r_j} B & \Longleftrightarrow & B = E_{ij}(k)A; \\
& \Longrightarrow & |B| = |E_{ij}(k)|\cdot|A| = 1\cdot|A| = |A|.
\end{eqnarray*}
\end{list}
\end{solution}

\begin{eg}
三类初等矩阵都是可逆矩阵,且
\begin{eqnarray*}
E_i(k)^{-1} & = & E_i(1/k) \quad (k\neq 0); \\
E_{i,j}^{-1} & = & E_{i,j}; \\
E_{i,j}(k)^{-1} & = & E_{i,j}(-k).
\end{eqnarray*}
\end{eg}

\begin{eg}
考虑矩阵$A = \begin{bmatrix} 1 & -1 \\ 0 & 0 \end{bmatrix}$,对任意$2$阶方阵$B= \begin{bmatrix} a & b \\ c & d \end{bmatrix}$有
$$AB = \begin{bmatrix} 1 & -1 \\ 0 & 0 \end{bmatrix} \begin{bmatrix} a & b \\ c & d \end{bmatrix} = \begin{bmatrix} a-c & b-d \\ 0 & 0 \end{bmatrix} \neq \begin{bmatrix} 1 & 0 \\ 0 & 1 \end{bmatrix} = I_2,$$
从而,并不是所有的方阵都可逆。
\end{eg}

\begin{eg}
若二阶方阵的行列式$|A| \neq 0$,于是
\begin{eqnarray*}
A & = & \begin{bmatrix} a_{11} & a_{12} \\ a_{21} & a_{22} \end{bmatrix} \Rightarrow A^{\ast} = \begin{bmatrix} A_{11} & A_{21} \\ A_{12} & A_{22} \end{bmatrix} = \begin{bmatrix} a_{22} & -a_{12} \\ -a_{21} & a_{11} \end{bmatrix} \\
& \Rightarrow & A^{-1} = |A|^{-1}A^{\ast} = \frac{1}{a_{11}a_{22} - a_{12}a_{21}} \begin{bmatrix} a_{22} & -a_{12} \\ -a_{21} & a_{11} \end{bmatrix}.
\end{eqnarray*}
\end{eg}

\begin{eg}
求方阵$A = \begin{bmatrix} 1 & 2 & 3 \\ 2 & 2 & 1 \\ 3 & 4 & 3 \end{bmatrix}$的逆矩阵。
\end{eg}

\begin{solution}
因为$|A| = \begin{vmatrix} 1 & 2 & 3 \\ 2 & 2 & 1 \\ 3 & 4 & 3 \end{vmatrix} = 2 \neq 0$,所以可以用伴随矩阵求逆法。
$$A_{11} = \begin{vmatrix} 2 & 1 \\ 4 & 3 \end{vmatrix} = 2, A_{12} = -\begin{vmatrix} 2 & 1 \\ 3 & 3 \end{vmatrix} = -3, A_{13} = \begin{vmatrix} 2 & 2 \\ 3 & 4 \end{vmatrix} = 2,$$
同理可得$A_{21} = 6, A_{22} = -6, A_{23} = 2, A_{31} = -4, A_{32} = 5, A_{33} = -2$。故
$$A^{-1} = \frac{1}{|A|}A^{\ast} = \frac12 \begin{bmatrix} 2 & 6 & -4 \\ -3 & -6 & 5 \\ 2 & 2 & -2 \end{bmatrix} = \begin{bmatrix} 1 & 3 & -2 \\ -3/2 & -3 & 5/2 \\ 1 & 1 & -1 \end{bmatrix}.$$
\end{solution}

\begin{eg}
设$A = \begin{bmatrix} a_{11} & & & \\ & a_{22} & & \\ & & \ddots & \\ & & & a_{nn} \end{bmatrix}$,其中$a_{ii} \neq 0, 1 \leqslant i \leqslant n$,求$A^{-1}$。
\end{eg}

\begin{solution}
若$AB = I_n$,其中$A$与$I_n$均为对角阵,猜测$B$也是($n$阶)对角阵;再由条件$a_{ii}\neq 0$,用定义可验证结果。

由于$a_{ii} \neq 0$,有
\begin{eqnarray*}
& \begin{bmatrix} a_{11} & & & \\ & a_{22} & & \\ & & \ddots & \\ & & & a_{nn} \end{bmatrix} & \begin{bmatrix} a_{11}^{-1} & & & \\ & a_{22}^{-1} & & \\ & & \ddots & \\ & & & a_{nn}^{-1} \end{bmatrix} = \begin{bmatrix} a_{11}a_{11}^{-1} & & & \\ & a_{22}a_{22}^{-1} & & \\ & & \ddots & \\ & & & a_{nn}a_{nn}^{-1} \end{bmatrix} = I_n \\
& \Longrightarrow & A^{-1} = \begin{bmatrix} a_{11}^{-1} & & & \\ & a_{22}^{-1} & & \\ & & \ddots & \\ & & & a_{nn}^{-1} \end{bmatrix}.
\end{eqnarray*}
\end{solution}

\begin{eg}
设$A = \begin{bmatrix} a_{11} & a_{12} \\ 0 & a_{22} \end{bmatrix}$,其中$a_{11}, a_{22} \neq 0$,求$A^{-1}$。
\end{eg}

\begin{solution}
若$AB = I_2$,其中$A$与$I_2$均为上三角阵,猜测$B$也是$2$阶上三角阵,再由待定系数法,可求出$B$。

设$B = \begin{bmatrix} b_{11} & b_{12} \\ 0 & b_{22} \end{bmatrix}$,若
\begin{eqnarray*}
& & I_2 = \begin{bmatrix} a_{11} & a_{12} \\ 0 & a_{22} \end{bmatrix} \begin{bmatrix} b_{11} & b_{12} \\ 0 & b_{22} \end{bmatrix} = \begin{bmatrix} a_{11}b_{11} & a_{11}b_{12} + a_{12}b_{22} \\ 0 & a_{22}b_{22} \end{bmatrix} \\
& \Longrightarrow & \begin{cases} a_{11}b_{11} = 1 \\ a_{22}b_{22} = 1 \\ a_{11}b_{12} + a_{12}b_{22} = 0 \end{cases} \\
& \Longrightarrow & \begin{cases} b_{11} = a_{11}^{-1} \\ b_{22} = a_{22}^{-1} \\ b_{12} = -a_{11}^{-1}a_{12}a_{22}^{-1} \end{cases} \\
& \Longrightarrow & A^{-1} = \begin{bmatrix} a_{11}^{-1} & -a_{11}^{-1}a_{12}a_{22}^{-1} \\ 0 & a_{22}^{-1} \end{bmatrix}
\end{eqnarray*}
\end{solution}

\begin{eg}
设方阵$A$满足$A^2-2A+4I=O$,证明:$A+I$和$A-3I$都可逆,并求它们的逆矩阵。
\end{eg}

\begin{solution}
关于$A$是多项式,$A^i$与$A^j$乘法可交换,可按通常方法分解多项式,希望能凑出$A+I$和$A-3I$的项。

\begin{eqnarray*}
& & A^2-2A+4I = (A+I)(A-3I)+7I = O \\
& \Longrightarrow & -\frac17 (A+I)(A-3I) = I.
\end{eqnarray*}
由定义知
$$(A+I)^{-1} = -\frac17 (A-3I), \quad (A-3I)^{-1} = -\frac17 (A+I).$$
\end{solution}

\begin{eg}
用伴随矩阵求下面方阵$A$的逆矩阵:$A = \begin{bmatrix} 2 & 3 & -1 \\ 1 & 2 & 0 \\ 1 & 2 & -2 \end{bmatrix}.$
\end{eg}

\begin{solution}
$\because \ |A| = \begin{vmatrix} 2 & 3 & -1 \\ 1 & 2 & 0 \\ 1 & 2 & -2 \end{vmatrix} = -2 \neq 0,$ 所以方阵$A$可逆。
$$A_{11} = \begin{vmatrix} 2 & 0 \\ 2 & -2 \end{vmatrix} = -4, A_{21} = -\begin{vmatrix} 3 & -1 \\ 2 & -2 \end{vmatrix} = 4, A_{31} = \begin{vmatrix} 3 & -1 \\ 2 & 0 \end{vmatrix} = 2,$$
$$A_{12} = -\begin{vmatrix} 1 & 0 \\ 1 & -2 \end{vmatrix} = 2, A_{22} = \begin{vmatrix} 2 & -1 \\ 1 & -2 \end{vmatrix} = -3, A_{32} = -\begin{vmatrix} 2 & -1 \\ 1 & 0 \end{vmatrix} = -1,$$
$$A_{13} = \begin{vmatrix} 1 & 2 \\ 1 & 2 \end{vmatrix} = 0, A_{23} = -\begin{vmatrix} 2 & 3 \\ 1 & 2 \end{vmatrix} = -1, A_{32} = \begin{vmatrix} 2 & 3 \\ 1 & 2 \end{vmatrix} = 1,$$
从而
$$A^{\ast} = \begin{bmatrix} -4 & 4 & 2 \\ 2 & -3 & -1 \\ 0 & -1 & 1 \end{bmatrix},$$
于是
$$A^{-1} = |A|^{-1}A^{\ast} = \begin{bmatrix} 2 & -2 & -1 \\ -1 & 3/2 & 1/2 \\ 0 & 1/2 & -1/2 \end{bmatrix}.$$
\end{solution}

\begin{eg}
用初等行变换求矩阵$A$的逆矩阵:$A = \begin{bmatrix} 0 & 2 & -1 \\ 1 & 1 & 2 \\ -1 & -1 & -1 \end{bmatrix}$。
\end{eg}

\begin{solution}
先将$A$化为阶梯形矩阵,再化为单位阵:
\begin{eqnarray*}
[A,I] & = &
  \setlength{\dashlinegap}{2pt}
  \left[ \begin{array}{ccc:ccc}
    0 & 2 & -1 & 1 & 0 & 0 \\ 1 & 1 & 2 & 0 & 1 & 0 \\ -1 & -1 & -1 & 0 & 0 & 1
  \end{array} \right] \xrightarrow[r_1+r_3]{r_1\leftrightarrow r_2}
  \left[ \begin{array}{ccc:ccc}
     1 & 1 & 2 & 0 & 1 & 0 \\ 0 & 2 & -1 & 1 & 0 & 0 \\ 0 & 0 & 1 & 0 & 1 & 1
  \end{array} \right] \\
  & \longrightarrow & \left[ \begin{array}{ccc:ccc}
     1 & 0 & 0 & -\frac12 & -\frac32 & -\frac52 \\ 0 & 1 & 0 & \frac12 & \frac12 & \frac12 \\ 0 & 0 & 1 & 0 & 1 & 1 \end{array} \right], \\
\Longrightarrow & A^{-1} & = \frac12\begin{bmatrix} -1 & -3 & -5 \\ 1 & 1 & 1 \\ 0 & 2 & 2 \end{bmatrix}
\end{eqnarray*}
\end{solution}

\begin{eg}
设$A = \begin{bmatrix} 1 & 1 & 2 \\ 0 & 2 & -1 \\ -1 & 1 & -3 \end{bmatrix}$,试判断$A$是否可逆。
\end{eg}

\begin{solution}
\begin{eqnarray*}
[A,I] & = &
  \setlength{\dashlinegap}{2pt}
  \left[ \begin{array}{ccc:ccc}
    1 & 1 & 2 & 1 & 0 & 0 \\ 0 & 2 & -1 & 0 & 1 & 0 \\ -1 & 1 & -3 & 0 & 0 & 1
  \end{array} \right] \longrightarrow
  \left[ \begin{array}{ccc:ccc}
     1 & 1 & 2 & 1 & 0 & 0 \\ 0 & 2 & -1 & 0 & 1 & 0 \\ 0 & 2 & -1 & 1 & 0 & 1
  \end{array} \right] \\
  & \longrightarrow & \left[ \begin{array}{ccc:ccc}
      1 & 1 & 2 & 1 & 0 & 0 \\ 0 & 2 & -1 & 0 & 1 & 0 \\ 0 & 0 & 0 & 1 & -1 & 1
   \end{array} \right],
\end{eqnarray*}
此时,$A$经初等行变换化为阶梯形矩阵时,出现全零行,则$A$的行列式为零,故$A$不可逆。
\end{solution}

\begin{eg}
求如下矩阵方程$XA=C$,其中$A = \begin{bmatrix} 1 & 2 \\ 3 & 4 \end{bmatrix}, C = \begin{bmatrix} 0 & 2 \\ 5 & 6 \\ 7 & 8 \end{bmatrix}$。
\end{eg}

\begin{solution}
由$|A|=-2$知,$A$可逆,则$X=CA^{-1}$,对如下分块矩阵进行初等列变换:
\begin{eqnarray*}
\begin{bmatrix} A \\ C \end{bmatrix} & = &
  \setlength{\dashlinegap}{2pt}
  \left[ \begin{array}{cc}
    1 & 2 \\ 3 & 4 \\ \hdashline  0 & 2 \\ 5 & 6 \\ 7 & 8 \end{array} \right]
  \xrightarrow{(-2)c_1+c_2}
  \left[ \begin{array}{cc}
    1 & 0 \\ 3 & -2 \\ \hdashline  0 & 2 \\ 5 & -4 \\ 7 & -6 \end{array} \right]
  \xrightarrow{(-\frac12)c_2}
  \left[ \begin{array}{cc}
    1 & 0 \\ 3 & 1 \\ \hdashline  0 & -1 \\ 5 & 2 \\ 7 & 3 \end{array} \right] \\
  & \xrightarrow{(-3)c_2 + c_1} &
  \left[ \begin{array}{cc}
    1 & 0 \\ 0 & 1 \\ \hdashline  3 & -1 \\ -1 & 2 \\ -2 & 3 \end{array} \right]
  = \begin{bmatrix} I \\ CA^{-1} \end{bmatrix} \\
\Longrightarrow & X=CA^{-1} & = \left[ \begin{array}{cc}
    3 & -1 \\ -1 & 2 \\ -2 & 3 \end{array} \right].
\end{eqnarray*}
\end{solution}

\begin{eg}
试判断矩阵$A = \begin{bmatrix} 2 & 0 & 0 & 0 \\ 1 & 2 & 1 & 0 \\ 0 & 0 & 1 & 0 \\ 0 & 0 & 1 & 1 \end{bmatrix}$是否可逆?若可逆,求出$A^{–1}$。
\end{eg}

\begin{solution}
设$A = \begin{bmatrix} A_{11} & A_{12} \\ 0 & A_{22} \end{bmatrix}$,其中
$A_{11} = \begin{bmatrix} 2 & 0 \\ 1 & 2 \end{bmatrix}, A_{12} = \begin{bmatrix} 0 & 0 \\ 1 & 0 \end{bmatrix}, A_{22} = \begin{bmatrix} 1 & 0 \\ 1 & 1 \end{bmatrix}$,且$|A_{11}| = 4, |A_{22}| = 1$,从而$|A| = |A_{11}|\cdot|A_{22}| = 4$,所以$A, A_{11}, A_{22}$均可逆。

设$A^{–1} = \begin{bmatrix} X & Y \\ 0 & Z \end{bmatrix}$,则
$$\begin{bmatrix} A_{11} & A_{12} \\ 0 & A_{22} \end{bmatrix}\begin{bmatrix} X & Y \\ 0 & Z \end{bmatrix} = I \Longrightarrow
\begin{cases} X = A_{11}^{–1} \\ Z = A_{22}^{–1} \\ A_{11}Y + A_{12}A_{22}^{–1} = 0 \end{cases} \Longrightarrow
Y = -A_{11}^{–1}A_{12}A_{22}^{–1},$$
所以,
$$A^{–1} = \begin{bmatrix} A_{11}^{–1} & -A_{11}^{–1}A_{12}A_{22}^{–1} \\ 0 & A_{22}^{–1} \end{bmatrix} = \begin{bmatrix} \frac12 & 0 & 0 & 0 \\ -1 & \frac12 & -\frac12 & 0 \\ 0 & 0 & 1 & 0 \\ 0 & 0 & -1 & 1 \end{bmatrix}.$$
\end{solution}

\begin{eg}
设$M = \begin{bmatrix} A & B \\ C & D \end{bmatrix}$,其中$A$可逆,$D$为方阵,试证$|M| = |A||D-CA^{-1}B|$。进而证明,$M$可逆$\Longleftrightarrow$ $D-CA^{-1}B$可逆,并求$M$的逆。
\end{eg}

\begin{proof}[证明]
\begin{eqnarray*}
& & \begin{bmatrix} I & 0 \\ CA^{-1} & I \end{bmatrix} \begin{bmatrix} A & B \\ C & D \end{bmatrix} \begin{bmatrix} I & -A^{-1}B \\ 0 & I \end{bmatrix} = \begin{bmatrix} A & 0 \\ 0 & D-CA^{-1}B \end{bmatrix} \\
& \Longrightarrow & |M| = \begin{vmatrix} A & 0 \\ 0 & D-CA^{-1}B \end{vmatrix} = |A||D-CA^{-1}B|.
\end{eqnarray*}
从而$M$可逆$\Longleftrightarrow$ $D-CA^{-1}B$可逆,且
$$M^{-1} = \begin{bmatrix} I & -A^{-1}B \\ 0 & I \end{bmatrix} \begin{bmatrix} A^{-1} & 0 \\ 0 & (D-CA^{-1}B)^{-1} \end{bmatrix} \begin{bmatrix} I & 0 \\ -CA^{-1} & I \end{bmatrix}.$$
\end{proof}

%%%%%%%%%%%%%%%%%%%%%%%%%%%%%%%%%%%%%%%%%%%%%%%%%%%%%%%%%%%%%%%%%%%%%%%%%%%%%%%%%%%%%%%%%%%%

\section{课后习题}

\begin{ex} \label{ex:3.1}
分别给出$3\times3$阶的对角阵,上三角阵,对称阵,反对称阵的一个例子。
\end{ex}

\begin{ex} \label{ex:3.2}
对$n \geqslant 2$,构造$n$阶方阵$A,B$,使得$\det(A+B)\neq\det(A)+\det(B)$。
\end{ex}

\begin{ex} \label{ex:3.3}
设$A = \begin{bmatrix} 1 & -2 & -2 \\ -1 & 1 & -2 \\ 1 & 1 & -2 \end{bmatrix}, B = \begin{bmatrix} 3 & -1 & 5 \\ 1 & -3 & 1 \\ -4 & 5 & -1 \end{bmatrix}$,求$AB,BA$以及$AB-BA$。
\end{ex}

\begin{ex} \label{ex:3.4}
设$A = \begin{bmatrix} 1 & 2 & -1 \\ 2 & 3 & 2 \\ -1 & 0 & 2 \end{bmatrix}, B = \begin{bmatrix} 0 & 1 & 2 \\ 2 & -1 & 0 \\ -1 & -1 & 3 \end{bmatrix}$,计算$A^T,B^T,AB,A^TB^T$。
\end{ex}

\begin{ex} \label{ex:3.5}
设$A = \begin{bmatrix} 1 & -3 & 4 \\ 0 & 1 & -2 \\ 2 & -3 & 3 \end{bmatrix}$,求$A$的伴随矩阵。
\end{ex}

\begin{ex} \label{ex:3.6}
以下方阵是否可逆?如果可逆,求其逆矩阵。

$$\begin{bmatrix} \cos\varphi & -\sin\varphi \\ \sin\varphi & \cos\varphi \end{bmatrix}$$
\end{ex}

\begin{ex} \label{ex:3.7}
求以下可逆阵的逆矩阵。

\enum
\item[(1)] $\begin{bmatrix} a & b \\ c & d \end{bmatrix}$,满足$ad-bc\neq 0$
\item[(2)] $\begin{bmatrix}  \cos\varphi & 0 & -\sin\varphi \\ 0 & 1 & 0 \\ \sin\varphi & 0 & \cos\varphi \end{bmatrix}$
\end{list}
\end{ex}

\begin{ex}\  \label{ex:3.8}

\enum
\item[(1)] 设$A\in M_n(\mathbb{R}), B, E\in M_{n\times m}(\mathbb{R}), C, F\in M_{m\times n}(\mathbb{R}), D\in M_m(\mathbb{R})$,证明:矩阵$\begin{bmatrix} A & B \\ C & D \end{bmatrix}$ 与矩阵$\begin{bmatrix} A + EC & B + ED \\ C & D \end{bmatrix},$ $\begin{bmatrix} A & B \\ FA + C & FB + D \end{bmatrix},$ $\begin{bmatrix} A & AE + B \\ C & CE + D \end{bmatrix},$ $\begin{bmatrix} A + BF & B \\ C + DF & D \end{bmatrix}$均有相同的行列式。

\item[(2)] 设$U,V\in M_n(\mathbb{R})$,证明$\det(I_n - UV) = \det(I_n-VU)$。

\item[(3)] 设$P\in M_{n\times (n-1)}(\mathbb{R}), Q\in M_{(n - 1)\times n}(\mathbb{R})$。证明$PQ \in M_n(\mathbb{R})$的行列式为$0$。

\item[(4)] 设$\alpha, \beta$为$n$阶列向量,证明$\det(I_n - \alpha\beta^T) = 1 - \alpha^T\beta$。

\item[(5)] 计算$\begin{bmatrix} \lambda & 1 & \cdots & 1 \\ 1 & \lambda & & \vdots \\ \vdots & & \ddots & 1 \\ 1 & \cdots & 1 & \lambda \end{bmatrix} \in M_n(\mathbb{R})$的行列式。
\end{list}
\end{ex}

\begin{ex} \label{ex:3.9}
设$A, B$为同阶方阵,证明$\begin{vmatrix} A & B \\ B & A \end{vmatrix} = |A + B||A - B|$。
\end{ex}

\begin{ex} \label{ex:3.10}
对$n\geqslant 1$,构造$n$阶方阵$A, B$,使得$A, B, A+B$均可逆,但$(A+B)^{-1}\neq A^{-1}+B^{-1}$。
\end{ex}

\begin{ex} \label{ex:3.11}
设$A = \begin{bmatrix} 0 & -1 & 2 \\ -1 & 0 & -1 \end{bmatrix}, B = \begin{bmatrix} 2 & -1 & -3 \\ 1 & -3 & -3 \\ 1 & 0 & -3 \end{bmatrix}, C = \begin{bmatrix} 2 & -2 \\ 1 & -2 \\ 2 & 2 \end{bmatrix}$, 求$ABC$。
\end{ex}

\begin{ex} \label{ex:3.12}
计算

\enum
\item[(1)] $\begin{bmatrix} \cos\varphi & -\sin\varphi \\ \sin\varphi & \cos\varphi \end{bmatrix}^n$

\item[(2)] $\begin{bmatrix} 1 & \alpha & \beta \\ & 1 & \alpha \\ & & 1 \end{bmatrix}^{n}$

\item[(3)] $\begin{bmatrix} 0 & 1 & 0 & 0 \\ 0 & 0 & 1 & 0 \\ 0 & 0 & 0 & 1 \\ 0 & 0 & 0 & 0\end{bmatrix}^3$

\item[(4)] $\begin{bmatrix} \lambda & 1 & 0 & 0 \\ 0 & \lambda & 1 & 0 \\ 0 & 0 & \lambda & 1 \\ 0 & 0 & 0 & \lambda\end{bmatrix}^3$
\end{list}
\end{ex}

\begin{ex} \label{ex:3.13}
解矩阵方程:

\enum
\item[(1)] $\begin{bmatrix} 1 & 1 \\ 2 & 3 \end{bmatrix} X = \begin{bmatrix} 1 & 1 & 1 \\ 1 & 1 & 1 \end{bmatrix}$

\item[(2)] $\begin{bmatrix} 1 & 0 & 4 \\ -1 & 1 & -2 \\ 0 & 0 & 1 \end{bmatrix} X \begin{bmatrix} -1 & -3 & 0 \\ 1 & 4 & 4 \\ -1 & -3 & -1 \end{bmatrix} = \begin{bmatrix} 1 & 0 & -6 \\ -5 & -13 & 3 \\ -1 & -4 & -3 \end{bmatrix}$
\end{list}
\end{ex}

\begin{ex} \label{ex:3.14}
设$A = \begin{bmatrix} c & 0 & 0 \\ 1 & c & 0 \\ 0 & 1 & c \end{bmatrix}, f(\lambda) = \lambda^2 - 2\lambda - 1$。求$f(A)$。
\end{ex}

\begin{ex} \label{ex:3.15}
设$A\in M_n(\mathbb{R})$为一个$n$阶方阵,

\enum
\item[(1)] 证明:如果$A$是对角阵且其主对角线上的元素各不相同,则任一与$A$乘法可交换的矩阵也是对角阵。

\item[(2)] 设$P\in M_n(\mathbb{R})$, $P$可逆,$PAP^{-1}$为对角阵且其对角线上元素各不相同。设$B\in M_n(\mathbb{R}), AB = BA$。证明$PBP^{-1}$为对角阵。
\end{list}
\end{ex}

\begin{ex} \label{ex:3.16}
求平方等于单位阵的所有二阶方阵。
\end{ex}

\begin{ex} \label{ex:3.17}
求幂等(即平方等于自身)的所有二阶方阵。
\end{ex}

\begin{ex} \label{ex:3.18}
设$A = (a_{ij})$为$n$阶上三角方阵且对角线上的元素均为$0$,求$A^{n-1}, A^n$。
\end{ex}

\begin{ex} \label{ex:3.19}
证明:

\enum
\item[(1)] 对方阵$A = (a_{ij}) \in M_n(\mathbb{R})$,定义$A$的迹(trace)为$tr(A) = \sum\limits_{i=1}^n a_i$。证明:任取方阵$A, B\in M_n(\mathbb{R})$,总有$tr(AB) = tr(BA)$。

\item[(2)] 对任意方阵$A, B\in M_n(\mathbb{R})$, $AB - BA \neq I_n$。
\end{list}
\end{ex}

\begin{ex} \label{ex:3.20}
所有与下列方阵$A$可交换的方阵$B$(即满足$AB = BA$):

\enum
\item[(1)] $A = \begin{bmatrix} 1 & 1 \\ 0 & 0 \end{bmatrix}$ \item[(2)] $\begin{bmatrix} 0 & 1 & 0 & 0 \\ 0 & 0 & 1 & 0 \\ 0 & 0 & 0 & 1 \\ 0 & 0 & 0 & 0 \end{bmatrix}$
\item[(3)] $\begin{bmatrix} 1 & & & \\ & 1 & & \\ & & 2 & \\ & & & 2 \end{bmatrix}$
\end{list}
\end{ex}

\begin{ex} \label{ex:3.21}
证明:与所有$n$阶方阵均可交换的方阵必为数量阵(即$aI_n, a\in\mathbb{R}$)。
\end{ex}

\begin{ex} \label{ex:3.22}
设$J = \begin{bmatrix} & I_n \\ -I_n & \end{bmatrix} \in M_{2n}(\mathbb{R})$,求

\enum
\item[(1)] $\det J$
\item[(2)] $J^2$
\end{list}
\end{ex}

\begin{ex} \label{ex:3.23}
设$A$为实对称矩阵,若$A^2 = 0$,求证$A = 0$。
\end{ex}

\begin{ex} \label{ex:3.24}
证明:任一方阵均可表示为一个对称阵和一个反对称阵之和。
\end{ex}

\begin{ex} \label{ex:3.25}
证明:若$A, B, A + B$均为可逆阵,证明$A^{-1} + B^{-1}$也为可逆阵。
\end{ex}

\begin{ex} \label{ex:3.26}
设$A$是$n$阶方阵,满足$A^k = 0, A^{k-1} \neq 0$,(这样的方阵被称为一个幂零阵),证明$I_n - A$可逆。
\end{ex}

\begin{ex} \label{ex:3.27}
设$A$是$n$阶方阵,满足$A^2 = A$(这样的方阵被称为一个幂等阵),证明$A + I_n$可逆。
\end{ex}

\begin{ex} \label{ex:3.28}
证明:
\enum
\item[(1)] 上(下)三角阵的逆矩阵(如果存在)也是上(下)三角阵。
\item[(2)] 对称阵的逆矩阵(如果存在)也是对称阵。
\item[(3)] 反对称阵的逆矩阵(如果存在)也是反对称阵。
\end{list}
\end{ex}

\begin{ex} \label{ex:3.29}
设$A$是$n$阶可逆阵,若$A$的每一行元素之和等于常数$c$,求证:$c\neq 0$,且$A^{-1}$的每一行元素之和等于$c^{-1}$。
\end{ex}

\begin{ex} \label{ex:3.30}
设$A$为$n$阶可逆阵,$u,v$是$n$维列向量,且$1 + v^T A^{-1}u \neq 0$,求证
$$(A+uv^T)^{-1} = A^{-1} - \frac{A^{-1}uv^T A^{-1}}{1 + v^T A^{-1}u}.$$
\end{ex}

\begin{ex} \label{ex:3.31}
设$A$为$n$阶可逆阵,$U$是$n\times k$阶矩阵,$V$是$k\times n$阶矩阵,$C$是$k\times k$阶矩阵,且$ C^{-1}+VA^{-1}U$可逆,求证
$$ \left(A+UCV \right)^{-1} = A^{-1} - A^{-1}U \left(C^{-1}+VA^{-1}U \right)^{-1} VA^{-1}.$$
\end{ex}

\newpage

%%%%%%%%%%%%%%%%%%%%%%%%%%%%%%%%%%%%%%%%%%%%%%%%%%%%%%%%%%%%%%%%%%%%%%%%%%%%%%%%%

\section{习题答案}

\textbf{习题\ref{ex:3.1} 解答:}

例如
\begin{eqnarray*}
\text{对角阵} & : & \begin{bmatrix} 1 & & \\ & 2 & \\ & & 3 \end{bmatrix}, \\
\text{上三角阵} & : & \begin{bmatrix} 1 & 1 & 0 \\ 0 & 1 & 0 \\ 0 & 0 & 3 \end{bmatrix}, \\
\text{对称阵} & : & \begin{bmatrix} 1 & 2 & 3\\ 2 & 2 & 1 \\ 3 & 1 & -1 \end{bmatrix}, \\
\text{反对称阵} & : & \begin{bmatrix} 0 & 2 & 1\\ -2 & 0 & -1 \\ -1 & 1 & 0 \end{bmatrix}
\end{eqnarray*}

\vspace{1.5em}

\textbf{习题\ref{ex:3.2} 解答:}

$n \geqslant 2$时,取$B = A$,且$A$可逆,即有$\det(A+B) = \det(2A) = 2^n\det A \neq 2\det A = \det(A)+\det(B)$。

\vspace{1.5em}

\textbf{习题\ref{ex:3.3} 解答:}

$
AB = \begin{bmatrix} 9 & -5 & 5 \\ 6 & -12 & -2 \\ 12 & -14 & 8 \end{bmatrix},
BA = \begin{bmatrix} 9 & -2 & -14 \\ 5 & -4 & 2 \\ -10 & 12 & 0 \end{bmatrix},
AB - BA = \begin{bmatrix} 0 & -3 & 19 \\  1 & -8 & -4 \\ 22 & -26 & 8 \end{bmatrix}.
$

\vspace{1.5em}

\textbf{习题\ref{ex:3.4} 解答:}

$A^T = \begin{bmatrix} 1 & 2 & -1 \\ 2 & 3 & 0 \\ -1 & 2 & 2 \end{bmatrix}, B^T = \begin{bmatrix} 0 & 2 & -1 \\ 1 & -1 & -1 \\ 2 & 0 & 3 \end{bmatrix}, AB = \begin{bmatrix} 5 & 0 & -1 \\ 4  & -3 & 10 \\ -2 & -3 & 4 \end{bmatrix}, A^TB^T = \begin{bmatrix} 0 & 0 & -6 \\ 3 & 1 & -5 \\ 6 & -4 & 5 \end{bmatrix}$。

\vspace{1.5em}

\textbf{习题\ref{ex:3.5} 解答:}

设$A^{\ast} = (b_{ij})_{1 \leqslant i,j \leqslant 3}$,那么$b_{ij} = A_{ji}$,$A_{ji}$为矩阵$A$的第$(j,i)$位元素的代数余子式。$A_{11} = \begin{vmatrix} 1 & -2 \\ -3 & 3 \end{vmatrix} = -3$,类似可以算出其他矩阵元素的值。最后的结论是:
$$A^{\ast} = \begin{bmatrix} -3 & -3 & 2 \\ -4 & -5 & 2 \\ -2 & -3 & 1 \end{bmatrix}$$

\vspace{1.5em}

\textbf{习题\ref{ex:3.6} 解答:}

行列式为$\cos^2\varphi + \sin^2\varphi = 1$,所以可逆。逆为$\begin{bmatrix} \cos\varphi & \sin\varphi \\ -\sin\varphi & \cos\varphi \end{bmatrix}$。

\vspace{1.5em}

\textbf{习题\ref{ex:3.7} 解答:}

\enum
\item[(1)] $\begin{bmatrix} a & b \\ c & d \end{bmatrix}^{-1} = \frac{1}{ad-bc}\begin{bmatrix} d & -b \\ -c & a \end{bmatrix}$
\item[(2)] $\begin{bmatrix}  \cos\varphi & 0 & -\sin\varphi \\ 0 & 1 & 0 \\ \sin\varphi & 0 & \cos\varphi \end{bmatrix}^{-1} = \begin{bmatrix}  \cos\varphi & 0 & \sin\varphi \\ 0 & 1 & 0 \\ -\sin\varphi & 0 & \cos\varphi \end{bmatrix}$
\end{list}

\vspace{1.5em}

\textbf{习题\ref{ex:3.8} 解答:}

\enum
\item[(1)] $\det\begin{bmatrix} A & B \\ C & D \end{bmatrix} = \det(\begin{bmatrix} I_n & E \\ 0 & I_m \end{bmatrix} \begin{bmatrix} A & B \\ C & D \end{bmatrix}) = \det \begin{bmatrix} A + EC & B + ED \\ C & D \end{bmatrix}$。其余类似可以证明。

\item[(2)] 由第(1)小题,$\det(I_n - UV) = \det\begin{bmatrix} I_n - UV & 0 \\ V & I_n \end{bmatrix} = \det\begin{bmatrix} I_n & U \\ V & I_n \end{bmatrix} = \det\begin{bmatrix} I_n & 0 \\ V & I_n - VU \end{bmatrix} = \det(I_n-VU)$。

\item[(3)] %因为$P,Q$的秩都至多为$n-1$,所以$PQ$的秩至多为$n-1$,不满秩,所以行列式为0。
假设$\det PQ \neq 0,$ 那么齐次线性方程组$PQX = 0, X = \begin{bmatrix} x_1 \\ \vdots \\ x_n \end{bmatrix},$ 只有零解。但是$QX=0$是$n-1$个方程,$n$个变元的齐次线性方程组,必有非零解$X_0,$ 于是$PQX_0 = P(QX_0) = 0,$ 与$PQX = 0$只有零解矛盾。因此$PQ$的行列式为0。

\item[(4)] 由第(1)小题,$\det(I_n - \alpha\beta^T) = \det\begin{bmatrix} I_n - \alpha\beta^T & \alpha \\ 0 & 1 \end{bmatrix} = \det\begin{bmatrix} I_n & \alpha \\ \beta^T & 1 \end{bmatrix} = \det\begin{bmatrix} I_n & 0 \\ \beta^T & 1 - \alpha^T\beta \end{bmatrix} = 1 - \alpha^T\beta$。

\item[(5)] $\begin{vmatrix} \lambda & 1 & \cdots & 1 \\ 1 & \lambda & & \vdots \\ \vdots & & \ddots & 1 \\ 1 & \cdots & 1 & \lambda \end{vmatrix} = \left| (\lambda - 1)I_n + \begin{bmatrix} 1 & 1 & \cdots & 1 \\ 1 & 1 & & \vdots \\ \vdots & & \ddots & 1 \\ 1 & \cdots & 1 & 1 \end{bmatrix} \right| = \left| (\lambda - 1)I_n + \begin{bmatrix} 1 \\ 1 \\ \vdots \\ 1 \end{bmatrix}(1,1,\cdots,1) \right|\\
= (\lambda-1)^{n-1}(\lambda - 1 + n)$。
\end{list}

\vspace{1.5em}

\textbf{习题\ref{ex:3.9} 解答:}

$\begin{vmatrix} A & B \\ B & A \end{vmatrix} = \begin{vmatrix} A+B & B+A \\ B & A \end{vmatrix} = \begin{vmatrix} A+B & 0 \\ B & A-B \end{vmatrix} = |A + B||A - B|$

\vspace{1.5em}

\textbf{习题\ref{ex:3.10} 解答:}

例如$A = \begin{bmatrix} 3 & & & \\ & 1 & & \\ & & \ddots & \\ & & & 1 \end{bmatrix}, B = \begin{bmatrix} 2 & & & \\ & 1 & & \\ & & \ddots & \\ & & & 1 \end{bmatrix}$,那么
$$(A+B)^{-1} = \begin{bmatrix} 5 & & & \\ & 1 & & \\ & & \ddots & \\ & & & 1 \end{bmatrix}^{-1} = \begin{bmatrix} \frac15 & & & \\ & 1 & & \\ & & \ddots & \\ & & & 1 \end{bmatrix},$$
而
$$A^{-1}+B^{-1} = \begin{bmatrix} \frac13 & & & \\ & 1 & & \\ & & \ddots & \\ & & & 1 \end{bmatrix} + \begin{bmatrix} \frac12 & & & \\ & 1 & & \\ & & \ddots & \\ & & & 1 \end{bmatrix} = \begin{bmatrix} \frac56 & & & \\ & 1 & & \\ & & \ddots & \\ & & & 1 \end{bmatrix},$$
二者并不相等。

\vspace{1.5em}

\textbf{习题\ref{ex:3.11} 解答:}

直接矩阵相乘,$ABC = \begin{bmatrix} -1 & -14 \\ 7 & 16 \end{bmatrix}$。

\vspace{1.5em}

\textbf{习题\ref{ex:3.12} 解答:}

\enum
\item[(1)] $\begin{bmatrix} \cos n\varphi & -\sin n\varphi \\ \sin n\varphi & \cos n\varphi \end{bmatrix}$

\item[(2)] $\begin{bmatrix} 1 & n\alpha & \frac{n(n-1)}{2}\alpha^2 + n\beta \\ & 1 & n\alpha \\ & & 1 \end{bmatrix}$

\item[(3)] $\begin{bmatrix} 0 & 0 & 0 & 1 \\ 0 & 0 & 0 & 0 \\ 0 & 0 & 0 & 0 \\ 0 & 0 & 0 & 0\end{bmatrix}$

\item[(4)] $\begin{bmatrix} \lambda^3 & 3\lambda^2 & 3\lambda & 1 \\ 0 & \lambda^3 & 3\lambda^2 & 3\lambda \\ 0 & 0 & \lambda^3 & 3\lambda^2 \\ 0 & 0 & 0 & \lambda^3\end{bmatrix}$
\end{list}

\vspace{1.5em}

\textbf{习题\ref{ex:3.13} 解答:}

\enum
\item[(1)] $X = \begin{bmatrix} 1 & 1 \\ 2 & 3 \end{bmatrix}^{-1} \begin{bmatrix} 1 & 1 & 1 \\ 1 & 1 & 1 \end{bmatrix} = \begin{bmatrix} 2 & 2 & 2 \\ -1 & -1 & -1 \end{bmatrix}$,

\item[(2)] $X = \begin{bmatrix} 1 & 0 & 4 \\ -1 & 1 & -2 \\ 0 & 0 & 1 \end{bmatrix}^{-1} \cdot \begin{bmatrix} 1 & 0 & -6 \\ -5 & -13 & 3 \\ -1 & -4 & -3 \end{bmatrix} \cdot \begin{bmatrix} -1 & -3 & 0 \\ 1 & 4 & 4 \\ -1 & -3 & -1 \end{bmatrix}^{-1} = \begin{bmatrix} -2 & 1 & -2 \\ 2 & 1 & 1 \\ 1 & -1 & -1 \end{bmatrix}$。
\end{list}

\vspace{1.5em}

\textbf{习题\ref{ex:3.14} 解答:}

$f(A) = \begin{bmatrix} c^2 - 2c - 1 & 0 & 0 \\ 2c - 2 & c^2 - 2c - 1 & 0 \\ 1 & 2c - 2 & c^2 - 2c - 1 \end{bmatrix}$。

\vspace{1.5em}

\textbf{习题\ref{ex:3.15} 解答:}

\enum
\item[(1)] 设$A = diag(\lambda_1,\cdots,\lambda_n)$,$B = (b_{ij})$与$A$可交换。分别考察$AB$与$BA$的第$(i,j)$位元素,便有$\lambda_ib_{ij} = \lambda_jb_{ij}$。由假设,如果$i\neq j$,便有$\lambda_i\neq\lambda_j$,因此必须有$b_{ij} = 0$。也就是说$B$必须为对角阵。

\item[(2)] 由于$AB = BA$,所以$PBP^{-1}$可以与$PAP^{-1}$交换,由第(1)小题结论以及第(2)小题题设即可知$PBP^{-1}$也为对角阵。
\end{list}

\vspace{1.5em}

\textbf{习题\ref{ex:3.16} 解答:}

设$A = \begin{bmatrix} a & b \\ c & d \end{bmatrix}$是平方等于单位阵的二阶方阵,那么我们会有
\begin{numcases}{ }
  a^2 + bc = 1  \notag \\
  ab + bd = 0 \notag \\
  ac + cd = 0 \notag \\
  d^2 + bc = 1 \notag
\end{numcases}
由$ab + bd = 0$知$b = 0$或$a + d = 0$。如果$b = 0$,那么由$a^2 + bc = 1,d^2 + bc = 1$知$a=\pm 1, d=\pm 1$,再由$ac + cd = 0$知当$a,d$符号相同时,$c = 0$,符号不同时,$c$可以任取。如果$b \neq 0$,那么必须有$a + d = 0$。只要再满足$a^2 + bc = 1$,其他两式自动成立。所以平方等于单位阵的二阶方阵有$\pm I_2$,以及
$$\begin{bmatrix} a & b \\ c & -a \end{bmatrix}, \quad a^2 + bc = 1$$

\vspace{1.5em}

\textbf{习题\ref{ex:3.17} 解答:}

类似第16题可以得幂等的二阶方阵有$0, I_2$,以及
$$\begin{bmatrix} a & b \\ c & 1-a \end{bmatrix}, \quad a^2 + bc = a$$

\vspace{1.5em}

\textbf{习题\ref{ex:3.18} 解答:}

$A^n = 0$。

$A^{n-1}$为除了其第$(1,n)$位元素外其余元素都为0的矩阵(第$(1,n)$位元素也可能为0):
$$\begin{bmatrix}
0 & \cdots & 0 & \ast \\ 0 & \cdots & 0 & 0 \\ \vdots & & \vdots & \vdots \\ 0 & \cdots & 0 & 0
\end{bmatrix}$$

\vspace{1.5em}

\textbf{习题\ref{ex:3.19} 解答:}

\enum
\item[(1)] 设$A = (a_{ij}), B = (b_{ij})$,那么
$$tr(AB) = \sum\limits_{i=0}^n\sum\limits_{j=0}^n{a_{ij}b_{ji}} = \sum\limits_{i=0}^n\sum\limits_{j=0}^n{b_{ij}a_{ji}} = tr(BA)$$

\item[(2)] 对任意方阵$A, B\in M_n(\mathbb{R})$, 有$tr(AB - BA) = tr(AB) - tr(BA)= 0 \neq n = tr(I_n)$,所以必然有$AB - BA \neq I_n$。
\end{list}

\vspace{1.5em}

\textbf{习题\ref{ex:3.20} 解答:}

\enum
\item[(1)] 设$B = \begin{bmatrix} b_1 & b_2 \\ b_3 & b_4 \end{bmatrix}$与$A$可交换,那么有
$$\begin{bmatrix} 1 & 1 \\ 0 & 0 \end{bmatrix} \begin{bmatrix} b_1 & b_2 \\ b_3 & b_4 \end{bmatrix} =  \begin{bmatrix} b_1 & b_2 \\ b_3 & b_4 \end{bmatrix} \begin{bmatrix} 1 & 1 \\ 0 & 0 \end{bmatrix}$$
即$\begin{bmatrix} b_1 + b_3 & b_2 + b_4 \\ 0 & 0 \end{bmatrix} =  \begin{bmatrix} b_1 & b_1 \\ b_3 & b_3 \end{bmatrix}$,那么$b_3 = 0, b_1 = b_2 + b_4$即可。

\item[(2)] 类似第(1)小题可得答案为形如$\begin{bmatrix} a & b & c & d \\ & a & b & c \\ & & a & b \\ & & & a \end{bmatrix}, a, b, c, d \in \mathbb{R}$,的矩阵。

\item[(3)] $\begin{bmatrix} A_2 & 0 \\ 0 & B_2 \end{bmatrix}$,其中$A_2, B_2$是2个2阶方阵。
\end{list}

\vspace{1.5em}

\textbf{习题\ref{ex:3.21} 解答:}

设方阵$A = (a_{ij})_{1 \leqslant i,j \leqslant n}$与所有$n$阶方阵均可交换,特别的,$A$与方阵$E_{st}$可交换,其中$E_{st}$为第$(s,t)$位元素为1,其余为0的方阵。那么
\begin{eqnarray*}
AE_{st} & = & \begin{bmatrix} 0 & \cdots & 0 & a_{1s} & 0 & \cdots & 0 \\ \vdots & & \vdots & \vdots & \vdots & & \vdots \\ 0 & \cdots & 0 & a_{ns} & 0 & \cdots & 0 \end{bmatrix} \qquad (a_{1s},\cdots,a_{ns}\text{ 位于第$t$列}) \\
E_{st}A & = & \begin{bmatrix} 0 & \cdots & 0 \\ \vdots & & \vdots \\ 0 & \cdots & 0 \\ a_{t1} & \cdots & a_{tn} \\ 0 & \cdots & 0 \\ \vdots & & \vdots \\ 0 & \cdots & 0 \end{bmatrix} \qquad (a_{t1}, \cdots, a_{tn}\text{ 位于第$s$行})
\end{eqnarray*}
$AE_{st} = E_{st}A \Rightarrow a_{ss} = a_{tt}$;以上两个矩阵其余元素都是$0$。取遍所有的$E_{st}$,便可以知道$A$的对角线元素都相等,其余元素都必须为$0$,也就是说$A$必为数量阵。

\vspace{1.5em}

\textbf{习题\ref{ex:3.22} 解答:}

\enum
\item[(1)] 将矩阵$J$的第$1$列与第$n+1$列交换,第$2$列与第$n+2$列交换,……,第$n$列与第$2n$列交换,得到的矩阵为$\begin{bmatrix} I_n & \\ & -I_n \end{bmatrix}$,所以
$$\det J = (-1)^n \det\begin{bmatrix} I_n & \\ & -I_n \end{bmatrix} = (-1)^n\times(-1)^n = 1$$

\item[(2)] $J^2 = \begin{bmatrix} & I_n \\ -I_n & \end{bmatrix} \cdot \begin{bmatrix} & I_n \\ -I_n & \end{bmatrix} = \begin{bmatrix} -I_n & \\ & -I_n \end{bmatrix} = -I_{2n}$
\end{list}

\vspace{1.5em}

\textbf{习题\ref{ex:3.23} 解答:}

若$A$为实对称矩阵,那么$A = A^T.$ 由$A^2 = 0$知$AA^T = 0.$ 记$A = (a_{ij})_{1 \leqslant i,j \leqslant n},$ $AA^T = (c_{ij})_{1 \leqslant i,j \leqslant n},$ 那么
$$c_{ii} = \sum\limits_{j = 1}^n a_{ij}^2 = 0, \quad i = 1, \cdots n,$$
因此$a_{ij} = 0,$ 对任意的$1 \leqslant i,j \leqslant n,$ 也就是说$A=0.$
%若$A$为实对称矩阵,那么存在可逆方阵$P$,使得$P^TAP = diag(\lambda_1,\cdots,\lambda_n)$为对角矩阵。若$A^2 = 0$,那么$P^TAP = 0$,从而有$diag(\lambda_1^2,\cdots,\lambda_n^2) = 0$,所以每个$\lambda_i$都必须为0。所以$A = 0$。

\vspace{1.5em}

\textbf{习题\ref{ex:3.24} 解答:}

任取方阵$A$,有
$$A = \frac{A+A^T}{2} + \frac{A-A^T}{2},$$
$\frac{A+A^T}{2}$为对称阵,$\frac{A-A^T}{2}$为反对称阵。

\vspace{1.5em}

\textbf{习题\ref{ex:3.25} 解答:}

$A^{-1} + B^{-1} = A^{-1}(A + B)B^{-1}$,能表示成三个可逆阵之积,所以也是可逆阵。

\vspace{1.5em}

\textbf{习题\ref{ex:3.26} 解答:}

直接验证$(I_n + A + A^2 + \cdots A^{k-1})(I_n - A) = I_n$。

\vspace{1.5em}

\textbf{习题\ref{ex:3.27} 解答:}

直接验证$(A+I_n)(-\dfrac{A}{2}+I_n) = -\dfrac{A^2}{2} + \dfrac{A}{2} + I_n = I_n.$
%因为$A^2 = A$,所以$A$的特征值只可能有0和1,也就是说存在可逆阵$P$,使得$P^{-1}AP = \Lambda$,其中$\Lambda$为一个对角线元素为0或1的上三角矩阵。那么$P^{-1}(A+I_n)P$便是一个对角线元素为1或2的上三角矩阵,自然是可逆阵。

\vspace{1.5em}

\textbf{习题\ref{ex:3.28} 解答:}

\enum
\item[(1)] 根据矩阵的乘法,很容易看出。

\item[(2)] 设$A$是$n$阶对称阵可逆矩阵,那么$I_n = (AA^{-1})^T = (A^{-1})^T A^T = (A^{-1})^T A$,由于矩阵逆是唯一的,所以$A^{-1} = (A^{-1})^T$。

\item[(3)] 与第(2)小题类似可做。
\end{list}

\vspace{1.5em}

\textbf{习题\ref{ex:3.29} 解答:}

设$A = (a_{ij})_{1\leqslant i,j \leqslant n}$,那么把第$2,3,\cdots,n$列加到第1列上,我们有
\begin{align*}
\det A & = \begin{vmatrix}
a_{11} & a_{12} & \cdots & a_{1n} \\ \vdots & \vdots & & \vdots \\ a_{n1} & a_{n2} & \cdots & a_{nn}
\end{vmatrix} = \begin{vmatrix}
a_{11} + a_{12} + \cdots + a_{1n} & a_{12} & \cdots & a_{1n} \\ \vdots & \vdots & & \vdots \\ a_{n1} + a_{n2} + \cdots + a_{nn} & a_{n2} & \cdots & a_{nn}
\end{vmatrix} \\
& = \begin{vmatrix}
c & a_{12} & \cdots & a_{1n} \\ \vdots & \vdots & & \vdots \\ c & a_{n2} & \cdots & a_{nn}
\end{vmatrix} = c \cdot \begin{vmatrix}
1 & a_{12} & \cdots & a_{1n} \\ \vdots & \vdots & & \vdots \\ 1 & a_{n2} & \cdots & a_{nn}
\end{vmatrix}
\end{align*}
所以必须有$c\neq 0$,否则$\det A = 0$,与$A$是$n$阶可逆阵矛盾。

将$\begin{vmatrix}
1 & a_{12} & \cdots & a_{1n} \\ \vdots & \vdots & & \vdots \\ 1 & a_{n2} & \cdots & a_{nn}
\end{vmatrix}$按第一列展开,我们有
$$\det A = c\cdot(A_{11} + A_{21} + \cdots + A_{n1}),$$
其中$A_{ij}$为$a_{ij}$的代数余子式,正好是$A$的伴随矩阵$A^{\ast}$的第$(j,i)$位元素,而$A^{\ast} = \det A \cdot A^{-1}$。也就是说我们有
$$\det A = c\cdot(\det A \cdot A^{-1}\text{的第一行元素之和}).$$
即有
$$1 = c\cdot(A^{-1}\text{的第一行元素之和}).$$

同样地,如果把$A$的非$i$列元素都加到第$i$列上,再按第$i$列展开,我们有
$$1 = c\cdot(A^{-1}\text{的第$i$行元素之和})$$。
所以$A^{-1}$的每一行元素之和都等于$c^{-1}$。

\vspace{1.5em}

\textbf{习题\ref{ex:3.30} 解答:}

可以直接验证:
\begin{eqnarray*}
& & (A+uv^T) (A^{-1} - {A^{-1}uv^T A^{-1} \over 1 + v^T A^{-1}u}) \\
& = & AA^{-1} +  uv^T A^{-1} - {AA^{-1}uv^T A^{-1} + uv^T A^{-1}uv^T A^{-1} \over 1 + v^TA^{-1}u} \\
& = & I +  uv^T A^{-1} - {uv^T A^{-1} + uv^T A^{-1}uv^T A^{-1} \over 1 + v^T A^{-1}u} \\
& = & I + uv^T A^{-1} - {u(1 + v^T A^{-1}u) v^T A^{-1} \over 1 + v^T A^{-1}u} \\
& = & I + uv^T A^{-1} - {1 + v^T A^{-1}u \over 1 + v^T A^{-1}u}uv^T A^{-1} \\
& = & I + uv^T A^{-1} - uv^T A^{-1} \\
& = & I_n.
\end{eqnarray*}

另解:也可以通过观察,发现$(A+uv^T)^{-1}$是以下方程
$$\begin{bmatrix} A & u \\ v^T & -1 \end{bmatrix}\begin{bmatrix} X \\ Y \end{bmatrix} = \begin{bmatrix} I \\ 0 \end{bmatrix}.$$
的解$X$,然后通过解矩阵方程组
$$\begin{cases}
AX + uY = I \\
v^TX - Y = 0
\end{cases}$$
由第一个方程有$X = A^{-1}(I-uY)$,代入第二个方程得$v^TA^{-1}(I-uY) = Y$。合并同类项得$(1 + v^TA^{-1}u)^{-1}v^TA^{-1} = Y$。再将$Y$代回到第一个方程,就有
$$(A+uv^T)^{-1} = X = A^{-1} - {A^{-1}uv^T A^{-1} \over 1 + v^T A^{-1}u}.$$

\vspace{1.5em}

\textbf{习题\ref{ex:3.31} 解答:}

这是上一题的推广,证明方法类似。

%%%%%%%%%%%%%%%%%%%%%%%%%%%%%%%%%%%%%%%%%%%%%%%%%%%%%%%%%%%%%%%%%%%%%%%%%%%%%%%%%%%%%%%
%%%%%%%%%%%%%%%%%%%%%%%%%%%%%%%%%%%%%%%%%%%%%%%%%%%%%%%%%%%%%%%%%%%%%%%%%%%%%%%%%%%%%%%
