\documentclass[a4paper]{book}
\usepackage{amsfonts,amssymb}
\usepackage{amsmath}
\usepackage{amsthm}
\usepackage[all]{xy}
\usepackage{indentfirst}
\usepackage{systeme}
\usepackage[a4paper,left=0.9in,right=1.1in,top=1.3in,bottom=1.5in]{geometry}
\usepackage{cases}
\usepackage{extarrows}
\usepackage{arydshln,leftidx}
\usepackage{float}
\def\overleaf{overleaf}
%----------------------------------------------------------------------------
% overleaf专有,在本地编译把这部分注释掉
%%% For accessing system, OTF and TTF fonts
%%% (would have been loaded by polylossia anyway)
\usepackage{fontspec}
%%% For language switching -- like babel, but for xelatex
\usepackage{polyglossia}
%%% For those cool-looking menus and keystrokes
\usepackage{menukeys}
%%% For the xelatex (and other LaTeX friends) logos
\usepackage{hologo}
%%% For the awesome fontawesome icons!
\usepackage{fontawesome}
\usepackage[hyphens]{url}
\setmainlanguage{english}
%%% You'll probably want these lines if
%%% you are also using tikz-related packages with
%%% RTL languages. Put these lines *after* you
%%% loaded the RTL languages.
\makeatletter
    \let\pgfpicture\origin@pgfpicture%
    \let\endpgfpicture\origin@endpgfpicture%
\makeatother
% Main serif font for English (Latin alphabet) text
\setmainfont[Ligatures=TeX]{TeX Gyre Termes}
\setsansfont{Lato}
\setmonofont{Inconsolata}
%%% CJK needs a different treatment
\usepackage[space]{xeCJK}
%%% Assuming Chinese is the main CJK language...
\setCJKmainfont[
  BoldFont=WenQuanYi Zen Hei,
  ItalicFont=AR PL KaitiM GB]
  {AR PL SungtiL GB}
\setCJKsansfont{WenQuanYi Zen Hei}
\setCJKmonofont{cwTeXFangSong}

%--------------------------------------------------------------------------
%在本地编译把下面这两行注释去掉
%\usepackage{CJK}
%\begin{CJK}{GBK}{song}
%--------------------------------------------------------------------------

\setlength{\baselineskip}{15pt}
\setlength{\itemindent}{1in}
\setlength{\parindent}{0pt}
\setlength{\parskip}{2ex}
\setlength{\arraycolsep}{2pt}
\setlength{\arraycolsep}{4pt}

\allowdisplaybreaks

\newtheorem{prop}{性质}[chapter]
\newtheorem{Def}{定义}[chapter]
\newtheorem{cor}{系}[chapter]
\newtheorem{thm}{定理}[chapter]
\newtheorem{rmk}{注记}[chapter]
\newtheorem{eg}{例题}[chapter]
\newtheorem{ex}{习题}[chapter]
\newtheorem*{solution}{解}

\newcommand{\sanhao}{\fontsize{12.75pt}{\baselineskip}\selectfont}
\newcommand{\enum}{\begin{list}{}{\setlength{\leftmargin}{0pt} \setlength{\itemindent}{2.5em} \setlength{\listparindent}{2em}}}
\renewcommand{\contentsname}{目录}

\begin{document}

\title{\Huge \bf MOOCAP 线性代数习题集} \normalfont
\date{}
\author{MoocAP线性代数编委会}
\maketitle
\tableofcontents

%\sanhao
\noindent

\part{线性代数}

\chapter{线性方程组}

\section{知识点解析}

\begin{Def}[二阶行列式]
$$\begin{vmatrix} a_{11} & a_{12} \\ a_{21} & a_{22} \end{vmatrix} = a_{11}a_{22} - a_{12}a_{21}$$
被称为二阶行列式。
\end{Def}

\begin{Def}[三阶行列式]
$$\begin{vmatrix} a_{11} & a_{12} & a_{13} \\ a_{21} & a_{22} & a_{23} \\ a_{31} & a_{32} & a_{33} \end{vmatrix} = a_{11}a_{22}a_{33} + a_{12}a_{23}a_{31} + a_{13}a_{21}a_{32} - a_{11}a_{23}a_{32} - a_{12}a_{21}a_{33} - a_{13}a_{22}a_{31}$$
被称为三阶行列式。
\end{Def}

\begin{prop}[含参二元一次方程组的解]\

当系数行列式$D = \begin{vmatrix} a_{11} & a_{12} \\ a_{21} & a_{22} \end{vmatrix} \neq 0$时,二元线性方程组
$$\left\{ \begin{array}{rcl} a_{11}x_1 + a_{12}x_2 & = & b_1 \\ a_{21}x_1 + a_{22}x_2 & = & b_2 \end{array}\right.$$
的解为
$$x_1 = \frac{D_1}{D}, \quad x_2 = \frac{D_2}{D},$$
其中
$$D_1 = \begin{vmatrix} b_1 & a_{12} \\ b_2 & a_{22} \end{vmatrix}, \quad D_2 = \begin{vmatrix} a_{11} & b_1 \\ a_{21} & b_2 \end{vmatrix}.$$
\end{prop}

\begin{prop}[含参三元一次方程组的解]\

当系数行列式$D = \begin{vmatrix} a_{11} & a_{12} & a_{13} \\ a_{21} & a_{22} & a_{23} \\ a_{31} & a_{32} & a_{33} \end{vmatrix} \neq 0$时,三元线性方程组
$$\left\{ \begin{array}{rcl} a_{11}x_1 + a_{12}x_2 + a_{13}x_3 & = & b_1 \\ a_{21}x_1 + a_{22}x_2 + a_{23}x_3 & = & b_2 \\ a_{31}x_1 + a_{32}x_2 + a_{33}x_3 & = & b_3 \end{array}\right.$$
的解为
$$x_1 = \frac{D_1}{D}, \quad x_2 = \frac{D_2}{D}, \quad x_3 = \frac{D_3}{D}$$
其中
$$D_1 = \begin{vmatrix} b_1 & a_{12} & a_{13} \\ b_2 & a_{22} & a_{23} \\ b_3 & a_{32} & a_{33} \end{vmatrix}, \quad D_2 = \begin{vmatrix} a_{11} & b_1 & a_{13} \\ a_{21} & b_2 & a_{23}  \\ a_{31} & b_3 & a_{33} \end{vmatrix}, \quad D_3 = \begin{vmatrix} a_{11} & a_{12} & b_1 \\ a_{21} & a_{22} & b_2   \\ a_{31} & a_{32} & b_3 \end{vmatrix}.$$
\end{prop}

\begin{Def}[一般的线性(一次)方程组]\

关于$n$个未知量$x_1,\cdots,x_n$的线性方程组,形式如下
\begin{equation} \label{eq:2.1}
\left\{ \begin{array}{rcl} a_{11}x_1 + a_{12}x_2 + \cdots + a_{1n}x_n & = & b_1 \\ a_{21}x_1 + a_{22}x_2 + \cdots + a_{2n}x_n & = & b_2 \\ \hdotsfor{3} \\ a_{m1}x_1 + a_{m2}x_2 + \cdots + a_{mn}x_n & = & b_m \end{array}\right.
\end{equation}
其中
\enum
\item[(1)] $m \in \mathbb{N}$为方程组方程的个数,
\item[(2)] $a_{ij}\in\mathbb{R}(1\leqslant i \leqslant m, 1\leqslant j \leqslant n)$称为系数,
\item[(3)] $b_i\in\mathbb{R}(1\leqslant i \leqslant m)$称为常数项。
\end{list}

若对所有$i=1,2,…,m$,均有$b_i=0$,则称为上述为齐次线性方程组, 否则,称为非齐次线性方程组。
\end{Def}

\begin{Def}
对线性方程组进行的如下操作:
\enum
\item[(1)] 交换两个方程的位置(对换),
\item[(2)] 用一个非零数乘以某个方程(倍乘),
\item[(3)] 把一个方程的倍数加到另一个方程上(倍加),
\end{list}
统称为方程组的初等变换。
\end{Def}

\begin{prop}
线性方程组的初等变换不改变方程组的解。
\end{prop}

\begin{Def}[高斯消元法]\

设方程组\eqref{eq:2.1}中$x_1$的系数不全为零,总可以通过对换,使得$a_{11}\neq0$,于是,把第一个方程的$-\frac{a_{j1}}{a_{11}}$倍加到第$j$个方程上$(2 \leqslant j \leqslant m)$,即可在第$2\sim m$个方程中消去未知量$x_1$。按类似的步骤,考察第$2\sim m$个方程,对其他未知量继续做下去。以此类推,便可求解线性方程组。

这样的计算方法就称为高斯消元法.

\end{Def}

\begin{Def}
由$mn$个实数排成行列的矩形数表, 用圆(或方)括号括起来,即
$$A = \begin{bmatrix}
a_{11} & a_{12} & \cdots & a_{1n} \\ a_{21} & a_{22} & \cdots & a_{2n} \\ \vdots & \vdots & \vdots & \vdots \\ a_{m1} & a_{m2} & \cdots & a_{mn}
\end{bmatrix}$$
称为$m\times n$型的矩阵,简记为$A = (a_{ij})_{m\times n}$,其中横排称为矩阵的行,竖排称为矩阵的列。$a_{ij}$称为矩阵的元素,其第一下标表示所在的行数,第二下标表示其所在的列数。全体$m\times n$型的矩阵组成的集合,记为$M_{m\times n}(\mathbb{R})$。

特别地,行数与列数相同的矩阵(即$m = n$),称为$n$阶方阵,全体n阶方阵组成的集合,记为$M_n(\mathbb{R})$。
\end{Def}

\begin{Def}
由线性方程组
\begin{equation*}
\left\{ \begin{array}{rcl} a_{11}x_1 + a_{12}x_2 + \cdots + a_{1n}x_n & = & b_1 \\ a_{21}x_1 + a_{22}x_2 + \cdots + a_{2n}x_n & = & b_2 \\ \hdotsfor{3} \\ a_{m1}x_1 + a_{m2}x_2 + \cdots + a_{mn}x_n & = & b_m \end{array}\right.
\end{equation*}
未知元前的系数给出的矩阵
$$
\begin{bmatrix}
a_{11} & a_{12} & \cdots & a_{1n} \\ a_{21} & a_{22} & \cdots & a_{2n} \\ \vdots & \vdots & & \vdots \\ a_{m1} & a_{m2} & \cdots & a_{mn}
\end{bmatrix}
$$
被称为该线性方程组的系数矩阵。

把常数项列添加到系数矩阵最后一列之后得到的矩阵
$$
\begin{bmatrix}
a_{11} & a_{12} & \cdots & a_{1n}  & b_1 \\ a_{21} & a_{22} & \cdots & a_{2n} & b_2 \\ \vdots & \vdots & & \vdots & \vdots \\ a_{m1} & a_{m2} & \cdots & a_{mn} & b_m
\end{bmatrix}
$$
被称为该线性方程组的增广(系数)矩阵。
\end{Def}

\begin{Def}
一个矩阵若满足下列条件,称其为阶梯形矩阵:
\enum
\item[(1)]矩阵若有零行(即元素全为0的行),则零行一定全在矩阵的下方。
\item[(2)]对于矩阵的每一个非零行,从左起第1个非零元素称为此行的主元。矩阵下面行的主元所在列一定在上面行的主元所在列的右端。
\end{list}
\end{Def}

\begin{Def}
一个阶梯矩阵若满足下列条件,称其为简化的阶梯形矩阵:
\enum
\item[(1)]主元都是$1$。
\item[(2)]每个主元所在的列中,除主元外其他的元素都是0。
\end{list}
\end{Def}

\begin{prop}
任一矩阵$A$都可以通过矩阵的初等行变换化为阶梯形矩阵, 进而可再化为简化的阶梯形矩阵。
\end{prop}

\begin{thm}
设含$n$个未知量的线性方程组的增广系数矩阵为$\overline{A}$,对$\overline{A}$作初等行变换, 化为阶梯形矩阵$B$。若$B$有一个主元在最后一列,则方程组无解. 若$B$的主元都不在最后一列,则方程组有解。进一步地,若这时主元个数$r=n$,则方程组有唯一解。若$r<n$,则方程组有无穷多组解。
\end{thm}

\begin{cor}
对于齐次线性方程组,有如下结论成立:

\enum
\item[(1)]增广系数矩阵$\overline{A}$的最后一列全为0,故作初等行变换后,最后一列的元素也全为0,于是主元素不可能出现在最后一列,因此一定有解。
\item[(2)]$x_i=0 (i=1,2,…,m)$一定是一组解,称为零解。
\item[(3)]若有非零解,则它就必有无穷多个解。
\item[(4)]作高斯消元法时,只需对系数矩阵$A$操作即可。
\end{list}
\end{cor}

\begin{thm}
若齐次线性方程组的方程个数$m$小于未知量的个数$n$,即$m<n$时,齐次线性方程一定有非零解。
\end{thm}

%%%%%%%%%%%%%%%%%%%%%%%%%%%%%%%%%%%%%%%%%%%%%%%%%%%%%%%%%%%%%%%%%%%%%%%%%%%%%%%%%%%%%%

\section{例题讲解}

\begin{eg}
《孙子算经》中著名的数学问题,其内容是:“今有雉(鸡)兔同笼,上有三十五头,下有九十四足。问雉兔各几何。”
\end{eg}
\begin{solution}
设鸡和兔的数量分别为$x, y,$  则
$$\systeme{ x + y = 35, 2x + 4y = 94}$$
因为$94-35-35=24$,故兔子数量$y=24/2=12,$ 则鸡的数量$x=35-12=23$。(实际上,就是用方程2 $-$(方程1)$\times 2$,消去$x,$求出$y$后,代回求得$x$)
\end{solution}

\vspace{1.5em}

\begin{eg}
解线性方程组
$\systeme{x_1 - 2x_2 + x_3 = -2 , 2x_1 + x_2 - 3x_3 = 1 , -x_1 + x_2 - x_3 = 0}$
\end{eg}

\begin{solution}
\begin{align*}
D & = \begin{vmatrix} 1 & -2 & 1 \\ 2 & 1 & -3 \\ -1 & 1 & -1 \end{vmatrix} \\
& = 1\times 1\times (-1) + (-2)\times (-3)\times (-1) + 2\times 1\times 1 - 1\times 1\times (-1) \\
& \quad - (-2)\times 2\times (-1) - 1\times (-3)\times 1 \\
& = -5 \neq 0, \\
D_1 & = \begin{vmatrix} -2 & -2 & 1 \\ 1 & 1 & -3 \\ 0 & 1 & -1 \end{vmatrix} = (-2)\times 1\times (-1) + 1\times 1\times 1 - (-2)\times 1\times (-3) \\
& \quad - (-2)\times 1\times (-1) \\
& = -5, \\
D_2 & = \begin{vmatrix} 1 & -2 & 1 \\ 2 & 1 & -3 \\ -1 & 0 & -1 \end{vmatrix}
= 1\times 1\times (-1) + (-2)\times (-3)\times (-1) \\
& \quad - (-1)\times 1\times 1 - (-2)\times 2\times (-1) \\
& = -10, \\
D_3 & = \begin{vmatrix} 1 & -2 & -2 \\ 2 & 1 & 1 \\ -1 & 1 & 0 \end{vmatrix}
 = (-2)\times 1\times (-1) + (-2)\times 2\times 1 \\
& \quad - (-1)\times 1\times (-2) - 1\times 1\times 1 \\
& = -5,
\end{align*}

故方程组的解为:
$$x_1 = \frac{D_1}{D} = 1, \quad x_2 = \frac{D_2}{D} = 2, \quad x_3 = \frac{D_3}{D} = 1.$$
\end{solution}

\vspace{1.5em}

\begin{eg}
解三元线性方程组$\systeme{3x_2 + x_3 = 9, x_1 - 2x_2 + x_3 = 0 , 3x_1 + 3x_2 - x_3 = 6}$
\end{eg}
\begin{solution}\

\begin{eqnarray*}
& & \systeme{3x_2 + x_3 = 9 , x_1 - 2x_2 + x_3 = 0 , 3x_1 + 3x_2 - x_3 = 6} \xrightarrow{(1)\leftrightarrow(2)} \systeme{x_1 - 2x_2 + x_3 = 0 , 3x_2 + x_3 = 9 , 3x_1 + 3x_2 - x_3 = 6} \\
& \xrightarrow{-3\cdot(1) + (3)} & \systeme{x_1 - 2x_2 + x_3 = 0, 3x_2 + x_3 = 9,  9x_2 - 4x_3 = 6} \xrightarrow{[-3\cdot(2)+(3)]/7} \systeme{x_1 - 2x_2 + x_3 = 0 , 3x_2 + x_3 = 9 , x_3 = 3} \\ & \xrightarrow{\substack{-1\cdot(3)+(1) \\ -1\cdot(3)+(2)}} & \systeme{x_1 - 2x_2 = -3 , 3x_2 = 6, x_3 = 3} \\
& \xrightarrow{\substack{\frac13\cdot(2) \\ \frac23\cdot(2)+(1)}} & \systeme{x_1 = 1, x_2 = 2, x_3 = 3}
\end{eqnarray*}
\end{solution}

\vspace{1.5em}

\begin{eg}
解线性方程组$\systeme{x_1 - x_2 - 3x_3 + x_4 = 1 , x_1 - x_2 + 2x_3 - x_4 = 3 , 2x_1 - 2x_2 + 3x_3 - 4x_4 = 0 , 3x_1 - 3x_2 + 5x_3 = -1 }$
\end{eg}
\begin{solution}
\begin{eqnarray*}
& & \begin{vmatrix} 1 & -1 & -3 & 1 & 1 \\ 1 & -1 & 2 & -1 & 3 \\ 2 & -2 & 3 & -4 & 0 \\ 3 & -3 & 5 & 0 & -1 \end{vmatrix} \longrightarrow \begin{vmatrix} 1 & -1 & -3 & 1 & 1 \\ 0 & 0 & 5 & -2 & 2 \\ 0 & 0 & 9 & -6 & -2 \\ 0 & 0 & 14 & -3 & -4 \end{vmatrix} \\
&\longrightarrow & \begin{vmatrix} 1 & -1 & -3 & 1 & 1 \\ 0 & 0 & 5 & -2 & 2 \\ 0 & 0 & 45 & -30 & -10 \\ 0 & 0 & 70 & -15 & -20 \end{vmatrix}
\longrightarrow \begin{vmatrix} 1 & -1 & -3 & 1 & 1 \\ 0 & 0 & 5 & -2 & 2 \\ 0 & 0 & 0 & -12 & -28 \\ 0 & 0 & 0 & 13 & -48 \end{vmatrix} \\ &\longrightarrow & \begin{vmatrix} 1 & -1 & -3 & 1 & 1 \\ 0 & 0 & 5 & -2 & 2 \\ 0 & 0 & 0 & 3 & 7 \\ 0 & 0 & 0 & 0 & 1 \end{vmatrix}
\end{eqnarray*}

上面最后一个矩阵的最后一行对应的方程是
$$0\cdot x_1 + 0\cdot x_2 + 0\cdot x_3 + 0\cdot x_4 = 1$$
不管$x_1, x_2, x_3, x_4$取何值,上式均不可能成立,所以原方程组无解。
\end{solution}

\vspace{1.5em}

\begin{eg}
解线性方程组$\systeme{x_1 - x_2 - 3x_3 + x_4 = 1 , x_1 - x_2 + 2x_3 - x_4 = 3 , 4x_1 - 4x_2 + 3x_3 - 2x_4 = 10 , 2x_1 - 2x_2 - 11x_3 + 4x_4 = 0}$
\end{eg}

\begin{solution}
\begin{eqnarray*}
& & \begin{vmatrix} 1 & -1 & -3 & 1 & 1 \\ 1 & -1 & 2 & -1 & 3 \\ 4 & -4 & 3 & -2 & 10 \\ 2 & -2 & -11 & 4 & 0 \end{vmatrix} \longrightarrow \begin{vmatrix} 1 & -1 & -3 & 1 & 1 \\ 0 & 0 & 5 & -2 & 2 \\ 0 & 0 & 15 & -6 & 6 \\ 0 & 0 & -5 & 2 & -2 \end{vmatrix} \\
& \longrightarrow & \begin{vmatrix} 1 & -1 & -3 & 1 & 1 \\ 0 & 0 & 5 & -2 & 2 \\ 0 & 0 & 0 & 0 & 0 \\ 0 & 0 & 0 & 0 & 0 \end{vmatrix}
\longrightarrow \begin{vmatrix} 1 & -1 & -3 & 1 & 1 \\ 0 & 0 & 1 & -\frac25 & \frac25 \\ 0 & 0 & 0 & 0 & 0 \\ 0 & 0 & 0 & 0 & 0 \end{vmatrix} \\
& \longrightarrow & \begin{vmatrix} 1 & -1 & 0 & -\frac15 & \frac{11}{5} \\ 0 & 0 & 1 & -\frac25 & \frac25 \\ 0 & 0 & 0 & 0 & 0 \\ 0 & 0 & 0 & 0 & 0 \end{vmatrix}
\end{eqnarray*}

这时方程组化为
$$\left\{ \begin{array}{rcl} x_1 - x_2 - \frac15 x_4 & = & \frac{11}{5} \\ x_3 - \frac25x_4 & = & \frac25 \\ 0 & = & 0 \\ 0 & = & 0 \end{array}\right.$$
或写为
\begin{equation} \label{eq:2.4}
\left\{ \begin{array}{rcl} x_1 & = & x_2 + \frac15 x_4 + \frac{11}{5} \\ x_3 & = & \frac25x_4 + \frac25 \end{array}\right.
\end{equation}

可以看出,对于未知量$x_2, x_4$的任一组取值, 都可以唯一决定出$x_1, x_3$的值。称$x_1, x_3$为主变量,$x_2, x_4$为自由未知量。用自由未知量表示主变量的\eqref{eq:2.4}式称为方程组的一般解,或者把\eqref{eq:2.4}式表示为如下形式
$$\systeme*{x_1 = s + \frac15 t + \frac{11}{5}, x_2 = s, x_3 = \frac25 t + \frac25, x_4 = t}$$
$\forall s,t \in \mathbb{R}$(有无穷多个解)。
\end{solution}

\vspace{1.5em}

\begin{eg}
求解齐次线性方程组
$\systeme{2x_1 - x_2 + 5x_3 - 3x_4 = 0, x_1 - 5x_2 + 3x_3 + 2x_4 = 0, 3x_1 - 4x_2 + 7x_3 - x_4 = 0, 9x_1 - 7x_2 + 15x_3 + 4x_4 = 0}$
\end{eg}
\begin{solution}
只需对系数矩阵$A$作初等行变换:
\begin{align*}
& \begin{vmatrix} 2 & -1 & 5 & -3 \\ 1 & -5 & 3 & 2 \\ 3 & -4 & 7 & -1 \\ 9 & -7 & 15 & 4 \end{vmatrix} \longrightarrow \begin{vmatrix} 1 & -5 & 3 & 2 \\ 2 & -1 & 5 & -3 \\ 3 & -4 & 7 & -1 \\ 9 & -7 & 15 & 4 \end{vmatrix} \longrightarrow \begin{vmatrix} 1 & -5 & 3 & 2 \\ 0 & 9 & -1 & -7 \\ 0 & 11 & -2 & -7 \\ 0 & 38 & -12 & -14 \end{vmatrix} \\
\longrightarrow & \begin{vmatrix} 1 & -5 & 3 & 2 \\ 0 & 9 & -1 & -7 \\ 0 & 11 & -2 & -7 \\ 0 & 19 & -6 & -7 \end{vmatrix} \longrightarrow \begin{vmatrix} 1 & -5 & 3 & 2 \\ 0 & 9 & -1 & -7 \\ 0 & 11 & -2 & -7 \\ 0 & 1 & -4 & 7 \end{vmatrix} \longrightarrow \begin{vmatrix} 1 & -5 & 3 & 2 \\ 0 & 1 & -4 & 7 \\ 0 & 9 & -1 & -7 \\ 0 & 11 & -2 & -7 \end{vmatrix} \\
\longrightarrow & \begin{vmatrix} 1 & -5 & 3 & 2 \\ 0 & 1 & -4 & 7 \\ 0 & 0 & 42 & -84 \\ 0 & 0 & 35 & -70 \end{vmatrix} \longrightarrow \begin{vmatrix} 1 & -5 & 3 & 2 \\ 0 & 1 & -4 & 7 \\ 0 & 0 & 1 & -2 \\ 0 & 0 & 0 & 0 \end{vmatrix}\longrightarrow \begin{vmatrix} 1 & -5 & 0 & 8 \\ 0 & 1 & 0 & -1 \\ 0 & 0 & 1 & -2 \\ 0 & 0 & 0 & 0 \end{vmatrix} \\
\longrightarrow & \begin{vmatrix} 1 & 0 & 0 & 3 \\ 0 & 1 & 0 & -1 \\ 0 & 0 & 1 & -2 \\ 0 & 0 & 0 & 0 \end{vmatrix}
\end{align*}
所以方程组有无穷多组解,它的一般解为
$$\begin{cases}
x_1 = -3x_4 \\ x_2 = x_4 \\ x_3 = 2x_4,
\end{cases} \qquad (x_4\in\mathbb{R}),$$
其中,$x_4$为自由未知量,可取任意实数。
\end{solution}

%%%%%%%%%%%%%%%%%%%%%%%%%%%%%%%%%%%%%%%%%%%%%%%%%%%%%%%%%%%%%%%%%%%%%%%%%%%%%%%%%%%%%%%%%%%%

\section{课后习题}

\begin{ex} \label{ex:1.1}
求下列线性方程组系数矩阵的行列式并求解线性方程组。
\enum
\item[(1)] $\systeme{2x + 5y = 3 , 3x + 7y = 2}$
\item[(2)] $\systeme{3x + 5y = 0 , 4x + 7y = 2}$
\item[(3)] $\left\{ \begin{array}{rcl} x + y + z & = & 1 \\ 2x - y & = & 2 \\ x + 2y + 2z & = & 0\end{array}\right.$
\end{list}
\end{ex}

\begin{ex} \label{ex:1.2}
证明:对任意整数$a,b,c$,以下方程组有唯一解。
$$\left\{ \begin{array}{rcl} 2x + y & = & a \\ x + z & = & b \\ x - 3y - 6z & = & c\end{array}\right.$$
\end{ex}

\begin{ex} \label{ex:1.3}
若以$x,y,z$为未知元的方程组
$$\left\{ \begin{array}{rcl} ax + y + z & = & 0 \\ x + ay + z & = & 0 \\ x + y + az & = & 0\end{array}\right.$$
有无穷多组解,求$a$的值。
\end{ex}

\begin{ex} \label{ex:1.4}
讨论$x_1,x_2,x_3,x_4$为未知元的线性方程组
$$\left\{ \begin{array}{rcl} x_1 + x_2 + 4x_3 & = & a \\ -x_1 - 2x_3 - x_4 & = & 1 \\ x_3 - 2x_4 & = & -3 \\ 2x_1 + x_2 + 4x_3 + 5x_4 & = & 4 \end{array}\right.$$
的解的情况。
\end{ex}

\begin{ex} \label{ex:1.5}
讨论$x_1,x_2,x_3,x_4$为未知元的线性方程组
$$\left\{ \begin{array}{rcl} x_1 + x_2 + 4x_3 & = & 1 \\ -x_1 - 2x_3 - x_4 & = & 1 \\ ax_1 + bx_2 + x_3 - 2x_4 & = & -3 \\ 2x_1 + x_2 + 4x_3 + 5x_4 & = & 4 \end{array}\right.$$
的解的情况。
\end{ex}

\begin{ex} \label{ex:1.6}
用高斯消元法求解下列齐次线性方程组
\enum
\item[(1)] $\left\{ \begin{array}{rcl} x_1 + x_2 + 2x_3 - x_4 & = & 0 \\ 2x_1 + x_2 + x_3 - x_4 & = & 0 \\ 2x_1 + 2x_2 + x_3 + 2x_4 & = & 0 \end{array}\right.$
\item[(2)] $\left\{ \begin{array}{rcl} - x_2 -x_3 + x_4 & = & 0 \\ x_1 + x_2 + 3x_3 - x_4 & = & 0 \\ -2x_1 + 3x_2 - x_4 & = & 0 \\ -x_1 - 4x_3 - 4x_4 & = & 0\end{array}\right.$
\item[(3)] $\left\{ \begin{array}{rcl} 4x_1 + 3x_2 - 3x_3 - 2x_4 & = & 0 \\ x_1 + x_3 + 3x_4 & = & 0 \\ -3x_1 - 2x_2 + 2x_3 + x_4 & = & 0 \\ x_1 + 2x_2 - 3x_3 - 4x_4 & = & 0\end{array}\right.$
\end{list}
\end{ex}

\begin{ex} \label{ex:1.7}
用高斯消元法求解下列非齐次线性方程组
\enum
\item[(1)] $\left\{ \begin{array}{rcl} x_1 + 3x_3 & = & 3 \\ x_2 - 2x_3 & = & -4 \\ 2x_1 + 4x_3 & = & 4 \end{array}\right.$
\item[(2)] $\left\{ \begin{array}{rcl} x_1 + x_2 + x_3 & = & 0 \\ x_2 + x_3 & = & -2 \\ -x_1 - 2x_3 & = & -2 \end{array}\right.$
\item[(3)] $\left\{ \begin{array}{rcl} - x_2 -x_3 + 3x_4 & = & 0 \\ x_1 + 3x_2 - 4x_4 & = & 5 \\  x_2 + 2x_3 - 4x_4 & = & -1 \\ -2x_1 - x_2- 4x_3 + 3x_4 & = & -3\end{array}\right.$
\end{list}
\end{ex}

\begin{ex} \label{ex:1.8}
设$A = \begin{bmatrix} -2 & 0 & -1 \\ 2 & 1 & 3 \\ 3 & 0 & 1 \\ 1 & 1 & 3 \end{bmatrix}$,对什么样的$b = (b_1, b_2, b_3, b_4)^T$, $Ax = b$有解?其中$x = (x_1, x_2, x_3, x_4)^T$为未知元。
\end{ex}

\begin{ex} \label{ex:1.9}
已知$a_1 = (0,1,0)^T, a_2 = (-3,2,2)^T$是线性方程组
$$\left\{ \begin{array}{rcl} x_1 - x_2 + 2x_3 & = & -1 \\ 3x_1 + x_2 + 4x_3 & = & 1 \\ ax_1 + bx_2 + cx_3 & = & d\end{array}\right.$$
的两个解,求此方程组的全部解。
\end{ex}

\begin{ex} \label{ex:1.10}
求平面上$n$个点$(x_1,y_1),(x_2,y_2),\cdots,(x_n,y_n)$位于同一条直线上的充分必要条件。
\end{ex}

\newpage

%%%%%%%%%%%%%%%%%%%%%%%%%%%%%%%%%%%%%%%%%%%%%%%%%%%%%%%%%%%%%%%%%%%%%%%%%%%%%%%%%%%%%%%%%%%%

\section{习题答案}

\textbf{习题\ref{ex:1.1} 解答:}

\enum
\item[(1)] 系数矩阵的行列式$D = \begin{vmatrix} 2 & 5 \\ 3 & 7 \end{vmatrix} = 2\times 7 - 5\times 3 = -1$。另外算得$D_1 = \begin{vmatrix} 3 & 5 \\ 2 & 7 \end{vmatrix} = 11, D_2 = \begin{vmatrix} 2 & 3 \\ 3 & 2 \end{vmatrix} = -5$,所以原解线性方程组解为
$$\begin{cases} x = \frac{D_1}{D} = -11 \\ y = \frac{D_2}{D} = 5 \end{cases}$$

\item[(2)] 系数矩阵的行列式$\begin{vmatrix} 3 & 5 \\ 4 & 7 \end{vmatrix} = 3\times 7 - 5\times 4 = 1$。类似第(1)小题可算得$D_1 = -10, D_2 = 6$,原解线性方程组解为
$$\begin{cases} x = -10 \\ y = 6 \end{cases}$$

\item[(3)] 系数矩阵的行列式$\begin{vmatrix} 1 & 1 & 1 \\ 2 & -1 & 0 \\ 1 & 2 & 2 \end{vmatrix} = 1\times (-1) \times 2 + 1 \times 0 \times 1 + 1 \times 2 \times 2 - 2\times 1 \times 2 - 1 \times 2 \times 0 - 1\times (-1) \times 1 = -1$。而$D_1 = -2, D_2 = -2, D_3 = 3$原解线性方程组解为
$$\begin{cases} x = 2 \\ y = 2 \\ z = -3 \end{cases}$$
\end{list}

\vspace{1.5em}

\textbf{习题\ref{ex:1.2} 解答:}

因为该线性方程组系数矩阵$A = \begin{bmatrix} 2 & 1 & 0 \\ 1 & 0 & 1 \\ 1 & -3 & -6 \end{bmatrix}$的行列式$D$为13,不等于0,所以它只有
$$x_1 = \frac{D_1}{D}, \quad x_2 = \frac{D_2}{D}, \quad x_3 = \frac{D_3}{D}$$
这一组解。

或者也可以利用高斯消元法,把原线性方程组的增广矩阵化为阶梯形
$$\begin{bmatrix}
1 & 0 & 0 & \frac{3}{13} a + \frac{6}{13} b + \frac{1}{13} c \\
0 & 1 & 0 & \frac{7}{13} a - \frac{12}{13} b - \frac{2}{13} c \\
0 & 0 & 1 & -\frac{3}{13} a + \frac{7}{13} b - \frac{1}{13} c
\end{bmatrix},$$
发现其主元都不在最后一列且主元个数等于未知元个数,所以原线性方程组有唯一解。

学过后面的内容可以知道$A$是可逆的,而且容易求出$A^{-1} = \frac{1}{13}\begin{bmatrix} 3 & 6 & 1 \\ 7 & -12 & -2 \\ -3 & 7 & -1 \end{bmatrix}$,原线性方程组有唯一的解
$$\frac{1}{13}\begin{bmatrix} 3 & 6 & 1 \\ 7 & -12 & -2 \\ -3 & 7 & -1 \end{bmatrix}\cdot\begin{bmatrix} a \\ b \\ c \end{bmatrix}$$

\vspace{1.5em}

\textbf{习题\ref{ex:1.3} 解答:}

容易求得系数矩阵的行列式为$a^3 - 3a + 2$,要使原线性方程组有无穷多解,就必须有其系数矩阵的行列式为0。解方程$a^3 - 3a + 2 = 0$得$a = -2$ 或 $1$。

\vspace{1.5em}

\textbf{习题\ref{ex:1.4} 解答:}

用高斯消元法可将原线性方程组化为
$$\begin{cases} x_1 + 5x_4 = 5 \\ x_2 + 3x_4 = a+7 \\ x_3 - 2x_4 = -3 \\ 0 = -a-1 \end{cases}$$
所以当$a = -1$时,原线性方程组有无穷多组解;当$a\neq -1$时,原线性方程组无解。

\vspace{1.5em}

\textbf{习题\ref{ex:1.5} 解答:}

用高斯消元法可将原线性方程组化为
$$\begin{cases} x_1 + 5x_4 = 3 \\ x_2 + 3x_4 = 6 \\ x_3 - 2x_4 = -2 \\ (5 a + 3 b)x_4 = 3 a + 6 b + 1 \end{cases}$$
所以当$5a \neq -3b$时,原线性方程组有唯一解;当$a = \frac17, b = -\frac{5}{21}$时,原线性方程组有无穷多组解;其余情况无解。

\vspace{1.5em}

\textbf{习题\ref{ex:1.6} 解答:}

\enum
\item[(1)] 用高斯消元法化简系数矩阵:
\begin{eqnarray*}
& & \begin{bmatrix} 1 & 1 & 2 & -1 \\ 2 & 1 & 1 & -1 \\ 2 & 2 & 1 & 2 \end{bmatrix} \xrightarrow[r_3-2r_1]{r_2-2r_1} \begin{bmatrix} 1 & 1 & 2 & -1 \\ 0 & -1 & -3 & 1 \\ 0 & 0 & -3 & 4 \end{bmatrix} \xrightarrow[r_3\times(-\frac13)]{r_2\times(-1)} \begin{bmatrix} 1 & 1 & 2 & -1 \\ 0 & 1 & 3 & -1 \\ 0 & 0 & 1 & -\frac43 \end{bmatrix} \\
& \xrightarrow{r_1-r_2} & \begin{bmatrix} 1 & 0 & -1 & 0 \\ 0 & 1 & 3 & -1 \\ 0 & 0 & 1 & -\frac43 \end{bmatrix} \xrightarrow[r_2-3r_3]{r_1+r_3} \begin{bmatrix} 1 & 0 & 0 & -\frac43 \\ 0 & 1 & 0 & 3 \\ 0 & 0 & 1 & -\frac43 \end{bmatrix}
\end{eqnarray*}
所以方程组有无穷多组解,它的一般解为
$$\begin{cases} x_1 = \frac43 x_4 \\ x_2 = -3x_4 \\ x_3 = \frac43 x_4 \end{cases},$$
其中,$x_4$为自由未知量,可取任意实数。

\item[(2)] 类似第(1)小题用高斯消元法化成阶梯形为
$$\systeme{ x_1 - 4x_4 = 0 , x_2 - 3x_4 = 0 , x_3 + 2x_4 = 0 },$$
即方程组有无穷多组解:
$$\begin{cases} x_1 = 4x_4 \\ x_2 = 3x_4 \\ x_3 = -2x_4 \end{cases},$$
其中,$x_4$为自由未知量,可取任意实数。

\item[(3)] 类似第(1)小题用高斯消元法,系数矩阵最终可化为
$$\begin{bmatrix}
1 & & & \\ & 1 & & \\ & & 1 & \\ & & & 1
\end{bmatrix},$$
所以只有零解。
\end{list}

\vspace{1.5em}

\textbf{习题\ref{ex:1.7} 解答:}

\enum
\item[(1)] 用高斯消元法将增广矩阵化成阶梯形:
$$\begin{bmatrix} 1 & 0 & 3 & 3 \\ 0 & 1 & -2 & -4 \\ 2 & 0 & 4 & 4 \end{bmatrix} \longrightarrow \begin{bmatrix} 1 & 0 & 3 & 3 \\ 0 & 1 & -2 & -4 \\ 0 & 0 & -2 & -2 \end{bmatrix}
\longrightarrow \begin{bmatrix} 1 & 0 & 3 & 3 \\ 0 & 1 & -2 & -4 \\ 0 & 0 & 1 & 1 \end{bmatrix}
\longrightarrow \begin{bmatrix} 1 & 0 & 0 & 0 \\ 0 & 1 & 0 & -2 \\ 0 & 0 & 1 & 1 \end{bmatrix}$$
最后的阶梯形为
$$\systeme{ x_1 = 0, x_2 = -2 , x_3 = 1 },$$
同时这也是原线性方程组的解。

\item[(2)] 类似第(1)小题用高斯消元法化阶梯形可求得解为
$$\begin{cases}
x_1 = 2 \\ x_2 = -2 \\ x_3 = 0
\end{cases}.$$

\item[(3)] 类似第(1)小题用高斯消元法化阶梯形可求得解为
$$\begin{cases}
x_1 = 6 \\ x_2 = -3 \\ x_3 = -3 \\ x_4 = -2
\end{cases}.$$
\end{list}

\vspace{1.5em}

\textbf{习题\ref{ex:1.8} 解答:}

用高斯消元法化成
$$\begin{bmatrix} 1 & 0 &  0 \\ 0 & 1 & 0 \\ 0 & 0 & 1 \\ 0 & 0 & 0 \end{bmatrix}x = \begin{bmatrix} b_1 + b_3 \\ 7b_1 + b_2 + 4b_3 \\ -3b_1 - 2b_3 \\ b_1 - b_2 + b_3 + b_4 \end{bmatrix}$$
所以要使原线性方程组有解,必须有$b_1 - b_2 + b_3 + b_4 = 0$。

\vspace{1.5em}

\textbf{习题\ref{ex:1.9} 解答:}

把$a_1 = (0,1,0)^T, a_2 = (-3,2,2)^T$带入第三个方程$ax_1 + bx_2 + cx_3 = d$得到关于$a,b,c,d$为未知元的线性方程组
$$\left\{ \begin{array}{rcl} b & = & d \\ -3a + 2b + 2c & = & d\end{array}\right.,$$
得$d = b, b = 3a - 2c$,代入原线性方程组得
$$\left\{ \begin{array}{rcl} x_1 - x_2 + 2x_3 & = & -1 \\ 3x_1 + x_2 + 4x_3 & = & 1 \\ ax_1 + 2(3a - 2c)x_2 + cx_3 & = & (3a - 2c)\end{array}\right.$$
用高斯消元法化成阶梯形为
$$\left\{ \begin{array}{rcl} x_1 + \frac32 x_3 & = & 0 \\ x_2 - \frac12 x_3 & = & 1 \end{array}\right.$$
所以原线性方程组全部解为$(0,1,0)^T + k\cdot(-3,1,2)^T, k\in\mathbb{R}$。

另解:由于$a_1 = (0,1,0)^T, a_2 = (-3,2,2)^T$是线性无关的两个解,且原线性方程组的前2个方程中任何一个都不是另一个的倍乘,所以原线性方程组对应的齐次线性方程组的解空间最多为1维,所以所以原线性方程组全部解为$(0,1,0)^T + k\cdot((-3,2,2)^T - (0,1,0)^T), k\in\mathbb{R}$。

\vspace{1.5em}

\textbf{习题\ref{ex:1.10} 解答:}

平面上一般的直线方程为$ax+by+c=0$,$a,b$不同时为0。如果$(x_1,y_1),(x_2,y_2),\cdots,(x_n,y_n)$都位于这条直线上,那么他们都必须满足这个直线方程,也就是说以$a,b,c$为未知元的线性方程组
$$\begin{cases} x_1a+y_1b+c = 0 \\ x_2a+y_2b+c = 0 \\ \cdots\cdots\cdots\cdots \\ x_na+y_nb+c = 0\end{cases} \qquad (\ast)$$
必须有$a,b$不同时为0的解。

反之,如果这个$a,b,c$为未知元的线性方程组有$a,b$不同时为0的解$(a,b,c) = (a_0,b_0,c_0)$,那么说明这$n$个点都满足直线方程$ax_0+by_0+c_0=0$。

线性方程组$(\ast)$用高斯消元法化简为
$$\begin{cases} c + x_1a + y_1b = 0 \\ (x_2-x_1)a + (y_2-y_1)b = 0 \\ \cdots\cdots\cdots\cdots \\ (x_n-x_1)a + (y_n-y_1)b = 0 \end{cases} \qquad (\ast\ast)$$
假设有一个$x_i \neq x_1 (i > 1),$ 不妨设其为$x_2,$ 那么可继续化简为
$$\begin{cases} c + \left[ y_1 - x_1\dfrac{y_2-y_1}{x_2-x_1} \right] b = 0 \\ a + \dfrac{y_2-y_1}{x_2-x_1}b = 0 \\ \left[ y_3-y_1 - (x_3-x_1)\dfrac{y_2-y_1}{x_2-x_1} \right] b = 0 \\ \cdots\cdots\cdots\cdots \\ \left[ y_n-y_1 - (x_n-x_1)\dfrac{y_2-y_1}{x_2-x_1} \right] b = 0 \end{cases}$$
因此,必须有$(y_3-y_1) - (x_3-x_1)\dfrac{y_2-y_1}{x_2-x_1} = \cdots = (y_n-y_1) - (x_n-x_1)\dfrac{y_2-y_1}{x_2-x_1} = 0,$ 才能保证齐次原线性方程组有非零解。

如果所有$x_i$值都相等,那么方程组$(\ast\ast)$就变为
$$\begin{cases} c + x_1a + y_1b = 0 \\ (y_2-y_1)b = 0 \\ \cdots\cdots\cdots\cdots \\ (y_n-y_1)b = 0 \end{cases}$$
这个方程无论$y_i$如何取值,总会有$a,b$不同时为0的解。

%%%%%%%%%%%%%%%%%%%%%%%%%%%%%%%%%%%%%%%%%%%%%%%%%%%%%%%%%%%%%%%%%%%%%%%%%%%%%%%%%%%%%%%%%%%%
%%%%%%%%%%%%%%%%%%%%%%%%%%%%%%%%%%%%%%%%%%%%%%%%%%%%%%%%%%%%%%%%%%%%%%%%%%%%%%%%%%%%%%%%%%%%

\chapter{行列式}

\section{知识点解析}

\begin{prop}[二阶行列式的性质]\

\enum
\item[性质$1$.] 行列互换,二阶行列式的值不变,即
$$\begin{vmatrix} a_{11} & a_{12} \\ a_{21} & a_{22} \end{vmatrix} = \begin{vmatrix} a_{11} & a_{21} \\ a_{12} & a_{22} \end{vmatrix}$$
\item[性质$2$.] 若二阶行列式中某行(列)每个元素分成两个数之和,则该行列式可关于该行(列)拆开成两个行列式之和,拆开时其他行均保持不变,即
$$\begin{vmatrix} a_{11} + b_{11} & a_{12} + + b_{12} \\ a_{21} & a_{22} \end{vmatrix} = \begin{vmatrix} a_{11} & a_{12} \\ a_{21} & a_{22} \end{vmatrix} + \begin{vmatrix} b_{11} & b_{12} \\ a_{21} & a_{22} \end{vmatrix}$$
\item[性质$3$.] 两行(列)互换,行列式的值变号,即
$$\begin{vmatrix} a_{11} & a_{12} \\ a_{21} & a_{22} \end{vmatrix} = -\begin{vmatrix} a_{21} & a_{22} \\ a_{11} & a_{12} \end{vmatrix}$$
\item[性质$4$.] 二阶行列式中某行(列)有公因子$k$时,$k$可以提出公因式外,即
$$\begin{vmatrix} ka_{11} & ka_{12} \\ a_{21} & a_{22} \end{vmatrix} = k\begin{vmatrix} a_{11} & a_{12} \\ a_{21} & a_{22} \end{vmatrix}$$
\item[性质$5$.] 二阶行列式中某一行(列)加上另一行(列)的倍时,其值不变, 即
$$\begin{vmatrix} a_{11} + ka_{21} & a_{12} + ka_{22} \\ a_{21} & a_{22} \end{vmatrix} = \begin{vmatrix} a_{11} & a_{12} \\ a_{21} & a_{22} \end{vmatrix}$$
\end{list}
\end{prop}

\begin{prop}[三阶行列式的性质]\

\enum
\item[性质$1'$.] 行列互换,三阶行列式的值不变。
%$$\begin{vmatrix} a_{11} & a_{12} & a_{13} \\ a_{21} & a_{22} & a_{23} \\ a_{31} & a_{32} & a_{33} \end{vmatrix} = \begin{vmatrix} a_{11} & a_{21} & a_{31} \\ a_{12} & a_{22} & a_{32} \\ a_{13} & a_{23} & a_{33} \end{vmatrix}$$
\item[性质$2'$.] 若三阶行列式某行(列)各个元素分成两个数的和,则该行列式可关于该行(列)拆开成两个行列式之和,拆开时其他行(列)均保持不变。
%$$\begin{vmatrix} a_{11} + b_{11} & a_{12} + + b_{12} \\ a_{21} & a_{22} \end{vmatrix} = \begin{vmatrix} a_{11} & a_{12} \\ a_{21} & a_{22} \end{vmatrix} + \begin{vmatrix} b_{11} & b_{12} \\ a_{21} & a_{22} \end{vmatrix}$$
\item[性质$3'$.] 两行(列)互换,三阶行列式的值变号。
%(只给出行列式的前2行变换的情形,其他情形类似)。
%$$\begin{vmatrix} a_{11} & a_{12} & a_{13} \\ a_{21} & a_{22} & a_{23} \\ a_{31} & a_{32} & a_{33} \end{vmatrix} = \begin{vmatrix} a_{21} & a_{22} & a_{23} \\ a_{11} & a_{12} & a_{13} \\ a_{31} & a_{32} & a_{33} \end{vmatrix}$$
\item[性质$4'$.] 三阶行列式的某一行(列)的公因式  可以提到行列式的外面. 特别的,若行列式有一行(列)为零,则行列式的值为0。
\item[性质$5'$.] 把一行(列)的倍数加到另一行(列)上,三阶行列式的值不变。
\end{list}
\end{prop}

\begin{Def}
设$A = (a_{ij})_n$为一个$n$阶方阵,$A$划去第$i$行第$j$列后所剩下的$n-1$阶行列式称为元素$a_{ij}$的余子式,记为$M_{ij}$。再令$A_{ij} = (-1)^{i+j}M_{ij}$,称之为元素$a_{ij}$的代数余子式。
\end{Def}

\begin{thm}
二,三阶行列式等于它的任一行(或列)元素与自己的代数余子式乘积之和。
\end{thm}

\begin{Def}
由$1,2,\cdots,n$组成的有序数组称为一个$n$元排列,记为$j_1j_2\cdots j_n$。全体$n$元排列组成的集合记为$P_n$。
\end{Def}

\begin{Def}
在一个$n$元排列$j_1j_2\cdots j_n$中,如果一个大数排在小数面前,即当$s<t$时,有$j_s>j_t$,则称这一对数$j_sj_t$构成一个逆序。此排列的逆序总数称为它的逆序数,记为$\tau(j_1j_2\cdots j_n)$。
\end{Def}

\begin{Def}
逆序数为偶数的排列称为偶排列;逆序数为奇数的排列称为奇排列。
\end{Def}

\begin{Def}
在一个排列中把两个数$i$与$j$互换位置,这样的操作称为对换,记为$(i, j)$。
\end{Def}

\begin{thm}
对换改变排列奇偶性。
\end{thm}

\begin{cor}
全部$n(\geqslant 2)$元排列中,奇偶排列各占一半。
\end{cor}

\begin{cor}
存在$t$,使得$j_1j_2\cdots j_n$经过$t$次对换变为$12\cdots n$,且$t$与$\tau(j_1j_2\cdots j_n)$同奇偶。
\end{cor}

\begin{Def}[$n$阶行列式]
$$\begin{vmatrix}
a_{11} & a_{12} & \cdots & a_{1n} \\ a_{21} & a_{22} & \cdots & a_{2n} \\ \vdots & \vdots & \ddots & \vdots \\ a_{n1} & a_{n2} & \cdots & a_{nn}
\end{vmatrix} := \sum\limits_{j_1j_2\cdots j_n \in P_n} (-1)^{\tau(j_1j_2\cdots j_n)} a_{1j_1}a_{2j_2}\cdots a_{nj_n}.$$
\end{Def}

\begin{rmk}\
我们可以等价地定义$n$阶行列式为
$$\begin{vmatrix}
a_{11} & a_{12} & \cdots & a_{1n} \\ a_{21} & a_{22} & \cdots & a_{2n} \\ \vdots & \vdots & \ddots & \vdots \\ a_{n1} & a_{n2} & \cdots & a_{nn}
\end{vmatrix} := \sum\limits_{i_1i_2\cdots i_n \in P_n} (-1)^{\tau(i_1i_2\cdots i_n)} a_{i_11}a_{i_22}\cdots a_{i_nn}.$$
\end{rmk}

\begin{rmk}\

\enum
\item[$\bullet$] 行列式定义式是$n!$项代数和;
\item[$\bullet$] 每项为选自不同行,不同列的$n$个元素之积;
\item[$\bullet$] 每项符号:行下标按自然排列排好后,列下标排列的奇偶性决定正负号;
\item[$\bullet$] 行列式可视为对方阵$A=(a_{ij})_{n}$的一种运算,也记作$\det(A)$或$|A|$。
\end{list}
\end{rmk}

\begin{Def}
记行列式
$$D = \begin{vmatrix}
a_{11} & a_{12} & \cdots & a_{1n} \\ a_{21} & a_{22} & \cdots & a_{2n} \\ \vdots & \vdots & \ddots & \vdots \\ a_{n1} & a_{n2} & \cdots & a_{nn}
\end{vmatrix}.$$
将行列式中的行与列互换,所得新的行列式
$$\begin{vmatrix}
a_{11} & a_{21} & \cdots & a_{n1} \\ a_{12} & a_{22} & \cdots & a_{n2} \\ \vdots & \vdots & \ddots & \vdots \\ a_{1n} & a_{2n} & \cdots & a_{nn}
\end{vmatrix},$$
称为转置行列式,记为$D^T$。
\end{Def}

\begin{prop}[行列式的性质]\
\enum
\item[性质$1$.] 行列式与它的转置行列式相等,即$D = D^T$。
\item[性质$2$.] 某列(行)相加可拆项。具体地说,若行列式的某一列(行)的元素都是两数之和,例如
$$D = \begin{vmatrix}
a_{11} & a_{12} & \cdots & a_{1n} \\ \vdots & \vdots & & \vdots \\ a_{i1}+a_{i1}' & a_{i2}+a_{i2}' & \cdots & a_{in}+a_{in}' \\ \vdots & \vdots & & \vdots \\ a_{n1} & a_{n2} & \cdots & a_{nn}
\end{vmatrix},$$
则$D$等于下列两个行列式之和:
$$D = \begin{vmatrix}
a_{11} & a_{12} & \cdots & a_{1n} \\ \vdots & \vdots & & \vdots \\ a_{i1} & a_{i2} & \cdots & a_{in} \\ \vdots & \vdots & & \vdots \\ a_{n1} & a_{n2} & \cdots & a_{nn}
\end{vmatrix} + \begin{vmatrix}
a_{11} & a_{12} & \cdots & a_{1n} \\ \vdots & \vdots & & \vdots \\ a_{i1}' & a_{i2}' & \cdots & a_{in}' \\ \vdots & \vdots & & \vdots \\ a_{n1} & a_{n2} & \cdots & a_{nn}
\end{vmatrix}.$$
\item[性质$2'$.] 有时,为了方便,我们把行列式$D$按列表示为
$$D = \left|\alpha_1, \alpha_2, \cdots, \alpha_n \right| \text{ 或 } \det(\alpha_1, \alpha_2, \cdots, \alpha_n)$$
其中$\alpha_j$表示行列式$D$的第$j$列。于是,利用行列等价性,性质(2)还可以表示成以下形式:
$$\left|\alpha_1, \cdots, \alpha_j + \beta_j, \cdots, \alpha_n \right| = \left|\alpha_1, \cdots, \alpha_j, \cdots, \alpha_n \right| + \left|\alpha_1, \cdots, \beta_j, \cdots, \alpha_n \right|,$$
或
$$\det(\alpha_1, \cdots, \alpha_j + \beta_j, \cdots, \alpha_n) = \det(\alpha_1, \cdots, \alpha_j, \cdots, \alpha_n) + \det(\alpha_1, \cdots, \beta_j, \cdots, \alpha_n).$$
\item[性质$3$.] 行列式某一列(或行)的公因子可以提到行列式外,即
$$\begin{vmatrix}
a_{11} & a_{12} & \cdots & a_{1n} \\ \vdots & \vdots & & \vdots \\ ka_{i1} & ka_{i2} & \cdots & ka_{in} \\ \vdots & \vdots & & \vdots \\ a_{n1} & a_{n2} & \cdots & a_{nn}
\end{vmatrix} = k\begin{vmatrix}
a_{11} & a_{12} & \cdots & a_{1n} \\ \vdots & \vdots & & \vdots \\ a_{i1} & a_{i2} & \cdots & a_{in} \\ \vdots & \vdots & & \vdots \\ a_{n1} & a_{n2} & \cdots & a_{nn}
\end{vmatrix}$$
或者
$$\left|\alpha_1, \cdots, k\alpha_i, \cdots, \alpha_n \right| = k \left|\alpha_1, \cdots, \alpha_i, \cdots, \alpha_n \right|.$$
\item[性质$4$.] 交换任意两行(列)的位置,行列式的值变号,即
$$\begin{vmatrix}
a_{11} & a_{12} & \cdots & a_{1n} \\ \vdots & \vdots & & \vdots \\ a_{s1} & a_{s2} & \cdots & a_{sn} \\ \vdots & \vdots & & \vdots \\ a_{i1} & a_{i2} & \cdots & a_{in} \\ \vdots & \vdots & & \vdots \\ a_{n1} & a_{n2} & \cdots & a_{nn}
\end{vmatrix} = - \begin{vmatrix}
a_{11} & a_{12} & \cdots & a_{1n} \\ \vdots & \vdots & & \vdots \\ a_{i1} & a_{i2} & \cdots & a_{in} \\ \vdots & \vdots & & \vdots \\ a_{s1} & a_{s2} & \cdots & a_{sn} \\ \vdots & \vdots & & \vdots \\ a_{n1} & a_{n2} & \cdots & a_{nn}
\end{vmatrix}.$$
或者
$$\left|\alpha_1, \cdots, \alpha_i, \cdots, \alpha_s, \cdots, \alpha_n \right| = -\left|\alpha_1, \cdots, \alpha_s, \cdots, \alpha_i, \cdots, \alpha_n \right|.$$
\item[性质$5$.] 把某一列(行)的常数倍加到另一列(行)上,行列式值不变,即
$$\begin{vmatrix}
a_{11} & \cdots & (a_{1i} + ka_{1j}) & \cdots & a_{1j} & \cdots & a_{1n} \\ a_{21} & \cdots & (a_{2i} + ka_{2j}) & \cdots & a_{2j} & \cdots & a_{2n} \\ \vdots & & \vdots & & \vdots & & \vdots \\ a_{n1} & \cdots & (a_{ni} + ka_{nj}) & \cdots & a_{nj} & \cdots & a_{nn}
\end{vmatrix} = \begin{vmatrix}
a_{11} & \cdots & a_{1i} & \cdots & a_{1j} & \cdots & a_{1n} \\ a_{21} & \cdots & a_{2i} & \cdots & a_{2j} & \cdots & a_{2n} \\ \vdots & & \vdots & & \vdots & & \vdots \\ a_{n1} & \cdots & a_{ni} & \cdots & a_{nj} & \cdots & a_{nn}
\end{vmatrix}.$$
或者
$$\left|\alpha_1, \cdots, \alpha_i + k\alpha_j, \cdots, \alpha_j, \cdots, \alpha_n \right| = \left|\alpha_1, \cdots, \alpha_i, \cdots, \alpha_j, \cdots, \alpha_n \right|.$$
\end{list}
\end{prop}

\begin{cor}\

\enum
\item[(1)] 行列式有一列(或行)元素全为零,则行列式的值为零。
\item[(2)] 列式中有两列(行)元素相同,则行列式的值为零。
\item[(3)] 列式中有两列(行)成比例,则行列式的值为零
\end{list}
\end{cor}

\begin{thm} \label{thm:det_minor1}
$n$阶行列式$D$等于它的任意一行的所有元素与它们的代数余子式的乘积之和,即
$$D = a_{i1}A_{i1} + a_{i2}A_{i2} + \cdots + a_{in}A_{in} = \sum\limits_{j=1}^n a_{ij}A_{ij}, \quad (i=1,2,\cdots,n).$$
上式称为$n$阶行列式按第$i$行展开公式。完全类似可以得到按列的展开公式:
$$D = a_{1j}A_{1j} + a_{2j}	A_{2j} + \cdots + a_{nj}A_{nj} = \sum\limits_{i=1}^n a_{ij}A_{ij}, \quad (j=1,2,\cdots,n).$$
\end{thm}

\begin{thm} \label{thm:det_minor2}
$n$阶行列式$D$的某一行的元素与另一行对应代数余子式的乘积之和等于零,即
$$a_{i1}A_{k1} + a_{i2}A_{k2} + \cdots + a_{in}A_{kn} = \sum\limits_{j=1}^n a_{ij}A_{kj} = 0, \quad (i\neq k).$$
由行列对称性,对列来说也有类似的结论:
$$a_{1j}A_{1k} + a_{2j}A_{2k} + \cdots + a_{nj}A_{nk} = \sum\limits_{i=1}^n a_{ij}A_{ik} = 0, \quad (j\neq k).$$
\end{thm}

\begin{rmk}
综合定理\ref{thm:det_minor1}与定理\ref{thm:det_minor2}的结论,可以得到:
\begin{eqnarray*}
a_{i1}A_{k1} + a_{i2}A_{k2} + \cdots + a_{in}A_{kn} & = & \sum\limits_{j=1}^n a_{ij}A_{kj} = \begin{cases}
D, & i = k; \\ 0, & i \neq k.
\end{cases} \\
a_{1j}A_{1k} + a_{2j}A_{2k} + \cdots + a_{nj}A_{nk} & = & \sum\limits_{i=1}^n a_{ij}A_{ik} = \begin{cases}
D, & j = k; \\ 0, & j \neq k.
\end{cases}
\end{eqnarray*}
引入克罗内克(Kronecker)符号:
$$\delta_{ik} =
\begin{cases}
1, & i = k; \\ 0, & i \neq k.
\end{cases},$$
上述公式可统一书写为如下形式:
$$\sum\limits_{j=1}^n a_{ij}A_{kj} = \delta_{ik}D;$$ $$\sum\limits_{i=1}^n a_{ij}A_{ik} = \delta_{jk}D.$$
\end{rmk}

\begin{thm}[Cramer法则:n元线性方程组的求解公式]\

若线性方程组
$$\left\{ \begin{array}{rcl} a_{11}x_1 + a_{12}x_2 + \cdots + a_{1n}x_n & = & b_1 \\ a_{21}x_1 + a_{22}x_2 + \cdots + a_{2n}x_n & = & b_2 \\ \hdotsfor{3} \\ a_{n1}x_1 + a_{n2}x_2 + \cdots + a_{nn}x_n & = & b_n \end{array}\right.$$
的系数行列式$D\neq 0$,则方程组有唯一解:
$$x_j = \frac{D_j}{D}, \quad j = 1,2,\cdots,n.$$
其中$D_j$是由线性方程组的常数列替换$D$中第$j$列元素所得到的行列式。
\end{thm}

\begin{cor}
若$n$个变量,$n$个方程的齐次线性方程组
$$\left\{ \begin{array}{rcl} a_{11}x_1 + a_{12}x_2 + \cdots + a_{1n}x_n & = & 0 \\ a_{21}x_1 + a_{22}x_2 + \cdots + a_{2n}x_n & = & 0 \\ \hdotsfor{3} \\ a_{n1}x_1 + a_{n2}x_2 + \cdots + a_{nn}x_n & = & 0 \end{array}\right.$$
的系数行列式$D\neq 0$,则方程组只有零解:
$$x_j = 0, \quad j = 1,2,\cdots,n.$$
\end{cor}

%%%%%%%%%%%%%%%%%%%%%%%%%%%%%%%%%%%%%%%%%%%%%%%%%%%%%%%%%%%%%%%%%%%%%%%%%%%%%%%%%%%%%%%%%%%%

\section{例题讲解}

\begin{eg}
试证$\begin{vmatrix} a_1+b_1 & b_1+c_1 & c_1+a_1 \\ a_2+b_2 & b_2+c_2 & c_2+a_2 \\ a_3+b_3 & b_3+c_3 & c_3+a_3 \end{vmatrix} = 2\begin{vmatrix} a_1 & b_1 & c_1 \\ a_2 & b_2 & c_2 \\ a_3 & b_3 & c_3 \end{vmatrix}$
\end{eg}
\begin{proof}[证明]
\begin{align*}
\text{左端} & = \begin{vmatrix} a_1 & b_1+c_1 & c_1+a_1 \\ a_2 & b_2+c_2 & c_2+a_2 \\ a_3 & b_3+c_3 & c_3+a_3 \end{vmatrix} + \begin{vmatrix} b_1 & b_1+c_1 & c_1+a_1 \\ b_2 & b_2+c_2 & c_2+a_2 \\ b_3 & b_3+c_3 & c_3+a_3 \end{vmatrix} \\
& = \begin{vmatrix} a_1 & b_1+c_1 & c_1 \\ a_2 & b_2+c_2 & c_2 \\ a_3 & b_3+c_3 & c_3 \end{vmatrix} + 0 + 0 + \begin{vmatrix} b_1 & c_1 & c_1+a_1 \\ b_2 & c_2 & c_2+a_2 \\ b_3 & c_3 & c_3+a_3 \end{vmatrix} \\
& = \begin{vmatrix} a_1 & b_1 & c_1 \\ a_2 & b_2 & c_2 \\ a_3 & b_3 & c_3 \end{vmatrix} + 0 + 0 + \begin{vmatrix} b_1 & c_1 & a_1 \\ b_2 & c_2 & a_2 \\ b_3 & c_3 & a_3 \end{vmatrix} \\
& = 2\begin{vmatrix} a_1 & b_1 & c_1 \\ a_2 & b_2 & c_2 \\ a_3 & b_3 & c_3 \end{vmatrix} = \text{右端}
\end{align*}
\end{proof}

\begin{eg}
试证$\begin{vmatrix} a_1 & 0 & 0 \\ b_1 & b_2 & b_3 \\ c_1 & c_2 & c_3 \end{vmatrix} = a_1 \begin{vmatrix} b_2 & b_3 \\ c_2 & c_3	 \end{vmatrix}$
\end{eg}
\begin{proof}[证明]
把左端行列式按第一行展开即得。
\end{proof}

\begin{eg}
计算下列行列式:
\enum
\item[(1)] $\begin{vmatrix} 1 & -1 & 2 \\ 3 & 2 & 1 \\ 0 & 1 & 4 \end{vmatrix}$
\item[(2)] $\begin{vmatrix} x & y & x+y \\ y & x+y & x \\ x+y & x & y \end{vmatrix}$
\item[(3)] $\begin{vmatrix} 1 & 1 & 1 \\ a_1 & a_2 & a_3 \\ a_1^2 & a_2^2 & a_3^2 \end{vmatrix}$
\end{list}
\end{eg}

\begin{solution}\

\enum
\item[(1)]
$$\begin{vmatrix} 1 & -1 & 2 \\ 3 & 2 & 1 \\ 0 & 1 & 4 \end{vmatrix} \xrightarrow{(-3)r1+r2} \begin{vmatrix} 1 & -1 & 2 \\ 0 & 5 & -5 \\ 0 & 1 & 4 \end{vmatrix} = \begin{vmatrix} 5 & -5 \\ 1 & 4 \end{vmatrix} = 25$$
\item[(2)]
\begin{align*}
& \begin{vmatrix} x & y & x+y \\ y & x+y & x \\ x+y & x & y \end{vmatrix} \xrightarrow{(r3+r2)+r1} \begin{vmatrix} 2(x+y) & 2(x+y) & 2(x+y) \\ y & x+y & x \\ x+y & x & y \end{vmatrix} \\
= & 2(x+y)\begin{vmatrix} 1 & 1 & 1 \\ y & x+y & x \\ x+y & x & y \end{vmatrix} \xrightarrow{\substack{-c1+c2 \\ -c1+c3}} 2(x+y)\begin{vmatrix} 1 & 0 & 0 \\ y & x & x-y \\ x+y & -y & -x \end{vmatrix} \\
= & 2(x+y)\begin{vmatrix} x & x-y \\ -y & -x \end{vmatrix} = -2(x+y)(x^2-xy+y^2) \\
= & -2(x^3+y^3)
\end{align*}
\item[(3)]
\begin{align*}
& \begin{vmatrix} 1 & 1 & 1 \\ a_1 & a_2 & a_3 \\ a_1^2 & a_2^2 & a_3^2 \end{vmatrix} \xrightarrow{\substack{-c1+c2 \\ -c1+c3}} \begin{vmatrix} 1 & 0 & 0 \\ a_1 & a_2 - a_1 & a_3 - a_1 \\ a_1^2 & a_2^2 - a_1^2 & a_3^2 - a_1^2 \end{vmatrix} \\
= & \begin{vmatrix} a_2 - a_1 & a_3 - a_1 \\ (a_2 - a_1)(a_2 + a_1) & (a_3 - a_1)(a_3 + a_1) \end{vmatrix} \\
= & (a_2 - a_1)(a_3 - a_1)\begin{vmatrix} 1 & 1 \\ a_2 + a_1 & a_3 + a_1 \end{vmatrix} \\
= & (a_2 - a_1)(a_3 - a_1)(a_3 - a_2)
\end{align*}
\end{list}
\end{solution}

\begin{eg}
分别求$12\cdots n$以及$n(n-1)\cdots 21$的逆序数。
\end{eg}

\begin{solution}
$\tau(12\cdots n) = 0$,而$\tau(n(n-1)\cdots 21) = (n-1)+(n-2)+\cdots+1 = \frac{n}{2}(n-1)$。
\end{solution}

\begin{eg}
写出四阶行列式展开式中同时含$a_{13}$与$a_{32}$的项, 并确定正负号。
\end{eg}

\begin{solution}
对四阶行列式
$$
\begin{vmatrix}
a_{11} & a_{12} & a_{13} & a_{14} \\ a_{21} & a_{22} & a_{23} & a_{24} \\ a_{31} & a_{32} & a_{33} & a_{34} \\ a_{41} & a_{42} & a_{43} & a_{44}
\end{vmatrix}
$$
按定义展开后,同时含$a_{13}$与$a_{32}$的项的一般形式为$a_{13}a_{2j_2}a_{32}a_{4j_4}$,其中$j_2j_4$为$1$与$4$的排列,\\
共两种:$14,41$,对应的项为
$$\begin{array}{rcl}
(-1)^{\tau(3124)} a_{13}a_{21}a_{32}a_{44} & = &  a_{13}a_{21}a_{32}a_{44}, \\
(-1)^{\tau(3421)} a_{13}a_{24}a_{32}a_{41} & = &  -a_{13}a_{24}a_{32}a_{41}
\end{array}$$
\end{solution}

\begin{eg}
已知$f(x) = \begin{vmatrix}
x & 1 & 1 & 2 \\ 1 & x & 1 & -1 \\ 3 & 2 & x & 1 \\ 1 & 1 & 2x & 1
\end{vmatrix}$,求$x^3$的系数。
\end{eg}

\begin{solution}
由不同行不同列的选取原则知,含$x^3$的系数有两项,即
$$(-1)^{\tau(1234)} a_{11}a_{22}a_{33}a_{44} + (-1)^{\tau(1243)} a_{11}a_{22}a_{34}a_{43} = x^3 - 2x^3 = -x^3$$
故$x^3$的系数为-1。
\end{solution}

\begin{eg}
证明对角行列式,上(下)三角行列式均等于其主对角元素的乘积,即均等于
$$a_{11}a_{22}\cdots a_{nn} = \prod\limits_{i=1}^n a_{ii}.$$
\end{eg}

\begin{proof}[证明]
只以“下三角行列式”为例来证明。先决定所有可能的非零项:
\begin{eqnarray*}
a_{1j_1}a_{2j_2}\cdots a_{nj_n} = a_{11}a_{22}\cdots a_{nn} \\
(\because j_1 = 1 \Rightarrow j_2 = 2 \Rightarrow \cdots \Rightarrow j_n = n)
\end{eqnarray*}
其次决定其非零项的符号:
$$\begin{vmatrix}
a_{11} & 0 & \cdots & 0 \\ a_{21} & a_{22} & \cdots & 0 \\ \vdots & \vdots & \ddots & \vdots \\ a_{n1} & a_{n2} & \cdots & a_{nn} \end{vmatrix} = (-1)^{\tau(12\cdots n)}a_{11}a_{22}\cdots a_{nn} = a_{11}a_{22}\cdots a_{nn}.$$
\end{proof}

\begin{eg}
计算$\begin{vmatrix}
a_{11} & a_{12} & a_{13} & a_{14} & a_{15} \\a_{21} & a_{22} & a_{23} & a_{24} & a_{25} \\ a_{31} & a_{32} & 0 & 0 & 0 \\ a_{41} & a_{42} & 0 & 0 & 0 \\ a_{51} & a_{52} & 0 & 0 & 0 \end{vmatrix}$
\end{eg}

\begin{solution}
当$j\geqslant 3$时,$a_{3j}, a_{4j}, a_{5j}$均为零,对行列式展开式中每一项$a_{1j_1}a_{2j_2}a_{3j_3}a_{4j_4}a_{5j_5}$,$j_1,j_2,j_3$互不相同,必然有一个$\geqslant 3$,而每项乘积均为0。故整个行列式为0。
\end{solution}

\begin{eg}
利用行列式, 可以将如下行列式拆分为哪几项之和?
$$D = \begin{vmatrix}
a_1 + b_1 & a_2 + b_2 \\ a_3 + b_3 & a_4 + b_4 \end{vmatrix}$$
\end{eg}

\begin{solution}
$$D = \begin{vmatrix}
a_1 & a_2 + b_2 \\ a_3 & a_4 + b_4 \end{vmatrix} + \begin{vmatrix}
b_1 & a_2 + b_2 \\ b_3 & a_4 + b_4 \end{vmatrix} = \begin{vmatrix}
a_1 & a_2 \\ a_3 & a_4 \end{vmatrix} + \begin{vmatrix}
a_1 & b_2 \\ a_3 & b_4 \end{vmatrix} + \begin{vmatrix}
b_1 & a_2 \\ b_3 & a_4 \end{vmatrix} + \begin{vmatrix}
b_1 & b_2 \\ b_3 & b_4 \end{vmatrix}.$$
\end{solution}

\begin{eg} 利用二阶行列式性质4,我们有
\begin{eqnarray*}
\begin{vmatrix} -15 & 30 \\ 2 & 3 \end{vmatrix} & = & 15 \times \begin{vmatrix} -1 & 2 \\ 2 & 3 \end{vmatrix} = 15 \times (-3-4) = -105 \\
& = & 3 \times \begin{vmatrix} -15 & 10 \\ 2 & 1 \end{vmatrix} = 3 \times 5 \times \begin{vmatrix} -3 & 2 \\ 2 & 1 \end{vmatrix} = -105
\end{eqnarray*}
\end{eg}

\begin{eg}
计算下面的四阶行列式$D = \begin{vmatrix} 1 & 2 & -2 & 3 \\ -1 & -2 & 4 & -2 \\ 0 & 1 & 2 & -1 \\ 2 & 3 & -3 & 10 \end{vmatrix}$。
\end{eg}

\begin{solution}
通过行变换将$D$化为上三角行列式
\begin{eqnarray*}
D & \xlongequal[(-2)\times r_1 + r_4]{r_1 + r_2} & \begin{vmatrix} 1 & 2 & -2 & 3 \\ 0 & 0 & 2 & 1 \\ 0 & 1 & 2 & -1 \\ 0 & -1 & 1 & 4 \end{vmatrix} \xlongequal{r_3 + r_4} \begin{vmatrix} 1 & 2 & -2 & 3 \\ 0 & 0 & 2 & 1 \\ 0 & 1 & 2 & -1 \\ 0 & 0 & 3 & 3 \end{vmatrix} \xrightarrow{r_2 \leftrightarrow r_3} -\begin{vmatrix} 1 & 2 & -2 & 3 \\ 0 & 1 & 2 & -1 \\ 0 & 0 & 2 & 1 \\ 0 & 0 & 3 & 3 \end{vmatrix} \\
& \xrightarrow{(-3/2)\times r_3 + r_4} & -\begin{vmatrix} 1 & 2 & -2 & 3 \\ 0 & 1 & 2 & -1 \\ 0 & 0 & 2 & 1 \\ 0 & 0 & 0 & 3/2 \end{vmatrix} = -3
\end{eqnarray*}
\end{solution}

\begin{eg}
计算$n$阶行列式$D_n = \begin{vmatrix}
a & b & \cdots & b \\ b & a & \cdots & b \\ \vdots & \vdots & \ddots & \vdots \\ b & b & \cdots & a \end{vmatrix}$
\end{eg}

\begin{solution}
分析特点:每行元素之和均为$a+(n-1)b$。

操作:把第$2\sim n$列加到第$1$列,提出公因子,然后将第$1$行的$-1$倍加到其余各行,即可化简。
\begin{eqnarray*}
D_n & = & [a+(n-1)b] \begin{vmatrix} 1 & b & \cdots & b \\ 1 & a & \cdots & b \\ \vdots & \vdots & \ddots & \vdots \\ 1 & b & \cdots & a \end{vmatrix} = [a+(n-1)b] \begin{vmatrix} 1 & b & \cdots & b \\ & a-b & & \\ & & \ddots & \\ & & & a-b \end{vmatrix} \\
& = & [a+(n-1)b](a-b)^{n-1}
\end{eqnarray*}
\end{solution}

\begin{eg}
计算下列行列式(其中$a_i \neq 0, i = 1,…, n$):
$$D_{n+1} = \begin{vmatrix}
a_0 & b_1 & b_2 & \cdots & b_{n-1} & b_n \\ c_1 & a_1 & 0 & \cdots & 0 & 0 \\ c_2 & 0 & a_2 & \cdots & 0 & 0 \\ \vdots & \vdots & \vdots & \ddots & \vdots & \vdots \\ c_n & 0 & 0 & \cdots & 0 & a_n \end{vmatrix}.$$
这种行列式称为\underline{箭形行列式}。
\end{eg}

\begin{solution}
对箭形行列式有固定方法,即把第$i+1$列的$-c_i/a_i$倍加到第1列$(i=1,2,…,n)$,就可以把这个行列式化为三角行列式:
\begin{eqnarray*}
D_{n+1} & = & \begin{vmatrix}
a_0 - \sum\limits_{i=1}^n\frac{b_ic_i}{a_i} & b_1 & b_2 & \cdots & b_{n-1} & b_n \\ 0 & a_1 & 0 & \cdots & 0 & 0 \\ 0 & 0 & a_2 & \cdots & 0 & 0 \\ \vdots & \vdots & \vdots & \ddots & \vdots & \vdots \\ 0 & 0 & 0 & \cdots & 0 & a_n \end{vmatrix} = a_1\cdots a_n\left( a_0 - \sum\limits_{i=1}^n \dfrac{b_ic_i}{a_i} \right) \\
& = & \prod\limits_{j=0}^n a_j - \sum\limits_{i=1}^n \widetilde{a}_ib_ic_i \quad (\text{其中 } \widetilde{a}_i := a_1\cdots a_n / a_i).
\end{eqnarray*}
\end{solution}

\begin{eg}
计算行列式$D = \begin{vmatrix}
1+a_1 & 1 & \cdots & 1 \\ 1 & 1+a_2 & \cdots & 1 \\ \vdots & \vdots & \ddots & \vdots \\ 1 & 1 & \cdots & 1+a_n \end{vmatrix}$,其中$a_i \neq 0, i = 1,…, n$。
\end{eg}

\begin{solution}[解法一]
$D$的第$2\sim n$行均减第$1$行,可化成箭形行列式:
\begin{eqnarray*}
D & = & \begin{vmatrix} 1+a_1 & 1 & \cdots & 1 \\ -a_1 & a_2 & & \\ \vdots & & \ddots & \\ -a_1 & & & a_n \end{vmatrix} = \begin{vmatrix} 1+a_1+\sum\limits_{i=2}^n\frac{a_1}{a_i} & 1 & \cdots & 1 \\ 0 & a_2 & & \\ \vdots & & \ddots & \\ 0 & & & a_n \end{vmatrix} \\
& = & (1+a_1+\sum\limits_{i=2}^n\frac{a_1}{a_i})a_2\cdots a_n = (\prod\limits_{k=1}^n a_k)(1+\sum\limits_{i=1}^n\frac{1}{a_i})
\end{eqnarray*}
\end{solution}

\begin{solution}[解法二]
除主对角线外,将每项的$1$写成$1+0$,将$D$拆成$2^n$个行列式,只有如下的$n+1$个非$0$:
\begin{eqnarray*}
D & = & \begin{vmatrix} 1 & & & \\ 1 & a_2 & & \\ \vdots & & \ddots & \\ 1 & & & a_n \end{vmatrix} + \begin{vmatrix} a_1 & 1 & & \\ & 1 & & & \\ & \vdots & \ddots & \\ & 1 & & a_n \end{vmatrix} + \cdots + \begin{vmatrix} a_1 & & & 1 \\ & a_2 & & 1 \\ & & \ddots & \vdots \\  & & & 1 \end{vmatrix} + \begin{vmatrix} a_1 & & & \\ & a_2 & & \\ & & \ddots & \\ & & & a_n \end{vmatrix} \\
& = & \frac{1}{a_1}\prod\limits_{k=1}^n a_k + \frac{1}{a_2}\prod\limits_{k=1}^n a_k + \cdots + \frac{1}{a_n}\prod\limits_{k=1}^n a_k + \prod\limits_{k=1}^n a_k \\
& = & (\prod\limits_{k=1}^n a_k)(1+\sum\limits_{i=1}^n\frac{1}{a_i})
\end{eqnarray*}
\end{solution}

\begin{solution}[解法三]
在$D$的上面加一行$(1,1,\cdots,1)$,左边加一列$(1,0,\cdots,0)^T$,得到$n+1$阶行列式$D_1$,这种方法叫加边法(升阶法):
\begin{eqnarray*}
D & = & D_1 = \begin{vmatrix}
1 & 1 & \cdots & 1 \\ 0 & 1+a_1 & \cdots & 1 \\ \vdots & \vdots & \ddots & \vdots \\ 0 & 1 & \cdots & 1+a_n \end{vmatrix} = \begin{vmatrix}
1 & 1 & \cdots & 1 \\ -1 & a_1 & \cdots & \\ \vdots & & \ddots & \\ -1 & & & a_n \end{vmatrix} \quad (\text{化为了箭形行列式}) \\
& = & \begin{vmatrix} 1+\sum\limits_{i=1}^n\frac{1}{a_i} & 1 & \cdots & 1 \\ 0 & a_1 & & \\ \vdots & & \ddots & \\ 0 & & & a_n \end{vmatrix} = (1+\sum\limits_{i=1}^n\frac{1}{a_i})(\prod\limits_{k=1}^n a_k)
\end{eqnarray*}
\end{solution}

\begin{eg}
对于$2$阶行列式$\begin{vmatrix} a_{11} & a_{12} \\ a_{21} & a_{22} \end{vmatrix}$,有
$$M_{11} = A_{11} = a_{22}, M_{12} = a_{21}, A_{12} = -a_{21},$$
$$\Rightarrow D = a_{11}A_{11} + a_{12}A_{12} = a_{11}a_{22} - a_{12}a_{21}.$$
\end{eg}

\begin{eg}
\begin{eqnarray*}
D & = & \begin{vmatrix}
5 & 3 & -1 & 2 & 0 \\  1 & 7 & 2 & 5 & 2 \\ 0 & -2 & 3 & 1 & 0 \\ 0 & -4 & -1 & 4 & 0 \\ 0 & 2 & 3 & 5 & 0
\end{vmatrix} \xlongequal{\text{按最后一列展开}} (-1)^{2+5}2 \begin{vmatrix}
5 & 3 & -1 & 2 \\ 0 & -2 & 3 & 1 \\ 0 & -4 & -1 & 4 \\ 0 & 2 & 3 & 5
\end{vmatrix} \\
& \xlongequal{\text{按第一列展开}} & -2\cdot 5 \begin{vmatrix}
-2 & 3 & 1 \\ -4 & -1 & 4 \\ 2 & 3 & 5
\end{vmatrix} \xlongequal[r_1+r_3]{-2r_1+r_2} -10 \begin{vmatrix}
-2 & 3 & 1 \\ 0 & -7 & 2 \\ 0 & 6 & 6
\end{vmatrix} \\
& \xlongequal{\text{按第一列展开}} & -10\cdot(-2) \begin{vmatrix}
-7 & 2 \\ 6 & 6
\end{vmatrix} = 20(-42-12) = -1080
\end{eqnarray*}
\end{eg}

\begin{eg}
证明$n$阶范德蒙(Vandermonde)行列式$(n \geqslant 2)$
$$D_n = \begin{vmatrix}
1 & 1 & 1 & \cdots & 1 \\
a_1 & a_2 & a_3 & \cdots & a_n \\
a_1^2 & a_2^2 & a_3^2 & \cdots & a_n^2 \\
\vdots & \vdots & \vdots & \ddots & \vdots \\
a_1^{n-1} & a_2^{n-1} & a_3^{n-1} & \cdots & a_n^{n-1} \\
\end{vmatrix} = \prod_{1\leqslant j < i \leqslant n} (a_i - a_j).$$
\end{eg}

\begin{proof}[证明]
用数学归纳法。

当$n=2$时,$D_n = \begin{vmatrix} 1 & 1\\ a_1 & a_2 \end{vmatrix} = a_2 - a_1 = \prod_{1\leqslant j < i \leqslant 2} (a_i - a_j)$。

假设结论对$n=k-1$成立,即
$$D_{k-1} = \begin{vmatrix}
1 & 1 & 1 & \cdots & 1 \\
a_1 & a_2 & a_3 & \cdots & a_{k-1} \\
\vdots & \vdots & \vdots & \ddots & \vdots \\
a_1^{k-2} & a_2^{k-2} & a_3^{k-2} & \cdots & a_{k-1}^{k-2} \\
\end{vmatrix} = \prod_{1\leqslant j < i \leqslant k-1} (a_i - a_j)$$
则当$n=k$时,观察$D_k$的第一列,下面元素是上面元素的$a_1$倍。从第$n$行到第$2$行,依次将前一行的$(-a_1)$倍加到本行上,得
\begin{eqnarray*}
D_{k} & = & \begin{vmatrix}
1 & 1 & 1 & \cdots & 1 \\
0 & a_2-a_1 & a_3-a_1 & \cdots & a_k-a_1 \\
0 & a_2(a_2-a_1) & a_3(a_3-a_1) & \cdots & a_k(a_k-a_1) \\
\vdots & \vdots & \vdots & \ddots & \vdots \\
0 & a_1^{k-2}(a_2-a_1) & a_2^{k-2}(a_3-a_1) & \cdots & a_{k-1}^{k-2}(a_k-a_1) \\
\end{vmatrix} \\
& = & (a_k - a_1)\cdots(a_2-a_1)\begin{vmatrix}
1 & 1 & 1 & \cdots & 1 \\
a_1 & a_2 & a_3 & \cdots & a_{k-1} \\
\vdots & \vdots & \vdots & \ddots & \vdots \\
a_1^{k-2} & a_2^{k-2} & a_3^{k-2} & \cdots & a_{k-1}^{k-2} \\
\end{vmatrix} \\
& = & (a_k - a_1)\cdots(a_2-a_1)\prod_{2\leqslant j < i \leqslant k} (a_i - a_j) \\
& = & \prod_{1\leqslant j < i \leqslant k} (a_i - a_j).
\end{eqnarray*}
\end{proof}

\begin{eg}
计算下列行列式
\enum
\item[(1)] $\begin{vmatrix}
a_1^2 & a_2^2 & a_3^2 \\ a_1 & a_2 & a_3 \\ 1 & 1 & 1
\end{vmatrix}$
\item[(2)]$\begin{vmatrix}
1 & 1 & 4 & a \\ 3 & \frac12 & 8 & aq \\ 9 & \frac14 & 16 & aq^2 \\ 27 & \frac18 & 32 & aq^3
\end{vmatrix}$
\end{list}
\end{eg}

\begin{solution}\

\enum
\item[(1)] $\begin{vmatrix}
a_1^2 & a_2^2 & a_3^2 \\ a_1 & a_2 & a_3 \\ 1 & 1 & 1
\end{vmatrix} \xlongequal{r_1 \leftrightarrow r_3} - \begin{vmatrix}
1 & 1 & 1 \\ a_1 & a_2 & a_3 \\ a_1^2 & a_2^2 & a_3^2
\end{vmatrix} = -(a_2 - a_1)(a_3 - a_1)(a_3 - a_2)$。
\item[(2)]$\begin{vmatrix}
1 & 1 & 4 & a \\ 3 & \frac12 & 8 & aq \\ 9 & \frac14 & 16 & aq^2 \\ 27 & \frac18 & 32 & aq^3
\end{vmatrix} = 4a \begin{vmatrix}
1 & 1 & 1 & 1 \\ 3 & \frac12 & 2 & q \\ 3^2 & (\frac12)^2 & 2^2 & q^2 \\ 3^3 & (\frac12)^3 & 2^3 & q^3
\end{vmatrix} = 15a(q-\frac12)(q-3)(q-2)$。
\end{list}
\end{solution}

\begin{eg}
计算$D = \begin{vmatrix}
1 & 1 & 1 \\ a_1 & a_2 & a_3 \\ a_1^3 & a_2^3 & a_3^3
\end{vmatrix}$.
\end{eg}

\begin{solution}
考虑
$$D' = \begin{vmatrix}
1 & 1 & 1 & 1 \\ x &  a_1 & a_2 & a_3 \\ x^2 & a_1^2 & a_2^2 & a_3^2 \\ x^3 & a_1^3 & a_2^3 & a_3^3
\end{vmatrix}.$$
一方面,$D'$是一个标准范德蒙行列式,有
$$D' = (a_1-x)(a_2-x)(a_3-x)\prod_{1\leqslant j < i \leqslant 3} (a_i - a_j).$$
故$D'$是关于$x$的$3$次多项式,其中$x^2$的系数为
$$(a_1+a_2+a_3)\prod_{1\leqslant j < i \leqslant 3} (a_i - a_j).$$
另一方面,将$D'$按第一列展开,可知$x^2$的系数为
$$(-1)^{3+1}D = D.$$
比较两种方法所得$x^2$的系数,有
$$D = (a_1+a_2+a_3)\prod_{1\leqslant j < i \leqslant 3} (a_i - a_j) = \left(\sum\limits_{k=1}^n a_k\right)\cdot\prod_{1\leqslant j < i \leqslant 3} (a_i - a_j).$$

本例中的$D$称为\underline{超范德蒙行列式},且结果可以推广到一般情况。
\end{solution}

\begin{eg}
计算$n$阶三对角行列式(递推公式法)
$$D_n = \begin{vmatrix}
\alpha + \beta & \alpha\beta & & & & \\ 1 & \alpha + \beta & \alpha\beta & & & \makebox(0,0){\text{\Huge0}} \\ & 1 & \alpha + \beta & \alpha\beta & & \\ & & \ddots & \ddots & \ddots & \\ & \makebox(0,0){\text{\Huge0}} & & \ddots & \ddots & \alpha\beta \\ & & & & 1 & \alpha + \beta
\end{vmatrix}$$
\end{eg}

\begin{solution}
将$D_n$按第一列展开,得
\begin{eqnarray*}
& D_n = & (\alpha + \beta) D_{n-1} - \begin{vmatrix}
\alpha\beta & 0 & \cdots & \cdots & 0 \\ 1 & \alpha + \beta & \alpha\beta & & \\ & 1 & \ddots & \ddots & \\ & & \ddots & \ddots & \alpha\beta \\ & & & 1 & \alpha + \beta
\end{vmatrix} \\
& \Longrightarrow & D_n = (\alpha + \beta) D_{n-1} - \alpha\beta D_{n-2}
\end{eqnarray*}
可化为
$$D_n - \beta D_{n-1}= \alpha D_{n-1} - \alpha\beta D_{n-2} = \alpha(D_{n-1} - \beta D_{n-2})$$
令$d_n = D_n - \beta D_{n-1}$得
$$d_n = \alpha d_{n-1} = \alpha^2 d_{n-2} = \cdots = \alpha^{n-2} d_2 = \alpha^{n-2}(D_2 - \beta D_1) = \alpha^n,$$
从而有
\begin{eqnarray*}
D_n & = & \beta D_{n-1} + \alpha^n = \beta (\beta D_{n-2} + \alpha^{n-1}) + \alpha^n \\
& = & \beta^2 D_{n-2} + \alpha^{n-1}\beta + \alpha^n \\
& \cdots & \\
& = & \beta^n + \beta^{n-1}\alpha + \beta^{n-2}\alpha^2 + \cdots + \beta\alpha^{n-1} + \alpha^n
\end{eqnarray*}
\end{solution}

\begin{eg} \label{eg:block_matrix}
证明
\[
  \setlength{\dashlinegap}{2pt}
  \left| \begin{array}{ccc:ccc}
    a_{11} & \cdots & a_{1r} & 0 & \cdots & 0 \\
    \vdots & & \vdots & \vdots & & \vdots \\
    a_{r1} & \cdots & a_{rr} & 0 & \cdots & 0 \\
    \hdashline
    c_{11} & \cdots & c_{1r} & b_{11} & \cdots & b_{1s} \\
    \vdots & & \vdots & \vdots & & \vdots \\
    c_{s1} & \cdots & c_{sr} & b_{s1} & \cdots & b_{ss} \\
  \end{array} \right| =
\begin{vmatrix}
a_{11} & \cdots & a_{1r} \\
\vdots & & \vdots \\
a_{r1} & \cdots & a_{rr}
\end{vmatrix} \cdot
\begin{vmatrix}
b_{11} & \cdots & b_{1s} \\
\vdots & & \vdots \\
b_{s1} & \cdots & b_{ss}
\end{vmatrix}
\]
\end{eg}

\begin{proof}[证明]
设$A=(a_{ij})_{r\times r}, B=(b_{ij})_{s\times s}, C=(c_{ij})_{s\times r}, O$为零矩阵,则要证明的结论可简写为
$$\begin{vmatrix} A & 0 \\ C & B\end{vmatrix} = |A|\cdot|B|.$$

对$r$作归纳。当$r=1$时,按第一行展开知,结论成立。

假设结论对$r-1$成立,则对$r$的情形,仍把行列式按第一行展开,且将$A$中去掉第$1$行第$j$列的矩阵记为$A_j$,将$C$中去掉第$j$列的矩阵记为$C_j, 1 \leqslant j \leqslant r$,则有
$$\begin{vmatrix} A & 0 \\ C & B\end{vmatrix} = a_{11}\begin{vmatrix} A_1 & 0 \\ C_1 & B\end{vmatrix} + (-1)^{1+2}a_{12}\begin{vmatrix} A_2 & 0 \\ C_2 & B\end{vmatrix} + \cdots + (-1)^{1+r}a_{1r}\begin{vmatrix} A_r & 0 \\ C_r & B\end{vmatrix}.$$
由于$A_j$均为$r-1$阶方阵,根据归纳假设,有$\begin{vmatrix} A_j & 0 \\ C_j & B\end{vmatrix} = |A_j|\cdot|B|$。故
$$\begin{vmatrix} A & 0 \\ C & B\end{vmatrix} = \sum\limits_{j=1}^r (-1)^{1+j}a_{1j}|A_j|\cdot|B| = (\sum\limits_{j=1}^r (-1)^{1+j}a_{1j}|A_j|)\cdot|B| = |A|\cdot|B|.$$

类似地,可得如下结论:
\enum
\item[(1)] $\begin{vmatrix} A_{r\times r} & 0 \\ C_{s\times r} & B_{s\times s}\end{vmatrix} = \begin{vmatrix} A_{r\times r} & C_{r\times s} \\ 0 & B_{s\times s}\end{vmatrix} = |A_{r\times r}|\cdot|B_{s\times s}|$。
\item[(2)] $\begin{vmatrix} 0 &  A_{r\times r} \\ B_{s\times s} & C_{s\times r} \end{vmatrix} = \begin{vmatrix} C_{r\times s} & A_{r\times r}  \\ B_{s\times s} & 0 \end{vmatrix} = (-1)^{rs}|A_{r\times r}|\cdot|B_{s\times s}|$。

利用上述结论,可尽量使行列式的整块化零,从而降阶,简化计算。这样的方法称为:分块三角阵法或分块打洞法。
\end{list}
\end{proof}

\begin{eg}
用克拉默法则解方程组$\begin{cases}
3x_1 + 5x_2 + 2x_3 + x_4 = 3 \\
3x_2 + 4x_4 = 4 \\
x_1 + x_2 + x_3 + x_4 = 11/6 \\
x_1 - x_2 - 3x_3 + 2x_4 = 5/6
\end{cases}$
\end{eg}

\begin{solution}
$D = \begin{vmatrix} 3 & 5 & 2 & 1 \\ 0 & 3 & 0 & 4 \\ 1 & 1 & 1 & 1 \\ 1 & -1 & -3 & 2 \end{vmatrix} = 67 \neq 0$,
$D_1 = \begin{vmatrix} 3 & 5 & 2 & 1 \\ 4 & 3 & 0 & 4 \\ 11/6 & 1 & 1 & 1 \\ 5/6 & -1 & -3 & 2 \end{vmatrix} = 67/3$,
$D_2 = \begin{vmatrix} 3 & 3 & 2 & 1 \\ 0 & 4 & 0 & 4 \\ 1 & 11/6 & 1 & 1 \\ 1 & 5/6 & -3 & 2 \end{vmatrix} = 0$,
$D_3 = \begin{vmatrix} 3 & 5 & 3 & 1 \\ 0 & 3 & 4 & 4 \\ 1 & 1 & 11/6 & 1 \\ 1 & -1 & 5/6 & 2 \end{vmatrix} = 67/2$,
$D_4 = \begin{vmatrix} 3 & 5 & 2 & 3 \\ 0 & 3 & 0 & 4 \\ 1 & 1 & 1 & 11/6 \\ 1 & -1 & -3 & 5/6 \end{vmatrix} =67$。即可算得:
$$
\begin{cases}
x_1 = D_1/D = 1/3 \\ x_2 = D_2/D = 0 \\ x_3 = D_3/D = 1/2 \\ x_4 = D_4/D = 1
\end{cases}$$
\end{solution}

\begin{eg}
设$P_i = (x_i ,y_i),$ $i=1, 2, 3$,为平面上不共线的三点,且$x_1,$ $x_2,$ $x_3$两两互不相等,求过点$P_1,$ $P_2,$ $P_3$的二次曲线$y = ax^2+bx+c$。
\end{eg}

\begin{solution}
将$P_i = (x_i ,y_i)$代入曲线方程$y = ax^2+bx+c$,得
$$\begin{cases}
ax_1^2 + bx_1 + c = y_1 \\
ax_2^2 + bx_2 + c = y_2 \\
ax_3^2 + bx_3 + c = y_3
\end{cases}$$
这是一个以$a, b, c$为未知量的线性方程组,其系数行列式为
$$D = \begin{vmatrix} x_1^2 & x_1 & 1 \\ x_2^2 & x_2 & 1 \\ x_3^2 & x_3 & 1 \end{vmatrix} \neq 0 \quad(\text{因为$x_1, x_2, x_3$互不相等}).$$
$$D_1 = \begin{vmatrix} y_1 & x_1 & 1 \\ y_2 & x_2 & 1 \\ y_3 & x_3 & 1 \end{vmatrix},
D_2 = \begin{vmatrix} x_1^2 & y_1 & 1 \\ x_2^2 & y_2 & 1 \\ x_3^2 & y_3 & 1 \end{vmatrix},
D_3 = \begin{vmatrix} x_1^2 & x_1 & y_1 \\ x_2^2 & x_2 & y_2 \\ x_3^2 & x_3 & y_3 \end{vmatrix},$$
因为$P_1, P_2, P_3$不共线,所以$D_1\neq 0$,所以
$$a = D_1/D\neq 0, b = D_2/D, c = D_3/D,$$
代入二次曲线方程,有
$$y = ax^2+bx+c = \begin{vmatrix} x_1^2 & x_1 & 1 \\ x_2^2 & x_2 & 1 \\ x_3^2 & x_3 & 1 \end{vmatrix}^{-1}\cdot\begin{vmatrix} x^2 & x & 1 & 0 \\ x_1^2 & x_1 & 1 & y_1 \\ x_2^2 & x_2 & 1 & y_2 \\ x_3^2 & x_3 & 1 & y_3 \end{vmatrix}.$$
\end{solution}

%%%%%%%%%%%%%%%%%%%%%%%%%%%%%%%%%%%%%%%%%%%%%%%%%%%%%%%%%%%%%%%%%%%%%%%%%%%%%%%%%%%%%%%%%%%%

\section{课后习题}

\begin{ex} \label{ex:2.1}
求以下排列的逆序数:

\enum
\item[(1)] $(1374625)$
\item[(2)] $(21436587)$
\end{list}
\end{ex}

\begin{ex} \label{ex:2.2}
证明:任意一个$1,2,\cdots,n$的排列,可经过不多于$n$次对换,变成$(123\cdots n)$。
\end{ex}

\begin{ex} \label{ex:2.3}
在数$1,2,\cdots,n$组成的任意一个排列中,逆序数和正序数之和等于多少?
\end{ex}

\begin{ex} \label{ex:2.4}
设$1,2,\cdots,n$组成的排列$a_1,\cdots,a_n$的逆序数为$k$,请问$a_n,\cdots,a_1$逆序数等于多少?
\end{ex}

\begin{ex} \label{ex:2.5}
如果$n$阶行列式中所有元素都变号,则行列式值如何变化?
\end{ex}

\begin{ex} \label{ex:2.6}
设$n$阶行列式$|A|$的值为$D$,若将$|A|$的每个元素$a_{ij}$变为$(-1)^{i+j}a_{ij}$,则行列式值如何变化?
\end{ex}

\begin{ex} \label{ex:2.7}
求下列2阶行列式:

\enum
\item[(1)] $\begin{vmatrix} 0.7 & 0.3 \\ 0.2 & 0.8 \end{vmatrix}$
\item[(2)] $\begin{vmatrix} \cos\theta & -\sin\theta \\ \sin\theta & \cos\theta \end{vmatrix}$
\item[(3)] $\begin{vmatrix} a & b \\ c & d \end{vmatrix}$
\end{list}
\end{ex}

\begin{ex} \label{ex:2.8}
求下列3阶行列式:

\enum
\item[(1)] $\begin{vmatrix} 1 & 4 & 4 \\ -1 & -3 & -2 \\ 1 & 3 & 3 \end{vmatrix}$
\item[(2)] $\begin{vmatrix} 1 & 1 & 0 \\ 0 & 1 & 4 \\ -1 & -1 & 1 \end{vmatrix}$
\item[(3)] $\begin{vmatrix} -2 & -1 & -4 \\ -3 & -2 & -4 \\ -2 & -1 & -3 \end{vmatrix}$
\end{list}
\end{ex}

\begin{ex} \label{ex:2.9}
求下列4阶行列式:

\enum
\item[(1)] $\begin{vmatrix} 0 & -1 & 4 & 0 \\ 1 & 0 & -3 & 0 \\ 1 & 0 & -2 & -1 \\ 3 & -3 & 0 & 4 \end{vmatrix}$
\item[(2)] $\begin{vmatrix} 1 & 2 & -1 & -2 \\ 0 & 1 & 0 & -3 \\ 0 & 0 & 1 & -4 \\ 0 & 0 & 1 & -3 \end{vmatrix}$
\item[(3)] $\begin{vmatrix} 2 & 4 & 3 & 4 \\ -2 & -3 & -3 & 0 \\ -1 & -2 & -1 & -4 \\ 0 & 2 & 3 & -3 \end{vmatrix}$
\end{list}
\end{ex}

\begin{ex} \label{ex:2.10}
求下列5阶行列式:

\enum
\item[(1)] $\begin{vmatrix} -3 & 5 & 0 & 1 & -1 \\ 1 & -2 & 7 & -3 & 1 \\ 0 & 0 & 1 & -3 & 0 \\ 0 & 0 & 1 & 2 & 3 \\ 0 & 0 & 0 & -2 & -1 \end{vmatrix}$
\item[(2)] $\begin{vmatrix} 1 & 0 & 0 & 0 & 0 \\ 2 & 4 & 3 & -3 & -2 \\ 3 & -1 & 0 & -1 & -3 \\ 4 & -3 & -2 & 2 & 1 \\ 5 & 1 & 2 & -3 & -4 \end{vmatrix}$
\end{list}
\end{ex}

\begin{ex} \label{ex:2.11}
计算行列式$\begin{vmatrix} 103 & 100 & 204 \\ 199 & 200 & 395 \\ 301 & 300 & 600 \end{vmatrix}$。
\end{ex}

\begin{ex} \label{ex:2.12}
利用范德蒙行列式的结果,计算以下行列式

\enum
\item[(1)] $\begin{vmatrix} 1 & 1 & 1 \\ 2 & 3 & 4 \\ 4 & 9 & 16 \end{vmatrix}$
\item[(2)] $\begin{vmatrix} 1 & 2 & 4 & 8 \\ -1 & -3 & -9 & -27 \\ 1 & 4 & 16 & 64 \\ -1 & 5 & -25 & 125 \end{vmatrix}$
\item[(3)] $\begin{vmatrix} a_1 + a_2 & a_2 + a_3 & \cdots & a_n + a_1 \\ a_1^2 + a_2^2 & a_2^2 + a_3^2 & \cdots & a_n^2 + a_1^2 \\ \vdots & \vdots & \ddots & \vdots \\ a_1^n + a_2^n & a_2^n + a_3^n & \cdots & a_n^n + a_1^n \end{vmatrix}$
\end{list}
\end{ex}

\begin{ex}\  \label{ex:2.13}

\enum
\item[(1)] 计算行列式$\begin{vmatrix} 1 & 1 & 1 & 1 \\ x & a & b & c \\ x^2 & a^2 & b^2 & c^2 \\ x^3 & a^3 & b^3 & c^3 \end{vmatrix}$。
\item[(2)] 结合展开公式,求行列式$\begin{vmatrix} 1 & 1 & 1 \\ a & b & c \\ a^3 & b^3 & c^3 \end{vmatrix}$以及$\begin{vmatrix} 1 & 1 & 1 \\ a^2 & b^2 & c^2 \\ a^3 & b^3 & c^3 \end{vmatrix}$。
\end{list}
\end{ex}

\begin{ex} \label{ex:2.14}
令$\{F_n\}$为Fibonacci数列,由
$$
\begin{cases}
F_1 = 1, F_2 = 2 \\
F_n = F_{n-1} + F_{n-2}, & n \geqslant 3
\end{cases}
$$
给出。证明
$F_n = \begin{vmatrix} 1 & 1 & & & \\ -1 & 1 & \ddots & & \\ & \ddots & \ddots & \ddots & \\ & & \ddots & \ddots & 1 \\ & & & -1 & 1 \end{vmatrix}_{n}$。
\end{ex}

\begin{ex} \label{ex:2.15}
展开以下行列式:

\enum
\item[(1)] $\begin{vmatrix} a & x & b & x \\ x & a & x & b \\ b & x & b & x \\ x & b & x & b \end{vmatrix}$
\item[(2)] $\begin{vmatrix} 1 & x & x^2 & x^3 \\ x^3 & 1 & x & x^2 \\ x^2 & x^3 & 1 & x \\ x & x^2 & x^3 & 1 \end{vmatrix}$
\end{list}
\end{ex}

\begin{ex} \label{ex:2.16}
证明下列恒等式:

\enum
\item[(1)] $\begin{vmatrix} a^2 & ab & b^2 \\ 2a & a + b & 2b \\ 1 & 1 & 1 \end{vmatrix} = (a - b)^3$

\item[(2)] $\begin{vmatrix} a_1 + b_1x & a_1x + b_1 & c_1 \\ a_2 + b_2x & a_2x + b_2 & c_2 \\ a_3 + b_3x & a_3x + b_3 & c_3 \end{vmatrix} = (1 - x^2)\begin{vmatrix} a_1 & b_1 & c_1 \\ a_2 & b_2 & c_2 \\ a_3 & b_3 & c_3 \end{vmatrix}$

\item[(3)] $\begin{vmatrix} x & -1 & & & \\ & x & -1 & & \\ & & \ddots & \ddots & \\ & & & x & -1 \\ a_n & a_{n-1} & \cdots & a_2 & x + a_1 \end{vmatrix} = x^n + \sum\limits_{k=1}^n a_kx^{n-k}$

\item[(4)] $\begin{vmatrix} \cos\theta & 1 & & & \\ 1 & 2\cos\theta & 1 & & \\ & \ddots & \ddots & \ddots & \\ & & 1 & 2\cos\theta & 1 \\ & & & 1 & 2\cos\theta \end{vmatrix}_n = \cos n\theta$

\item[(5)] $\begin{vmatrix} 0 & \cdots & 0 & a_{1n} \\ 0 & \cdots & a_{2,n-1} & a_{2n} \\ \vdots & \ddots & \vdots & \vdots \\ a_{n1} & \cdots & a_{n,n-1} & a_{nn} \end{vmatrix} = (-1)^{\frac{n(n-1)}{2}}a_{1n}a_{2,n-1}\cdots a_{n1}$

%\item[(6)] $\begin{vmatrix} 1 + a_1 & 1 & \cdots & 1 \\ 1 & 1 + a_2 & \cdots & 1 \\ \vdots & \vdots & \ddots & \vdots \\ 1 & 1 & \cdots & 1 + a_n \end{vmatrix} = (1 + \sum\limits_{i=1}^n\frac{1}{a_i}) \prod\limits_{i=1}^n a_i, \quad a_i \neq 0, i = 1, 2, \cdots, n$
\end{list}
\end{ex}

\begin{ex} \label{ex:2.17}
设
$$D = \begin{vmatrix} a_{11} & \cdots & a_{1n} \\ \vdots & & \vdots \\ a_{n1} & \cdots & a_{nn} \end{vmatrix},$$
求证:
$$D' = \begin{vmatrix} a_{11}+x_1 & a_{12}+x_2 & \cdots & a_{1n}+x_n \\ a_{21}+x_1 & a_{22}+x_2 & \cdots & a_{2n}+x_n \\ \vdots & \vdots & & \vdots \\ a_{n1}+x_1 & a_{n2}+x_2 & \cdots & a_{nn}+x_n \end{vmatrix} = D + \sum\limits_{j=1}^n x_j \sum\limits_{i=1}^n A_{ij},$$
其中$A_{ij}$为$D$中元素$a_{ij}$的代数余子式。
\end{ex}

\begin{ex} \label{ex:2.18}
用Cramer法则解下述线性方程组(首先验证Cramer法则的适用条件是否满足)

\enum
\item[(1)] $\begin{bmatrix} -1 & 0 & 0 \\ -5 & -2 & 6 \\ -5 & -3 & 7 \end{bmatrix} \begin{bmatrix} x \\ y \\ z \end{bmatrix} = \begin{bmatrix} -2 \\ 1 \\ -1 \end{bmatrix}$

\item[(2)] $\begin{bmatrix} -2 & 1 & 1 \\ -2 & -2 & 2 \\ -2 & 0 & 2 \end{bmatrix} \begin{bmatrix} x \\ y \\ z \end{bmatrix} = \begin{bmatrix} -1 \\ 0 \\ 1 \end{bmatrix}$

\item[(3)] $\begin{bmatrix} 0 & -2 & 0 & 1 \\ 0 & 0 & 1 & 0 \\ 2 & -2 & 0 & 0 \\ 0 & 0 & 0 & 1 \end{bmatrix} \begin{bmatrix} x_1 \\ x_2 \\ x_3 \\ x_4 \end{bmatrix} = \begin{bmatrix} 1 \\ 2 \\ 3 \\ 4 \end{bmatrix}$
\end{list}
\end{ex}

\begin{ex} \label{ex:2.19}
假设平面上三个点$(x_1, y_1), (x_2, y_2), (x_3, y_3)$不共线。证明:由方程
 $$\begin{vmatrix} 1 & x & y & x^2 + y^2 \\ 1 & x_1 & y_1 & x_1^2 + y_1^2 \\ 1 & x_2 & y_2 & x_2^2 + y_2^2 \\ 1 & x_3 & y_3 & x_3^2 + y_3^2 \end{vmatrix} = 0$$ ($x,y$为未知元) 给出的解集是平面上过$(x_1, y_1), (x_2, y_2), (x_3, y_3)$这三个点的圆。
\end{ex}

\begin{ex} \label{ex:2.20}
设$\left\{ \begin{array}{rcl} L_1: \alpha x + \beta y + \gamma = 0 \\ L_2: \gamma x + \alpha y + \beta = 0 \\ L_3: \beta x + \gamma y + \alpha = 0\end{array}\right.$是三条不同的直线,若$L_1, L_2, L_3$交于一点,试证$\alpha + \beta + \gamma = 0$。
\end{ex}

\begin{ex} \label{ex:2.21}
设$A$为$n$阶方阵,$\alpha,\beta$为$n$维列向量,$b,c$为实数。假设有$\begin{vmatrix} A & \alpha \\ \beta^T & b \end{vmatrix} = 0$,求证$$\begin{vmatrix} A & \alpha \\ \beta^T & c \end{vmatrix} = (c-b)|A|.$$
\end{ex}

\begin{ex}\ \label{ex:2.22}

\enum
\item[(1)] 设$n > 1,$ 由
$$\begin{vmatrix} 1 & 1 & \cdots & 1 \\ 1 & 1 & \cdots & 1 \\ \vdots & \vdots &  & \vdots \\ 1 & 1 & \cdots & 1 \end{vmatrix}_{n\times n} = 0,$$
证明数$1,2,\cdots,n$组成的所有排列中,奇偶排列各占一半。

\item[(2)] 计算$\sum\limits_{i_1i_2\cdots i_n\in P_n} \begin{vmatrix} a_{1i_1} & a_{1i_2} & \cdots & a_{1i_n} \\ a_{2i_1} & a_{2i_2} & \cdots & a_{2i_n} \\ \vdots & \vdots &  & \vdots \\ a_{ni_1} & a_{ni_2} & \cdots & a_{ni_n} \end{vmatrix}$。
\end{list}
\end{ex}

\begin{ex} \label{ex:2.23}
设$|A| = \det (a_{ij})_{n\times n}$为$n$阶行列式。如果$a_{ii}>0 (i = 1,2,\cdots,n)$,$a_{ij}<0 (i\neq j)$,又设$\sum\limits_{i=1}^n a_{ij} > 0 (j = 1,2,\cdots,n)$。试证行列式
$$D_n = \begin{vmatrix} a_{11} & \cdots & a_{1n} \\ \vdots & & \vdots \\ a_{n1} & \cdots & a_{nn} \end{vmatrix} > 0$$
\end{ex}

\begin{ex} \label{ex:2.24}
设$n(n \geqslant 2)$阶行列式$\det A$的所有元素为$1$或者$-1$,

\enum
\item[(1)] 求证:$\det A$的绝对值小于等于$n!$。
\item[(2)] 求证:$\det A$被$2^{n-1}$整除。
\item[(3)] 求证:$\det A$的绝对值小于等于$(n-1)!\cdot2(n-1)$。
\item[(4)] 求证:$n > 2$时,$\det A$的绝对值小于等于$(n-1)!\cdot(n-1)$。
\end{list}
\end{ex}

\newpage

%%%%%%%%%%%%%%%%%%%%%%%%%%%%%%%%%%%%%%%%%%%%%%%%%%%%%%%%%%%%%%%%%%%%%%%%%%%%%%%%%

\section{习题答案}

\textbf{习题\ref{ex:2.1} 解答:}

\enum
\item[(1)] 逆序数为$0 + 1 + 4 + 1 + 2 + 0 (\text{分别为1, 3, 7, 4, 6, 2的逆序数})= 8$。

\item[(2)] 逆序数为4。
\end{list}

\vspace{1.5em}

\textbf{习题\ref{ex:2.2} 解答:}

可以用数学归纳法证明。

也可以直接证明:如果$1$不在排列的第一位,则通过一次对换把$1$换到第一位。如果$2$不在新得到的排列的第二位,则通过一次对换把$2$换到第二位……如此便可以可经过不多于$n$次对换,变
成$(123\cdots n)$。

\vspace{1.5em}

\textbf{习题\ref{ex:2.3} 解答:}

任意一个$1,2,\cdots,n$的排列,可经过不多于$n$次有限次对换,变成自然排列$(123\cdots n)$。每次对换,正序数增加(减少)的个数恰好等于逆序数减少(增加)的个数,所以逆序数和正序数之和等于自然排列$(123\cdots n)$的正序数(其逆序数为0),等于$\frac{n(n-1)}{2}$。

\vspace{1.5em}

\textbf{习题\ref{ex:2.4} 解答:}

因为排列$a_1,\cdots,a_n$的逆序数为$k$,所以排列$a_n,\cdots,a_1$的正序数为$k$。由上一题知数$1,2,\cdots,n$组成的任意一个排列中,逆序数和正序数之和等于$\frac{n(n-1)}{2}$,所以$a_n,\cdots,a_1$ 的逆序数便等于$\frac{n(n-1)}{2}-k$。

\vspace{1.5em}

\textbf{习题\ref{ex:2.5} 解答:}

变为原行列式值乘以$(-1)^n$。

从行列式定义式即可以看出新的行列式定义式的每一个求和项都是原行列式的相应的项的$(-1)^n$倍。也可以利用行列式的性质,把新行列式每行都提出一个$(-1)$,总共提出$n$个$(-1)$,即变为原行列式乘以$(-1)^n$。

\vspace{1.5em}

\textbf{习题\ref{ex:2.6} 解答:}

设$D' = \det ((-1)^{i+j}a_{ij})_{n\times n}$,那么根据行列式定义,有
\begin{eqnarray*}
D' & = & \sum\limits_{i_1i_2\cdots i_n\in P_n} (-1)^{\tau(i_1i_2\cdots i_n)} ((-1)^{1+i_1}a_{1i_1}) ((-1)^{2+i_2}a_{2i_2}) \cdots ((-1)^{n+i_n}a_{ni_n}) \\
& = & \sum\limits_{i_1i_2\cdots i_n\in P_n} (-1)^{\tau(i_1i_2\cdots i_n)} (-1)^{1+i_1+2+i_2+\cdots+n+i_n} a_{1i_1} a_{2i_2} \cdots a_{ni_n} \\
& = & \sum\limits_{i_1i_2\cdots i_n\in P_n} (-1)^{\tau(i_1i_2\cdots i_n)} ((-1)^{1+2+\cdots+n})^2 a_{1i_1} a_{2i_2} \cdots a_{ni_n} \\
& = &  \sum\limits_{i_1i_2\cdots i_n\in P_n} (-1)^{\tau(i_1i_2\cdots i_n)} a_{1i_1} a_{2i_2} \cdots a_{ni_n} \\
& = & D
\end{eqnarray*}
所以行列式值不变。

\vspace{1.5em}

\textbf{习题\ref{ex:2.7} 解答:}

直接通过行列式的定义做即可。第(1)小题:
$$\begin{vmatrix} 0.7 & 0.3 \\ 0.2 & 0.8 \end{vmatrix} = 0.7\times 0.8 - 0.3\times 0.2 = 0.56 - 0.06 = 0.5.$$
其他小题的做法类似,答案分别为$1, ad-bc.$

\vspace{1.5em}

\textbf{习题\ref{ex:2.8} 解答:}

答案都是1,通过行或列的初等变换把行列式化成容易计算的形式。第(1)小题:
$$\begin{vmatrix} 1 & 4 & 4 \\ -1 & -3 & -2 \\ 1 & 3 & 3 \end{vmatrix} = \begin{vmatrix} 1 & 4 & 4 \\ 0 & 1 & 2 \\ 0 & -1 & -1 \end{vmatrix} = \begin{vmatrix} 1 & 4 & 4 \\ 0 & 1 & 2 \\ 0 & 0 & 1 \end{vmatrix} = 1.$$
其他小题的做法类似。直接通过行列式的定义做也可以。

\vspace{1.5em}

\textbf{习题\ref{ex:2.9} 解答:}

答案都是1,通过行或列的初等变换把行列式化成容易计算的形式,与上一题做法相似。

\vspace{1.5em}

\textbf{习题\ref{ex:2.10} 解答:}

答案都是1,要注意利用例题\ref{eg:block_matrix}的结论进行化简计算。

\vspace{1.5em}

\textbf{习题\ref{ex:2.11} 解答:}

\begin{eqnarray*}
& & \begin{vmatrix} 103 & 100 & 204 \\ 199 & 200 & 395 \\ 301 & 300 & 600 \end{vmatrix}
=\begin{vmatrix} 100+3 & 100 & 200+4 \\ 200-1 & 200 & 400-5 \\ 300+1 & 300 & 600+0 \end{vmatrix} = \begin{vmatrix} 3 & 100 & 4 \\ -1 & 200 & -5 \\ 1 & 300 & 0 \end{vmatrix} \\
& = & 100\begin{vmatrix} 3 & 1 & 4 \\ -1 & 2 & -5 \\ 1 & 3 & 0 \end{vmatrix} = 2000
\end{eqnarray*}

\vspace{1.5em}

\textbf{习题\ref{ex:2.12} 解答:}

\enum
\item[(1)] $\begin{vmatrix} 1 & 1 & 1 \\ 2 & 3 & 4 \\ 4 & 9 & 16 \end{vmatrix} = (4-2)\times(4-3)\times(3-2) = 2$

\item[(2)] $\begin{vmatrix} 1 & 2 & 4 & 8 \\ -1 & -3 & -9 & -27 \\ 1 & 4 & 16 & 64 \\ -1 & 5 & -25 & 125 \end{vmatrix} = \det\begin{vmatrix} 1 & 2 & 4 & 8 \\ 1 & 3 & 9 & 27 \\ 1 & 4 & 16 & 64 \\ 1 & -5 & 25 & -125 \end{vmatrix} \\ = (-5-4)\times(-5-3)\times(-5-2)\times(4-3) \times(4-2)\times(3-2)= -1008$

\item[(3)] 设$v_i = \begin{bmatrix} a_i \\ a_i^2\\ \vdots\\ a_i^n\end{bmatrix}$,那么
$$\begin{vmatrix} a_1 + a_2 & a_2 + a_3 & \cdots & a_n + a_1 \\ a_1^2 + a_2^2 & a_2^2 + a_3^2 & \cdots & a_n^2 + a_1^2 \\ \vdots & \vdots & \ddots & \vdots \\ a_1^n + a_2^n & a_2^n + a_3^n & \cdots & a_n^n + a_1^n \end{vmatrix} = \begin{vmatrix} v_1 + v_2, v_2 + v_3, \cdots, v_n + v_1 \end{vmatrix}.$$
利用行列式对列的线性性,展开上式,发现除了 $\begin{vmatrix} v_1, v_2, \cdots, v_n\end{vmatrix}$ 和 $\begin{vmatrix} v_2, v_3, \cdots, v_n, v_1\end{vmatrix},$ 其他项都有至少一对相同的列,所以
\begin{eqnarray*}
& & \begin{vmatrix} a_1 + a_2 & a_2 + a_3 & \cdots & a_n + a_1 \\ a_1^2 + a_2^2 & a_2^2 + a_3^2 & \cdots & a_n^2 + a_1^2 \\ \vdots & \vdots & \ddots & \vdots \\ a_1^n + a_2^n & a_2^n + a_3^n & \cdots & a_n^n + a_1^n \end{vmatrix} \\
& = & \begin{vmatrix} v_1, v_2, \cdots, v_n\end{vmatrix} + \begin{vmatrix} v_2, v_3, \cdots, v_n, v_1\end{vmatrix} \\
& = & (1 + (-1)^{n-1}) \cdot \begin{vmatrix} a_1 & a_2& \cdots & a_n \\ a_1^2 & a_2^2 & \cdots & a_n^2 \\ \vdots & \vdots & \ddots & \vdots \\ a_1^n & a_2^n & \cdots & a_n^n \end{vmatrix} \\
& = & (1 + (-1)^{n-1}) \cdot \prod\limits_{1\leqslant i < j \leqslant n}(a_j - a_i)
\end{eqnarray*}

\end{list}

\vspace{1.5em}

\textbf{习题\ref{ex:2.13} 解答:}

根据范德蒙行列式的结果,
$$\begin{vmatrix} 1 & 1 & 1 & 1 \\ x & a & b & c \\ x^2 & a^2 & b^2 & c^2 \\ x^3 & a^3 & b^3 & c^3 \end{vmatrix} = (x-a)(x-b)(x-c)(a-b)(a-c)(b-c).$$

在第一列展开的话,$\begin{vmatrix} 1 & 1 & 1 \\ a & b & c \\ a^3 & b^3 & c^3 \end{vmatrix}$和$-\begin{vmatrix} 1 & 1 & 1 \\ a^2 & b^2 & c^2 \\ a^3 & b^3 & c^3 \end{vmatrix}$分别为第(1)小题中求到的行列式表达式中$x^2$和$x$的系数,所以
\begin{eqnarray*}
\begin{vmatrix} 1 & 1 & 1 \\ a & b & c \\ a^3 & b^3 & c^3 \end{vmatrix} & = & -(a+b+c)(a-b)(a-c)(b-c) \\
\begin{vmatrix} 1 & 1 & 1 \\ a^2 & b^2 & c^2 \\ a^3 & b^3 & c^3 \end{vmatrix} & = & -(ab+ac+bc)(a-b)(a-c)(b-c)
\end{eqnarray*}

\vspace{1.5em}

\textbf{习题\ref{ex:2.14} 解答:}

设$n \geqslant 3$,将$F_n$按第一列展开,很容易得到递推公式:
\begin{eqnarray*}
F_n & = & \begin{vmatrix} 1 & 1 & & & \\ -1 & 1 & \ddots & & \\ & \ddots & \ddots & \ddots & \\ & & \ddots & \ddots & 1 \\ & & & -1 & 1 \end{vmatrix}_{n} \\
& = & (-1)^{1+1}\begin{vmatrix} 1 & 1 & & & \\ -1 & 1 & \ddots & & \\ & \ddots & \ddots & \ddots & \\ & & \ddots & \ddots & 1 \\ & & & -1 & 1 \end{vmatrix}_{n-1} + (-1)^{2+1}\cdot(-1)\begin{vmatrix} 1 & 0 & & & & & \\ -1 & 1 & 1 & & & \\ 0 & -1 & 1 & \ddots & & \\ & & \ddots & \ddots & \ddots & \\ & & & \ddots & \ddots & 1 \\ & & & & -1 & 1 \end{vmatrix}_{n-1} \\
& = & F_{n-1} + (-1)^{2+1}\cdot(-1)\cdot(-1)^{1+1}\begin{vmatrix} 1 & 1 & & & \\ -1 & 1 & \ddots & & \\ & \ddots & \ddots & \ddots & \\ & & \ddots & \ddots & 1 \\ & & & -1 & 1 \end{vmatrix}_{n-2} \\
& = & F_{n-1} + F_{n-2}
\end{eqnarray*}
剩下的只要验证$n = 1, 2$的特殊情形了。

\vspace{1.5em}

\textbf{习题\ref{ex:2.15} 解答:}

\enum
\item[(1)] $a^{2} b^{2} - 2 a b^{3} + b^{4} -  a^{2} x^{2} + 2 a b x^{2} -  b^{2} x^{2}$

\item[(2)] $-x^{12} + 3x^8 - 3x^4 + 1$
\end{list}

\vspace{1.5em}

\textbf{习题\ref{ex:2.16} 解答:}

\enum
\item[(1)]
\begin{eqnarray*}
\begin{vmatrix} a^2 & ab & b^2 \\ 2a & a + b & 2b \\ 1 & 1 & 1 \end{vmatrix} & = & \begin{vmatrix} a^2 - 2ab + b^2 & ab & b^2 \\ 0 & a + b & 2b \\ 0 & 1 & 1 \end{vmatrix} = (a^2 - 2ab + b^2) \begin{vmatrix} a + b & 2b \\ 1 & 1 \end{vmatrix} \\
& = & (a - b)^3
\end{eqnarray*}

\item[(2)]
\begin{eqnarray*}
& & \begin{vmatrix} a_1 + b_1x & a_1x + b_1 & c_1 \\ a_2 + b_2x & a_2x + b_2 & c_2 \\ a_3 + b_3x & a_3x + b_3 & c_3 \end{vmatrix} = \begin{vmatrix} a_1 + b_1x & (b_1 - a_1)(1 - x) & c_1 \\ a_2 + b_2x & (b_2 - a_2)(1 - x) & c_2 \\ a_3 + b_3x & (b_3 - a_3)(1 - x) & c_3 \end{vmatrix} \\
& = & (1 - x)\cdot\begin{vmatrix} a_1 + b_1x & b_1 - a_1 & c_1 \\ a_2 + b_2x & b_2 - a_2 & c_2 \\ a_3 + b_3x & b_3 - a_3 & c_3 \end{vmatrix} = (1 - x)\cdot\begin{vmatrix} b_1(1+x) & b_1 - a_1 & c_1 \\ b_2(1+x) & b_2 - a_2 & c_2 \\ b_3(1+x) & b_3 - a_3 & c_3 \end{vmatrix} \\
& = & (1 - x^2)\cdot\begin{vmatrix} b_1 & b_1 - a_1 & c_1 \\ b_2 & b_2 - a_2 & c_2 \\ b_3 & b_3 - a_3 & c_3 \end{vmatrix} = (1 - x^2)\cdot\begin{vmatrix} b_1 & -a_1 & c_1 \\ b_2 & -a_2 & c_2 \\ b_3 & -a_3 & c_3 \end{vmatrix}\\
& = & (1 - x^2)\begin{vmatrix} a_1 & b_1 & c_1 \\ a_2 & b_2 & c_2 \\ a_3 & b_3 & c_3 \end{vmatrix}
\end{eqnarray*}

\item[(3)] 按最后一列展开。行列式降阶之后继续按最后一列展开,直到阶数降到2,能够直接计算。

\item[(4)] 按最后一列展开,利用数学归纳法证明:设
$D_n = \begin{vmatrix} \cos\theta & 1 & & & \\ 1 & 2\cos\theta & 1 & & \\ & \ddots & \ddots & \ddots & \\ & & 1 & 2\cos\theta & 1 \\ & & & 1 & 2\cos\theta \end{vmatrix}_n = \cos n\theta$, 那么有
\begin{align*}
    D_{n+1} & = (-1)^{n+1+n+1} \cdot 2\cos\theta \cdot \begin{vmatrix} \cos\theta & 1 & & & \\ 1 & 2\cos\theta & 1 & & \\ & \ddots & \ddots & \ddots & \\ & & 1 & 2\cos\theta & 1 \\ & & & 1 & 2\cos\theta \end{vmatrix}_{n} - (-1)^{n+1+n}\begin{vmatrix} \cos\theta & 1 & & & \\ 1 & 2\cos\theta & 1 & & \\ & \ddots & \ddots & \ddots & \\ & & 1 & 2\cos\theta & 1 \\ & & & & 1 \end{vmatrix}_{n} \\
    & = 2\cos\theta \cdot D_{n} + \quad \begin{vmatrix} \cos\theta & 1 & & & \\ 1 & 2\cos\theta & 1 & & \\ & \ddots & \ddots & \ddots & \\ & & 1 & 2\cos\theta & 1 \\ & & & 1 & 2\cos\theta \end{vmatrix}_{n-1} = 2\cos\theta \cdot D_{n} - D_{n-1} \\
    & = 2\cos\theta \cos n\theta - \cos (n-1)\theta \\
    & = 2\cos\theta \cos n\theta - (\cos n\theta \cos\theta + \sin n\theta \sin\theta) \\ 
    & = \cos\theta \cos n\theta - \sin n\theta \sin\theta \\
    & = \cos (n+1)\theta
\end{align*}

\item[(5)] 利用行列式定义式,发现只有$(-1)^{\frac{n(n-1)}{2}}a_{1n}a_{2,n-1}\cdots a_{n1}$项是非零的。

%\item[(6)] 将最后一行乘以$-1$加到前$n-1$行,
%$$\text{原式左边} = \begin{vmatrix} a_1 & 0 & 0 & \cdots & -a_n \\ 0 & 0a_2 & 0 & \cdots & -a_n \\ \vdots & & \ddots & & \vdots \\ \vdots &  &  & a_{n-1} & -a_n \\ 1 & \cdots & \cdots & 1 & 1+a_n \end{vmatrix},$$
%再把第$i$行($i = 1, 2, \cdots, n-1$)乘以$-\frac{1}{a_i}$加到最后一行,得$$\begin{vmatrix} a_1 & 0 & 0 & \cdots & -a_n \\ 0 & 0a_2 & 0 & \cdots & -a_n \\ \vdots & & \ddots & & \vdots \\ \vdots &  &  & a_{n-1} & a_n \\ 0 & \cdots & \cdots & 0 & 1+a_n+\frac{a_n}{a_1}+\cdots+\frac{a_n}{a_{n-1}} \end{vmatrix} = \text{原式右边}.$$
\end{list}

\vspace{1.5em}

\textbf{习题\ref{ex:2.17} 解答:}

\begin{eqnarray*}
D' & = & \begin{vmatrix} a_{11} & a_{12}+x_2 & \cdots & a_{1n}+x_n \\ a_{21} & a_{22}+x_2 & \cdots & a_{2n}+x_n \\ \vdots & \vdots & & \vdots \\ a_{n1} & a_{n2}+x_2 & \cdots & a_{nn}+x_n \end{vmatrix} + \begin{vmatrix} x_1 & a_{12}+x_2 & \cdots & a_{1n}+x_n \\ x_1 & a_{22}+x_2 & \cdots & a_{2n}+x_n \\ \vdots & \vdots & & \vdots \\ x_1 & a_{n2}+x_2 & \cdots & a_{nn}+x_n \end{vmatrix} \\
& = & \begin{vmatrix} a_{11} & a_{12}+x_2 & \cdots & a_{1n}+x_n \\ a_{21} & a_{22}+x_2 & \cdots & a_{2n}+x_n \\ \vdots & \vdots & & \vdots \\ a_{n1} & a_{n2}+x_2 & \cdots & a_{nn}+x_n \end{vmatrix} + x_1 \begin{vmatrix} 1 & a_{12}+x_2 & \cdots & a_{1n}+x_n \\ 1 & a_{22}+x_2 & \cdots & a_{2n}+x_n \\ \vdots & \vdots & & \vdots \\ 1 & a_{n2}+x_2 & \cdots & a_{nn}+x_n \end{vmatrix} \\
& = & \begin{vmatrix} a_{11} & a_{12}+x_2 & \cdots & a_{1n}+x_n \\ a_{21} & a_{22}+x_2 & \cdots & a_{2n}+x_n \\ \vdots & \vdots & & \vdots \\ a_{n1} & a_{n2}+x_2 & \cdots & a_{nn}+x_n \end{vmatrix} + x_1 \sum\limits_{i=1}^n A_{i1} \\
& = & \begin{vmatrix} a_{11} & a_{12} & \cdots & a_{1n}+x_n \\ a_{21} & a_{22} & \cdots & a_{2n}+x_n \\ \vdots & \vdots & & \vdots \\ a_{n1} & a_{n2} & \cdots & a_{nn}+x_n \end{vmatrix} + \begin{vmatrix} a_{11} & x_2 & \cdots & a_{1n}+x_n \\ a_{21} & x_2 & \cdots & a_{2n}+x_n \\ \vdots & \vdots & & \vdots \\ a_{n1} & x_2 & \cdots & a_{nn}+x_n \end{vmatrix} + x_1 \sum\limits_{i=1}^n A_{i1} \\
& \cdots & \cdots\cdots\cdots \\
& = & D + x_n \sum\limits_{i=1}^n A_{in} + \cdots + x_1 \sum\limits_{i=1}^n A_{i1} \\
& = & D + \sum\limits_{j=1}^n x_j \sum\limits_{i=1}^n A_{ij}
\end{eqnarray*}

\vspace{1.5em}

\textbf{习题\ref{ex:2.18} 解答:}

先验证(1)的行列式为$-4$,(2)的行列式为$4$,(3)的行列式为$-4$,都满足Cramer法则的适用条件,然后再应用Cramer法则即可。

\enum
\item[(1)] $D_1 = \begin{vmatrix} -2 & 0 & 0 \\ 1 & -2 & 6 \\ -1 & -3 & 7 \end{vmatrix} = -8,$ 类似可算得$D_2 = -23, D_3 = -15$。所以解为
$$\begin{cases} x=2 \\ y=\frac{23}{4} \\ z=\frac{15}{4} \end{cases}$$
\item[(2)] 类似第(1)小题可求得解为
$$\begin{cases} x=2 \\ y=\frac12 \\ z=\frac52 \end{cases}$$
\item[(3)] 类似第(1)小题可求得解为
$$\begin{cases} x_1=3 \\ x_2=\frac32 \\ x_3=2 \\ x_4=4 \end{cases}$$
\end{list}

\vspace{1.5em}

\textbf{习题\ref{ex:2.19} 解答:}

$\begin{vmatrix} 1 & x & y & x^2 + y^2 \\ 1 & x_1 & y_1 & x_1^2 + y_1^2 \\ 1 & x_2 & y_2 & x_2^2 + y_2^2 \\ 1 & x_3 & y_3 & x_3^2 + y_3^2 \end{vmatrix}$按第一行展开的话,因为这三点不共线,所以$(x^2+y^2)$的系数$-\begin{vmatrix} 1 & x_1 & y_1 \\ 1 & x_2 & y_2 \\ 1 & x_3 & y_3 \end{vmatrix}$不为0,因此可以看出这是一个圆方程。把$(x, y)$分别用这三个点$(x_1, y_1), (x_2, y_2), (x_3, y_3)$代入,容易看出左边行列式都是0,等式成立,也就是说这个圆过这3个点。

\vspace{1.5em}

\textbf{习题\ref{ex:2.20} 解答:}

若$L_1, L_2, L_3$交于一点,则线性方程组
$$\left\{ \begin{array}{rcl} \alpha x + \beta y & = & -\gamma \\ \gamma x + \alpha y & = & -\beta \\ \beta x + \gamma y & = & -\alpha \end{array}\right. \qquad (\ast)$$
有唯一解。因此它的增广矩阵(方阵)的行列式必须为$0,$ 否则化为阶梯形矩阵,它会有主元在最后一列,导致无解。也就是说我们必须有
\begin{eqnarray*}
\begin{vmatrix}
\alpha & \beta & -\gamma \\ \gamma & \alpha & -\beta \\ \beta & \gamma & -\alpha \end{vmatrix} & = & \alpha^3 + \beta^3 + \gamma^3 - 3\alpha\beta\gamma \\
& = & (\alpha + \beta + \gamma)(\alpha^2 + \beta^2 + \gamma^2 - \alpha\beta - \beta\gamma - \alpha\gamma) \\
& = & \dfrac{1}{2}(\alpha + \beta + \gamma)((\alpha - \beta)^2 + (\gamma - \beta)^2 + (\gamma - \alpha)^2 + \alpha^2 + \beta^2 + \gamma^2)
\end{eqnarray*}
由于题设$L_1, L_2, L_3$是三条不同的直线,所以
$$((\alpha - \beta)^2 + (\gamma - \beta)^2 + (\gamma - \alpha)^2 + \alpha^2 + \beta^2 + \gamma^2)\neq 0,$$
所以必须有
$$\alpha + \beta + \gamma = 0.$$
反过来,如果$\alpha + \beta + \gamma = 0,$ 把线性方程组$(\ast)$的前两个方程加到第三个方程上,有
$$\left\{ \begin{array}{rcl} \alpha x + \beta y & = & -\gamma \\ \gamma x + \alpha y & = & -\beta \\ (\alpha + \beta + \gamma) x + (\alpha + \beta + \gamma) y & = & -(\alpha + \beta + \gamma) \end{array}\right.$$
即
$$\left\{ \begin{array}{rcl} \alpha x + \beta y & = & -\gamma \\ \gamma x + \alpha y & = & -\beta \\ 0 & = & 0 \end{array}\right.$$
注意到$\alpha,\beta$不能同时为$0,$ 那么系数方阵行列式为
$$\begin{vmatrix}
\alpha & \beta \\ -(\alpha+\beta) & \alpha \end{vmatrix} = \alpha^2 + \beta^2 + \alpha\beta >0,$$
把线性方程组$(\ast)$便有唯一解。
%线性方程组$(\ast)$可化为
%$$\left\{ \begin{array}{rcl} (\alpha + \beta + \gamma) x + (\alpha + \beta + \gamma) y & = & -(\alpha + \beta + \gamma) \\ \gamma x + \alpha y & = & -\beta \\ \beta x + \gamma y & = & -\alpha \end{array}\right.$$
%如果$\alpha + \beta + \gamma \neq 0,$ 那么线性方程组化为
%$$\left\{ \begin{array}{rcl} x + y & = & -1 \\ \gamma x + \alpha y & = & -\beta \\ \beta x + \gamma y & = & -\alpha \end{array}\right.$$
%进一步可化简为
%$$\left\{ \begin{array}{rcl} x + y & = & -1 \\ (\alpha - \gamma) y & = & \gamma - \beta \\ (\gamma - \beta) y & = & \beta - \alpha \end{array}\right.$$
%由于题设$L_1, L_2, L_3$是三条不同的直线,所以$\alpha - \gamma, \gamma - \beta$至少有一个不等于$0$,否则会有$\alpha = \gamma = \beta,$ $L_1, L_2, L_3$是同一条直线。
%这等价于$\alpha + \beta + \gamma = 0$。

\vspace{1.5em}

\textbf{习题\ref{ex:2.21} 解答:}

$$\begin{vmatrix} A & \alpha \\ \beta^T & c \end{vmatrix} = \begin{vmatrix} A & \alpha+0 \\ \beta^T & b+(c-b) \end{vmatrix} = \begin{vmatrix} A & \alpha \\ \beta^T & b \end{vmatrix} + \begin{vmatrix} A & 0 \\ \beta^T & c-b \end{vmatrix} = (c-b)|A|$$

\vspace{1.5em}

\textbf{习题\ref{ex:2.22} 解答:}

\enum
\item[(1)] 由行列式定义,我们有
$$\begin{vmatrix} 1 & 1 & \cdots & 1 \\ 1 & 1 & \cdots & 1 \\ \vdots & \vdots &  & \vdots \\ 1 & 1 & \cdots & 1 \end{vmatrix}_{n\times n} = \sum\limits_{i_1i_2\cdots i_n\in P_n} (-1)^{\tau(i_1i_2\cdots i_n)}a_{1i_1}a_{2i_2}\cdots a_{ni_n},$$
其中$a_{ji_1} = a_{ji_2} = \cdots = a_{ji_n} = 1 (j = 1,2,\cdots,n)$。所以
$$\begin{vmatrix} 1 & 1 & \cdots & 1 \\ 1 & 1 & \cdots & 1 \\ \vdots & \vdots &  & \vdots \\ 1 & 1 & \cdots & 1 \end{vmatrix}_{n\times n} = \sum\limits_{i_1i_2\cdots i_n\in P_n} (-1)^{\tau(i_1i_2\cdots i_n)}.$$
由于原行列式为$0$,所以有$\sum\limits_{i_1i_2\cdots i_n\in P_n} (-1)^{\tau(i_1i_2\cdots i_n)} = 0$,故知数$1,2,\cdots,n$组成的所有排列中,奇偶排列各占一半。

\item[(2)]
\begin{eqnarray*}
& & \sum\limits_{i_1i_2\cdots i_n\in P_n} \begin{vmatrix} a_{1i_1} & a_{1i_2} & \cdots & a_{1i_n} \\ a_{2i_1} & a_{2i_2} & \cdots & a_{2i_n} \\ \vdots & \vdots &  & \vdots \\ a_{ni_1} & a_{ni_2} & \cdots & a_{ni_n} \end{vmatrix} = \sum\limits_{i_1i_2\cdots i_n\in P_n} (-1)^{\tau(i_1i_2\cdots i_n)}  \begin{vmatrix} a_{11} & a_{12} & \cdots & a_{1n} \\ a_{21} & a_{22} & \cdots & a_{2n} \\ \vdots & \vdots &  & \vdots \\ a_{n1} & a_{n2} & \cdots & a_{nn} \end{vmatrix} \\
& = & \begin{vmatrix} a_{11} & a_{12} & \cdots & a_{1n} \\ a_{21} & a_{22} & \cdots & a_{2n} \\ \vdots & \vdots &  & \vdots \\ a_{n1} & a_{n2} & \cdots & a_{nn} \end{vmatrix} \cdot (\sum\limits_{i_1i_2\cdots i_n\in P_n} (-1)^{\tau(i_1i_2\cdots i_n)}) = \begin{vmatrix} a_{11} & a_{12} & \cdots & a_{1n} \\ a_{21} & a_{22} & \cdots & a_{2n} \\ \vdots & \vdots &  & \vdots \\ a_{n1} & a_{n2} & \cdots & a_{nn} \end{vmatrix} \cdot 0 \\
& = & 0
\end{eqnarray*}
\end{list}

\vspace{1.5em}

\textbf{习题\ref{ex:2.23} 解答:}

对行列式的阶数$n$用归纳法。$n=2$时,由于$a_{11},a_{22}$大于0,$a_{12},a_{21}$小于0,所以$D_2 = a_{11}a_{22} - a_{12}a_{21} > 0$成立。假设题目的结论对所有满足题设条件的$n-1$阶行列式行列式都成立,那么对$D_n$,把第一列乘以$(-\frac{a_{1j}}{a_{11}})$加到第$j$列上去($j=2,\cdots,n$),于是
$$D_n = \begin{vmatrix} a_{11} & 0 & \cdots & 0 \\ a_{21} & a_{22}' & \cdots & a_{2n}' \\ \vdots & \vdots & & \vdots \\ a_{n1} & a_{n2}' & \cdots & a_{nn}' \end{vmatrix} = a_{11} \begin{vmatrix}  a_{22}' & \cdots & a_{2n}' \\ \vdots & & \vdots \\ a_{n2}' & \cdots & a_{nn}' \end{vmatrix},$$
其中$a_{ij}' = a_{ij} - a_{i1}\frac{a_{1j}}{a_{11}}$,我们需要证明上面这个$n-1$阶行列式仍然满足题设条件:
\enum
\item[(1)] $a_{ii}' = a_{ii} - a_{i1}\frac{a_{1i}}{a_{11}} = \frac{a_{ii}a_{11} - a_{i1}a_{1i}}{a_{11}} > 0,$
\item[(2)] $a_{ij}' = a_{ij} - a_{i1}\frac{a_{1i}}{a_{11}} < 0,$
\item[(3)] $\sum\limits_{i=2}^n a_{ij}' = \sum\limits_{i=2}^n a_{ij} - \sum\limits_{i=2}^n a_{i1}\frac{a_{1j}}{a_{11}} > -a_{1j} - a_{1j}\sum\limits_{i=2}^n \frac{a_{i1}}{a_{11}} = -a_{1j}(a_{11} + \sum\limits_{i=2}^n a_{i1}) / a_{11} > 0.$
\end{list}
所以由归纳假设知
$$\begin{vmatrix}  a_{22}' & \cdots & a_{2n}' \\ \vdots & & \vdots \\ a_{n2}' & \cdots & a_{nn}' \end{vmatrix} > 0,$$
所以有$D_n > 0$。

\vspace{1.5em}

\textbf{习题\ref{ex:2.24} 解答:}

\enum
\item[(1)] 设$A = (a_{ij})_{n\times n}$,那么
$$\det A = \sum\limits_{i_1i_2\cdots i_n\in P_n} (-1)^{\tau(i_1i_2\cdots i_n)}a_{1i_1}a_{2i_2}\cdots a_{ni_n}.$$
所以(以下$|\cdot|$表示绝对值)
\begin{eqnarray*}
|\det A| & = & |\sum\limits_{i_1i_2\cdots i_n\in P_n} (-1)^{\tau(i_1i_2\cdots i_n)}a_{1i_1}a_{2i_2}\cdots a_{ni_n}| \\
& \leqslant & \sum\limits_{i_1i_2\cdots i_n\in P_n} |(-1)^{\tau(i_1i_2\cdots i_n)}||a_{1i_1}||a_{2i_2}|\cdots |a_{ni_n}| \\
& = & \sum\limits_{i_1i_2\cdots i_n\in P_n}1 = n!
\end{eqnarray*}
\item[(2)] 把第一列中元素为$-1$的行都乘以$-1$后,第一行的$-1$倍加到各行上,有
$$\det A = \pm \begin{vmatrix} 1 & \ast & \cdots & \ast \\ 0 & & & \\ \vdots & & \makebox(0,0){\huge $A_{n-1}$} & \\ 0 & & & \end{vmatrix} = \det A_{n-1},$$
其中$\ast$表示元素值未知,$\det A_{n-1}$是一个元素由$\pm2,0$组成的行列式,所以
$$\det A_{n-1} = 2^{n-1}\det B_{n-1},$$
$\det B_{n-1}$是$\det A_{n-1}$每行除以2所得的行列式(若$\det A_{n-1}$某行全为$0$,此时$\det A_{n-1} = 0$,也可被看成被$2^{n-1}$整除)。所以$\det A$被$2^{n-1}$整除。
\item[(3)] 对行列式的阶数$n$用归纳法。根据行列式定义式很容易看出当$n=2$时,$|\det A| \leqslant 2$。假设当阶数等于$n$时,所有满足题设的行列式的绝对值都小于等于$(n-1)!\cdot2(n-1)$,那么任取一个所有元素为$1$或者$-1$的$n+1$阶行列式$\det A$,把他按第一行展开,有$\det A = \sum\limits_{i=1}^{n+1} a_{1i}A_{1i}$,其中$A_{1i}$是$a_{1i}$的代数余子式,是一个所有元素为$1$或者$-1$的$n$ 阶行列式乘以$(-1)^{1+i}$。所以依据归纳假设有
\begin{eqnarray*}
|\det A| & = & \left|\sum\limits_{i=1}^{n+1} a_{1i}A_{1i}\right| \leqslant \sum\limits_{i=1}^{n+1} \left|A_{1i}\right| \leqslant \sum\limits_{i=1}^{n+1} (n-1)! \cdot 2(n-1) \\
& = & (n+1)\cdot\left[(n-1)! \cdot 2(n-1)\right] \\
& = & (n-1)! \cdot 2(n^2-1) \\
& < & (n-1)! \cdot 2n^2 = n!\cdot 2n.
\end{eqnarray*}
所以原命题成立。
\item[(4)] 类似本题第(3)小题,我们可以用归纳法证明,唯一的区别是还要对$n = 3$进行验证。以下设$A$为一个$3$阶行列式。我们验证必定有$|\det A| \leqslant 4$。由本题第(1)小题知,$|\det A| \leqslant 3! = 6$,又由本题第(2)小题知$\det A$被$2^{3-1} = 4$整除,所以$|\det A|$不能取$5$或$6$,所以必须有$|\det A| \leqslant 4$。
\end{list}

%%%%%%%%%%%%%%%%%%%%%%%%%%%%%%%%%%%%%%%%%%%%%%%%%%%%%%%%%%%%%%%%%%%%%%%%%%%%%%%%%%%%%%%%%%%%
%%%%%%%%%%%%%%%%%%%%%%%%%%%%%%%%%%%%%%%%%%%%%%%%%%%%%%%%%%%%%%%%%%%%%%%%%%%%%%%%%%%%%%%%%%%%

\chapter{矩阵}

\section{知识点解析}

\begin{Def}
由$mn$个实数排成行列的矩形数表, 用圆(或方)括号括起来,即
$$A = \begin{bmatrix}
a_{11} & a_{12} & \cdots & a_{1n} \\ a_{21} & a_{22} & \cdots & a_{2n} \\ \vdots & \vdots & \vdots & \vdots \\ a_{m1} & a_{m2} & \cdots & a_{mn}
\end{bmatrix}$$
称为$m\times n$型的矩阵,简记为$A = (a_{ij})_{m\times n}$,其中横排称为矩阵的行,竖排称为矩阵的列。$a_{ij}$称为矩阵的元素,其第一下标表示所在的行数,第二下标表示其所在的列数。全体$m\times n$型的矩阵组成的集合,记为$M_{m\times n}(\mathbb{R})$。
\end{Def}

\begin{rmk}
几类特殊矩阵:
\enum
\item[(1)] 方阵:行数与列数都等于$n$的矩阵称为$n$阶方阵。全体$n$阶方阵组成的集合,记为$M_n(\mathbb{R})$。
\item[(2)] 行矩阵,列矩阵:

行数$m=1$时,$(a_1,a_2,\cdots,a_n)$称为行矩阵(或行向量)。

列数$n=1$时,$\begin{bmatrix} a_1 \\ a_2 \\ \vdots \\ a_n \end{bmatrix}$称为列矩阵(或列向量)。
\item[(3)] 零矩阵:元素全为零的矩阵称为零矩阵,记为$O$或$0_{m\times n}$。
\item[(4)] 负矩阵:$-A = -(a_{ij})_{m\times n} = (-a_{ij})_{m\times n}$
\item[(5)] 三角形矩阵:

上三角矩阵:主对角线下方的元素全为零的方阵称为上三角形矩阵。

下三角矩阵:主对角线上方的元素全为零的方阵称为下三角形矩阵。
\item[(6)] 对角矩阵:除主对角线上元素外,全为零的方阵。
$$\begin{bmatrix} a_{11} & & \\ & \ddots & \\ & & a_{nn} \end{bmatrix} =: diag(a_{11},\cdots,a_{nn})$$
特别地有:
\begin{eqnarray*}
\text{纯量矩阵} & : & \begin{bmatrix} c & & \\ & \ddots & \\ & & c \end{bmatrix} = diag(c,\cdots,c) \\
\text{单位矩阵} & : & \begin{bmatrix} 1 & & \\ & \ddots & \\ & & 1 \end{bmatrix} = diag(1,\cdots,1) =: I_n
\end{eqnarray*}
\item[(7)] 阶梯形矩阵(零行在最下方,非零行左端的零严格增加)。

简化阶梯形矩阵(阶梯形矩阵,主元素$=1$,主元所在列的其它元均为$0$)。
\end{list}
\end{rmk}

\begin{Def}\

\enum
\item[(1)] 如果两个矩阵行数相同且列数相同,则称这两个矩阵是同型的;
\item[(2)] 两个矩阵同型且对应元素相等,则称这两个矩阵是相等的。
\end{list}
\end{Def}

\begin{Def}[矩阵的线性运算]\

\enum
\item[(1)] 加法:同型矩阵对应分量相加。
$$A = (a_{ij})_{m\times n}, B = (b_{ij})_{m\times n} \Rightarrow A + B := (a_{ij} + b_{ij})_{m\times n}$$
\item[(2)] 减法:同型矩阵对应分量相减。
$$A - B := A + (-B) = (a_{ij} - b_{ij})_{m\times n}$$
\item[(3)] 数量乘法(或数乘):矩阵的每个分量同乘一个数。
$$\text{设 }k\in\mathbb{R}, A = (a_{ij})_{m\times n}, \text{则 } kA = k(a_{ij})_{m\times n} := (ka_{ij})_{m\times n}$$
\end{list}
\end{Def}

\begin{rmk}
由数的乘法可交换,知$kA = (ka_{ij})_{m\times n} = (a_{ij}k)_{m\times n} = Ak$。一般统一都把数$k$乘在矩阵$A$的左边。
\end{rmk}

\begin{prop}[矩阵线性运算(加法与数乘)的运算律]\

加法运算律:
\enum
\item[(1)] 交换律:$A+B = B+A$。
\item[(2)] 结合律:$A+(B+C) = (A+B)+C$。
\item[(3)] 零矩阵:$A+0 = A$。
\item[(4)] 负矩阵:$A+(-A) = 0$。
\end{list}

数乘运算律
\enum
\item[(5)] 单位:$1\cdot A = A$。
\item[(6)] 结合律:$k(lA) = (kl)A$。
\item[(7)] 分配律1:$k(A+B) = kA+kB$。
\item[(8)] 分配律2:$(k+l)A = kA+lA$。
\end{list}
\end{prop}

\begin{Def}[矩阵的乘法]
设$A = (a_{ij})_{m\times s}, B = (b_{ij})_{t\times n}$。如果矩阵$A$的列数等于矩阵$B$的行数,即$s=t$,则$A$与$B$ 可以相乘,记为$AB$。$AB$为一个$m\times n$阶的矩阵,其$(i,j)$位置上的元素是$A$ 的第$i$行的元素与$B$的第$j$列对应位置上的元素乘积之和,即如果记$AB = (c_{ij})_{m\times n}$,那么
$$c_{ij} = a_{i1}b_{1j} + a_{i2}b_{2j} + \cdots + a_{is}b_{sj} = \sum\limits_{k=1}^s a_{ik}b_{kj}, \quad (i=1,\cdots,m; j=1,\cdots,n)$$
\end{Def}

\begin{prop}[矩阵乘法的一些``反常''性质]\

\enum
\item[(1)] 不满足交换律:$AB\neq BA$。
\item[(2)] 存在零因子:
\begin{eqnarray*}
A \neq 0, B \neq 0 & \not\Rightarrow & AB \neq 0; \\
AB = 0 & \not\Rightarrow & A = 0 \text{ 或 } B = 0.
\end{eqnarray*}
\item[(3)] 不满足消去律:$AB = AC \not\Rightarrow B = C$。
\end{list}
\end{prop}

\begin{prop}[矩阵乘法的运算律]\
设$A\in M_{m\times n}(\mathbb{R})$为一个$m\times n$阶矩阵设,且取$B, C$使得下列运算可行,则
\enum
\item[(1)] 零矩阵:$0_{k\times m}A = 0_{k\times n}, A0_{n\times l} = 0_{m\times l}$。
\item[(2)] 单位阵:$I_mA = A, AI_n = A$。
\item[(3)] 数乘:$(kA)B = A(kB) = k(AB)$。
\item[(4)] 左右分配律:$A(B+C) = AB + AC, (B+C)A = BA+CA$。
\item[(5)] 结合律:$A(BC) = (AB)C$。
\end{list}
\end{prop}

\begin{Def}
对于$n$阶方阵$A$,定义$A$的方幂为:$A^k = \underbrace{A\cdots A}_{k \text{ 个}}, k\in \mathbb{Z}$。

另对$A^0$,我们规定$A^0 = I_n$。
\end{Def}

\begin{prop}[矩阵方幂的性质]
设$A$为$n$阶方阵,$k,l$为非负整数,则
\enum
\item[$\bullet$] $A^k A^l = A^{k+l} = A^l A^k$。
\item[$\bullet$] $(A^k)^l = A^{kl}$。
\end{list}
\end{prop}

\begin{rmk}
对同一个$n$阶方阵$A$反复进行线性运算与乘法(方幂)运算,经过合并化简后,可得
$$a_mA^m + a_{m-1}A^{m-1} + \cdots + a_1A + a_0I_n$$
把上式记为$f(A)$,称为关于矩阵$A$的$m$次多项式。

由于$A^k$与$A^l$乘法可交换,对任意多项式$f(x)$与$g(x)$,有矩阵多项式$f(A)$与$g(A)$乘法可交换,即
$$f(A)g(A) = g(A)f(A).$$
\end{rmk}

\begin{Def}
设矩阵$A=(a_{ij})_{m\times n}$,将$A$的行与列互换得到的矩阵称为$A$的转置,记为$A^T$。
\end{Def}

\begin{rmk}
$A^T$为$n\times m$阶矩阵,若记$A^T = (a_{ij}')_{n\times m}$,则有$a_{ij}' = a_{ji}$。
\end{rmk}

\begin{prop}[矩阵转置的运算律]
设矩阵$A,B$和实数$\lambda$使得下述相关运算有定义,则
\enum
\item[(1)] 两次还原:$(A^T)^T = A$。
\item[(2)] 加法相容:$(A+B)^T = A^T + B^T$。
\item[(3)] 数乘相容:$(\lambda A)^T = \lambda A^T$。
\item[(4)] 乘法反序:$(AB)^T = B^T A^T$。
\end{list}
\end{prop}

\begin{Def}
设$A$为方阵。若$A^T = A$,则称$A$为对称阵;若$A^T = -A$,则称$A$为反对称阵。
\end{Def}

\begin{prop}[对称阵与反对称阵的运算律]\

线性运算保持矩阵的(反)对称性,即
\enum
\item[(1)] 如果$A, B$是同阶对称矩阵,则$A+B, kA$也是对称矩阵。
\item[(2)] 如果$A, B$是同阶反对称矩阵,则$A+B, kA$也是反对称矩阵。
\end{list}
而乘法运算不一定保持(反)对称性。
\end{prop}

\begin{thm}
设$A,B\in M_n(\mathbb{R})$,则$\det(AB) = \det(A)\cdot\det(B)$。
\end{thm}

\begin{cor}
设$A_1,\cdots,A_s$为$s$个$n$阶方阵,则
$$\det(A_1\cdots A_s) = \det A_1\cdots\det A_s.$$
\end{cor}

\begin{prop}
方阵的行列式运算与其他运算有如下关系(设$A$为$n$阶方阵):
\enum
\item[$\bullet$] $\det A^T = \det A$。
\item[$\bullet$] $\det (kA) = k^n \det A$。
\end{list}
\end{prop}

\begin{Def}
对一个$m\times n$的矩阵$A$,用若干横线和竖线把$A$的行分成$p$个部分,列分成$q$个部分,整个矩阵$A$分成$pq$个小矩阵,即
$$A = \begin{bmatrix}
A_{11} & A_{12} & \cdots & A_{1q} \\
A_{21} & A_{22} & \cdots & A_{2q} \\
\vdots & \vdots & \ddots & \vdots \\
A_{p1} & A_{p2} & \cdots & A_{pq}
\end{bmatrix}$$
其中每个小矩阵$A_{ij}(1 \leqslant i \leqslant p, 1 \leqslant j \leqslant q)$称为矩阵$A$的子块,$A$也可视为由子块$A_{ij}$构成的$p\times q$阶矩阵,称为分块矩阵。
\end{Def}

\begin{rmk}
矩阵分块的三个原则:
\enum
\item[(1)] 体现原矩阵特点,按需划分。
\item[(2)] 能够把子块看作元素进行运算(除乘法的次序外)。
\item[(3)] 保持原有运算性质。
\end{list}
\end{rmk}

\begin{prop}[分块矩阵的运算]\

\enum
\item[(1)] 分块矩阵的加法:设分块矩阵$A$与$B$的行列数均相同(同型矩阵),且采用同样的分块方法,即
$$A = \begin{bmatrix}
A_{11} & \cdots & A_{1q} \\
\vdots & \ddots & \vdots \\
A_{p1} & \cdots & A_{pq}
\end{bmatrix},
B = \begin{bmatrix}
B_{11} & \cdots & B_{1q} \\
\vdots & \ddots & \vdots \\
B_{p1} & \cdots & B_{pq}
\end{bmatrix},
$$
其中$A_{ij}$与$B_{ij}$的行数和列数均相同(同型矩阵),则
$$A + B = \begin{bmatrix}
A_{11} + B_{11} & \cdots & A_{1q} + B_{1q}\\
\vdots & \ddots & \vdots \\
A_{p1} + B_{p1} & \cdots & A_{pq} + B_{pq}
\end{bmatrix}.
$$
\item[(2)] 分块矩阵的数乘:

设$A = \begin{bmatrix}
A_{11} & \cdots & A_{1q} \\
\vdots & \ddots & \vdots \\
A_{p1} & \cdots & A_{pq}
\end{bmatrix}$(可任意分块),$\lambda$是数,则
$$\lambda A = \begin{bmatrix}
\lambda A_{11} & \cdots & \lambda A_{1q} \\
\vdots & \ddots & \vdots \\
\lambda A_{p1} & \cdots & \lambda A_{pq}
\end{bmatrix}$$
\item[(3)] 分块矩阵的转置:

设$$A = \begin{bmatrix}
A_{11} & \cdots & A_{1q} \\
\vdots & \ddots & \vdots \\
A_{p1} & \cdots & A_{pq}
\end{bmatrix}$$,则分块矩阵A的转置为
$$A^T = \begin{bmatrix}
A_{11}^T & \cdots & A_{p1}^T \\
\vdots & \ddots & \vdots \\
A_{1q}^T & \cdots & A_{pq}^T
\end{bmatrix},$$
即,每一块子矩阵均做转置,且行列下标互换。
\item[(4)] 分块矩阵的乘法:

设$A_{m\times n} B_{n\times l}=A_{m\times l}$,要求
\enum
\item[$\bullet$] $A$的列数$= B$的行数$= n$。
\item[$\bullet$] $A$的列的分法$= B$的行的分法($n$的加法有序分拆)。
\end{list}

$$A = \begin{bmatrix}
A_{11} & \cdots & A_{1s} \\
\vdots & \ddots & \vdots \\
A_{r1} & \cdots & A_{rs}
\end{bmatrix},
B = \begin{bmatrix}
B_{11} & \cdots & B_{1t} \\
\vdots & \ddots & \vdots \\
B_{s1} & \cdots & B_{st}
\end{bmatrix},
C = \begin{bmatrix}
C_{11} & \cdots & C_{1t} \\
\vdots & \ddots & \vdots \\
C_{r1} & \cdots & C_{rt}
\end{bmatrix}.$$
对乘积矩阵$C$,有
\enum
\item[$\bullet$] $C$的行数及行分块法由$A$的行数及行分块法决定;$C$的列数及列分块法由$B$的列数及列分块法决定。
\item[$\bullet$] $C$的每个子块
$$C_{ij} = A_{i1}B_{1j} + A_{i2}B_{2j} +\cdots +  A_{is}B_{sj} = \sum\limits_{k=1}^s  A_{ik}B_{kj}, \quad (1 \leqslant i \leqslant r, 1 \leqslant j \leqslant t)$$
\end{list}
\end{list}
\end{prop}

\begin{rmk}
特殊的分块矩阵:
\enum
\item[$\bullet$] 准对角矩阵:

设$A$为方阵,若
$$A = \begin{bmatrix} A_{11} & & \\ & \ddots & \\ & & A_{nn} \end{bmatrix} = diag(A_{11},\cdots,A_{nn}),$$
其中$A_{11},\cdots,A_{nn}$都是小方阵,则称$A$为准对角矩阵。

准对角矩阵可作为对角矩阵的推广情形,是最简单的一类分块矩阵。

\item[$\bullet$] 准上三角矩阵:

设矩阵$A$的行与列均分为个$n$子块,且
$$A = \begin{bmatrix} A_{11} & A_{12} & \cdots & A_{1n} \\ 0 & A_{22} & \cdots & A_{2n} \\ \vdots & \vdots & \ddots & \vdots \\ 0 & 0 & \cdots & A_{nn} \end{bmatrix},$$
则称$A$为准上三角矩阵。

\item[$\bullet$] 准下三角矩阵:

设矩阵$A$的行与列均分为个$n$子块,且
$$A = \begin{bmatrix} A_{11} & 0 & \cdots & 0 \\ A_{21} & A_{22} & \cdots & 0 \\ \vdots & \vdots & \ddots & \vdots \\ A_{n1} & A_{n2} & \cdots & A_{nn} \end{bmatrix},$$
则称$A$为准下三角矩阵。

一般地,准三角形矩阵不一定是方阵。
\end{list}
\end{rmk}

\begin{thm}
在可运算的条件下,准上(下)三角形矩阵的加法,数乘与乘法仍是准上(下)三角形矩阵。
\end{thm}

\begin{cor}
在可运算的条件下,上(下)三角形矩阵的加法,数乘与乘法仍是上(下)三角形矩阵。
\end{cor}

\begin{Def}
单位矩阵$I$经过一次初等变换所得到的矩阵称为初等矩阵。

初等矩阵有以下三类:
\enum
\item[(1)] 单位矩阵$I$的第$i$行乘以非零数$k$(倍乘行变换)所得矩阵,记为$E_i(k)$,称为倍乘矩阵。它同时也是$I$的第$i$列乘以非零数$k$(倍乘行变换)所得矩阵。
$$I_n \xrightarrow[(kc_i)]{kr_i} \begin{bmatrix} 1 & & & & & & \\ & \ddots & & & & & \\ & & 1 & & & & \\ & & & k & & & \\ & & & & 1 & &  \\ & & & & & \ddots & \\ & & & & & & 1 \end{bmatrix} =: E_i(k)$$
\item[(2)] 将单位矩阵$I$第$i$行的$k$倍加到第$j$行,所得矩阵记为$E_{i,j}(k)$,称为倍加矩阵。它也是将$I$第$j$列的$k$倍加到第$i$列,所得的矩阵。
$$\begin{bmatrix} 1 & & & & & & \\ & \ddots & & & & & \\ & & 1 & & & & \\ & & \vdots & \ddots & & & \\ & & k & \cdots & 1 & &  \\ & & & & & \ddots & \\ & & & & & & 1 \end{bmatrix} =: E_{i,j}(k)$$
\item[(3)] 交换单位矩阵$I$的第$i$行与第$j$行,所得矩阵记为$E_{i,j}$,称为对换矩阵。它也是交换$I$的第$i$列与第$j$列,所得的矩阵。
%$$
%\begin{bmatrix} 1 & & & & & & & \\ & \ddots & & & & & & \\ & & 0 & & 1 & & \\ & & & \ddots & & & & \\ & & 1 & & 0 & & \\ & & & & & & \ddots & \\ & & & & & & & 1\end{bmatrix} =: E_{ij}$$
$$\left[\begin{array}{*{11}c}
1 & & & & & & & & &  & \\ & \ddots & & & & & & & & & \\ & & 1 & & & & & & & & \\ & & & 0 & \cdots & \cdots & \cdots & 1 & & & \\ & & & \vdots & 1 & & & \vdots & & & \\ & & & \vdots & & \ddots & & \vdots & & & \\ & & & \vdots & & & 1 & \vdots & & & \\ & & & 1 & \cdots & \cdots & \cdots & 0 & & & \\ & & & & & & & & 1 & & \\ & & & & & & & & & \ddots & \\ & & & & & & & & & & 1 \end{array}\right] =: E_{ij}$$
\end{list}
\end{Def}

\begin{prop}[初等矩阵的性质]\

\enum
\item[(1)] 初等矩阵的转置:
$$E_i(k)^T = E_i(k), \quad E_{ij}^T = \left[\begin{array}{*{11}c}
1 & & & & & & & & &  & \\ & \ddots & & & & & & & & & \\ & & 1 & & & & & & & & \\ & & & 0 & \cdots & \cdots & \cdots & 1 & & & \\ & & & \vdots & 1 & & & \vdots & & & \\ & & & \vdots & & \ddots & & \vdots & & & \\ & & & \vdots & & & 1 & \vdots & & & \\ & & & 1 & \cdots & \cdots & \cdots & 0 & & & \\ & & & & & & & & 1 & & \\ & & & & & & & & & \ddots & \\ & & & & & & & & & & 1 \end{array}\right] = E_{ij},$$
故倍乘矩阵、对换矩阵转置不变(对称阵),转置后仍为初等矩阵。

设$i < j$,则
$$E_{i,j}(k) = \begin{bmatrix} 1 & & & & & & \\ & \ddots & & & & & \\ & & 1 & & & & \\ & & \vdots & \ddots & & & \\ & & k & \cdots & 1 & &  \\ & & & & & \ddots & \\ & & & & & & 1 \end{bmatrix} = E_{j,i}(k)^T,$$
故倍加矩阵转置仍为倍加矩阵,但下标顺序交换。
\item[(2)] 初等矩阵的行列式:易知$|E_i(k)|=k\neq 0; |E_{i,j}(k)|=1$,而
\begin{eqnarray*}
|E_{ij}| & = & \left|\begin{array}{*{11}c}
1 & & & & & & & & &  & \\ & \ddots & & & & & & & & & \\ & & 1 & & & & & & & & \\ & & & 0 & \cdots & \cdots & \cdots & 1 & & & \\ & & & \vdots & 1 & & & \vdots & & & \\ & & & \vdots & & \ddots & & \vdots & & & \\ & & & \vdots & & & 1 & \vdots & & & \\ & & & 1 & \cdots & \cdots & \cdots & 0 & & & \\ & & & & & & & & 1 & & \\ & & & & & & & & & \ddots & \\ & & & & & & & & & & 1 \end{array}\right| \\
& = & (-1)^{\tau(1\cdots j\cdots i\cdots n)}a_{11}\cdots a_{i-1,i-1}a_{ij}a_{i+1,i+1}\cdots a_{j-1,j-1}a_{ji}a_{j+1,j+1}\cdots a_{nn} \\
& = & (-1)\times 1\times\cdots\times 1 = -1
\end{eqnarray*}
\item[(3)] 初等矩阵的分块表示:

设$e_i = (0,\cdots,0,1,0,\cdots,0)^T$ ($1$在第$i$位)列矩阵(列向量),则
$$I_n = (e_1,\cdots,e_i,\cdots,e_j,\cdots,e_n), \quad \text{ 或 }\ I_n = I_n^T = \begin{bmatrix} e_1^T \\ \vdots \\ e_i^T \\ \vdots \\ e_n^T \end{bmatrix},$$
从而,由初等矩阵的定义,有
\begin{eqnarray*}
E_{ij} & = & E_{ij}^T = (e_1,\cdots,e_j,\cdots,e_i,\cdots,e_n), \\
E_i(k) & = & E_i(k)^T = (e_1,\cdots,ke_i,\cdots,e_n), \quad (k\neq 0), \\
E_{ij}(k) & = & \begin{bmatrix} e_1^T \\ \vdots \\ e_i^T \\ \vdots \\ ke_i^T + e_j^T \\ \cdots \\ e_n^T \end{bmatrix} = (e_1,\cdots,ke_j + e_i,\cdots,e_j,\cdots,e_n)
\end{eqnarray*}
\end{list}
\end{prop}

\begin{prop}[初等矩阵与初等变换的关系]\

\enum
\item[(1)] 左乘倍乘矩阵$E_i(k)$,相当于对$A$的第$i$行做$k$倍的倍乘变换;右乘倍乘矩阵$E_i(k)$,相当于对$A$的第$i$列做$k$倍的倍乘变换。
\item[(2)] 左乘对换矩阵$E_{ij}$,相当于对$A$的第$i, j$行做对换变换;右乘对换矩阵$E_{ij}$,相当于对$A$的第$i, j$列做对换变换。
\item[(3)] 左乘倍加矩阵$E_{ij}(k)$,相当于对$A$做第$i$行的$k$倍加到第$j$行上的倍加变换;右乘倍加矩阵$E_{ij}(k)$,相当于对$A$做第$j$列的$k$倍加到第$i$列上的倍加变换。
\end{list}
\end{prop}

\begin{thm}
用初等矩阵左乘矩阵$A$,相当于对$A$进行一次相应的初等行变换。用初等矩阵右乘矩阵$A$,相当于对$A$进行一次相应的初等列变换。
\end{thm}

\begin{Def}
下面三种针对分块矩阵$M$的变形,统称为分块矩阵的初等变换:
\enum
\item[(1)] 倍乘:用特定矩阵$P$左(右)乘$M$的某一``行(列)'';
\item[(2)] 倍加:用矩阵$Q$乘$M$的某``行(列)''加到另外一``行(列)''。
\item[(3)] 对换:交换$M$的两``行''或``列''。
\end{list}
\end{Def}

\begin{Def}
将单位矩阵分块成准对角形矩阵$I = diag(I_s, I_t)$,对其进行一次初等变换,得到的分块矩阵称为分块初等矩阵:
\enum
\item[(1)] 分块倍乘矩阵:$\begin{bmatrix} P & 0 \\ 0 & I_t \end{bmatrix}, \begin{bmatrix} I_s & 0 \\ 0 & P \end{bmatrix}$(其中$P$为可逆方阵);
\item[(2)] 分块倍加矩阵:$\begin{bmatrix} I_s & 0 \\ Q & I_t \end{bmatrix}, \begin{bmatrix} I_s & Q \\ 0 & I_t \end{bmatrix}$;
\item[(3)] 分块对换矩阵:$\begin{bmatrix} 0 & I_t \\ I_s & 0 \end{bmatrix}, \begin{bmatrix} 0 & I_s \\ I_t & 0 \end{bmatrix}$。
\end{list}
\end{Def}

\begin{thm}
对分块矩阵进行一次初等行(列)变换,相当于给它左(右)乘以一个相应的分块初等矩阵。
\end{thm}

\begin{Def}
设$A$为$n$阶方阵,若存在$n$阶方阵$B$,使得 $AB = BA = I_n$,则称$A$为可逆矩阵或非奇异矩阵,而称$B$为$A$的逆矩阵,并记为$A^{-1}$。
\end{Def}

\begin{rmk}\

\enum
\item[(1)] 定义中矩阵$A$与$B$的地位相同,因而若$A$可逆,且$B$是$A$的逆,则$B$也可逆,$A$即是$B$的逆,并且$A$与$A^{-1}$乘法可交换。
\item[(2)] 对上述定义式两边取行列式知:若$A$为可逆矩阵,则
$$|A|\neq 0, |A^{-1}| = |A|^{-1} = 1 / |A|$$
\item[(3)] 若$A$有逆矩阵,则其逆是唯一的。
\end{list}
\end{rmk}

\begin{prop}
设$A, B, A_i$为$n$阶可逆矩阵,实数$k\neq 0$,则
\enum
\item[(1)] $A^{-1}$也可逆,且$(A^{-1})^{-1} = A$。
\item[(2)] $AB$也可逆,且$(AB)^{-1} = B^{-1}A^{-1}$。进一步有,$(A_1A_2\cdots A_s)^{-1} = A_s^{-1}\cdots A_1^{-1}$。
\item[(3)] $kA$也可逆,且$(kA)^{-1} = k^{-1}A^{-1}$。
\item[(4)] $A^T$也可逆,且$(A^T)^{-1} = (A^{-1})^T$。
\end{list}
\end{prop}

\begin{Def}
用$n$阶方阵$A$的元素的代数余子式$A_{ij}$组成的矩阵的转置
$$(A_{ij})^T_{\tiny \substack{1 \leqslant i \leqslant n \\ 1 \leqslant j \leqslant n}\normalfont} = \begin{bmatrix}
A_{11} & A_{21} & \cdots & A_{n1} \\
A_{12} & A_{22} & \cdots & A_{n2} \\
\vdots & \vdots & \ddots & \vdots \\
A_{1n} & A_{2n} & \cdots & A_{nn}
\end{bmatrix} =: A^{\ast}$$
称为矩阵$A$的伴随矩阵。
\end{Def}

\begin{rmk}\

\enum
\item[(1)] 对方阵$A, AA^{\ast} = A^{\ast}A = |A|I$总是成立的(含$|A| = 0$时)。
\item[(2)] 若$|A| \neq 0$,则$A|A|^{-1}A^{\ast} = |A|^{-1}A^{\ast}A = I$,故$A$和$A^{\ast}$可逆,且
$$A^{-1} = |A|^{-1}A^{\ast}, \quad (A^{\ast})^{-1} = |A|^{-1}A.$$
\end{list}
\end{rmk}

\begin{thm}
$n$阶方阵$A$可逆的充要条件是$|A| \neq 0$。
\end{thm}

\begin{cor}
设$A$为$n$阶方阵,若存在$n$阶方阵$B$,使得$AB = I_n$(或 $BA = I_n$),则$A$可逆,且$B = A^{-1}$。
\end{cor}

\begin{thm}
$n$阶方阵$A$可逆 $\Longleftrightarrow$ $A$为若干个初等矩阵的乘积。
\end{thm}

\begin{cor}
$n$阶方阵$A$可逆 $\Longleftrightarrow$ 齐次线性方程组$AX=0$只有零解。
\end{cor}

\begin{prop}[求逆矩阵的初等变换法]\

\enum
\item[$\bullet$] 行变换法:构造一个$n\times(2n)$阶分块矩阵
$$[A,I] \xrightarrow{\text{若干初等行变换}} [I,A^{-1}]:$$
$$P_s\cdots P_1[A,I] = [P_s\cdots P_1A, P_s\cdots P_1I] = [A^{-1}A,A^{-1}I] = [I,A^{-1}].$$
\item[$\bullet$] 列变换法:构造一个$(2n)\times n$阶分块矩阵
$$\begin{bmatrix} A \\ I \end{bmatrix} \xrightarrow{\text{若干初等列变换}} \begin{bmatrix} I \\ A^{-1} \end{bmatrix}:$$
$$\begin{bmatrix} A \\ I \end{bmatrix}P_s\cdots P_1 = \begin{bmatrix} AP_s\cdots P_1 \\ IP_s\cdots P_1 \end{bmatrix} = \begin{bmatrix} AA^{-1} \\ IA^{-1} \end{bmatrix} = \begin{bmatrix} I \\ A^{-1} \end{bmatrix}.$$
\end{list}
\end{prop}

\begin{prop}
对准上三角分块矩阵有
$$A = \begin{bmatrix} A_{11} & A_{12} & \cdots & A_{1n} \\ 0 & A_{22} & \cdots & A_{1n} \\ \vdots & \vdots & \ddots & \vdots \\ 0 & 0 & \cdots & A_{nn} \end{bmatrix} \text{ 可逆 }
\Longleftrightarrow  A_{11},\cdots,A_{nn}\text{ 均可逆,}$$
此时,
$$A^{-1} = \begin{bmatrix} A_{11}^{-1} & \ast & \cdots & \ast \\ 0 & A_{22}^{-1} & \cdots & \ast \\ \vdots & \vdots & \ddots & \vdots \\ 0 & 0 & \cdots & A_{nn}^{-1} \end{bmatrix}.$$

特别地,若$A = diag(A_{11},\cdots,A_{nn})$,则$A^{-1} = diag(A_{11}^{-1},\cdots,A_{nn}^{-1})$。
\end{prop}

%%%%%%%%%%%%%%%%%%%%%%%%%%%%%%%%%%%%%%%%%%%%%%%%%%%%%%%%%%%%%%%%%%%%%%%%%%%%%%%%%%%%%%%%%%%%

\section{例题讲解}

\begin{eg}
设有如下两组变量替换
\begin{eqnarray*}
\begin{cases}
x_1 = a_{11}y_1 + a_{12}y_2 + a_{13}y_3 \\ x_2 = a_{21}y_1 + a_{22}y_2 + a_{23}y_3
\end{cases}
& \longrightarrow &
\begin{bmatrix}
a_{11} & a_{12} & a_{13} \\ a_{21} & a_{22} & a_{23}
\end{bmatrix} \\
\begin{cases}
y_1 = b_{11}z_1 + b_{12}z_2 \\ y_2 = b_{21}z_1 + b_{22}z_2 \\ y_3 = b_{31}z_1 + b_{32}z_2
\end{cases}
& \longrightarrow &
\begin{bmatrix}
b_{11} & b_{12} \\ b_{21} & b_{22} \\ b_{31} & b_{32}
\end{bmatrix}
\end{eqnarray*}
将第二组变量替换代入到第一组中,即可将$x_1, x_2$表示为$z_1, z_2$的形式:
$$
\begin{cases}
x_1 = (a_{11}b_{11} + a_{12}b_{21} + a_{13}b_{31})z_1 + (a_{11}b_{12} + a_{12}b_{22} + a_{13}b_{32})z_2 \\ x_2 = (a_{21}b_{11} + a_{22}b_{21} + a_{23}b_{31})z_1 + (a_{21}b_{12} + a_{22}b_{22} + a_{23}b_{32})z_2
\end{cases}
\longrightarrow
\begin{bmatrix}
c_{11} & c_{12} \\ c_{21} & c_{22}
\end{bmatrix}
$$
且$c_{ij} = a_{i1}b_{1j} + a_{i2}b_{2j} + a_{i3}b_{3j} = \sum\limits_{k=1}^3 a_{ik}b_{kj}, \ (1 \leqslant i,j \leqslant 2).$
\end{eg}

\begin{eg}
设$A = \begin{bmatrix} 2 & 3 & 1 \\ 1 & 2 & -1 \\ 0 & 3 & 1\end{bmatrix}, B = \begin{bmatrix} 1 \\ 0 \\ 2 \end{bmatrix}$,求$AB$。
\end{eg}

\begin{solution}
$\begin{bmatrix} 2 & 3 & 1 \\ 1 & 2 & -1 \\ 0 & 3 & 1\end{bmatrix} \begin{bmatrix} 1 \\ 0 \\ 2 \end{bmatrix} = \begin{bmatrix} 2\times 1 + 3\times 0 + 1\times 2 \\ 1\times 1 + 2\times 0 + (-1)\times 2 \\ 0\times 1 + 3\times 0 + 1\times 2\end{bmatrix} = \begin{bmatrix} 4 \\ -1 \\ 2 \end{bmatrix}$
\end{solution}

\begin{eg}
考虑如下线性方程组
$$\left\{ \begin{array}{rcl} a_{11}x_1 + a_{12}x_2 + \cdots + a_{1n}x_n & = & b_1 \\ a_{21}x_1 + a_{22}x_2 + \cdots + a_{2n}x_n & = & b_2 \\ \hdotsfor{3} \\ a_{m1}x_1 + a_{m2}x_2 + \cdots + a_{mn}x_n & = & b_m \end{array}\right.$$
记$A = (a_{ij})_{m\times n} = \begin{bmatrix} a_{11} & \cdots & a_{1n} \\ \vdots & & \vdots \\ a_{m1} & \cdots & a_{mn} \end{bmatrix}, X_{n\times 1} = \begin{bmatrix} x_1 \\ \vdots \\ x_n \end{bmatrix} B_{m\times 1} = \begin{bmatrix} b_1 \\ \vdots \\ b_m \end{bmatrix}$。则由矩阵的乘法定义可知,线性方程组可以写为:$AX = B$,其中$A$为系数矩阵,$\left( A \middle| B \right)_{m\times(n+1)}$为增广系数矩阵。
\end{eg}

\begin{eg}
设$A = \begin{bmatrix} a_1 \\ a_2 \end{bmatrix}, B = \begin{bmatrix} b_1 & b_2 \end{bmatrix}$,则$AB = \begin{bmatrix} a_1b_1 & a_1b_2 \\ a_2b_1 & a_2b_2 \end{bmatrix}, BA = (b_1a_1 + b_2a_2)$。
\end{eg}

\begin{eg}
设$A = \begin{bmatrix} 1 & 1 \\ -1 & -1 \end{bmatrix}, B = \begin{bmatrix} 1 & -1 \\ -1 & 1 \end{bmatrix}$,则$AB = \begin{bmatrix} 0 & 0 \\0 & 0 \end{bmatrix}, BA = \begin{bmatrix} 2 & 2 \\ -2 & -2 \end{bmatrix}$。
\end{eg}

\begin{eg}
  设$A = \begin{bmatrix} 1 & 2 \\ 2 & 4 \end{bmatrix}, B = \begin{bmatrix} -1 & 3 \\ -2 & 1 \end{bmatrix}, C = \begin{bmatrix} -7 & 1 \\ 1 & 2 \end{bmatrix}$,求$AB$以及$AC$。
\end{eg}

\begin{solution}
$$\left. \begin{array}{c}
AB = \begin{bmatrix} 1 & 2 \\ 2 & 4 \end{bmatrix} \begin{bmatrix} -1 & 3 \\ -2 & 1 \end{bmatrix} = \begin{bmatrix} -5 & 5 \\ -10 & 10 \end{bmatrix} \\
AC = \begin{bmatrix} 1 & 2 \\ 2 & 4 \end{bmatrix} \begin{bmatrix} -7 & 1 \\ 1 & 2 \end{bmatrix} = \begin{bmatrix} -5 & 5 \\ -10 & 10 \end{bmatrix}
\end{array} \right\} \Rightarrow AB = AC, B \neq C.$$
\end{solution}

\begin{eg}
考虑对角阵与矩阵的乘积:
\begin{eqnarray*}
\begin{bmatrix}
k_1 & & & \\ & k_2 & & \\ & & \ddots & \\ & & & k_m
\end{bmatrix}_{m\times m}
\begin{bmatrix}
a_{11} & a_{12} & \cdots & a_{1n} \\ a_{21} & a_{22} & \cdots & a_{2n} \\ \vdots & \vdots & \vdots & \vdots \\ a_{m1} & a_{m2} & \cdots & a_{mn}
\end{bmatrix}_{m\times n}
& = & \begin{bmatrix}
k_1 a_{11} & k_1 a_{12} & \cdots & k_1 a_{1n} \\ k_2 a_{21} & k_2 a_{22} & \cdots & k_2 a_{2n} \\ \vdots & \vdots & \vdots & \vdots \\ k_m a_{m1} & k_m a_{m2} & \cdots & k_m a_{mn}
\end{bmatrix}_{m\times n} \\
\begin{bmatrix}
a_{11} & a_{12} & \cdots & a_{1n} \\ a_{21} & a_{22} & \cdots & a_{2n} \\ \vdots & \vdots & \vdots & \vdots \\ a_{m1} & a_{m2} & \cdots & a_{mn}
\end{bmatrix}_{m\times n}
\begin{bmatrix}
k_1 & & & \\ & k_2 & & \\ & & \ddots & \\ & & & k_n
\end{bmatrix}_{n\times n}
& = & \begin{bmatrix}
k_1 a_{11} & k_2 a_{12} & \cdots & k_n a_{1n} \\ k_1 a_{21} & k_2 a_{22} & \cdots & k_n a_{2n} \\ \vdots & \vdots & \vdots & \vdots \\ k_1 a_{m1} & k_2 a_{m2} & \cdots & k_n a_{mn}
\end{bmatrix}_{m\times n}
\end{eqnarray*}
\end{eg}

\begin{eg}
设$X = \begin{bmatrix} x_1 \\ x_2 \\ x_3 \end{bmatrix}, Y = \begin{bmatrix} y_1 & y_2 & y_3 \end{bmatrix}$,求$(XY)^{100}$。
\end{eg}

\begin{solution}
\begin{eqnarray*}
(XY)^{100} & = & (XY)(XY)\cdots (XY) \\
& = & X(YX)(YX)\cdots (YX)Y \\
& = & X(YX)^{99}Y \\
& = & (YX)^{99}XY \\
& = & (x_1y_1 + x_2y_2 + x_3y_3)^{99} \begin{bmatrix} x_1y_1 & x_1y_2 & x_1y_3 \\ x_2y_1 & x_2y_2 & x_2y_3 \\ x_3y_1 & x_3y_2 & x_3y_3 \end{bmatrix}.
\end{eqnarray*}
\end{solution}

\begin{eg}
已知$A = \begin{bmatrix} 2 & 0 & -1 \\ 1 & 3 & 2 \end{bmatrix}, B = \begin{bmatrix} 1 & 7 & -1 \\ 4 & 2 & 3 \\ 2 & 0 & 1 \end{bmatrix}$,求$(AB)^T$。
\end{eg}

\begin{solution}[解法一]
\begin{eqnarray*}
& AB & = \begin{bmatrix} 2 & 0 & -1 \\ 1 & 3 & 2 \end{bmatrix} \begin{bmatrix} 1 & 7 & -1 \\ 4 & 2 & 3 \\ 2 & 0 & 1 \end{bmatrix} = \begin{bmatrix} 0 & 14 & -3 \\ 17 & 13 & 10 \end{bmatrix}\\
\Longrightarrow & (AB)^T & = \begin{bmatrix} 0 & 17 \\ 14 & 13 \\ -3 & 10 \end{bmatrix}
\end{eqnarray*}
\end{solution}

\begin{solution}[解法二]
$$(AB)^T = B^T A^T =  \begin{bmatrix} 1 & 4 & 2 \\ 7 & 2 & 0 \\ -1 & 3 & 1 \end{bmatrix} \begin{bmatrix} 2 & 1 \\ 0 & 3 \\ -1 & 2 \end{bmatrix} = \begin{bmatrix} 0 & 17 \\ 14 & 13 \\ -3 & 10 \end{bmatrix}.$$
\end{solution}

\begin{eg}
试证:奇数阶反对称矩阵的行列式一定为$0$。
\end{eg}

\begin{proof}[证明]
设$n$为奇数,$A$为$n$阶反对称矩阵,则
\begin{eqnarray*}
& |A| & = |A^T| = |-A| = (-1)^n|A| = -|A| \\
\Longrightarrow & 2|A| & = 0 \Longrightarrow |A| = 0.
\end{eqnarray*}
\end{proof}

\begin{eg}
矩阵的分块运算:
\[
  \setlength{\dashlinegap}{2pt}
  A = \left[ \begin{array}{c:cc}
    2 & 0 & 0 \\
    1 & 0 & 0 \\
    \hdashline
    3 & 1 & 0 \\
    2 & 0 & 1
  \end{array} \right],
  B = \left[ \begin{array}{cc:ccc}
    1 & 2 & 0 & 0 & 0\\
    \hdashline
    2 & 3 & 1 & 3 & 1 \\
    0 & 2 & 2 & 0 & 1
  \end{array} \right], \text{ 那么 }
  AB = \left[ \begin{array}{cc:ccc}
    2 & 4 & 0 & 0 & 0\\
    1 & 2 & 0 & 0 & 0\\
    \hdashline
    5 & 9 & 1 & 3 & 1 \\
    2 & 6 & 2 & 0 & 1
  \end{array} \right]
\]
\end{eg}

\begin{eg}
矩阵的分块运算:
\[
  \setlength{\dashlinegap}{2pt}
  A = \left[ \begin{array}{c:cc}
    2 & 0 & 0 \\
    1 & 0 & 0 \\
    \hdashline
    3 & 1 & 0 \\
    2 & 0 & 1
  \end{array} \right],
  A^T = \left[ \begin{array}{cc:cc}
    2 & 1 & 0 & 0 \\
    \hdashline
    0 & 0 & 1 & 0 \\
    0 & 0 & 0 & 1
  \end{array} \right],
\]
将$A$与$A^T$如下分块,有
\begin{eqnarray*}
A = \begin{bmatrix}
A_{11} & A_{12} \\ A_{21} & A_{22}
\end{bmatrix},
& \text{ 其中 } &
A_{11} = \begin{bmatrix} 2 \\ 1 \end{bmatrix}, A_{12} = 0_{2\times 2}, A_{21} = \begin{bmatrix} 3 \\ 2 \end{bmatrix}, A_{22} = I_2, \\
A^T = \begin{bmatrix}
B_{11} & B_{12} \\ B_{21} & B_{22}
\end{bmatrix},
& \text{ 其中 } &
B_{11} = \begin{bmatrix} 2 & 1 \end{bmatrix}, B_{12} = \begin{bmatrix} 3 & 2 \end{bmatrix}, B_{21} = 0_{2\times 2}, B_{22} = I_2.
\end{eqnarray*}
\begin{eqnarray*}
& \Longrightarrow & \qquad A_{11} = B_{11}^T, A_{12} = B_{21}^T = 0_{2\times 2}, A_{21} = B_{12}^T, A_{22} = B_{22}^T = I_2 \\
& \Longrightarrow & \qquad A_{ij} = B_{ji}^T, (1 \leqslant i,j \leqslant 2).
\end{eqnarray*}
\end{eg}

\begin{eg}
用初等矩阵与初等变换的关系,再次验证行列式的性质
\end{eg}

\begin{solution}[验证]\

\enum
\item[(1)] 逐行保数乘:
\begin{eqnarray*}
A \xrightarrow{kr_i} B & \Longleftrightarrow & B = E_i(k)A; \\
& \Longrightarrow & |B| = |E_i(k)|\cdot|A| = k|A|;
\end{eqnarray*}
\item[(2)] 交错性:
\begin{eqnarray*}
A \xrightarrow{r_i\leftrightarrow r_j} B & \Longleftrightarrow & B = E_{ij}A; \\
& \Longrightarrow & |B| = |E_{ij}|\cdot|A| = (-1)|A| = -|A|;
\end{eqnarray*}
\item[(3)] 倍加不变性:
\begin{eqnarray*}
A \xrightarrow{kr_i + r_j} B & \Longleftrightarrow & B = E_{ij}(k)A; \\
& \Longrightarrow & |B| = |E_{ij}(k)|\cdot|A| = 1\cdot|A| = |A|.
\end{eqnarray*}
\end{list}
\end{solution}

\begin{eg}
三类初等矩阵都是可逆矩阵,且
\begin{eqnarray*}
E_i(k)^{-1} & = & E_i(1/k) \quad (k\neq 0); \\
E_{i,j}^{-1} & = & E_{i,j}; \\
E_{i,j}(k)^{-1} & = & E_{i,j}(-k).
\end{eqnarray*}
\end{eg}

\begin{eg}
考虑矩阵$A = \begin{bmatrix} 1 & -1 \\ 0 & 0 \end{bmatrix}$,对任意$2$阶方阵$B= \begin{bmatrix} a & b \\ c & d \end{bmatrix}$有
$$AB = \begin{bmatrix} 1 & -1 \\ 0 & 0 \end{bmatrix} \begin{bmatrix} a & b \\ c & d \end{bmatrix} = \begin{bmatrix} a-c & b-d \\ 0 & 0 \end{bmatrix} \neq \begin{bmatrix} 1 & 0 \\ 0 & 1 \end{bmatrix} = I_2,$$
从而,并不是所有的方阵都可逆。
\end{eg}

\begin{eg}
若二阶方阵的行列式$|A| \neq 0$,于是
\begin{eqnarray*}
A & = & \begin{bmatrix} a_{11} & a_{12} \\ a_{21} & a_{22} \end{bmatrix} \Rightarrow A^{\ast} = \begin{bmatrix} A_{11} & A_{21} \\ A_{12} & A_{22} \end{bmatrix} = \begin{bmatrix} a_{22} & -a_{12} \\ -a_{21} & a_{11} \end{bmatrix} \\
& \Rightarrow & A^{-1} = |A|^{-1}A^{\ast} = \frac{1}{a_{11}a_{22} - a_{12}a_{21}} \begin{bmatrix} a_{22} & -a_{12} \\ -a_{21} & a_{11} \end{bmatrix}.
\end{eqnarray*}
\end{eg}

\begin{eg}
求方阵$A = \begin{bmatrix} 1 & 2 & 3 \\ 2 & 2 & 1 \\ 3 & 4 & 3 \end{bmatrix}$的逆矩阵。
\end{eg}

\begin{solution}
因为$|A| = \begin{vmatrix} 1 & 2 & 3 \\ 2 & 2 & 1 \\ 3 & 4 & 3 \end{vmatrix} = 2 \neq 0$,所以可以用伴随矩阵求逆法。
$$A_{11} = \begin{vmatrix} 2 & 1 \\ 4 & 3 \end{vmatrix} = 2, A_{12} = -\begin{vmatrix} 2 & 1 \\ 3 & 3 \end{vmatrix} = -3, A_{13} = \begin{vmatrix} 2 & 2 \\ 3 & 4 \end{vmatrix} = 2,$$
同理可得$A_{21} = 6, A_{22} = -6, A_{23} = 2, A_{31} = -4, A_{32} = 5, A_{33} = -2$。故
$$A^{-1} = \frac{1}{|A|}A^{\ast} = \frac12 \begin{bmatrix} 2 & 6 & -4 \\ -3 & -6 & 5 \\ 2 & 2 & -2 \end{bmatrix} = \begin{bmatrix} 1 & 3 & -2 \\ -3/2 & -3 & 5/2 \\ 1 & 1 & -1 \end{bmatrix}.$$
\end{solution}

\begin{eg}
设$A = \begin{bmatrix} a_{11} & & & \\ & a_{22} & & \\ & & \ddots & \\ & & & a_{nn} \end{bmatrix}$,其中$a_{ii} \neq 0, 1 \leqslant i \leqslant n$,求$A^{-1}$。
\end{eg}

\begin{solution}
若$AB = I_n$,其中$A$与$I_n$均为对角阵,猜测$B$也是($n$阶)对角阵;再由条件$a_{ii}\neq 0$,用定义可验证结果。

由于$a_{ii} \neq 0$,有
\begin{eqnarray*}
& \begin{bmatrix} a_{11} & & & \\ & a_{22} & & \\ & & \ddots & \\ & & & a_{nn} \end{bmatrix} & \begin{bmatrix} a_{11}^{-1} & & & \\ & a_{22}^{-1} & & \\ & & \ddots & \\ & & & a_{nn}^{-1} \end{bmatrix} = \begin{bmatrix} a_{11}a_{11}^{-1} & & & \\ & a_{22}a_{22}^{-1} & & \\ & & \ddots & \\ & & & a_{nn}a_{nn}^{-1} \end{bmatrix} = I_n \\
& \Longrightarrow & A^{-1} = \begin{bmatrix} a_{11}^{-1} & & & \\ & a_{22}^{-1} & & \\ & & \ddots & \\ & & & a_{nn}^{-1} \end{bmatrix}.
\end{eqnarray*}
\end{solution}

\begin{eg}
设$A = \begin{bmatrix} a_{11} & a_{12} \\ 0 & a_{22} \end{bmatrix}$,其中$a_{11}, a_{22} \neq 0$,求$A^{-1}$。
\end{eg}

\begin{solution}
若$AB = I_2$,其中$A$与$I_2$均为上三角阵,猜测$B$也是$2$阶上三角阵,再由待定系数法,可求出$B$。

设$B = \begin{bmatrix} b_{11} & b_{12} \\ 0 & b_{22} \end{bmatrix}$,若
\begin{eqnarray*}
& & I_2 = \begin{bmatrix} a_{11} & a_{12} \\ 0 & a_{22} \end{bmatrix} \begin{bmatrix} b_{11} & b_{12} \\ 0 & b_{22} \end{bmatrix} = \begin{bmatrix} a_{11}b_{11} & a_{11}b_{12} + a_{12}b_{22} \\ 0 & a_{22}b_{22} \end{bmatrix} \\
& \Longrightarrow & \begin{cases} a_{11}b_{11} = 1 \\ a_{22}b_{22} = 1 \\ a_{11}b_{12} + a_{12}b_{22} = 0 \end{cases} \\
& \Longrightarrow & \begin{cases} b_{11} = a_{11}^{-1} \\ b_{22} = a_{22}^{-1} \\ b_{12} = -a_{11}^{-1}a_{12}a_{22}^{-1} \end{cases} \\
& \Longrightarrow & A^{-1} = \begin{bmatrix} a_{11}^{-1} & -a_{11}^{-1}a_{12}a_{22}^{-1} \\ 0 & a_{22}^{-1} \end{bmatrix}
\end{eqnarray*}
\end{solution}

\begin{eg}
设方阵$A$满足$A^2-2A+4I=O$,证明:$A+I$和$A-3I$都可逆,并求它们的逆矩阵。
\end{eg}

\begin{solution}
关于$A$是多项式,$A^i$与$A^j$乘法可交换,可按通常方法分解多项式,希望能凑出$A+I$和$A-3I$的项。

\begin{eqnarray*}
& & A^2-2A+4I = (A+I)(A-3I)+7I = O \\
& \Longrightarrow & -\frac17 (A+I)(A-3I) = I.
\end{eqnarray*}
由定义知
$$(A+I)^{-1} = -\frac17 (A-3I), \quad (A-3I)^{-1} = -\frac17 (A+I).$$
\end{solution}

\begin{eg}
用伴随矩阵求下面方阵$A$的逆矩阵:$A = \begin{bmatrix} 2 & 3 & -1 \\ 1 & 2 & 0 \\ 1 & 2 & -2 \end{bmatrix}.$
\end{eg}

\begin{solution}
$\because \ |A| = \begin{vmatrix} 2 & 3 & -1 \\ 1 & 2 & 0 \\ 1 & 2 & -2 \end{vmatrix} = -2 \neq 0,$ 所以方阵$A$可逆。
$$A_{11} = \begin{vmatrix} 2 & 0 \\ 2 & -2 \end{vmatrix} = -4, A_{21} = -\begin{vmatrix} 3 & -1 \\ 2 & -2 \end{vmatrix} = 4, A_{31} = \begin{vmatrix} 3 & -1 \\ 2 & 0 \end{vmatrix} = 2,$$
$$A_{12} = -\begin{vmatrix} 1 & 0 \\ 1 & -2 \end{vmatrix} = 2, A_{22} = \begin{vmatrix} 2 & -1 \\ 1 & -2 \end{vmatrix} = -3, A_{32} = -\begin{vmatrix} 2 & -1 \\ 1 & 0 \end{vmatrix} = -1,$$
$$A_{13} = \begin{vmatrix} 1 & 2 \\ 1 & 2 \end{vmatrix} = 0, A_{23} = -\begin{vmatrix} 2 & 3 \\ 1 & 2 \end{vmatrix} = -1, A_{32} = \begin{vmatrix} 2 & 3 \\ 1 & 2 \end{vmatrix} = 1,$$
从而
$$A^{\ast} = \begin{bmatrix} -4 & 4 & 2 \\ 2 & -3 & -1 \\ 0 & -1 & 1 \end{bmatrix},$$
于是
$$A^{-1} = |A|^{-1}A^{\ast} = \begin{bmatrix} 2 & -2 & -1 \\ -1 & 3/2 & 1/2 \\ 0 & 1/2 & -1/2 \end{bmatrix}.$$
\end{solution}

\begin{eg}
用初等行变换求矩阵$A$的逆矩阵:$A = \begin{bmatrix} 0 & 2 & -1 \\ 1 & 1 & 2 \\ -1 & -1 & -1 \end{bmatrix}$。
\end{eg}

\begin{solution}
先将$A$化为阶梯形矩阵,再化为单位阵:
\begin{eqnarray*}
[A,I] & = &
  \setlength{\dashlinegap}{2pt}
  \left[ \begin{array}{ccc:ccc}
    0 & 2 & -1 & 1 & 0 & 0 \\ 1 & 1 & 2 & 0 & 1 & 0 \\ -1 & -1 & -1 & 0 & 0 & 1
  \end{array} \right] \xrightarrow[r_1+r_3]{r_1\leftrightarrow r_2}
  \left[ \begin{array}{ccc:ccc}
     1 & 1 & 2 & 0 & 1 & 0 \\ 0 & 2 & -1 & 1 & 0 & 0 \\ 0 & 0 & 1 & 0 & 1 & 1
  \end{array} \right] \\
  & \longrightarrow & \left[ \begin{array}{ccc:ccc}
     1 & 0 & 0 & -\frac12 & -\frac32 & -\frac52 \\ 0 & 1 & 0 & \frac12 & \frac12 & \frac12 \\ 0 & 0 & 1 & 0 & 1 & 1 \end{array} \right], \\
\Longrightarrow & A^{-1} & = \frac12\begin{bmatrix} -1 & -3 & -5 \\ 1 & 1 & 1 \\ 0 & 2 & 2 \end{bmatrix}
\end{eqnarray*}
\end{solution}

\begin{eg}
设$A = \begin{bmatrix} 1 & 1 & 2 \\ 0 & 2 & -1 \\ -1 & 1 & -3 \end{bmatrix}$,试判断$A$是否可逆。
\end{eg}

\begin{solution}
\begin{eqnarray*}
[A,I] & = &
  \setlength{\dashlinegap}{2pt}
  \left[ \begin{array}{ccc:ccc}
    1 & 1 & 2 & 1 & 0 & 0 \\ 0 & 2 & -1 & 0 & 1 & 0 \\ -1 & 1 & -3 & 0 & 0 & 1
  \end{array} \right] \longrightarrow
  \left[ \begin{array}{ccc:ccc}
     1 & 1 & 2 & 1 & 0 & 0 \\ 0 & 2 & -1 & 0 & 1 & 0 \\ 0 & 2 & -1 & 1 & 0 & 1
  \end{array} \right] \\
  & \longrightarrow & \left[ \begin{array}{ccc:ccc}
      1 & 1 & 2 & 1 & 0 & 0 \\ 0 & 2 & -1 & 0 & 1 & 0 \\ 0 & 0 & 0 & 1 & -1 & 1
   \end{array} \right],
\end{eqnarray*}
此时,$A$经初等行变换化为阶梯形矩阵时,出现全零行,则$A$的行列式为零,故$A$不可逆。
\end{solution}

\begin{eg}
求如下矩阵方程$XA=C$,其中$A = \begin{bmatrix} 1 & 2 \\ 3 & 4 \end{bmatrix}, C = \begin{bmatrix} 0 & 2 \\ 5 & 6 \\ 7 & 8 \end{bmatrix}$。
\end{eg}

\begin{solution}
由$|A|=-2$知,$A$可逆,则$X=CA^{-1}$,对如下分块矩阵进行初等列变换:
\begin{eqnarray*}
\begin{bmatrix} A \\ C \end{bmatrix} & = &
  \setlength{\dashlinegap}{2pt}
  \left[ \begin{array}{cc}
    1 & 2 \\ 3 & 4 \\ \hdashline  0 & 2 \\ 5 & 6 \\ 7 & 8 \end{array} \right]
  \xrightarrow{(-2)c_1+c_2}
  \left[ \begin{array}{cc}
    1 & 0 \\ 3 & -2 \\ \hdashline  0 & 2 \\ 5 & -4 \\ 7 & -6 \end{array} \right]
  \xrightarrow{(-\frac12)c_2}
  \left[ \begin{array}{cc}
    1 & 0 \\ 3 & 1 \\ \hdashline  0 & -1 \\ 5 & 2 \\ 7 & 3 \end{array} \right] \\
  & \xrightarrow{(-3)c_2 + c_1} &
  \left[ \begin{array}{cc}
    1 & 0 \\ 0 & 1 \\ \hdashline  3 & -1 \\ -1 & 2 \\ -2 & 3 \end{array} \right]
  = \begin{bmatrix} I \\ CA^{-1} \end{bmatrix} \\
\Longrightarrow & X=CA^{-1} & = \left[ \begin{array}{cc}
    3 & -1 \\ -1 & 2 \\ -2 & 3 \end{array} \right].
\end{eqnarray*}
\end{solution}

\begin{eg}
试判断矩阵$A = \begin{bmatrix} 2 & 0 & 0 & 0 \\ 1 & 2 & 1 & 0 \\ 0 & 0 & 1 & 0 \\ 0 & 0 & 1 & 1 \end{bmatrix}$是否可逆?若可逆,求出$A^{–1}$。
\end{eg}

\begin{solution}
设$A = \begin{bmatrix} A_{11} & A_{12} \\ 0 & A_{22} \end{bmatrix}$,其中
$A_{11} = \begin{bmatrix} 2 & 0 \\ 1 & 2 \end{bmatrix}, A_{12} = \begin{bmatrix} 0 & 0 \\ 1 & 0 \end{bmatrix}, A_{22} = \begin{bmatrix} 1 & 0 \\ 1 & 1 \end{bmatrix}$,且$|A_{11}| = 4, |A_{22}| = 1$,从而$|A| = |A_{11}|\cdot|A_{22}| = 4$,所以$A, A_{11}, A_{22}$均可逆。

设$A^{–1} = \begin{bmatrix} X & Y \\ 0 & Z \end{bmatrix}$,则
$$\begin{bmatrix} A_{11} & A_{12} \\ 0 & A_{22} \end{bmatrix}\begin{bmatrix} X & Y \\ 0 & Z \end{bmatrix} = I \Longrightarrow
\begin{cases} X = A_{11}^{–1} \\ Z = A_{22}^{–1} \\ A_{11}Y + A_{12}A_{22}^{–1} = 0 \end{cases} \Longrightarrow
Y = -A_{11}^{–1}A_{12}A_{22}^{–1},$$
所以,
$$A^{–1} = \begin{bmatrix} A_{11}^{–1} & -A_{11}^{–1}A_{12}A_{22}^{–1} \\ 0 & A_{22}^{–1} \end{bmatrix} = \begin{bmatrix} \frac12 & 0 & 0 & 0 \\ -1 & \frac12 & -\frac12 & 0 \\ 0 & 0 & 1 & 0 \\ 0 & 0 & -1 & 1 \end{bmatrix}.$$
\end{solution}

\begin{eg}
设$M = \begin{bmatrix} A & B \\ C & D \end{bmatrix}$,其中$A$可逆,$D$为方阵,试证$|M| = |A||D-CA^{-1}B|$。进而证明,$M$可逆$\Longleftrightarrow$ $D-CA^{-1}B$可逆,并求$M$的逆。
\end{eg}

\begin{proof}[证明]
\begin{eqnarray*}
& & \begin{bmatrix} I & 0 \\ CA^{-1} & I \end{bmatrix} \begin{bmatrix} A & B \\ C & D \end{bmatrix} \begin{bmatrix} I & -A^{-1}B \\ 0 & I \end{bmatrix} = \begin{bmatrix} A & 0 \\ 0 & D-CA^{-1}B \end{bmatrix} \\
& \Longrightarrow & |M| = \begin{vmatrix} A & 0 \\ 0 & D-CA^{-1}B \end{vmatrix} = |A||D-CA^{-1}B|.
\end{eqnarray*}
从而$M$可逆$\Longleftrightarrow$ $D-CA^{-1}B$可逆,且
$$M^{-1} = \begin{bmatrix} I & -A^{-1}B \\ 0 & I \end{bmatrix} \begin{bmatrix} A^{-1} & 0 \\ 0 & (D-CA^{-1}B)^{-1} \end{bmatrix} \begin{bmatrix} I & 0 \\ -CA^{-1} & I \end{bmatrix}.$$
\end{proof}

%%%%%%%%%%%%%%%%%%%%%%%%%%%%%%%%%%%%%%%%%%%%%%%%%%%%%%%%%%%%%%%%%%%%%%%%%%%%%%%%%%%%%%%%%%%%

\section{课后习题}

\begin{ex} \label{ex:3.1}
分别给出$3\times3$阶的对角阵,上三角阵,对称阵,反对称阵的一个例子。
\end{ex}

\begin{ex} \label{ex:3.2}
对$n \geqslant 2$,构造$n$阶方阵$A,B$,使得$\det(A+B)\neq\det(A)+\det(B)$。
\end{ex}

\begin{ex} \label{ex:3.3}
设$A = \begin{bmatrix} 1 & -2 & -2 \\ -1 & 1 & -2 \\ 1 & 1 & -2 \end{bmatrix}, B = \begin{bmatrix} 3 & -1 & 5 \\ 1 & -3 & 1 \\ -4 & 5 & -1 \end{bmatrix}$,求$AB,BA$以及$AB-BA$。
\end{ex}

\begin{ex} \label{ex:3.4}
设$A = \begin{bmatrix} 1 & 2 & -1 \\ 2 & 3 & 2 \\ -1 & 0 & 2 \end{bmatrix}, B = \begin{bmatrix} 0 & 1 & 2 \\ 2 & -1 & 0 \\ -1 & -1 & 3 \end{bmatrix}$,计算$A^T,B^T,AB,A^TB^T$。
\end{ex}

\begin{ex} \label{ex:3.5}
设$A = \begin{bmatrix} 1 & -3 & 4 \\ 0 & 1 & -2 \\ 2 & -3 & 3 \end{bmatrix}$,求$A$的伴随矩阵。
\end{ex}

\begin{ex} \label{ex:3.6}
以下方阵是否可逆?如果可逆,求其逆矩阵。

$$\begin{bmatrix} \cos\varphi & -\sin\varphi \\ \sin\varphi & \cos\varphi \end{bmatrix}$$
\end{ex}

\begin{ex} \label{ex:3.7}
求以下可逆阵的逆矩阵。

\enum
\item[(1)] $\begin{bmatrix} a & b \\ c & d \end{bmatrix}$,满足$ad-bc\neq 0$
\item[(2)] $\begin{bmatrix}  \cos\varphi & 0 & -\sin\varphi \\ 0 & 1 & 0 \\ \sin\varphi & 0 & \cos\varphi \end{bmatrix}$
\end{list}
\end{ex}

\begin{ex}\  \label{ex:3.8}

\enum
\item[(1)] 设$A\in M_n(\mathbb{R}), B, E\in M_{n\times m}(\mathbb{R}), C, F\in M_{m\times n}(\mathbb{R}), D\in M_m(\mathbb{R})$,证明:矩阵$\begin{bmatrix} A & B \\ C & D \end{bmatrix}$ 与矩阵$\begin{bmatrix} A + EC & B + ED \\ C & D \end{bmatrix},$ $\begin{bmatrix} A & B \\ FA + C & FB + D \end{bmatrix},$ $\begin{bmatrix} A & AE + B \\ C & CE + D \end{bmatrix},$ $\begin{bmatrix} A + BF & B \\ C + DF & D \end{bmatrix}$均有相同的行列式。

\item[(2)] 设$U,V\in M_n(\mathbb{R})$,证明$\det(I_n - UV) = \det(I_n-VU)$。

\item[(3)] 设$P\in M_{n\times (n-1)}(\mathbb{R}), Q\in M_{(n - 1)\times n}(\mathbb{R})$。证明$PQ \in M_n(\mathbb{R})$的行列式为$0$。

\item[(4)] 设$\alpha, \beta$为$n$阶列向量,证明$\det(I_n - \alpha\beta^T) = 1 - \alpha^T\beta$。

\item[(5)] 计算$\begin{bmatrix} \lambda & 1 & \cdots & 1 \\ 1 & \lambda & & \vdots \\ \vdots & & \ddots & 1 \\ 1 & \cdots & 1 & \lambda \end{bmatrix} \in M_n(\mathbb{R})$的行列式。
\end{list}
\end{ex}

\begin{ex} \label{ex:3.9}
设$A, B$为同阶方阵,证明$\begin{vmatrix} A & B \\ B & A \end{vmatrix} = |A + B||A - B|$。
\end{ex}

\begin{ex} \label{ex:3.10}
对$n\geqslant 1$,构造$n$阶方阵$A, B$,使得$A, B, A+B$均可逆,但$(A+B)^{-1}\neq A^{-1}+B^{-1}$。
\end{ex}

\begin{ex} \label{ex:3.11}
设$A = \begin{bmatrix} 0 & -1 & 2 \\ -1 & 0 & -1 \end{bmatrix}, B = \begin{bmatrix} 2 & -1 & -3 \\ 1 & -3 & -3 \\ 1 & 0 & -3 \end{bmatrix}, C = \begin{bmatrix} 2 & -2 \\ 1 & -2 \\ 2 & 2 \end{bmatrix}$, 求$ABC$。
\end{ex}

\begin{ex} \label{ex:3.12}
计算

\enum
\item[(1)] $\begin{bmatrix} \cos\varphi & -\sin\varphi \\ \sin\varphi & \cos\varphi \end{bmatrix}^n$

\item[(2)] $\begin{bmatrix} 1 & \alpha & \beta \\ & 1 & \alpha \\ & & 1 \end{bmatrix}^{n}$

\item[(3)] $\begin{bmatrix} 0 & 1 & 0 & 0 \\ 0 & 0 & 1 & 0 \\ 0 & 0 & 0 & 1 \\ 0 & 0 & 0 & 0\end{bmatrix}^3$

\item[(4)] $\begin{bmatrix} \lambda & 1 & 0 & 0 \\ 0 & \lambda & 1 & 0 \\ 0 & 0 & \lambda & 1 \\ 0 & 0 & 0 & \lambda\end{bmatrix}^3$
\end{list}
\end{ex}

\begin{ex} \label{ex:3.13}
解矩阵方程:

\enum
\item[(1)] $\begin{bmatrix} 1 & 1 \\ 2 & 3 \end{bmatrix} X = \begin{bmatrix} 1 & 1 & 1 \\ 1 & 1 & 1 \end{bmatrix}$

\item[(2)] $\begin{bmatrix} 1 & 0 & 4 \\ -1 & 1 & -2 \\ 0 & 0 & 1 \end{bmatrix} X \begin{bmatrix} -1 & -3 & 0 \\ 1 & 4 & 4 \\ -1 & -3 & -1 \end{bmatrix} = \begin{bmatrix} 1 & 0 & -6 \\ -5 & -13 & 3 \\ -1 & -4 & -3 \end{bmatrix}$
\end{list}
\end{ex}

\begin{ex} \label{ex:3.14}
设$A = \begin{bmatrix} c & 0 & 0 \\ 1 & c & 0 \\ 0 & 1 & c \end{bmatrix}, f(\lambda) = \lambda^2 - 2\lambda - 1$。求$f(A)$。
\end{ex}

\begin{ex} \label{ex:3.15}
设$A\in M_n(\mathbb{R})$为一个$n$阶方阵,

\enum
\item[(1)] 证明:如果$A$是对角阵且其主对角线上的元素各不相同,则任一与$A$乘法可交换的矩阵也是对角阵。

\item[(2)] 设$P\in M_n(\mathbb{R})$, $P$可逆,$PAP^{-1}$为对角阵且其对角线上元素各不相同。设$B\in M_n(\mathbb{R}), AB = BA$。证明$PBP^{-1}$为对角阵。
\end{list}
\end{ex}

\begin{ex} \label{ex:3.16}
求平方等于单位阵的所有二阶方阵。
\end{ex}

\begin{ex} \label{ex:3.17}
求幂等(即平方等于自身)的所有二阶方阵。
\end{ex}

\begin{ex} \label{ex:3.18}
设$A = (a_{ij})$为$n$阶上三角方阵且对角线上的元素均为$0$,求$A^{n-1}, A^n$。
\end{ex}

\begin{ex} \label{ex:3.19}
证明:

\enum
\item[(1)] 对方阵$A = (a_{ij}) \in M_n(\mathbb{R})$,定义$A$的迹(trace)为$tr(A) = \sum\limits_{i=1}^n a_i$。证明:任取方阵$A, B\in M_n(\mathbb{R})$,总有$tr(AB) = tr(BA)$。

\item[(2)] 对任意方阵$A, B\in M_n(\mathbb{R})$, $AB - BA \neq I_n$。
\end{list}
\end{ex}

\begin{ex} \label{ex:3.20}
所有与下列方阵$A$可交换的方阵$B$(即满足$AB = BA$):

\enum
\item[(1)] $A = \begin{bmatrix} 1 & 1 \\ 0 & 0 \end{bmatrix}$ \item[(2)] $\begin{bmatrix} 0 & 1 & 0 & 0 \\ 0 & 0 & 1 & 0 \\ 0 & 0 & 0 & 1 \\ 0 & 0 & 0 & 0 \end{bmatrix}$
\item[(3)] $\begin{bmatrix} 1 & & & \\ & 1 & & \\ & & 2 & \\ & & & 2 \end{bmatrix}$
\end{list}
\end{ex}

\begin{ex} \label{ex:3.21}
证明:与所有$n$阶方阵均可交换的方阵必为数量阵(即$aI_n, a\in\mathbb{R}$)。
\end{ex}

\begin{ex} \label{ex:3.22}
设$J = \begin{bmatrix} & I_n \\ -I_n & \end{bmatrix} \in M_{2n}(\mathbb{R})$,求

\enum
\item[(1)] $\det J$
\item[(2)] $J^2$
\end{list}
\end{ex}

\begin{ex} \label{ex:3.23}
设$A$为实对称矩阵,若$A^2 = 0$,求证$A = 0$。
\end{ex}

\begin{ex} \label{ex:3.24}
证明:任一方阵均可表示为一个对称阵和一个反对称阵之和。
\end{ex}

\begin{ex} \label{ex:3.25}
证明:若$A, B, A + B$均为可逆阵,证明$A^{-1} + B^{-1}$也为可逆阵。
\end{ex}

\begin{ex} \label{ex:3.26}
设$A$是$n$阶方阵,满足$A^k = 0, A^{k-1} \neq 0$,(这样的方阵被称为一个幂零阵),证明$I_n - A$可逆。
\end{ex}

\begin{ex} \label{ex:3.27}
设$A$是$n$阶方阵,满足$A^2 = A$(这样的方阵被称为一个幂等阵),证明$A + I_n$可逆。
\end{ex}

\begin{ex} \label{ex:3.28}
证明:
\enum
\item[(1)] 上(下)三角阵的逆矩阵(如果存在)也是上(下)三角阵。
\item[(2)] 对称阵的逆矩阵(如果存在)也是对称阵。
\item[(3)] 反对称阵的逆矩阵(如果存在)也是反对称阵。
\end{list}
\end{ex}

\begin{ex} \label{ex:3.29}
设$A$是$n$阶可逆阵,若$A$的每一行元素之和等于常数$c$,求证:$c\neq 0$,且$A^{-1}$的每一行元素之和等于$c^{-1}$。
\end{ex}

\begin{ex} \label{ex:3.30}
设$A$为$n$阶可逆阵,$u,v$是$n$维列向量,且$1 + v^T A^{-1}u \neq 0$,求证
$$(A+uv^T)^{-1} = A^{-1} - \frac{A^{-1}uv^T A^{-1}}{1 + v^T A^{-1}u}.$$
\end{ex}

\begin{ex} \label{ex:3.31}
设$A$为$n$阶可逆阵,$U$是$n\times k$阶矩阵,$V$是$k\times n$阶矩阵,$C$是$k\times k$阶矩阵,且$ C^{-1}+VA^{-1}U$可逆,求证
$$ \left(A+UCV \right)^{-1} = A^{-1} - A^{-1}U \left(C^{-1}+VA^{-1}U \right)^{-1} VA^{-1}.$$
\end{ex}

\newpage

%%%%%%%%%%%%%%%%%%%%%%%%%%%%%%%%%%%%%%%%%%%%%%%%%%%%%%%%%%%%%%%%%%%%%%%%%%%%%%%%%

\section{习题答案}

\textbf{习题\ref{ex:3.1} 解答:}

例如
\begin{eqnarray*}
\text{对角阵} & : & \begin{bmatrix} 1 & & \\ & 2 & \\ & & 3 \end{bmatrix}, \\
\text{上三角阵} & : & \begin{bmatrix} 1 & 1 & 0 \\ 0 & 1 & 0 \\ 0 & 0 & 3 \end{bmatrix}, \\
\text{对称阵} & : & \begin{bmatrix} 1 & 2 & 3\\ 2 & 2 & 1 \\ 3 & 1 & -1 \end{bmatrix}, \\
\text{反对称阵} & : & \begin{bmatrix} 0 & 2 & 1\\ -2 & 0 & -1 \\ -1 & 1 & 0 \end{bmatrix}
\end{eqnarray*}

\vspace{1.5em}

\textbf{习题\ref{ex:3.2} 解答:}

$n \geqslant 2$时,取$B = A$,且$A$可逆,即有$\det(A+B) = \det(2A) = 2^n\det A \neq 2\det A = \det(A)+\det(B)$。

\vspace{1.5em}

\textbf{习题\ref{ex:3.3} 解答:}

$
AB = \begin{bmatrix} 9 & -5 & 5 \\ 6 & -12 & -2 \\ 12 & -14 & 8 \end{bmatrix},
BA = \begin{bmatrix} 9 & -2 & -14 \\ 5 & -4 & 2 \\ -10 & 12 & 0 \end{bmatrix},
AB - BA = \begin{bmatrix} 0 & -3 & 19 \\  1 & -8 & -4 \\ 22 & -26 & 8 \end{bmatrix}.
$

\vspace{1.5em}

\textbf{习题\ref{ex:3.4} 解答:}

$A^T = \begin{bmatrix} 1 & 2 & -1 \\ 2 & 3 & 0 \\ -1 & 2 & 2 \end{bmatrix}, B^T = \begin{bmatrix} 0 & 2 & -1 \\ 1 & -1 & -1 \\ 2 & 0 & 3 \end{bmatrix}, AB = \begin{bmatrix} 5 & 0 & -1 \\ 4  & -3 & 10 \\ -2 & -3 & 4 \end{bmatrix}, A^TB^T = \begin{bmatrix} 0 & 0 & -6 \\ 3 & 1 & -5 \\ 6 & -4 & 5 \end{bmatrix}$。

\vspace{1.5em}

\textbf{习题\ref{ex:3.5} 解答:}

设$A^{\ast} = (b_{ij})_{1 \leqslant i,j \leqslant 3}$,那么$b_{ij} = A_{ji}$,$A_{ji}$为矩阵$A$的第$(j,i)$位元素的代数余子式。$A_{11} = \begin{vmatrix} 1 & -2 \\ -3 & 3 \end{vmatrix} = -3$,类似可以算出其他矩阵元素的值。最后的结论是:
$$A^{\ast} = \begin{bmatrix} -3 & -3 & 2 \\ -4 & -5 & 2 \\ -2 & -3 & 1 \end{bmatrix}$$

\vspace{1.5em}

\textbf{习题\ref{ex:3.6} 解答:}

行列式为$\cos^2\varphi + \sin^2\varphi = 1$,所以可逆。逆为$\begin{bmatrix} \cos\varphi & \sin\varphi \\ -\sin\varphi & \cos\varphi \end{bmatrix}$。

\vspace{1.5em}

\textbf{习题\ref{ex:3.7} 解答:}

\enum
\item[(1)] $\begin{bmatrix} a & b \\ c & d \end{bmatrix}^{-1} = \frac{1}{ad-bc}\begin{bmatrix} d & -b \\ -c & a \end{bmatrix}$
\item[(2)] $\begin{bmatrix}  \cos\varphi & 0 & -\sin\varphi \\ 0 & 1 & 0 \\ \sin\varphi & 0 & \cos\varphi \end{bmatrix}^{-1} = \begin{bmatrix}  \cos\varphi & 0 & \sin\varphi \\ 0 & 1 & 0 \\ -\sin\varphi & 0 & \cos\varphi \end{bmatrix}$
\end{list}

\vspace{1.5em}

\textbf{习题\ref{ex:3.8} 解答:}

\enum
\item[(1)] $\det\begin{bmatrix} A & B \\ C & D \end{bmatrix} = \det(\begin{bmatrix} I_n & E \\ 0 & I_m \end{bmatrix} \begin{bmatrix} A & B \\ C & D \end{bmatrix}) = \det \begin{bmatrix} A + EC & B + ED \\ C & D \end{bmatrix}$。其余类似可以证明。

\item[(2)] 由第(1)小题,$\det(I_n - UV) = \det\begin{bmatrix} I_n - UV & 0 \\ V & I_n \end{bmatrix} = \det\begin{bmatrix} I_n & U \\ V & I_n \end{bmatrix} = \det\begin{bmatrix} I_n & 0 \\ V & I_n - VU \end{bmatrix} = \det(I_n-VU)$。

\item[(3)] %因为$P,Q$的秩都至多为$n-1$,所以$PQ$的秩至多为$n-1$,不满秩,所以行列式为0。
假设$\det PQ \neq 0,$ 那么齐次线性方程组$PQX = 0, X = \begin{bmatrix} x_1 \\ \vdots \\ x_n \end{bmatrix},$ 只有零解。但是$QX=0$是$n-1$个方程,$n$个变元的齐次线性方程组,必有非零解$X_0,$ 于是$PQX_0 = P(QX_0) = 0,$ 与$PQX = 0$只有零解矛盾。因此$PQ$的行列式为0。

\item[(4)] 由第(1)小题,$\det(I_n - \alpha\beta^T) = \det\begin{bmatrix} I_n - \alpha\beta^T & \alpha \\ 0 & 1 \end{bmatrix} = \det\begin{bmatrix} I_n & \alpha \\ \beta^T & 1 \end{bmatrix} = \det\begin{bmatrix} I_n & 0 \\ \beta^T & 1 - \alpha^T\beta \end{bmatrix} = 1 - \alpha^T\beta$。

\item[(5)] $\begin{vmatrix} \lambda & 1 & \cdots & 1 \\ 1 & \lambda & & \vdots \\ \vdots & & \ddots & 1 \\ 1 & \cdots & 1 & \lambda \end{vmatrix} = \left| (\lambda - 1)I_n + \begin{bmatrix} 1 & 1 & \cdots & 1 \\ 1 & 1 & & \vdots \\ \vdots & & \ddots & 1 \\ 1 & \cdots & 1 & 1 \end{bmatrix} \right| = \left| (\lambda - 1)I_n + \begin{bmatrix} 1 \\ 1 \\ \vdots \\ 1 \end{bmatrix}(1,1,\cdots,1) \right|\\
= (\lambda-1)^{n-1}(\lambda - 1 + n)$。
\end{list}

\vspace{1.5em}

\textbf{习题\ref{ex:3.9} 解答:}

$\begin{vmatrix} A & B \\ B & A \end{vmatrix} = \begin{vmatrix} A+B & B+A \\ B & A \end{vmatrix} = \begin{vmatrix} A+B & 0 \\ B & A-B \end{vmatrix} = |A + B||A - B|$

\vspace{1.5em}

\textbf{习题\ref{ex:3.10} 解答:}

例如$A = \begin{bmatrix} 3 & & & \\ & 1 & & \\ & & \ddots & \\ & & & 1 \end{bmatrix}, B = \begin{bmatrix} 2 & & & \\ & 1 & & \\ & & \ddots & \\ & & & 1 \end{bmatrix}$,那么
$$(A+B)^{-1} = \begin{bmatrix} 5 & & & \\ & 1 & & \\ & & \ddots & \\ & & & 1 \end{bmatrix}^{-1} = \begin{bmatrix} \frac15 & & & \\ & 1 & & \\ & & \ddots & \\ & & & 1 \end{bmatrix},$$
而
$$A^{-1}+B^{-1} = \begin{bmatrix} \frac13 & & & \\ & 1 & & \\ & & \ddots & \\ & & & 1 \end{bmatrix} + \begin{bmatrix} \frac12 & & & \\ & 1 & & \\ & & \ddots & \\ & & & 1 \end{bmatrix} = \begin{bmatrix} \frac56 & & & \\ & 1 & & \\ & & \ddots & \\ & & & 1 \end{bmatrix},$$
二者并不相等。

\vspace{1.5em}

\textbf{习题\ref{ex:3.11} 解答:}

直接矩阵相乘,$ABC = \begin{bmatrix} -1 & -14 \\ 7 & 16 \end{bmatrix}$。

\vspace{1.5em}

\textbf{习题\ref{ex:3.12} 解答:}

\enum
\item[(1)] $\begin{bmatrix} \cos n\varphi & -\sin n\varphi \\ \sin n\varphi & \cos n\varphi \end{bmatrix}$

\item[(2)] $\begin{bmatrix} 1 & n\alpha & \frac{n(n-1)}{2}\alpha^2 + n\beta \\ & 1 & n\alpha \\ & & 1 \end{bmatrix}$

\item[(3)] $\begin{bmatrix} 0 & 0 & 0 & 1 \\ 0 & 0 & 0 & 0 \\ 0 & 0 & 0 & 0 \\ 0 & 0 & 0 & 0\end{bmatrix}$

\item[(4)] $\begin{bmatrix} \lambda^3 & 3\lambda^2 & 3\lambda & 1 \\ 0 & \lambda^3 & 3\lambda^2 & 3\lambda \\ 0 & 0 & \lambda^3 & 3\lambda^2 \\ 0 & 0 & 0 & \lambda^3\end{bmatrix}$
\end{list}

\vspace{1.5em}

\textbf{习题\ref{ex:3.13} 解答:}

\enum
\item[(1)] $X = \begin{bmatrix} 1 & 1 \\ 2 & 3 \end{bmatrix}^{-1} \begin{bmatrix} 1 & 1 & 1 \\ 1 & 1 & 1 \end{bmatrix} = \begin{bmatrix} 2 & 2 & 2 \\ -1 & -1 & -1 \end{bmatrix}$,

\item[(2)] $X = \begin{bmatrix} 1 & 0 & 4 \\ -1 & 1 & -2 \\ 0 & 0 & 1 \end{bmatrix}^{-1} \cdot \begin{bmatrix} 1 & 0 & -6 \\ -5 & -13 & 3 \\ -1 & -4 & -3 \end{bmatrix} \cdot \begin{bmatrix} -1 & -3 & 0 \\ 1 & 4 & 4 \\ -1 & -3 & -1 \end{bmatrix}^{-1} = \begin{bmatrix} -2 & 1 & -2 \\ 2 & 1 & 1 \\ 1 & -1 & -1 \end{bmatrix}$。
\end{list}

\vspace{1.5em}

\textbf{习题\ref{ex:3.14} 解答:}

$f(A) = \begin{bmatrix} c^2 - 2c - 1 & 0 & 0 \\ 2c - 2 & c^2 - 2c - 1 & 0 \\ 1 & 2c - 2 & c^2 - 2c - 1 \end{bmatrix}$。

\vspace{1.5em}

\textbf{习题\ref{ex:3.15} 解答:}

\enum
\item[(1)] 设$A = diag(\lambda_1,\cdots,\lambda_n)$,$B = (b_{ij})$与$A$可交换。分别考察$AB$与$BA$的第$(i,j)$位元素,便有$\lambda_ib_{ij} = \lambda_jb_{ij}$。由假设,如果$i\neq j$,便有$\lambda_i\neq\lambda_j$,因此必须有$b_{ij} = 0$。也就是说$B$必须为对角阵。

\item[(2)] 由于$AB = BA$,所以$PBP^{-1}$可以与$PAP^{-1}$交换,由第(1)小题结论以及第(2)小题题设即可知$PBP^{-1}$也为对角阵。
\end{list}

\vspace{1.5em}

\textbf{习题\ref{ex:3.16} 解答:}

设$A = \begin{bmatrix} a & b \\ c & d \end{bmatrix}$是平方等于单位阵的二阶方阵,那么我们会有
\begin{numcases}{ }
  a^2 + bc = 1  \notag \\
  ab + bd = 0 \notag \\
  ac + cd = 0 \notag \\
  d^2 + bc = 1 \notag
\end{numcases}
由$ab + bd = 0$知$b = 0$或$a + d = 0$。如果$b = 0$,那么由$a^2 + bc = 1,d^2 + bc = 1$知$a=\pm 1, d=\pm 1$,再由$ac + cd = 0$知当$a,d$符号相同时,$c = 0$,符号不同时,$c$可以任取。如果$b \neq 0$,那么必须有$a + d = 0$。只要再满足$a^2 + bc = 1$,其他两式自动成立。所以平方等于单位阵的二阶方阵有$\pm I_2$,以及
$$\begin{bmatrix} a & b \\ c & -a \end{bmatrix}, \quad a^2 + bc = 1$$

\vspace{1.5em}

\textbf{习题\ref{ex:3.17} 解答:}

类似第16题可以得幂等的二阶方阵有$0, I_2$,以及
$$\begin{bmatrix} a & b \\ c & 1-a \end{bmatrix}, \quad a^2 + bc = a$$

\vspace{1.5em}

\textbf{习题\ref{ex:3.18} 解答:}

$A^n = 0$。

$A^{n-1}$为除了其第$(1,n)$位元素外其余元素都为0的矩阵(第$(1,n)$位元素也可能为0):
$$\begin{bmatrix}
0 & \cdots & 0 & \ast \\ 0 & \cdots & 0 & 0 \\ \vdots & & \vdots & \vdots \\ 0 & \cdots & 0 & 0
\end{bmatrix}$$

\vspace{1.5em}

\textbf{习题\ref{ex:3.19} 解答:}

\enum
\item[(1)] 设$A = (a_{ij}), B = (b_{ij})$,那么
$$tr(AB) = \sum\limits_{i=0}^n\sum\limits_{j=0}^n{a_{ij}b_{ji}} = \sum\limits_{i=0}^n\sum\limits_{j=0}^n{b_{ij}a_{ji}} = tr(BA)$$

\item[(2)] 对任意方阵$A, B\in M_n(\mathbb{R})$, 有$tr(AB - BA) = tr(AB) - tr(BA)= 0 \neq n = tr(I_n)$,所以必然有$AB - BA \neq I_n$。
\end{list}

\vspace{1.5em}

\textbf{习题\ref{ex:3.20} 解答:}

\enum
\item[(1)] 设$B = \begin{bmatrix} b_1 & b_2 \\ b_3 & b_4 \end{bmatrix}$与$A$可交换,那么有
$$\begin{bmatrix} 1 & 1 \\ 0 & 0 \end{bmatrix} \begin{bmatrix} b_1 & b_2 \\ b_3 & b_4 \end{bmatrix} =  \begin{bmatrix} b_1 & b_2 \\ b_3 & b_4 \end{bmatrix} \begin{bmatrix} 1 & 1 \\ 0 & 0 \end{bmatrix}$$
即$\begin{bmatrix} b_1 + b_3 & b_2 + b_4 \\ 0 & 0 \end{bmatrix} =  \begin{bmatrix} b_1 & b_1 \\ b_3 & b_3 \end{bmatrix}$,那么$b_3 = 0, b_1 = b_2 + b_4$即可。

\item[(2)] 类似第(1)小题可得答案为形如$\begin{bmatrix} a & b & c & d \\ & a & b & c \\ & & a & b \\ & & & a \end{bmatrix}, a, b, c, d \in \mathbb{R}$,的矩阵。

\item[(3)] $\begin{bmatrix} A_2 & 0 \\ 0 & B_2 \end{bmatrix}$,其中$A_2, B_2$是2个2阶方阵。
\end{list}

\vspace{1.5em}

\textbf{习题\ref{ex:3.21} 解答:}

设方阵$A = (a_{ij})_{1 \leqslant i,j \leqslant n}$与所有$n$阶方阵均可交换,特别的,$A$与方阵$E_{st}$可交换,其中$E_{st}$为第$(s,t)$位元素为1,其余为0的方阵。那么
\begin{eqnarray*}
AE_{st} & = & \begin{bmatrix} 0 & \cdots & 0 & a_{1s} & 0 & \cdots & 0 \\ \vdots & & \vdots & \vdots & \vdots & & \vdots \\ 0 & \cdots & 0 & a_{ns} & 0 & \cdots & 0 \end{bmatrix} \qquad (a_{1s},\cdots,a_{ns}\text{ 位于第$t$列}) \\
E_{st}A & = & \begin{bmatrix} 0 & \cdots & 0 \\ \vdots & & \vdots \\ 0 & \cdots & 0 \\ a_{t1} & \cdots & a_{tn} \\ 0 & \cdots & 0 \\ \vdots & & \vdots \\ 0 & \cdots & 0 \end{bmatrix} \qquad (a_{t1}, \cdots, a_{tn}\text{ 位于第$s$行})
\end{eqnarray*}
$AE_{st} = E_{st}A \Rightarrow a_{ss} = a_{tt}$;以上两个矩阵其余元素都是$0$。取遍所有的$E_{st}$,便可以知道$A$的对角线元素都相等,其余元素都必须为$0$,也就是说$A$必为数量阵。

\vspace{1.5em}

\textbf{习题\ref{ex:3.22} 解答:}

\enum
\item[(1)] 将矩阵$J$的第$1$列与第$n+1$列交换,第$2$列与第$n+2$列交换,……,第$n$列与第$2n$列交换,得到的矩阵为$\begin{bmatrix} I_n & \\ & -I_n \end{bmatrix}$,所以
$$\det J = (-1)^n \det\begin{bmatrix} I_n & \\ & -I_n \end{bmatrix} = (-1)^n\times(-1)^n = 1$$

\item[(2)] $J^2 = \begin{bmatrix} & I_n \\ -I_n & \end{bmatrix} \cdot \begin{bmatrix} & I_n \\ -I_n & \end{bmatrix} = \begin{bmatrix} -I_n & \\ & -I_n \end{bmatrix} = -I_{2n}$
\end{list}

\vspace{1.5em}

\textbf{习题\ref{ex:3.23} 解答:}

若$A$为实对称矩阵,那么$A = A^T.$ 由$A^2 = 0$知$AA^T = 0.$ 记$A = (a_{ij})_{1 \leqslant i,j \leqslant n},$ $AA^T = (c_{ij})_{1 \leqslant i,j \leqslant n},$ 那么
$$c_{ii} = \sum\limits_{j = 1}^n a_{ij}^2 = 0, \quad i = 1, \cdots n,$$
因此$a_{ij} = 0,$ 对任意的$1 \leqslant i,j \leqslant n,$ 也就是说$A=0.$
%若$A$为实对称矩阵,那么存在可逆方阵$P$,使得$P^TAP = diag(\lambda_1,\cdots,\lambda_n)$为对角矩阵。若$A^2 = 0$,那么$P^TAP = 0$,从而有$diag(\lambda_1^2,\cdots,\lambda_n^2) = 0$,所以每个$\lambda_i$都必须为0。所以$A = 0$。

\vspace{1.5em}

\textbf{习题\ref{ex:3.24} 解答:}

任取方阵$A$,有
$$A = \frac{A+A^T}{2} + \frac{A-A^T}{2},$$
$\frac{A+A^T}{2}$为对称阵,$\frac{A-A^T}{2}$为反对称阵。

\vspace{1.5em}

\textbf{习题\ref{ex:3.25} 解答:}

$A^{-1} + B^{-1} = A^{-1}(A + B)B^{-1}$,能表示成三个可逆阵之积,所以也是可逆阵。

\vspace{1.5em}

\textbf{习题\ref{ex:3.26} 解答:}

直接验证$(I_n + A + A^2 + \cdots A^{k-1})(I_n - A) = I_n$。

\vspace{1.5em}

\textbf{习题\ref{ex:3.27} 解答:}

直接验证$(A+I_n)(-\dfrac{A}{2}+I_n) = -\dfrac{A^2}{2} + \dfrac{A}{2} + I_n = I_n.$
%因为$A^2 = A$,所以$A$的特征值只可能有0和1,也就是说存在可逆阵$P$,使得$P^{-1}AP = \Lambda$,其中$\Lambda$为一个对角线元素为0或1的上三角矩阵。那么$P^{-1}(A+I_n)P$便是一个对角线元素为1或2的上三角矩阵,自然是可逆阵。

\vspace{1.5em}

\textbf{习题\ref{ex:3.28} 解答:}

\enum
\item[(1)] 根据矩阵的乘法,很容易看出。

\item[(2)] 设$A$是$n$阶对称阵可逆矩阵,那么$I_n = (AA^{-1})^T = (A^{-1})^T A^T = (A^{-1})^T A$,由于矩阵逆是唯一的,所以$A^{-1} = (A^{-1})^T$。

\item[(3)] 与第(2)小题类似可做。
\end{list}

\vspace{1.5em}

\textbf{习题\ref{ex:3.29} 解答:}

设$A = (a_{ij})_{1\leqslant i,j \leqslant n}$,那么把第$2,3,\cdots,n$列加到第1列上,我们有
\begin{align*}
\det A & = \begin{vmatrix}
a_{11} & a_{12} & \cdots & a_{1n} \\ \vdots & \vdots & & \vdots \\ a_{n1} & a_{n2} & \cdots & a_{nn}
\end{vmatrix} = \begin{vmatrix}
a_{11} + a_{12} + \cdots + a_{1n} & a_{12} & \cdots & a_{1n} \\ \vdots & \vdots & & \vdots \\ a_{n1} + a_{n2} + \cdots + a_{nn} & a_{n2} & \cdots & a_{nn}
\end{vmatrix} \\
& = \begin{vmatrix}
c & a_{12} & \cdots & a_{1n} \\ \vdots & \vdots & & \vdots \\ c & a_{n2} & \cdots & a_{nn}
\end{vmatrix} = c \cdot \begin{vmatrix}
1 & a_{12} & \cdots & a_{1n} \\ \vdots & \vdots & & \vdots \\ 1 & a_{n2} & \cdots & a_{nn}
\end{vmatrix}
\end{align*}
所以必须有$c\neq 0$,否则$\det A = 0$,与$A$是$n$阶可逆阵矛盾。

将$\begin{vmatrix}
1 & a_{12} & \cdots & a_{1n} \\ \vdots & \vdots & & \vdots \\ 1 & a_{n2} & \cdots & a_{nn}
\end{vmatrix}$按第一列展开,我们有
$$\det A = c\cdot(A_{11} + A_{21} + \cdots + A_{n1}),$$
其中$A_{ij}$为$a_{ij}$的代数余子式,正好是$A$的伴随矩阵$A^{\ast}$的第$(j,i)$位元素,而$A^{\ast} = \det A \cdot A^{-1}$。也就是说我们有
$$\det A = c\cdot(\det A \cdot A^{-1}\text{的第一行元素之和}).$$
即有
$$1 = c\cdot(A^{-1}\text{的第一行元素之和}).$$

同样地,如果把$A$的非$i$列元素都加到第$i$列上,再按第$i$列展开,我们有
$$1 = c\cdot(A^{-1}\text{的第$i$行元素之和})$$。
所以$A^{-1}$的每一行元素之和都等于$c^{-1}$。

\vspace{1.5em}

\textbf{习题\ref{ex:3.30} 解答:}

可以直接验证:
\begin{eqnarray*}
& & (A+uv^T) (A^{-1} - {A^{-1}uv^T A^{-1} \over 1 + v^T A^{-1}u}) \\
& = & AA^{-1} +  uv^T A^{-1} - {AA^{-1}uv^T A^{-1} + uv^T A^{-1}uv^T A^{-1} \over 1 + v^TA^{-1}u} \\
& = & I +  uv^T A^{-1} - {uv^T A^{-1} + uv^T A^{-1}uv^T A^{-1} \over 1 + v^T A^{-1}u} \\
& = & I + uv^T A^{-1} - {u(1 + v^T A^{-1}u) v^T A^{-1} \over 1 + v^T A^{-1}u} \\
& = & I + uv^T A^{-1} - {1 + v^T A^{-1}u \over 1 + v^T A^{-1}u}uv^T A^{-1} \\
& = & I + uv^T A^{-1} - uv^T A^{-1} \\
& = & I_n.
\end{eqnarray*}

另解:也可以通过观察,发现$(A+uv^T)^{-1}$是以下方程
$$\begin{bmatrix} A & u \\ v^T & -1 \end{bmatrix}\begin{bmatrix} X \\ Y \end{bmatrix} = \begin{bmatrix} I \\ 0 \end{bmatrix}.$$
的解$X$,然后通过解矩阵方程组
$$\begin{cases}
AX + uY = I \\
v^TX - Y = 0
\end{cases}$$
由第一个方程有$X = A^{-1}(I-uY)$,代入第二个方程得$v^TA^{-1}(I-uY) = Y$。合并同类项得$(1 + v^TA^{-1}u)^{-1}v^TA^{-1} = Y$。再将$Y$代回到第一个方程,就有
$$(A+uv^T)^{-1} = X = A^{-1} - {A^{-1}uv^T A^{-1} \over 1 + v^T A^{-1}u}.$$

\vspace{1.5em}

\textbf{习题\ref{ex:3.31} 解答:}

这是上一题的推广,证明方法类似。

%%%%%%%%%%%%%%%%%%%%%%%%%%%%%%%%%%%%%%%%%%%%%%%%%%%%%%%%%%%%%%%%%%%%%%%%%%%%%%%%%%%%%%%
%%%%%%%%%%%%%%%%%%%%%%%%%%%%%%%%%%%%%%%%%%%%%%%%%%%%%%%%%%%%%%%%%%%%%%%%%%%%%%%%%%%%%%%

\chapter{向量空间}

\section{知识点解析}
\begin{Def}
设 $\mathbb{R}$ 是实数集, $n$ 个数 $a_1, a_2 ,\dots, a_n \in\mathbb{R}$,  组成的有序数组 $(a_1, a_2 ,\dots, a_n)$ 称为 $n$ 维向量,  记作

\begin{displaymath}
\vec{\alpha}=(a_1,a_2,\dots,a_n) \ \mbox{或}\ \vec{\alpha}=\begin{bmatrix}a_1\\a_2\\ \vdots \\ a_n\end{bmatrix}=(a_1,a_2,\dots,a_n)^T
\end{displaymath}

其中 $a_i$ 称为向量的第 $i$ 个分量.  前一个表示式称为行向量,  后一个称为列向量.  一般用带箭头的希腊字母表示一个向量,且默认为列向量.

\end{Def}

\begin{Def}
设有两个 $n$ 维向量: $\vec{\alpha}=(a_1,a_2,\dots,a_n)^T, \vec{\beta}=(b_1,b_2,\dots,b_n)^T$, 称$\vec{\alpha}$与$\vec{\beta}$是相等的向量当且仅当它们对应分量全相等, 即

\begin{displaymath}
\vec{\alpha}=\vec{\beta}\Leftrightarrow a_i=b_i, (i=1,2,\dots,n).
\end{displaymath}

\end{Def}

\begin{Def}
设 $n$ 维向量: $\vec{\alpha}=(a_1,a_2,\dots,a_n)^T, \vec{\beta}=(b_1,b_2,\dots,b_n)^T$ 和数$k\in\mathbb{R}$, 则定义:

\begin{displaymath}\begin{aligned}
&(1)\vec{\alpha}+\vec{\beta}=(a_1+b_1,a_2+b_2,\dots,a_n+b_n)^T.\\
&(2)k\vec{\alpha}=(ka_1,ka_2,\dots,ka_n)^T.\\
&(3)-\vec{\alpha}=(-1)\vec{\alpha}=(-a_1,-a_2,\dots,-a_n)^T.
\end{aligned}\end{displaymath}
\end{Def}

\begin{Def}
全体$n$维向量组成的集合, 当定义了上述向量的加法及数乘运算之后, 就称为 $n$ 维向量空间 (vector space), 记作 $\mathbb{R}^n$.

\end{Def}

\begin{Def}
设$W$是$\mathbb{R}^n$中的非空子集合,满足

(1)对向量的加法是封闭的, 即: $\forall \vec{\alpha},\vec{\beta}\in W$, 有$\vec{\alpha}+\vec{\beta}\in W$.

(2) 对向量的数乘运算是封闭的, 即: $\forall \vec{\alpha}\in W, \forall k\in \mathbb{R}$, 有$k\vec{\alpha}\in W$.\\
则称$W$是$\mathbb{R}^n$的子空间.

\end{Def}

\begin{Def}
设 $\vec{\alpha}_1,\vec{\alpha}_2,\dots, \vec{\alpha}_s$是$s$个$n$维向量, $k_1,k_2,\dots,k_s\in\mathbb{R}$, 称$$k_1\vec{\alpha}_1+k_2\vec{\alpha}_2+\dots+k_s\vec{\alpha}_s$$是$\vec{\alpha}_1,\vec{\alpha}_2,\dots, \vec{\alpha}_s$ 的线性组合.

如果对于给定的向量$\vec{\beta}$而言, 若存在一组数$k_1,k_2,\dots,k_s\in\mathbb{R}$, 使得$$\vec{\beta}=k_1\vec{\alpha}_1+k_2\vec{\alpha}_2+\dots+k_s\vec{\alpha}_s$$
则称$\vec{\beta}$是$\vec{\alpha}_1,\vec{\alpha}_2,\dots, \vec{\alpha}_s$ 的一个线性组合, 也称 $\vec{\beta}$可由$\vec{\alpha}_1,\vec{\alpha}_2,\dots, \vec{\alpha}_s$线性表出.
\end{Def}

\begin{Def}
给定向量空间$\mathbb{R}^n$中$s$个向量$\vec{\alpha}_1,\vec{\alpha}_2,\dots, \vec{\alpha}_s$, 若存在不全为零的常数 $k_1, k_2,\dots, k_s$  使得
$$k_1\vec{\alpha}_1+k_2\vec{\alpha}_2+\dots+k_s \vec{\alpha}_s=\vec{0}.$$

则称这个向量组线性相关(linearly dependent);

否则, 称这个向量组线性无关(linearly independent).

\end{Def}

\begin{Def}
设有两个向量组 \\
(1) $\vec{\alpha}_1,\vec{\alpha}_2,\dots, \vec{\alpha}_s$\\
(2) $\vec{\beta}_1,\vec{\beta}_2,\dots,\vec{\beta}_t$.

如果向量组(1)中每个向量$\vec{\alpha}_i(i=1,2,3,\dots,s)$都可由向量组(2)中的向量$\vec{\beta}_1,\vec{\beta}_2,\dots,\vec{\beta}_t$线性表出, 则称向量组(1)可由向量组(2)线性表出.

如果同时向量组(1), (2)可以相互线性表出,  则称这两个向量组等价, 记为:
\begin{displaymath}
( \vec{\alpha}_1,\vec{\alpha}_2,\dots, \vec{\alpha}_s)\sim(\vec{\beta}_1,\vec{\beta}_2,\dots,\vec{\beta}_t)\ \ \mbox{或}\ \ (1)\sim(2).
\end{displaymath}

\end{Def}

\begin{thm}
设向量组 $\vec{\alpha}_1,\vec{\alpha}_2,\dots, \vec{\alpha}_s$可由向量组$\vec{\beta}_1,\vec{\beta}_2,\dots,\vec{\beta}_t$ 线性表出,

(1). 若$s>t$, 则$\vec{\alpha}_1,\vec{\alpha}_2,\dots, \vec{\alpha}_s$线性相关.

(2). 若$\vec{\alpha}_1,\vec{\alpha}_2,\dots, \vec{\alpha}_s$线性无关, 则$s\leq t$.
\end{thm}

\begin{Def}
如果向量组$\vec{\alpha}_1,\vec{\alpha}_2,\dots, \vec{\alpha}_s$ 中

(1) 存在$r$个线性无关的向量$\vec{\alpha}_{i_1},\vec{\alpha}_{i_2},\dots, \vec{\alpha}_{i_r}$;

(2)再加入任意一个向量$\vec{\alpha}_i(i=1,2,\dots,s)$都线性相关;\\
那么向量组$\vec{\alpha}_{i_1},\vec{\alpha}_{i_2},\dots, \vec{\alpha}_{i_r}$称为向量组$\vec{\alpha}_1,\vec{\alpha}_2,\dots, \vec{\alpha}_s$的极大线性无关组, 简称为极大无关组.


\end{Def}

\begin{thm}
如果矩阵 $A$ 经过初等行变换化为 $B$, 则$ A$ 与 $B$ 的列向量组的任何对应部分组有相同的线性相关性.

\end{thm}

\begin{Def}
向量组$\vec{\alpha}_1,\vec{\alpha}_2,\dots, \vec{\alpha}_s$的极大线性无关组中向量个数 $r$ 称为该向量组的秩, 记为$r(\vec{\alpha}_1,\vec{\alpha}_2,\dots, \vec{\alpha}_s)$或秩$(\vec{\alpha}_1,\vec{\alpha}_2,\dots, \vec{\alpha}_s)$.只含零向量的向量组的秩规定为0.
\end{Def}

\begin{thm}
若向量组 $A$ 可由向量组 $B$ 线性表出,  则 $r(A)\leq
r(B)$.
\end{thm}

\begin{Def}
设$V$表示$\mathbb{R}^n$或$\mathbb{R}^n$中的子空间, 则$\vec{\alpha}_1,\vec{\alpha}_2,\dots, \vec{\alpha}_s$是$V$中$s$个线性无关的向量, 且$V$中任何向量$\vec{\beta}$均可由$\vec{\alpha}_1,\vec{\alpha}_2,\dots, \vec{\alpha}_s$唯一地线性表示, 设为
$$\vec{\beta}=x_1\vec{\alpha}_1+x_2\vec{\alpha}_2+\dots+x_s \vec{\alpha}_s,$$
则我们称$\vec{\alpha}_1,\vec{\alpha}_2,\dots, \vec{\alpha}_s$为空间$V$的一组基, $x_1,x_2\dots,x_s$称为向量$\vec{\beta}$在$\vec{\alpha}_1,\vec{\alpha}_2,\dots, \vec{\alpha}_s$下的坐标, $s$称为$V$的维数, 记为$\dim V$.

\end{Def}

\begin{Def}
设$1\leq \min\{m,n\}$, 在矩阵$A=(a_{ij})_{m\times n}$中任取出 $k$ 行 $k$ 列, 位于这些行、列交叉处的$k^2$个元素, 按原次序组成的 $k$ 阶行列式, 称为矩阵 $A$ 的一个 $k$ 阶子式.

具体地, 设所取出的行为: 第$i_1,i_2,\dots,i_k$行, 满足$1\leq i_1\leq i_2\leq \dots\leq i_k\leq m;$ 设所取出的列为: 第$j_1,j_2,\dots,j_k$行, 满足$1\leq j_1\leq j_2\leq \dots\leq j_k\leq n.$ 将上述$k$行$k$列上的元素组成的$A$的子矩阵记为$A_{\left( i_1,i_2,\dots,i_k \atop j_1,j_2,\dots,j_k  \right)}$.
\end{Def}

\begin{Def}
矩阵$A=(a_{ij})_{m\times n}$中行向量组的秩称为矩阵$A$的行秩, 列向量组的秩称为矩阵$A$的列秩. 分别简记为: $r_r(A)$和$r_c(A)$.

\end{Def}

\begin{Def}
矩阵 $A$ 中非零子式的最高阶数称为 $A$ 的秩 (或行列式秩),  记为 $r(A)$.
\end{Def}

\begin{thm}
初等行变换不改变矩阵的秩.
\end{thm}

\begin{thm}
初等列变换不改变矩阵的秩.
\end{thm}

\begin{thm}
矩阵 $A$ 的秩=行秩=列秩, 即 $r(A) =r_r(A) =r_c(A)$.

\end{thm}

\begin{Def}
若$A\in M_{m\times n}$, 若$r(A)=m$, 则称$A$行满秩; 若 $r(A) = n$, 则称A列满秩. 若$A$既行满秩又列满秩, 则称$A$为满秩矩阵.

\end{Def}


\begin{thm}
设 $A$ 为 $n$ 阶方阵, 则下列命题等价:
\begin{enumerate}
\item $A$满秩$(r(A)=n)$;
\item $|A|\not=0$;
\item $A$可逆(称$A$为非奇异或非退化);
\item $A$的$n$个列(行)向量线性无关;
\item 齐次线性方程组 $A\vec{X}=\vec{0}$ 只有零解;
\item 对$\vec{b}\in \mathbb{R}^n$, 线性方程组$A\vec{X}=\vec{b}$ 有唯一解 $\vec{X}=A^{-1}\vec{b}$.
\end{enumerate}

\end{thm}

\begin{thm}
设以下运算可行, 则
\begin{enumerate}
\item 转置: $r(A^T)=r(A)$;
\item 求逆: $r(A^{-1})=r(A)$;
\item 加法: $r(A+B)\leq r(A)+r(B);$
\item 乘法: $r(A)+r(B)-n\leq r(A_{m\times n}B_{n\times p })\leq \min \{r(A),r(B)\}$.

\end{enumerate}

\end{thm}

\begin{thm}
设以下矩阵分块和运算可行, 则
\begin{enumerate}
\item 列并排: $\max\{r(A),r(B)\}\leq r(A|b)\leq  r(A)+r(B)$. (行并排也有相同结论)
\item 准对角: $r\begin{bmatrix} A&0\\0&B\end{bmatrix}=r(A)+r(B)$;
\item 准三角: $r\begin{bmatrix} A&0\\c&B\end{bmatrix}\geq r(A)+r(B)$.
\end{enumerate}
\end{thm}

%%%%%%%%%%%%%%%%%%%%%%%%%%%%%%%%%%%%%%%%%%%%%%%%%%%%%%%%%%%%%%%%%%%%%%%%%%%%%%%%%%

\section{例题讲解}

\begin{eg}
矩阵$$A=\begin{bmatrix}1&-1&2&3\\2&5&4&-1\\3&0&1&-2\end{bmatrix}=\begin{bmatrix}
\vec{\alpha}_1\\ \vec{\alpha}_2\\ \vec{\alpha}_3\end{bmatrix}=\begin{bmatrix} \vec{\beta}_1&\vec{\beta}_2&\vec{\beta}_3&\vec{\beta}_4\end{bmatrix}$$
有三个4维行向量:
\begin{displaymath}\begin{aligned}
&\vec{\alpha}_1=(1,-1,2,3)\\
&\vec{\alpha}_2=(2,5,4,-1)\\
&\vec{\alpha}_3=(3,0,1,-2)\end{aligned}
\end{displaymath}
有四个3维列向量:
\begin{displaymath}
\vec{\beta}_1=\begin{bmatrix}1\\2\\3\end{bmatrix},\ \vec{\beta}_2=\begin{bmatrix}-1\\5\\0\end{bmatrix},\ \vec{\beta}_3=\begin{bmatrix}2\\4\\1\end{bmatrix},\ \vec{\beta}_4=\begin{bmatrix}3\\-1\\-2\end{bmatrix}.
\end{displaymath}
\end{eg}

\begin{eg}
设 $V_1=\{(x,0,\dots,0)^T|x\in\mathbb{R}\}$, 显然$V_1\subset \mathbb{R}^n$, 对向量的加法和数乘运算都是封闭的,所以$V_1$是$\mathbb{R}^n$ 的子空间;

由零向量一个元素组成的集合$V_0=\{(0,0,\dots, 0)^T\}$,作向量的加法和数乘仍然为零向量,故对线性运算是封闭的,所以$V_0$是$\mathbb{R}^n$的子空间,称为零子空间.
\end{eg}


\begin{eg}
(1)若$$\vec{\alpha}_1=\begin{bmatrix}1\\0\end{bmatrix},\ \vec{\alpha}_2=\begin{bmatrix}1\\1\end{bmatrix},\ \vec{\beta}=\begin{bmatrix}-3\\2\end{bmatrix}.$$
则$\vec{\beta}$可唯一地由$\vec{\alpha}_1$与$\vec{\alpha}_2$线性表出:$$\vec{\beta}=-5\vec{\alpha}_1+2\vec{\alpha}_2.$$

(2)若$$\vec{\alpha}_1=\begin{bmatrix}0\\1\end{bmatrix},\ \vec{\alpha}_2=\begin{bmatrix}1\\2\end{bmatrix},\
\vec{\alpha}_3=\begin{bmatrix}-2\\4\end{bmatrix},\
\vec{\beta}=\begin{bmatrix}3\\5\end{bmatrix}.$$
有:
\begin{displaymath}\begin{aligned}
&\vec{\beta}=-\vec{\alpha}_1+3\vec{\alpha}_2-0\vec{\alpha}_3,\\
\mbox{或}\ \ &\vec{\beta}=7\vec{\alpha}_1+\vec{\alpha}_2-\vec{\alpha}_3,\\
\mbox{或}\ \ &\vec{\beta}=11\vec{\alpha}_1+0\vec{\alpha}_2-\frac{3}{2}\vec{\alpha}_3,\dots
\end{aligned}\end{displaymath}
故$\vec{\beta}$是$\vec{\alpha}_1,\vec{\alpha}_2,\vec{\alpha}_3$的线性组合,且线性表出的方式不是唯一的.

(3)若$$\vec{\alpha}_1=\begin{bmatrix}2\\0\end{bmatrix},\ \vec{\alpha}_2=\begin{bmatrix}3\\0\end{bmatrix},\ \vec{\beta}=\begin{bmatrix}0\\1\end{bmatrix}.$$
无论$k_1,k_2$取什么数, 均有$$k_1\vec{\alpha}_1+k_2\vec{\alpha}_2\not=\vec{\beta},$$
即$\vec{\beta}$不能由$\vec{\alpha}_1,\vec{\alpha}_2$线性表出.

\end{eg}

\begin{eg}
已知
$$\vec{\alpha}_1=\begin{bmatrix}-1\\0\\1\\2\end{bmatrix},\ \vec{\alpha}_2=\begin{bmatrix}3\\4\\-2\\5\end{bmatrix},\
\vec{\alpha}_3=\begin{bmatrix}1\\4\\0\\9\end{bmatrix},\  \vec{\beta}=\begin{bmatrix}5\\4\\-4\\1\end{bmatrix}.$$
问$\vec{\beta}$能否由$\vec{\alpha}_1,\vec{\alpha}_2,\vec{\alpha}_3$线性表出?如能线性表出写出表达式.
\end{eg}
解:设$$\vec{\beta}=x_1\vec{\alpha}_1+x_2\vec{\alpha}_2+x_3\vec{\alpha}_3.$$ 若$x_1, x_2, x_3$有解, 则$\vec{\beta}$能由$\vec{\alpha}_1,\vec{\alpha}_2,\vec{\alpha}_3$线性表出, 否则不能. 将表达式用分量形式写出, 得
\begin{displaymath}\left\{\begin{aligned}
&-x_1+3x_2+x_3=5,\\
&4x_2+4x_3=4,\\
&x_1-x_3=-4,\\
&2x_1+5x_2+9x_3=1.\end{aligned}\right.\Rightarrow\begin{bmatrix}-1&3&1&5\\0&4&4&4
\\ 1&-2&0&-4\\2&5&9&1\end{bmatrix}\rightarrow\begin{bmatrix}1&0&2&-2\\0&1&1&1\\0&0&0&0\\0&0&0&0
\end{bmatrix}.\end{displaymath}
解得$$x_1=-2-2t,\ x_2=1-t,\ x_3=t.$$
因此$\vec{\beta}$能由$\vec{\alpha}_1,\vec{\alpha}_2,\vec{\alpha}_3$线性表出, 且表示方法有无穷多种,表达式为
$$\vec{\beta}=(-2-2t)\vec{\alpha}_1+(1-t)\vec{\alpha}_2+t\vec{\alpha}_3.$$


\begin{eg}
设$\vec{\alpha}_1=\begin{bmatrix}1\\2\end{bmatrix},\ \vec{\alpha}_2=\begin{bmatrix}2\\4\end{bmatrix}$是$\mathbb{R}^2$中的两个向量. 易知$\vec{\alpha}_2=2\vec{\alpha}_1,$ 即$$2\vec{\alpha}_1-\vec{\alpha}_2=0,$$
组合系数$2,-1$不全为零. 由定义知$\vec{\alpha}_1,\vec{\alpha}2$线性相关.
\end{eg}

\begin{eg}
设$\vec{\alpha}_1=\begin{bmatrix}1\\0\\0\end{bmatrix},\ \vec{\alpha}_2=\begin{bmatrix}0\\1\\0\end{bmatrix},\ \vec{\alpha}_3=\begin{bmatrix}0\\0\\1\end{bmatrix}$ 是$\mathbb{R}^3$中的3个向量. 试判断它们的线性相关性.
\end{eg}
解:设$$k_1\begin{bmatrix}1\\0\\0\end{bmatrix}+k_2\begin{bmatrix}0\\1\\0\end{bmatrix}
 +k_3\begin{bmatrix}0\\0\\1\end{bmatrix}=\begin{bmatrix}0\\0\\0\end{bmatrix},$$
 即$\begin{bmatrix}k_1\\k_2\\k_3\end{bmatrix}=\begin{bmatrix}0\\0\\0\end{bmatrix}$. 则$k_1=k_2=k_3=0$, 所以$\vec{\alpha}_1,\vec{\alpha}_2,\vec{\alpha}_3$是线性无关的.


\begin{eg}
已知$\vec{\alpha}_1,\vec{\alpha}_2,\vec{\alpha}_3$线性无关, 试判断\\
(1) $\vec{\alpha}_1-\vec{\alpha}_2, \vec{\alpha}_2-\vec{\alpha}_3,\vec{\alpha}_3-\vec{\alpha}_1$;\\
(2) $\vec{\alpha}_1+\vec{\alpha}_2, \vec{\alpha}_2+\vec{\alpha}_3,\vec{\alpha}_3+\vec{\alpha}_1$的线性相关性, 并给出证明.
\end{eg}
解:(1) 线性相关. 因为
$$1(\vec{\alpha}_1-\vec{\alpha}_2)+1( \vec{\alpha}_2-\vec{\alpha}_3)+1(\vec{\alpha}_3-\vec{\alpha}_1)=\vec{0}.$$
系数$1, 1, 1$, 不全为$0$, 所以$\vec{\alpha}_1-\vec{\alpha}_2, \vec{\alpha}_2-\vec{\alpha}_3,\vec{\alpha}_3-\vec{\alpha}_1$线性相关.

(2) 线性无关. 设
$$k_1(\vec{\alpha}_1+\vec{\alpha}_2)+k_2( \vec{\alpha}_2+\vec{\alpha}_3)+k_3(\vec{\alpha}_3+\vec{\alpha}_1)=\vec{0}.$$
即$$(k_1+k_3)\vec{\alpha}_1+(k_1+k_2)\vec{\alpha}_2+(k_2+k_3)\vec{\alpha}_3=\vec{0}.$$
由于$\vec{\alpha}_1,\vec{\alpha}_2,\vec{\alpha}_3$线性无关, 有
\begin{displaymath}\left\{\begin{aligned}&k_1+k_3=0,\\&k_1+k_2=0,\\&k_2+k_3=0\end{aligned}
\right.\end{displaymath}
解这个齐次方程组得$k_1=k_2=k_3=0$, 因此$\vec{\alpha}_1+\vec{\alpha}_2, \vec{\alpha}_2+\vec{\alpha}_3,\vec{\alpha}_3+\vec{\alpha}_1$线性无关.


\begin{eg}
$a$取何值时, $$\vec{\beta}_1=\begin{bmatrix}1\\3\\6\\2\end{bmatrix},\  \vec{\beta}_2=\begin{bmatrix}2\\1\\2\\-1\end{bmatrix},\
\vec{\beta}_3=\begin{bmatrix}1\\-1\\a\\-2\end{bmatrix}$$线性无关?
\end{eg}
解:设$x_1\vec{\beta}_1+x_2\vec{\beta}_2+x_3\vec{\beta}_3=\vec{0}.$
\begin{displaymath}
(\vec{\beta}_1,\vec{\beta}_2,\vec{\beta}_3)=\begin{bmatrix}1&2&1\\3&1&-1\\6&2&
a\\2&-1&-2\end{bmatrix}\rightarrow\begin{bmatrix}1&2&1\\0&-5&-4\\0&-10&a-1\\0&5 &-4\end{bmatrix}\rightarrow\begin{bmatrix}1&2&1\\0&-5&-4\\0&0&a+2\\0&0&0\end{bmatrix}.
\end{displaymath}
当$a\not=2$时, 方程组只有零解$x_1=x_2=x_3=0$. 此时, $\vec{\beta}_1,\vec{\beta}_2,\vec{\beta}_3$线性无关.

\begin{eg}
判断下列命题是否正确?\\
(1) 若向量组线性相关, 则其中每一向量都是其余向量的线性组合.\\
(2) 若一个向量组线性无关,  则其中每一向量都不是其余向量的线性组合.\\
(3) 若$\vec{\alpha}_1,\vec{\alpha}_2$线性相关, $\vec{\beta}_1,\vec{\beta}_2$ 线性相关, 则$\vec{\alpha}_1+\vec{\beta}_1, \vec{\alpha}_2+\vec{\beta}_2$ 也线性相关.
\end{eg}
解:(1) 不正确. 例如非零向量$\vec{\alpha}_1$v线性无关, 再添一个向量$\vec{\alpha}_2=\vec{0}$就线性相关, 但$\vec{\alpha}_1$不能用 $\vec{\alpha}_2$ 线性表出.\\
(2) 正确. 用反证法: 若存在一向量是其余向量的线性组合,  则线性相关.\\
(3) 不正确. 例如$(1,0),(2,0)$线性相关, $(0,1),(0,3)$线性相关, 但$(1,1),(2,3)$ 线性无关.

\begin{eg}
设 $$\vec{\alpha}_1=\begin{bmatrix}1\\0\end{bmatrix},\ \vec{\alpha}_2=\begin{bmatrix}0\\1\end{bmatrix},\ \vec{\beta}_1=\begin{bmatrix}1\\1\end{bmatrix}, \ \vec{\beta}_2=\begin{bmatrix}1\\2\end{bmatrix}, \
\vec{\beta}_3=\begin{bmatrix}2\\2\end{bmatrix} $$
则\begin{displaymath}\begin{aligned}
&\vec{\alpha}_1=2\vec{\beta}_1-\vec{\beta}_2 , \ \ \vec{\alpha}_2=\vec{\beta}_2-\vec{\beta}_1,\\
& \vec{\beta}_1=\vec{\alpha}_1+\vec{\alpha}_2, \ \ \vec{\beta}_2=\vec{\alpha}_1+2\vec{\alpha}_2, \ \ \vec{\beta}_3=2\vec{\alpha}_1+2\vec{\alpha}_2.
\end{aligned}\end{displaymath}
因此, 向量组$\vec{\alpha}_1,\vec{\alpha}_2$与向量组$ \vec{\beta}_1, \vec{\beta}_2, \vec{\beta}_3$等价.
\end{eg}

\begin{eg}
考虑向量组$$\vec{\alpha}_1=\begin{bmatrix}1\\0\end{bmatrix},\ \vec{\alpha}_2=\begin{bmatrix}1\\2\end{bmatrix},\
\vec{\alpha}_3=\begin{bmatrix}2\\3\end{bmatrix}.$$
由于$\vec{\alpha}_1$与$\vec{\alpha}_2$不成比例, 故$\vec{\alpha}_1$, $\vec{\alpha}_2$线性无关. 且$$\vec{\alpha}_3=\frac{1}{2}\vec{\alpha}_1+\frac{3}{2}\vec{\alpha}_2,$$
则再添加$\vec{\alpha}_3$后就线性相关了, 由定义, $\{\vec{\alpha}_1, \vec{\alpha}_2\}$是一个极大线性无关组.

实际上,容易验证$\{\vec{\alpha}_1, \vec{\alpha}_3\}$与$\{\vec{\alpha}_2, \vec{\alpha}_3\}$也都是原向量组的极大线性无关组.
\end{eg}

\begin{eg}
线性无关的向量组是自身的极大无关组.

验证:(1)无关系;(2)极大性.
\end{eg}

\begin{eg}
(自然基)设$\vec{e}_i=(0,\dots,0,1,0,\dots,0)^T\ (1\leq i\leq n)$, 则向量组$\{\vec{e}_1,\vec{e}_2,\dots,\vec{e}_n\}$是$\mathbb{R}^n$中全体列向量集的极大无关组.

验证: (1) 无关性:(2) 极大性.
\end{eg}

\begin{eg}
对如下简化阶梯形矩阵,
\begin{displaymath}
C=\begin{bmatrix}1&0&\dots&0&c_{1,r+1}&\dots&c_{1n}\\0&1&\dots & 0&c_{2,r+1}&\dots&c_{2,n}\\ \vdots&\vdots&&\vdots&\vdots&&\vdots\\0&0&\dots&1&c_{r,r+1}&\dots&c_{rn}\\0& 0&\dots&0&0&\dots&0\\ \vdots&\vdots&&\vdots&\vdots&&\vdots \\0& 0&\dots&0&0&\dots&0\end{bmatrix}\end{displaymath}
主元素所在的列 ($1\sim r$列) 是线性无关的 (自然基的无关性). 其它列 ($r+1 \sim n$列),可表示为前$r$列的线性组合,  并且线性组合的系数就是该列的前$r$ 个元素; (表出性). 主元素所在的列 ($1\sim r$列) 就是$C$的列向量集的一个极大无关组.
\end{eg}

\begin{eg}
设$A=\begin{bmatrix}-1&-2&1&0\\1&4&1&4\\1&3&0&2\end{bmatrix}$, 求矩阵 $A$ 列向量组的一个极大无关组, 并把其余列向量用所求出的极大无关组表示出来.
\end{eg}
解:通过初等行变换把 $A$ 化为简化的阶梯形矩阵:
\begin{displaymath}
A=\begin{bmatrix}-1&-2&1&0\\1&4&1&4\\1&3&0&2\end{bmatrix}\rightarrow
\begin{bmatrix}-1&-2&1&0\\ 0&2&2&4\\0&1&1&2\end{bmatrix}\rightarrow
\begin{bmatrix}-1&0&3&4\\0&1&2&2\\0&0&0&0\end{bmatrix}\rightarrow
\begin{bmatrix}-1&0&3&4\\0&1&1&2\\0&0&0&0\end{bmatrix}=C\end{displaymath}
从而, $c_1$与$c_2$是$A$ 的列向量组的极大无关组; 且
\begin{displaymath}\begin{aligned}
&c_3=(-3)\times c_1+c_2;\\
&c_4=(-4)\times c_1+2\times c_2.\end{aligned}\end{displaymath}

\begin{eg}
(1) 若$r(\vec{\alpha}_1,\vec{\alpha}_2,\dots, \vec{\alpha}_s)=r$, 则$r<s$, 且其中任意$r$ 个线性无关的向量组都是一个极大无关组, 而其中任意 $r +1$ 个向量 (如果存在) 都线性相关.

(2) \begin{displaymath}
\begin{aligned}&\vec{\alpha}_1,\vec{\alpha}_2,\dots, \vec{\alpha}_s \mbox{ 线性无关}\Leftrightarrow r(\vec{\alpha}_1,\vec{\alpha}_2,\dots, \vec{\alpha}_s)=s. \\
&\vec{\alpha}_1,\vec{\alpha}_2,\dots, \vec{\alpha}_s\mbox{线性相关}\Leftrightarrow r(\vec{\alpha}_1,\vec{\alpha}_2,\dots, \vec{\alpha}_s)<s.\end{aligned}\end{displaymath}
\end{eg}

\begin{eg}
设$A=\begin{bmatrix}1&0&0&0\\0&1&0&0\\0&0&0&0\end{bmatrix}$, 求$A$的行秩与列秩.

易知: $r_1$与$r_2$是$A$的行向量集的极大无关组, 故 $r_r(A)=2$; $c_1$与$c_2$是$A$的列向量集的极大无关组, 故 $r_c(A)=2$.

\end{eg}

\begin{eg}
设$A=\begin{bmatrix}1&-1&0&1\\1&1&1&0\\2&0&1&1\end{bmatrix}$, 则$A$有$3\times 4=12$ 个$1$阶子式, 就是每个分量上的元素$a_{ij}$.

$A$有$C_3^2\dot C_4^2=3\times 6=18$个$2$阶子式, 例如$|A_{1,2\atop 1,2}|=\left|\begin{array}{cc}1&-1\\1&1\end{array}\right|=2$.

$A$有4个三阶子式, 如下:
\begin{displaymath}\begin{aligned}&
|A_{1,2,3\atop 1,2,3}|=\left|\begin{array}{ccc}1&-1&0\\1&1&1\\2&0&1\end{array}\right|.\\
&|A_{1,2,3\atop 1,2,4}|=\left|\begin{array}{ccc}1&-1&1\\1&1&0\\2&0&1\end{array}\right|.\\
&|A_{1,2,3\atop 1,3,4}|=\left|\begin{array}{ccc}1&\ 0&\ 1\\1&1&0\\2&1&1\end{array}\right|.\\
&|A_{1,2,3\atop 2,3,4}|=\left|\begin{array}{ccc}-1&0&1\\1&1&0\\0&1&1\end{array}\right|.\end{aligned}
\end{displaymath}

由于在上述$4$个三阶行列式中,均有 $r_3 = r_1+r_2$ , 故均等于$0$.

\end{eg}

\begin{eg}(续). 设$A=\begin{bmatrix}1&-1&0&1\\1&1&1&0\\2&0&1&1\end{bmatrix}$, 因为$A$的所有$3$ 阶子式均为$0$, 即:
\begin{displaymath}
|A_{1,2,3\atop 1,2,3}|=|A_{1,2,3\atop 1,2,4}|=|A_{1,2,3\atop 1,3,4}|=|A_{1,2,3\atop 2,3,4}|=0.\end{displaymath}
又因为$A$存在非零的$2$阶子式, 例如:$|A_{1,2\atop 1,2}|=\left|\begin{array}{cc}1&-1\\1&1\end{array}\right|=2\not=0$. $\Rightarrow r(A)=2$.

另一方面, 由 $r_3 = r_1+r_2$ 知, $r_r(A) = 2$. 由 $c_1 = c_3+c_4; c_2 = c_3-c_4$ 知, $r_c(A) = 2$.
\end{eg}

\begin{eg}
对于阶梯形矩阵
$$B=\begin{bmatrix}b_{11}&b_{12}&\dots &b_{1r}&b_{1,r+1}&\dots&b_{1n}\\ 0& b_{22}&\dots &b_{2,r}&b_{2,r+1}&\dots &b_{2n}\\ \vdots&\vdots&&\vdots&\vdots&&\vdots\\ 0&0&\dots&b_{rr}&b_{r,r+1}&\dots&b_{rn}\\ 0&0&\dots&0&0&\dots&0\\\vdots&\vdots&&\vdots&\vdots&&\vdots\\0&0&\dots&0&0&\dots&0
\end{bmatrix}.$$
一方面, $B$的任意$r+1$阶子式均会包含一个全零行, 则必为$0$.

另一方面, $B$的前$r$行$r$列组成的$r$阶子式, 为三角行列式, 由主元素$b_{ii}
not= 0$ 知, 此$r$ 阶子式非零.
 从而, $r(B) = r$, 即阶梯形矩阵的秩 = 非零行数 = 主元素个数.
\end{eg}

\begin{eg}
已知$A=\begin{bmatrix}1&1&1&1\\0&1&-1&b\\2&3&a&4\\3&5&1&7\end{bmatrix},\ r(A)=3$. 求参数$a,b$的值.
\end{eg}
解: 对$A$做初等行, 列变换, 得:
\begin{displaymath}
A\rightarrow\begin{bmatrix}1&1&1&1\\0&1&-1&b\\0&1&a-2&2\\0&2&-2&4\end{bmatrix}
\rightarrow\begin{bmatrix}1&1&1&1\\0&1&-1&b\\0&0&a-2&2-b\\0&0&0&4-2b\end{bmatrix}
\rightarrow\begin{bmatrix}1&0&0&0\\0&1&0&0\\0&0&a-1&0\\0&0&0&2-b\end{bmatrix}=B.
\end{displaymath}
由 $r(A)=r(B)=3$知,
$$a\not= 1, b=2\ \mbox{或}\ a=1,\not=2.$$

%%%%%%%%%%%%%%%%%%%%%%%%%%%%%%%%%%%%%%%%%%%%%%%%%%%%%%%%%%%%%%%%%%%%%%%%%%%%%%%%%%

\section{课后习题}

\begin{ex}\label{4.1}
判断下列子集是否为给定线性空间$\mathbb{R}^3$的子空间, 并说明理由。\\
(1)$V=\{(x_1,2,0)\in \mathbb{R}^3\}$\\
(2)$V=\{(x_1,0,x_3)\in \mathbb{R}^3\}$\\
(3)$V=\{(x_1,x_2,x_3)\in \mathbb{R}^3| x_1-3x_2+2x_3=0\}$\\
(4)$V=\{(x_1,x_2,x_3)\in \mathbb{R}^3| x_1-3x_2+2x_3=1\}$\\
(5)$V=\{(x_1,x_2,x_3)\in \mathbb{R}^3|\frac{x_1}{2}=\frac{x_2-4}{3}=\frac{x_3-3}{4}\}$\\
(6)$V=\{(x_1,x_2,x_3)\in \mathbb{R}^3|x_1=x_3\mbox{且}x_1+x_2+x_3=0\}$
\end{ex}

\begin{ex}\label{4.2}
已知向量$\vec{\alpha}_1^T=(4,1,-3,2), \vec{\alpha}_2^T=(1,0,3,2), \vec{\alpha}_3^T=(0,0,5,1)$。
求线性组合$5\vec{\alpha}_1+\vec{\alpha}_2+2\vec{\alpha}_3$ 的值。
\end{ex}

\begin{ex}\label{4.3}
已知$\vec{\alpha}_i(i=1, 2, 3)$满足
$$4(\vec{\alpha}_1+\vec{\alpha})+2(\vec{\alpha}_2+\vec{\alpha}_3)=5(\vec{\alpha}+\vec{\alpha}_1+\vec{\alpha}_2+\vec{\alpha}_3)$$
其中$\vec{\alpha}_1^T=(2,5,3), \vec{\alpha}_2^T=(1,0,0), \vec{\alpha}_3^T=(3,6,7)$。 求$\vec{\alpha}$。
\end{ex}

\begin{ex}\label{4.4}
$\vec{\alpha}_1^T=(1,0,1), \vec{\alpha}_2^T=(1,1,0), \vec{\alpha}_3^T=(3,-1,4)$。$\beta_1^T=(1,1,1), \beta_2^T=(1,3,1)$.\\
$\beta_1,\beta_2$是否可以用$\alpha_1,\alpha_2,\alpha_3$线性表出? 若可以, 表示是否唯一。
\end{ex}

\begin{ex}\label{4.5}
设$\vec{\alpha}_1,\ \vec{\alpha}_2,\ \vec{\alpha}_3\in \mathbb{R}^n$, $d_1,\ d_2,\ d_3 \in \mathbb{R}$。
若$d_1\vec{\alpha}_1+d_2\vec{\alpha}_2+d_3\vec{\alpha}_3=0$, 且$d_1d_2\not=0$.\\
证明:$L(\vec{\alpha}_1,\ \vec{\alpha}_3)=L(\vec{\alpha}_1,\ \vec{\alpha}_2).$
\end{ex}

\begin{ex}\label{4.6}
判断下列向量组的线性相关性.\\
(1) $(1,0,1)^T,\ (1,1,0)^T,\ (3,5,4)^T.$\\
(2) $(3,7,10)^T,\ (-1,0,1)^T,\ (1,7,8)^T.$\\
(3) $(2,2,0,0)^T,\  (2,0,2,0)^T,\ (0,0,2,2)^T,\ (0,2,0,2)^T.$\\
(4) $(1,3,-5,1)^T,\ (2,6,0,1)^T,\ (3,9,7,10)^T.$
\end{ex}

\begin{ex}\label{4.7}
证明:下三角阵$A=\begin{bmatrix} a&0&0\\b&d&0\\c&d&f\end{bmatrix}$ 的列向量线性相关的
充要条件是对角元素$a,d,f$中至少有一个为零元素.
\end{ex}

\begin{ex}\label{4.8}
求$(\vec{\alpha}_1,\vec{\alpha}_2,\vec{\alpha}_3)$ 和$(\vec{\alpha}_1,\vec{\alpha}_2,\vec{\alpha}_3,\vec{\beta})$ 的秩,\\
其中
$\vec{\alpha}_1=\begin{bmatrix}-1\\3\\0\\-5\end{bmatrix},\vec{\alpha}_2=\begin{bmatrix}2\\0\\7\\-3\end{bmatrix},
\vec{\alpha}_3=\begin{bmatrix}-4\\1\\-2\\6\end{bmatrix}$\\
(1)$\vec{\beta}=\begin{bmatrix} 8\\3\\-1\\-25\end{bmatrix}$;\\
(2)$\vec{\beta}=\begin{bmatrix} 1\\1\\1\\1\end{bmatrix}$
\end{ex}

\begin{ex}\label{4.9}
设$n$维向量组$\vec{\alpha}_1, \ \vec{\alpha}_2, \ \dots,\ \vec{\alpha}_n$线性无关,令
\begin{displaymath}
\left\{\begin{aligned}\vec{\beta}_1=a_{11}\vec{\alpha}_1+a_{12}\vec{\alpha}_2+\dots+a_{1n}\vec{\alpha}_n \\ \vec{\beta}_2=a_{21}\vec{\alpha}_1+a_{22}\vec{\alpha}_2+\dots+a_{2n}\vec{\alpha}_n \\ \dots \ \dots \ \dots
\ \dots \ \dots \ \dots  \ \dots \ \dots   \ \dots   \\ \vec{\beta}_n=a_{n1}\vec{\alpha}_1+a_{n2}\vec{\alpha}_2+\dots+a_{nn}\vec{\alpha}_n\end{aligned}\right.
\end{displaymath}
试证明:向量组$\vec{\beta}_1,\ \vec{\beta}_2,\ \dots,\ \vec{\beta}_n$ 线性无关$\Leftrightarrow$\ $|A|=\left|\begin{array}{cccc}a_{11}&a_{12}&\dots&a_{1n}
\\a_{21}&a_{22}&\dots&a_{2n}\\ \dots&\dots&\dots&\dots\\a_{n1}&a_{n2}&\dots&a_{nn}\end{array}\right|\not=0$
\end{ex}

\begin{ex}\label{4.10}
试证明:\\
(1) n维向量组$\vec{\alpha}_1, \ \vec{\alpha}_2, \ \dots,\ \vec{\alpha}_n(\ n\leq 2)\ $ 线性相关的充分必要条件是其中至少有一个向量可由其余向量线性表出;\\
(2) n维向量组$\vec{\alpha}_1, \ \vec{\alpha}_2, \ \dots,\ \vec{\alpha}_n(\ n\leq 2)\ $ 线性无关的充分必要条件是其中任何一个向量都不能由其余向量线性表出.
\end{ex}

\begin{ex}\label{4.11}
设有向量组$\vec{\alpha}_1, \ \vec{\alpha}_2, \ \dots,\ \vec{\alpha}_n(\ n\leq 2)$. 其中的任意$r$个$(1\leq r\leq s)$ 向量称为向量组$\vec{\alpha}_1, \ \vec{\alpha}_2, \ \dots,\ \vec{\alpha}_n$
的一个部分向量组. 试证明:\\
(1) 若$\vec{\alpha}_1, \ \vec{\alpha}_2, \ \dots,\ \vec{\alpha}_n$ 线性无关, 则它的任何一个部分向量组都线性无关;\\
(2) 若$\vec{\alpha}_1, \ \vec{\alpha}_2, \ \dots,\ \vec{\alpha}_n$的一个部分向量组线性相关, 则该向量组也线性相关。
\end{ex}

\begin{ex}\label{4.12}
设$n$维向量组$\vec{\alpha}_1, \ \vec{\alpha}_2, \ \dots,\ \vec{\alpha}n$ 线性无关, $\vec{\beta}$ 是一个$n$维向量,
试证明: 向量组$\vec{\beta},\ \vec{\alpha}_1, \ \dots,\ \vec{\alpha}_n$线性无关的充要条件是$\vec{\beta}$不可由向量组
$\vec{\alpha}_1, \ \vec{\alpha}_2, \ \dots,\ \vec{\alpha}_n$ 线性表出.
\end{ex}

\begin{ex}\label{4.13}
设向量组$\vec{\alpha}_1,\ \vec{\alpha}_2$ 线性无关,
证明向量组$\vec{\beta}_1=\vec{\alpha}_1+\vec{\alpha}_2, \vec{\beta}_2=\vec{\alpha}_1-\vec{\alpha}_2$ 也线性无关.
\end{ex}

\begin{ex}\label{4.14}
设$n$维向量组$\vec{\alpha}_1, \ \vec{\alpha}_2, \ \dots,\ \vec{\alpha}_n$ 线性无关, 若$\vec{\alpha}_{n+1}=\lambda_1\vec{\alpha}_1+\lambda_2\vec{\alpha}_2+\dots+\lambda_n\vec{\alpha}_n$ 且
$\lambda_i\not=0(i=1,2,\dots, n)$, 试证明: $\vec{\alpha}_1, \ \vec{\alpha}_2, \ \dots,\ \vec{\alpha}_n,\ \vec{\alpha}_{n+1}$ 中任意$n$ 个向量都线性无关.
\end{ex}

\begin{ex}\label{4.15}
设$n$维向量组$\vec{\alpha}_1, \ \vec{\alpha}_2, \ \dots,\ \vec{\alpha}_n$ 线性无关, 其中$\vec{\alpha}_i=(a_{i1},a_{i2},\dots,a_{in}),i=1,2,\dots,n.$ 试证明:存在不全为零的数
$k_1,k_2,\dots,k_n$使得$$k_1\vec{\alpha}_1+k_2\vec{\alpha}_2+\dots+k_n\vec{\alpha}_n=(c,0,\dots,0),\ \ \ \ c\not=\vec{0}.$$
\end{ex}

\begin{ex}\label{4.16}
$a$取什么值时,下列向量组线性相关?
\begin{displaymath}\vec{\alpha}_1=\begin{bmatrix}a\\1\\1\end{bmatrix},\ \ \ \  \ \  \vec{\alpha}_2=\begin{bmatrix}1\\a\\-1\end{bmatrix},\ \ \ \     \ \ \vec{\alpha}_3=\begin{bmatrix}1\\-1\\a\end{bmatrix}.
\end{displaymath}
\end{ex}

\begin{ex}\label{4.17}
设$\vec{\alpha}_1,\ \vec{\alpha}_2$ 线性无关, $\vec{\alpha}_1+\vec{\beta}, \ \vec{\alpha}_2+\vec{\beta}$线性相关,
求向量$\vec{\beta}$用$\vec{\alpha}_1,\ \vec{\alpha}_2$ 的线性表示式.
\end{ex}

\begin{ex}\label{4.18}
设$\vec{\alpha}_1,\ \vec{\alpha}_2$ 线性相关, $\vec{\beta}_1,\ \vec{\beta}_2$ 也线性相关,
问$\vec{\alpha}_1+\vec{\beta}_1,\ \vec{\alpha}_2+\vec{\beta}_2$是否一定线性相关? 试举例说明.
\end{ex}

\begin{ex}\label{4.19}
求矩阵$A=\begin{bmatrix}1&1&-1\\2&-1&0\\1&0&1\end{bmatrix}$ 的秩.
\end{ex}

\begin{ex}\label{4.20}
求下列矩阵的秩.\\
(1)$A=\left[\begin{array}{ccccc}4&3&-5&2&3\\8&6&-7&4&2\\4&3&-8&2&7\\4&3&1&2&-5\\8&6&-1&4&-6\end{array}\right] $\\
(2)$A=\left[\begin{array}{ccccc}0&1&1&1&1\\1&0&1&1&1\\1&1&0&1&1\\ 1&1&1&0&1\\1&1&1&1&0\end{array}\right]$\\
(3)$A=\begin{bmatrix}a_1b_1 & a_1b_2& \dots & a_1b_n\\a_2b_1 &a_2b_2& \dots & a_2b_n\\ \vdots & \vdots &\ddots  &\vdots \\ a_nb_1 &a_nb_2 &\dots& a_nb_n\end{bmatrix}$
\end{ex}

\begin{ex}\label{4.21}
求矩阵$\lambda E-A$的秩, 其中$\lambda=2$, $A=\begin{bmatrix}1&-1&1\\2&4&-2\\-3&-3&5\end{bmatrix}$.
\end{ex}

\begin{ex}\label{4.22}
求矩阵$\lambda E-A$的秩, 其中$A=\begin{bmatrix}4&-1&-1&1\\-1&4&1&-1\\-1&1&4&-1\\1&-1&-1&4\end{bmatrix}$.\\
(1)$\lambda=3$\\
(2)$\lambda=7$
\end{ex}

\begin{ex}\label{4.23}
求$A$和$(A\ b)$的秩,其中
$$A=\begin{bmatrix}3&1&-1&-2\\1&-5&2&1\\2&6&-3&-3\\-1&-11&5&4\end{bmatrix}$$
(1)$b=\begin{bmatrix}2\\-1\\3\\-4\end{bmatrix}$ \\
(2)$b=\begin{bmatrix}2\\-1\\c\\-4\end{bmatrix}$
\end{ex}

\begin{ex}\label{4.24}
证明$r(A+B)\leq r(A)+r(B).$
\end{ex}

\begin{ex}\label{4.25}
设$A,B$为$n\times n$矩阵.证明: 如果$AB=0$, 那么$$r(A)+r(B)\leq n.$$
\end{ex}

\begin{ex}\label{4.26}
设$A$为$n\times n$矩阵,证明:如果$A^2=E$, 那么$$r(A+E)+r(A-E)=n.$$
\end{ex}

\begin{ex}\label{4.27}
设$A$是$n\times n$矩阵, 且$r(A)=r$. 证明: 存在一个$n\times n$的可逆矩阵$P$, 使得$PAP^{-1}$ 的后$n-r$行全为零.
\end{ex}

\begin{ex}\label{4.28}
设$B$为一$r\times r$矩阵, $C$ 为一$r\times n$矩阵,且$r(C)=r$, 证明:\\
(1)如果$BC=0$, 那么$B=0$; \\
(2)如果$BC=C$, 那么$B=E$.
\end{ex}

\begin{ex}\label{4.29}
已知3阶矩阵$A$与3维列向量$x$ 满足$A^3 \vec{x}=3A\vec{x}-A^2\vec{x}$. 且向量组$\vec{x},\ A\vec{x},\ A^2\vec{x}$线性无关.\\
(1) 记$y=A\vec{x},\ \vec{z}=A\vec{y},\ P=(\vec{x},\ \vec{y},\ \vec{z})$. 求3 阶矩阵$B$, 使$AP=PB$;\\
(2) 求$|A|$
\end{ex}

%%%%%%%%%%%%%%%%%%%%%%%%%%%%%%%%%%%%%%%%%%%%%%%%%%%%%%%%%%%%%%%%%%%%%%%%%%%%%%%%%%

\section{习题答案}

\textbf{习题 \ref{4.1} 解答:}\\
(1)任取$(x,2,0),(y,2,0)\in V$, 则$$ (x,2,0)+(y,2,0)=(x+y,4,0)\not \in V.$$ 因此$V$ 不是$\mathbb{R}^3$的线性子空间.\\
(2)任取与(1)同样的方法, $V$ 对加法保持封闭. 对任意的$\vec{x}=(x_1,0,x_3)\in V$ 和$\lambda\in \mathbb{R}$, $$\lambda \vec{x}=(\lambda x_1,0,\lambda x_3)\in V.$$ 即$V$ 对数乘也保持封闭. 所以$V$是$\mathbb{R}^3$ 的线性子空间.\\
(3)任取$\vec{x}=(x_1,x_2,x_3)$和$\vec{y}=(y_1,y_2,y_3)\in V$, 以及$\lambda, \mu \in \mathbb{R}$, 则 $$(\lambda x_1+\mu y_1)-3(\lambda x_2+\mu y_2)+2(\lambda x_3+\mu y_3)=\lambda(x_1-3x_2+2x_3)+\mu(y_1-3y_2+2y_3)=0.$$
因此$$\lambda \vec{x}+\mu \vec{y}=(\lambda x_1+\mu y_1,\lambda x_2+\mu y_2,\lambda x_3+\mu y_3)\in V.$$ 所以$V$是$\mathbb{R}^3$ 的线性子空间.\\
(4)与(3)同样的方法, $V$不是$\mathbb{R}^3$的线性子空间.\\
(5)任取$\vec{x}=(x_1,x_2,x_3)$和$\vec{y}=(y_1,y_2,y_3)\in V$, 则$$\frac{x_1+y_1}{2}-\frac{x_2+y_2-4}{3}=\frac{x_1}{2}-\frac{x_2-4}{3}+\frac{y_1}{2}-\frac{y_2-4}{3}-\frac{4}{3}=-\frac{4}{3}.$$
即$$\frac{x_1+y_1}{2}\not =\frac{x_2+y_2-4}{3},$$ 因此$\vec{x}+\vec{y}\not\in V$. 所以$V$不是$\mathbb{R}^3$的线性子空间.\\
(6)\ 任取$\vec{x}=(x_1,x_2,x_3)$ 和$\vec{y}=(y_1,y_2,y_3)\in V$, 以及$\lambda, \mu \in \mathbb{R}$, 则$\lambda x_1+\mu y_1=\lambda x_3+\mu y_3$, 且 $$(\lambda x_1+\mu y_1)+(\lambda x_2+\mu y_2)+(\lambda x_3+\mu y_3)=0.$$ 因此$\lambda \vec{x}+\mu \vec{y}\in V$, 所以$V$是$\mathbb{R}^3$ 的线性子空间.\\
\textbf{习题 \ref{4.2} 解答:}\\
$$5\vec{\alpha}_1+\vec{\alpha}_2+2\vec{\alpha}_3=(20,5,-15,10)+(1,0,3,2)+(0,0,10,2)=(21, 5, -2, 14).$$
\textbf{习题 \ref{4.3} 解答:}\\
由关系式, 有$$\vec{\alpha}=-\vec{\alpha}_1-3\vec{\alpha}_2-3\vec{\alpha}_3=(-14,-23,-24)$$.
\textbf{习题 \ref{4.4} 解答:}\\
记$A=(\vec{\alpha}_1, \vec{\alpha}_2, \vec{\alpha}_3)$, 题目等价于解方程组$$A\vec{x}=\vec{\beta}_1\ \ \  \ \mbox{和}\ \ \ \ A\vec{x}=\vec{\beta}_2$$.
因此写成拓展矩阵的形式并对其做基础行变换:
\begin{displaymath}
\left[\begin{array}{ccccc}1 & 1 & 3 & 1 & 1 \\ 0 & 1 &-1 & 1 & 3\\ 1 & 0 & 4 & 1 & 1 \end{array}\right]\longrightarrow \left[\begin{array}{ccccc}1 & 1 & 3 & 1 & 1 \\ 0 & 1 &-1 & 1 & 3\\ 0 & -1 & 1 & 0 & 0 \end{array}\right]\longrightarrow
\left[\begin{array}{ccccc}1 & 1 & 3 & 1 & 1 \\ 0 & 1 &-1 & 1 & 3\\ 0 & 0 & 0 & 1 & 3 \end{array}\right]\end{displaymath}
$\vec{\beta}_1^{'}$和$\vec{\beta}_2^{'}$ 的第三个位置均不为零, 然而 $A^{'}$的第三行全部为零, 因此方程组$A\vec{x}=\vec{\beta}_1$ 和
$A\vec{x}=\vec{\beta}_2$均无解, 所以$\vec{\beta}_1$ 和$\vec{\beta}_2$ 均不可以用$\vec{\alpha}_1, \vec{\alpha}_2, \vec{\alpha}_3$线性表出.\\
\textbf{习题 \ref{4.5} 解答:}\\
因为$d_2\not=0, d_3\not =0$, 由关系式,得$$\vec{\alpha}_2=-\frac{d_1}{d_2}\vec{\alpha}_1-\frac{d_3}{d_2}\vec{\alpha}_3\in L(\vec{\alpha}_1,\vec{\alpha}_3).$$ 因此$$L(\vec{\alpha}_1,\vec{\alpha}_2)\subset L(\vec{\alpha}_1,\vec{\alpha}_3).$$
同理有$$L(\vec{\alpha}_1,\vec{\alpha}_3)\subset L(\vec{\alpha}_1,\vec{\alpha}_2).$$ 因此$L(\vec{\alpha}_1,\vec{\alpha}_2)=L(\vec{\alpha}_1,\vec{\alpha}_3)$.\\
\textbf{习题 \ref{4.6} 解答:}\\
(1)$$\begin{bmatrix}1&1&3\\0&1&5\\1&0&4\end{bmatrix}\longrightarrow \begin{bmatrix}1&1&3\\0&1&5\\0&-1&1\end{bmatrix} \longrightarrow \begin{bmatrix}1&1&3\\0&1&5\\0&0&6\end{bmatrix} .$$ 因此三个向量线性无关.\\
(2)$$A=\begin{bmatrix}-1&1&3\\0&7&7\\1&8&10\end{bmatrix}\longrightarrow \begin{bmatrix}-1&1&3\\0&7&7\\0&9&13\end{bmatrix}.$$ 所以矩阵$A$的行列式为$-28$, $A$ 非奇异, 因此三个向量线性无关.\\
(3)与(1)同样的方法, 所给的四个向量线性相关, 且极大线性无关组向量个数为3.\\
(4)与(1)同样的方法, 所给的三个向量线性无关.\\
\textbf{习题 \ref{4.7} 解答:}\\
"$\Rightarrow$" 若否, 即$a,d,f$ 均不为零, 此时解方程$$A\vec{x}=0$$ 得到唯一解$\vec{x}=0$. 即$A$的列向量线性无关, 矛盾.\\
"$\Leftarrow$" 若否, 则$A$的列向量线性无关, 因此$$|A|=adf\not=0.$$ 矛盾.\\
\textbf{习题 \ref{4.8} 解答:}\\
$$(\vec{\alpha}_1, \vec{\alpha}_2,\vec{\alpha}_3)=\begin{bmatrix}-1&2&-4\\ 3&0&1\\ 0&7&-2\\-5&-3&6\end{bmatrix}
\rightarrow \begin{bmatrix}-1&2&-4\\ 0&6&-11\\ 0&7&-2\\0&-13&26\end{bmatrix}
\rightarrow \begin{bmatrix}-1&2&-4\\ 0&1&-2\\ 0&0&12\\0&0&1\end{bmatrix}
\rightarrow \begin{bmatrix}-1&2&-4\\ 0&1&-2\\ 0&0&1\\0&0&0\end{bmatrix}$$
所以$r(\vec{\alpha}_1,\vec{\alpha}_2,\vec{\alpha}_3)=3$.\\
(1)\begin{displaymath}\begin{aligned}(\vec{\alpha}_1, \vec{\alpha}_2,\vec{\alpha}_3,\vec{\beta})=&\begin{bmatrix}-1&2&-4&8\\ 3&0&1&3\\ 0&7&-2&-1\\-5&-3&6&-25\end{bmatrix}
\rightarrow \begin{bmatrix}-1&2&-4&-8\\ 0&6&-11&27\\ 0&7&-2&-1\\0&-13&26&-65\end{bmatrix}\\
\rightarrow &\begin{bmatrix}-1&2&-4&-8\\ 0&1&-2&5\\ 0&0&12&-36\\0&0&1&-3\end{bmatrix}
\rightarrow \begin{bmatrix}-1&2&-4&-8\\ 0&1&-2&5\\ 0&0&1&-3\\0&0&0&0\end{bmatrix}\end{aligned}\end{displaymath}
所以$r(\vec{\alpha}_1, \vec{\alpha}_2,\vec{\alpha}_3,\vec{\beta})=3$.\\
(2)\begin{displaymath}\begin{aligned}(\vec{\alpha}_1, \vec{\alpha}_2,\vec{\alpha}_3,\vec{\beta})=&\begin{bmatrix}-1&2&-4&1\\ 3&0&1&1\\ 0&7&-2&1\\-5&-3&6&1\end{bmatrix}
\rightarrow \begin{bmatrix}-1&2&-4&-1\\ 0&6&-11&4\\ 0&7&-2&1\\0&-13&26&-4\end{bmatrix}\\
\rightarrow &\begin{bmatrix}-1&2&-4&-1\\ 0&1&-2&\frac{4}{13}\\ 0&0&12&\frac{-15}{13}\\0&0&1&\frac{28}{13}\end{bmatrix}
\rightarrow \begin{bmatrix}-1&2&-4&-1\\ 0&1&-2&\frac{4}{13}\\ 0&0&1&\frac{28}{13}3\\0&0&0&1\end{bmatrix}\end{aligned}\end{displaymath}
所以$r(\vec{\alpha}_1, \vec{\alpha}_2,\vec{\alpha}_3,\vec{\beta})=4$.

\textbf{习题 \ref{4.9} 解答:}\\
$"\Rightarrow"$ 向量组$\vec{\beta}_1,\ \vec{\beta}_2,\ \dots,\ \vec{\beta}_n$ 线性无关, 所以对任意的$x_1,x_2,\dots,x_n\in \mathbb{R}$, 若满足
$$x_1\vec{\beta}_1+x_2\vec{\beta}_2+\dots+x_n\vec{\beta}_n=\vec{0}$$
则必有$x_1=x_2=\dots=x_n=0$. 由$\beta_i$ 的表示式代入到上式,得到
$$(a_{11}x_1+a_{21}x_2+\dots+a_{n1}x_n)\vec{\alpha}_1+\dots+(a_{1n}x_1+a_{2n}x_2+\dots+a_{nn}x_n)\vec{\alpha}_n=0$$
又因为向量组$\vec{\alpha}_1, \ \vec{\alpha}_2, \ \dots,\ \vec{\alpha}_n$ 线性无关, 所以有
\begin{displaymath}
\left\{\begin{aligned}
&a_{11}x_1+a_{21}x_2+\dots+a_{n1}x_n=0\\
&a_{12}x_1+a_{22}x_2+\dots+a_{n2}x_n=0\\
&\dots\ \ \dots\ \ \dots\ \ \dots\ \ \dots \ \ \dots \ \ \dots\\
&a_{1n}x_1+a_{2n}x_2+\dots+a_{nn}x_n=0
\end{aligned}
\right.
\end{displaymath}
即$A^{*}\vec{x}=0$, 其中$\vec{x}=(x_1,x_2,\dots,x_n)$. 也就是说, 方程组$A^{*}\vec{x}=0$ 仅有零解$\Leftrightarrow$ 矩阵$A^{*}$ 非奇异$\Leftrightarrow|A^{*}|\not=0\Leftrightarrow |A|\not=0$.\\
$"\Leftarrow"$ 反设向量组$\vec{\beta}_1,\ \vec{\beta}_2,\ \dots,\ \vec{\beta}_n$ 线性相关, 即存在不全为零的$x_1,x_2,\dots,x_n\in \mathbb{R}$, 使得
$$x_1\vec{\beta}_1+x_2\vec{\beta}_2+\dots+x_n\vec{\beta}_n=0$$
由上面的分析, 知这等价于$A^{*}\vec{x}=0$,  $\vec{x}=(x_1,x_2,\dots,x_n)\not=0$. 即方程组$A^{*}\vec{x}=0$有非零解, 从而$|A^{*}|=0$, 因此$|A|=0$, 矛盾.

\textbf{习题 \ref{4.10} 解答:}\\
(1)$"\Rightarrow"\  \vec{\alpha}_1, \ \vec{\alpha}_2, \ \dots,\ \vec{\alpha}_n\ $ 线性相关, 因此存在不全为零的数$k_1,k_2,\dots,k_n$, 使得
$$k_1\vec{\alpha}_1+k_2\vec{\alpha}_2+dots+k_n\vec{\alpha}_n=\vec{0}.$$
设$k_i\not=0 $, 则由上式得到:$$\vec{\alpha}_i=-\frac{k_1}{k_i}\vec{\alpha}_1-\frac{k_2}{k_i}\vec{\alpha}_2-\dots-\frac{k_{i-1}}{k_i}\vec{\alpha}_{i-1}-\frac{k_{i+1}}{k_i}\vec{\alpha}_{
i+1}-\dots-\frac{k_n}{k_i}\vec{\alpha}_n.$$
即$\vec{\alpha}_i$可以由其余向量线性表出.\\
"$\Leftarrow$"\ 设$\vec{\alpha}_i$可以由其余向量表出, 存在$k_1,\dots,k_{i-1},k_{i+1},\dots,k_n$ 使得$$\vec{\alpha}_i=k_1\vec{\alpha}_1+\dots+k_{i-1}\vec{\alpha}_{i-1}+k_{i+1}\vec{\alpha}_{i+1}+\dots+k_{n}\vec{\alpha}_{n}.$$ 即$$
k_1\vec{\alpha}_1+\dots+k_{i-1}\vec{\alpha}_{i-1}-\vec{\alpha}_i+\dots+k_{i+1}\vec{\alpha}_{i+1}+\dots+k_{n}\vec{\alpha}_{n}=0.$$
即$\vec{\alpha}_1, \ \vec{\alpha}_2, \ \dots,\ \vec{\alpha}_n$ 线性相关.\\
(2)$"\Rightarrow"$\ 若不然, 则至少有一个向量可由其余向量线性表出, 由(1) 的结论, 知$\vec{\alpha}_1, \ \vec{\alpha}_2, \ \dots,\ \vec{\alpha}_n$线性相关, 矛盾.\\
$"\Leftarrow"$\ 若不然, 则$\vec{\alpha}_1, \ \vec{\alpha}_2, \ \dots,\ \vec{\alpha}_n$ 线性相关, 由(1), 至少有一个向量可由其余向量线性表出, 矛盾.\\
\textbf{习题 \ref{4.11} 解答:}\\
(1)任取$\vec{\alpha}_1, \ \vec{\alpha}_2, \ \dots,\ \vec{\alpha}_n$的一个部分向量组$\vec{\alpha}_{k_1}, \vec{\alpha}_{k_2}, \dots, \vec{\alpha}_{k_l}$, 其中$k_i\in\{1,2,\dots,n\}(i=1,\dots,l)$. 对
任意的一组参数$\lambda_1,\dots,\lambda_l$, 设
$$\lambda_1\vec{\alpha}_{k_1}+\lambda_2\vec{\alpha}_{k_2}+\dots+\lambda_l\vec{\alpha}_{k_l}=0.$$
由于向量组$\vec{\alpha}_1, \ \vec{\alpha}_2, \ \dots,\ \vec{\alpha}_n$ 线性无关, 因此$\lambda_1=\lambda_2=\dots=\lambda_l=0$. 即$\vec{\alpha}_{k_1}, \vec{\alpha}_{k_2}, \dots, \vec{\alpha}_{k_l}$ 线性无关.\\
(2)假设$\vec{\alpha}_1, \ \vec{\alpha}_2, \ \dots,\ \vec{\alpha}_n$的一个部分向量组$\vec{\alpha}_{k_1}, \vec{\alpha}_{k_2}, \dots, \vec{\alpha}_{k_l}$线性相关,
其中$k_i\in\{1,2,\dots,n\}(i=1,\dots,l)$. 即存在不全为零的参数$\lambda_1,\dots,\lambda_l$,
使得$$\lambda_1\vec{\alpha}_{k_1}+\lambda_2\vec{\alpha}_{k_2}+\dots+\lambda_l\vec{\alpha}_{k_l}=0.$$
因此$\vec{\alpha}_1, \ \vec{\alpha}_2, \ \dots,\ \vec{\alpha}_n$线性相关.\\
\textbf{习题 \ref{4.12} 解答:}\\
$"\Rightarrow"$\  是显然的, 若$\vec{\beta}$ 可由$\vec{\alpha}_1, \ \vec{\alpha}_2, \ \dots,\ \vec{\alpha}_n$ 线性表出, 由10 题中的结论,
则向量组$\vec{\beta},\ \vec{\alpha}_1, \ \dots,\ \vec{\alpha}_n$线性相关, 矛盾.\\
$"\Leftarrow"$\ 对任意的$k_0,k_1,\dots,k_n$, 若
$$k_0\vec{\beta}+k_1\vec{\alpha}_1+\dots+k_n\vec{\alpha}_n=0.$$
若$k_0\not=0$, 则$\vec{\beta}$可由$\vec{\alpha}_1, \ \vec{\alpha}_2, \ \dots,\ \vec{\alpha}_n$线性表出,矛盾. \\
因此$k_0=0$, 由$\vec{\alpha}_1, \ \vec{\alpha}_2, \ \dots,\ \vec{\alpha}_n$ 的线性无关性, 知$k_1=k_2=\dots=k_n=0$.\\
\textbf{习题 \ref{4.13} 解答:}\\
因为$\vec{\beta}_1=\vec{\alpha}_1+\vec{\alpha}_2, \vec{\beta}_2=\vec{\alpha}_1-\vec{\alpha}_2$,
所以\begin{displaymath}\begin{bmatrix}\vec{\beta}_1\\ \vec{\beta}_2\end{bmatrix}=
\begin{bmatrix}1&1\\1&-1\end{bmatrix}\begin{bmatrix}\vec{\alpha}_1\\ \vec{\alpha}_2\end{bmatrix}.\end{displaymath}
因为$\vec{\alpha}_1,\vec{\alpha}_2$ 线性无关, 因此$$r(\vec{\alpha}_1,\vec{\alpha}_2)=2.$$
又因为$$\left|\begin{array}{cc}1&1\\1&-1\end{array}\right|=-2\not=0,$$
所以矩阵$\begin{bmatrix}1&1\\1&-1\end{bmatrix}$ 可逆, 因此$$r(\vec{\beta}_1,\vec{\beta}_2)=r(\vec{\alpha}_1,\vec{\alpha}_2)=2.$$
因此$\vec{\beta}_1,\vec{\beta}_2$也是线性无关的.\\
\textbf{习题 \ref{4.14} 解答:}\\
(方法一)\\
在$\vec{\alpha}_1, \ \vec{\alpha}_2, \ \dots,\ \vec{\alpha}_n,\ \vec{\alpha}_{n+1}$ 中任取$n$ 个向量, 不失一般性, 我们不妨记这$n$ 个向量为$$\vec{\alpha}_1, \ \vec{\alpha}_2, \ \dots,\ \vec{\alpha}_{n-1},\ \vec{\alpha}_{n+1}.$$
对任意的$k_1,k_2,\dots,k_{n-1},k_{n+1}$, 设满足
$$k_1\vec{\alpha}_1+k_2\vec{\alpha}_2+\dots+k_{n-1}\vec{\alpha}_{n-1}+k_{n+1}\vec{\alpha}_{n+1}=0.$$
将条件$\vec{\alpha}_{n+1}=\lambda_1\vec{\alpha}_1+\lambda_2\vec{\alpha}_2+\dots+\lambda_n\vec{\alpha}_n$ 代入上式, 得:
$$(k_1+\lambda_1k_{n+1})\vec{\alpha}_1+(k_2+\lambda_2k_{n+1})\vec{\alpha}_2+\dots+(k_{n-1}+\lambda_{n-1}k_{n+1}\vec{\alpha}_{n-1}+
\lambda_nk_{n+1}\vec{\alpha}_n=\vec{0}.$$
因为向量组$\vec{\alpha}_1, \ \vec{\alpha}_2, \ \dots,\ \vec{\alpha}_n$ 线性无关, 所以有
\begin{displaymath}\left\{\begin{aligned}&k_1+\lambda_1k_{n+1}=0\\&k_2+\lambda_2k_{n+1}=0\\&\dots \\ &k_{n-1}+\lambda_{n-1}k_{n+1}=0 \\ &\lambda_nk_{n+1}=0\end{aligned}
\right.\end{displaymath}
因为$\lambda_n\not=0$, 由最后一个方程, 得$k_{n+1}=0$, 从而解出$$k_1=k_1=\dots=k_{n-1}.$$
所以向量组$\vec{\alpha}_1, \ \vec{\alpha}_2, \ \dots,\ \vec{\alpha}_{n-1},\ \vec{\alpha}_{n+1}$ 线性无关. 同样的道理,
可以知道$\vec{\alpha}_1, \ \vec{\alpha}_2, \ \dots,\ \vec{\alpha}_n,\ \vec{\alpha}_{n+1}$ 中任意$n$个向量都线性无关.\\
(方法二)\\
利用两个向量组的等价性来计算向量组的秩. 设
\begin{displaymath}\begin{aligned}&(A)\vec{\alpha}_1,  \ \vec{\alpha}_2, \ \dots,\ \vec{\alpha}_{n-1},\ \vec{\alpha}_{n+1}\\
&(B)\vec{\alpha}_1, \ \vec{\alpha}_2, \ \dots,\ \vec{\alpha}_{n-1},\ \vec{\alpha}_n,\ \vec{\alpha}_{n+1}\end{aligned}\end{displaymath}
显然向量组(A)中所有向量都可以由向量组(B) 线性表出. 而根据$$\vec{\alpha}_{n+1}=\lambda_1\vec{\alpha}_1+\lambda_2\vec{\alpha}_2+\dots+\lambda_n\vec{\alpha}_n$$ 以及$\lambda_i\not=0(i=1,
\dots,n)$,得
$$\vec{\alpha}_n=-\frac{\lambda_1}{\lambda_n}\vec{\alpha}_1-\frac{\lambda_2}{\lambda_n}\vec{\alpha}_2-\dots-
\frac{\lambda_{n-1}}{\lambda_n}\vec{\alpha}_{n-1}-\frac{1}{\lambda_n}\vec{\alpha}_{n+1}.$$
所以$\vec{\alpha}_n$可由(A)线性表出. 而(B) 中的其余向量都在(A) 中, 显然可以用(A) 线性表出. 也就是说, (B)可由(A)线性表出, 所以向量组(A)与(B)等价.\\
又因为$\vec{\alpha}_1, \ \vec{\alpha}_2, \ \dots,\ \vec{\alpha}_n$ 线性无关, $\vec{\alpha}_{n+1}=\lambda_1\vec{\alpha}_1+\lambda_2\vec{\alpha}_2+\dots+\lambda_n\vec{\alpha}_n$, 所以
$$r(B)=r(A)=n.$$因此$\vec{\alpha}_1, \ \vec{\alpha}_2, \ \dots,\ \vec{\alpha}_{n-1},\ \vec{\alpha}_{n+1}$ 线性无关.\\
(方法三)\\
等价证法二\ \ \ 利用矩阵的秩. 设矩阵$$A=(\vec{\alpha}_1,\vec{\alpha}_2,\dots,\vec{\alpha}_{n-1},\vec{\alpha}_{n+1}).$$ 则
\begin{displaymath}\begin{aligned}A&=(\vec{\alpha}_1,\vec{\alpha}_2,\dots,\vec{\alpha}_n)\begin{bmatrix}1&0&\dots&0&\lambda_1\\0&1&\dots&0\lambda_2\\
\vdots&\vdots&\ddots&\vdots&\vdots\\0&0&\dots&1&
\lambda_{n-1} \\0&0&\dots&0&\lambda_n\end{bmatrix}\\&=(\vec{\alpha}_1,\vec{\alpha}_2,\dots,\vec{\alpha}_n)C.\end{aligned}\end{displaymath}
其中$$|C|=\lambda_n\not=0$$即矩阵$C$可逆, 因此$$r(A)=r(\vec{\alpha}_1,\vec{\alpha}_2,\dots,\vec{\alpha}_n=n.$$
即$\vec{\alpha}_1,\vec{\alpha}_2,\dots,\vec{\alpha}_{n-1},\vec{\alpha}_{n+1}$ 线性无关.\\
\textbf{习题 \ref{4.15} 解答:}\\
反证法, 设$n$元线性方程组$k_1\vec{\alpha}_1+k_2\vec{\alpha}_2+\dots+k_n\vec{\alpha}_n=b$ 只有零解$k_1,\dots,k_n$,其中
$$a_i=(a_{i1},a_{i2},\dots,a_{in}),\ \ i=1,2,\dots,n$$
$$b=(c,0,\dots,0)$$
即
\begin{displaymath}\begin{bmatrix}a_{11}&a_{12}&\dots &a_{1n}\\a_{21}&a_{22}&\dots&a_{2n}\\ \vdots&\vdots&\ddots&\vdots\\ a_{n1}&a_{n2}&\dots&a_{nn}\end{bmatrix}
\begin{bmatrix}k_1\\k_2\\ \vdots\\ k_n\end{bmatrix}=\begin{bmatrix}c\\0\\ \vdots \\ 0\end{bmatrix}\end{displaymath}
由于系数矩阵的$n$个列向量线性无关, 因此$r(A)=n$. 于是方程组有唯一解, 而$c\not=0$, 因此此解非零. 与假设矛盾. 也就是说, 存在不全为零的数
$k_1,k_2,\dots,k_n$使得$$k_1\vec{\alpha}_1+k_2\vec{\alpha}_2+\dots+k_n\vec{\alpha}_n=(c,0,\dots,0),\ \ \ \ c\not=0.$$
\textbf{习题 \ref{4.16} 解答:}\\
令矩阵$A=(\vec{\alpha}_1,\vec{\alpha}_2,\vec{\alpha}_3)$, 三个向量线性相关等价于$|A|=0$, 而
$$|A|=\left|\begin{array}{ccc}a&1&1\\1&a&-1\\1&-1&a\end{array}\right|=(a+1)^2(a-2).$$
因此当$c=-1$或$c=2$时, $R(A)<3$, 此时向量组线性相关.\\
\textbf{习题 \ref{4.17} 解答:}\\
因为$\vec{\alpha}_1+\vec{\beta}, \ \vec{\alpha}_2+\vec{\beta}$线性相关, 故存在不全为零的数$\lambda_1$, $\lambda_2$, 使得
$$\lambda_1(\vec{\alpha}_1+\vec{\beta})+\lambda_2(\vec{\alpha}_2+\vec{\beta})=0.$$
由此得
$$\vec{\beta}=-\frac{\lambda_1}{\lambda_1+\lambda_2}\vec{\alpha}_1--\frac{\lambda_2}{\lambda_1+\lambda_2}\vec{\alpha}_2
=-\frac{\lambda_1}{\lambda_1+\lambda_2}\vec{\alpha}_1
-(1-\frac{\lambda_1}{\lambda_1+\lambda_2})\vec{\alpha}_2.$$
设$$c=-\frac{\lambda_1}{\lambda_1+\lambda_2}.$$
则$$\vec{\beta}=c\vec{\alpha}_1-(c+1)\vec{\alpha}_2,\ \ \ c\in\mathbb{R}$$
\textbf{习题 \ref{4.18} 解答:}\\
不一定. 例如
\begin{displaymath}\begin{aligned} &\vec{\alpha}_1=\begin{bmatrix}1\\2\end{bmatrix},\ \ \ \ \ &\vec{\alpha}_2=\begin{bmatrix}2\\4\end{bmatrix}\\&\vec{\beta}_1=\begin{bmatrix}-1\\-1\end{bmatrix}
, &\vec{\beta}_2=\begin{bmatrix}0\\0\end{bmatrix}\end{aligned}\end{displaymath}
此时,有
\begin{displaymath}\begin{aligned}&\vec{\alpha}_1+\vec{\beta}_1=\begin{bmatrix}1\\2\end{bmatrix}+\begin{bmatrix}-1\\-1\end{bmatrix}=\begin{bmatrix}0\\1\end{bmatrix}\\
&\vec{\alpha}_2+\vec{\beta}_2=\begin{bmatrix}2\\4\end{bmatrix}+\begin{bmatrix}0\\0\end{bmatrix}=\begin{bmatrix}2\\4\end{bmatrix}\end{aligned}\end{displaymath}
可以看到, $\vec{\alpha}_1+\vec{\beta}_1$ 和$\vec{\alpha}_2+\vec{\beta}_2$ 的对应分量不成比例, 是线性无关的.\\
\textbf{习题 \ref{4.19} 解答:}\\
方法一: 初等变换法.交换矩阵$A$的第一列和第三列, 不改变矩阵的秩, 得到$$A^{'}=\begin{bmatrix}-1&1&1\\0&-1&2\\1&0&1\end{bmatrix},\ \ \ \ r(A^{'})=r(A).$$ 再对矩阵$A^{'}$ 做初等行变换, 得到
$$\begin{bmatrix}-1&1&1\\0&-1&2\\1&0&1\end{bmatrix}\rightarrow \begin{bmatrix}-1&1&1\\0&-1&2\\0&1&2\end{bmatrix}\rightarrow \begin{bmatrix}-1&1&1\\0&-1&2\\0&0&4\end{bmatrix}$$
所以$r(A^{'})=3$, 从而$r(A)=3$.\\
方法二:主子式法. 由矩阵的(第二,三行)$\times$ (第二,三列)组成的主子式为$$\begin{bmatrix}-1&0\\0&1\end{bmatrix},$$ 其行列式为$-1$. 因此$r(A)\geq
2$. 而$|A|=-4$, 所以$r(A)=3$.\\
\textbf{习题 \ref{4.20} 解答:}\\
(1) 对$A$做初等行变换
$$A=\left[\begin{array}{ccccc}4&3&-5&2&3\\8&6&-7&4&2\\4&3&-8&2&7\\4&3&1&2&-5\\8&6&-1&4&-6\end{array}\right]\rightarrow \left[\begin{array}{ccccc}4&3&-5&2&3\\0&0&3&0&-4\\0&0&-3&0&4\\0&0&6&0&-8\\0&0&9&0&-12\end{array}\right]
\rightarrow \left[\begin{array}{ccccc}4&3&-5&2&3\\0&0&3&0&-4\\0&0&0&0&0\\0&0&0&0&0\\0&0&0&0&0\end{array}\right] $$
所以$r(A)=2$.\\
(2) 同样对$A$做行初等变换
\begin{displaymath}
\begin{aligned}A=&\left[\begin{array}{ccccc}0&1&1&1&1\\1&0&1&1&1\\1&1&0&1&1\\ 1&1&1&0&1\\1&1&1&1&0\end{array}\right]\rightarrow
\left[\begin{array}{ccccc}1&1&1&1&0 \\0&1&1&1&1\\1&0&1&1&1\\1&1&0&1&1\\ 1&1&1&0&1\end{array}\right]
\rightarrow \left[\begin{array}{ccccc}1&1&1&1&0 \\0&1&1&1&1\\0&-1&0&0&1\\0&0&-1&0&1\\ 0&0&0&-1&1\end{array}\right]\\
\rightarrow &\left[\begin{array}{ccccc}1&1&1&1&0 \\0&1&1&1&1\\0&0&1&1&2\\0&0&-1&0&1\\ 0&0&0&-1&1\end{array}\right]
\rightarrow \left[\begin{array}{ccccc}1&1&1&1&0 \\0&1&1&1&1\\0&0&1&1&2\\0&0&0&1&3\\ 0&0&0&-1&1\end{array}\right]
\rightarrow \left[\begin{array}{ccccc}1&1&1&1&0 \\0&1&1&1&1\\0&0&1&1&2\\0&0&0&1&3\\ 0&0&0&0&4\end{array}\right] \end{aligned}\end{displaymath}
因此$r(A)=5$.\\
(3)注意到$A=\begin{bmatrix}a_1\\a_2\\ \vdots\\a_n\end{bmatrix}\begin{bmatrix}b_1&b_2&\dots&b_n\end{bmatrix}$, 因此$r(A)=1$.
\textbf{习题 \ref{4.21} 解答:}\\
$$\lambda E-A=\begin{bmatrix}1&1&-1\\-2&-2&2\\3&3&-3\end{bmatrix}\rightarrow\begin{bmatrix}1&1&-1\\0&0&0\\0&0&0\end{bmatrix}.$$ 因此$r(\lambda E-A)=1$.\\
\textbf{习题 \ref{4.22} 解答:}\\
(1) $$3E-A=\begin{bmatrix}-1&1&1&-1\\1&-1&-1&1\\1&-1&-1&1\\-1&1&1&-1\end{bmatrix}\rightarrow \begin{bmatrix}-1&1&1&-1\\0&0&0&0\\0&0&0&0\\0&0&0&0\end{bmatrix}$$
因此$r(3E-A)=1.$\\
(2)\begin{displaymath}\begin{aligned}7E-A=&\begin{bmatrix}3&1&1&-1\\1&3&-1&1\\1&-1&3&1\\-1&1&1&3\end{bmatrix}
\rightarrow \begin{bmatrix}1&-1&-1&-3\\0&4&0&4\\0&0&4&4\\0&4&4&8\end{bmatrix}
\rightarrow  \begin{bmatrix}1&-1&-1&-3\\0&1&0&1\\0&0&1&1\\0&1&1&2\end{bmatrix}\\
\rightarrow & \begin{bmatrix}1&-1&-1&-3\\0&1&0&1\\0&0&1&1\\0&0&1&1\end{bmatrix}
\rightarrow  \begin{bmatrix}1&-1&-1&-3\\0&1&0&1\\0&0&1&1\\0&0&0&0\end{bmatrix}
\end{aligned}\end{displaymath}
因此$r(7E-A)=3.$\\
\textbf{习题 \ref{4.23} 解答:}\\
$$A=\begin{bmatrix}3&1&-1&-2\\1&-5&2&1\\2&6&-3&-3\\-1&-11&5&4\end{bmatrix}\rightarrow\begin{bmatrix}1&-5&2&1\\ 0&16&-7&-5\\0&16&-7&-5\\0&-16&7&5\end{bmatrix}
\rightarrow\begin{bmatrix}1&-5&2&1\\ 0&16&-7&-5\\0&0&0&0\\0&0&0&0\end{bmatrix}$$
所以$r(A)=2$.\\
(1) $$(A\ b)=\begin{bmatrix}3&1&-1&-2&2\\1&-5&2&1&-1\\2&6&-3&-3&3\\-1&-11&5&4&-4\end{bmatrix}\rightarrow
\begin{bmatrix}1&5&2&1&-1\\0&16&-7&-5&5\\0&16&-7&-5&5\\0&-16&7&5&-5\end{bmatrix}\rightarrow
\begin{bmatrix}1&5&2&1&-1\\0&16&-7&-5&5\\0&0&0&0&0\\0&0&0&0&0\end{bmatrix} $$
所以$r(A)=2$.\\
(2) $$(A\ b)=\begin{bmatrix}3&1&-1&-2&2\\1&-5&2&1&-1\\2&6&-3&-3&c\\-1&-11&5&4&-4\end{bmatrix}\rightarrow
\begin{bmatrix}1&5&2&1&-1\\0&16&-7&-5&5\\0&16&-7&-5&c+3\\0&-16&7&5&-5\end{bmatrix}\rightarrow
\begin{bmatrix}1&5&2&1&-1\\0&16&-7&-5&5\\0&16&-7&-5&c-3\\0&-16&7&5&-5\end{bmatrix}\rightarrow$$
若$c=3$, 则$r(A)=2$. 若$c\not=3$, 则$r(A)=3$.\\
\textbf{习题 \ref{4.24} 解答:}\\
设$$A=(\vec{a}_1,\vec{a}_2,\dots,\vec{a}_n),\ \ \ \ \mbox{以及}\ \ \ \   B=(\vec{b}_1,\vec{b}_2,\dots,\vec{b}_n).$$
则$$A+B=(\vec{a}_1+\vec{b}_1,\vec{a}_2+\vec{b}_2,\dots,\vec{a}_n+\vec{b}_n).$$
由于$\vec{a}_i$可以由$A$的最大线性无关组表出.$\vec{b}_i$可由$B$ 的最大线性无关组表出,因此$\vec{a}_i+\vec{b}_i$可由$A$的最大线性无关组和$B$的最大线性无关组表出.因此$r(A+B)\leq r(A)+r(B)$.\\
\textbf{习题 \ref{4.25} 解答:}\\
$AB=0,\ \Leftrightarrow \ B$ 的列向量是方程组$A\vec{x}=0$ 的解
  \quad  \quad  \quad \quad   $\Leftrightarrow\  r(B)\leq n-r(A)$
   \quad  \quad  \quad \quad  $\Leftrightarrow\ r(A)+r(B)\leq n.$\\
\textbf{习题 \ref{4.26} 解答:}\\
因为$A^2=E,$ 所以$A$非奇异, 因此$r(A)=n.$又有$$(A+E)(A-E)=A^2+A-A-E=0,$$ 由第25 题结论$$r(A+E)+r(A-E)\leq n.$$ 由第24题结论,  $$r(A+E)+r(A-E)\geq r(A+E+A-E)=r(A)=n.$$
因此有$$r(A+E)+r(A-E)=n.$$
\textbf{习题 \ref{4.27} 解答:}\\
由于$r(A)=r$, 因此经过有限次的初等行变换, 总能将$A$的后$n-r$行变为0, 即 矩阵$$P_sP_{s-1}\dots P_1A$$的后$n-r$行为0, 其中$P_i(i=1,2,\dots,s)$代表行初等变换. 令$$P=P_s\dots P_1,$$ 则
$P$为可逆矩阵,且$PA$的后$n-r$ 行为0. 由于$P^{-1}$正在$PA$后只对列做变换, 不改变后$n-r$ 行为0 的性质, 所以存在一个$n\times n$ 可逆矩阵$P$使得$PAP^{-1}$ 的后$n-r$ 行全为零.\\
\textbf{习题 \ref{4.28} 解答:}\\
(1)\ $BC=0\ \Leftrightarrow \ C$ 的列向量是方程组$B\vec{x}=0$的解
\quad \quad \quad \quad \quad  \ \ $ \Leftrightarrow B\vec{x}=0$ 的解空间的维数$\geq r$
\quad \quad \quad \quad \quad  \ \  $ \Leftrightarrow\ r(B)\leq r-r=0$
\quad \quad \quad \quad \quad  \ \  $ \Leftrightarrow\ r(B)=0$
所以$B=0$.\\
(2) $BC=C \ \Leftrightarrow \ (B-E)C=0$. 由(1) 中的证明,知$B-E=0$,即$B=E$.\\
\textbf{习题 \ref{4.29} 解答:}\\
(1) 因为\begin{displaymath}\begin{aligned}AP&=A(\vec{x},\vec{y},\vec{z})=(A\vec{x},A^2\vec{x},A^3\vec{x})=(A\vec{x},A^2\vec{x},3A\vec{x}-A^2\vec{x})\\
&=(\vec{x},A\vec{x},A^2\vec{x})\begin{bmatrix}0&0&0\\1&0&3\\0&1&-1\end{bmatrix}=P\begin{bmatrix}0&0&0\\1&0&3\\0&1&-1\end{bmatrix}\end{aligned}\end{displaymath}
所以\begin{displaymath}B=\begin{bmatrix}0&0&0\\1&0&3\\0&1&-1\end{bmatrix}\end{displaymath}
(2) 由于$A^3\vec{x}=3A\vec{x}-A^2\vec{x}$, 有$$A(3\vec{x}-A\vec{x}-A^2\vec{x})=\vec{0}$$
而向量组$\vec{x},A\vec{x},A^2\vec{x}$ 线性无关, 因此$$3\vec{x}-A\vec{x}-A^2\vec{x}\not=\vec{0}.$$
即线性方程组$A\vec{x}=0$有非零解, 因此$$r(A)<3,\ \ \  |A|=0$$

%%%%%%%%%%%%%%%%%%%%%%%%%%%%%%%%%%%%%%%%%%%%%%%%%%%%%%%%%%%%%%%%%%%%%%%%%%%%%%%%%%%%%%%
%%%%%%%%%%%%%%%%%%%%%%%%%%%%%%%%%%%%%%%%%%%%%%%%%%%%%%%%%%%%%%%%%%%%%%%%%%%%%%%%%%%%%%%

\chapter{线性方程组的解理论}

\section{知识点解析}

\begin{thm}
齐次线性方程组$A_{m\times n}\vec{X}=\vec{0}$的解集合为$N(A)$是$\mathbb{R}^n$中的子空间.
\end{thm}

\begin{Def}
解集$N(A)$称为齐次线性方程组 $A_{m\times n}\vec{X}=\vec{0}$ 的解空间;
           $N(A)$的一组基称为齐次线性方程组 $A_{m\times n}\vec{X}=\vec{0}$ 的一组基础解系.
\end{Def}

\begin{thm}
设$A$是 $m\times n$ 矩阵, $r(A) = r \leq n$, 则齐次线性方程组 $A_{m\times n}\vec{X}=\vec{0}$ 的基础解系含有 $n - r$ 个解向量,即$\dim N(A)= n - r$.
\end{thm}

\begin{thm}
\begin{displaymath}\left\{\begin{aligned}
&r(A,\vec{b})\not=r(A) \ \ \ \ \ \ \ \ \ \ \ \ \ \  \ \ \ \ \ \ \ \ \ \ \ \   \Leftrightarrow A_{m\times n}\vec{X}=\vec{b}\mbox{无解};\\
&r(A,\vec{b})=r(A)=n(\mbox{列数}) \ \ \ \ \ \ \ \ \ \ \ \ \ \Leftrightarrow A_{m\times n}\vec{X}=\vec{b}\mbox{有唯一解};\\
&r(A,\vec{b})=r(A)<n(\mbox{列数}) \ \ \ \ \ \ \ \ \ \ \ \ \ \Leftrightarrow A_{m\times n}\vec{X}=\vec{b}\mbox{有无穷多个解}.
\end{aligned}\right.\end{displaymath}

\end{thm}

\begin{thm}
对非齐次线性方程组 $A_{m\times n}\vec{X}=\vec{b}$,  设 $$r(A)=r(A,\vec{b})=r \leq n$$
且$\vec{\xi}_0\in  N(A, \vec{b})$是一个特解, 则
$$N(A,\vec{b})=\vec{\xi}_0+N(A).$$
进一步,设 $\vec{\xi}_1, \vec{\xi}_2,\dots,\vec{\xi}_{n-r}$ 是导出组 $A\vec{X}=\vec{0}$ 的一组基础解系, 则$A\vec{X}=\vec{b}$的通解(一般解)为:
$$\vec{\gamma}=\vec{\xi}_0+k_1\vec{\eta}_1+k_2\vec{\eta}_2+\dots+k_{n-r}\vec{\eta}_{n-r}$$
其中$k_1, k_2,\dots,k_{n-r}\in\mathbb{R}.$
\end{thm}

\begin{thm}
设$A$为$m\times n$型矩阵, $B$为 $m\times s$型矩阵, 则
\begin{displaymath}\left\{\begin{aligned}
&r(A)<r(A,B)\ \ \ \ \ \ \ \ \ \ \ \ \ \Leftrightarrow \mbox{矩阵方程}AX=B\mbox{无解};\\
&r(A)=r(A,B)=n\ \ \ \ \ \ \ \ \Leftrightarrow \mbox{矩阵方程}AX=B\mbox{有唯一解};\\
&r(A)=r(A,B)<n\ \ \ \ \ \ \ \Leftrightarrow \mbox{矩阵方程}AX=B\mbox{有无穷多解}\end{aligned}\right.\end{displaymath}
\end{thm}

%%%%%%%%%%%%%%%%%%%%%%%%%%%%%%%%%%%%%%%%%%%%%%%%%%%%%%%%%%%%%%%%%%%%%%%%%%%%%%%%

\section{例题讲解}

\begin{eg}
求解下列方程组:
\begin{displaymath}\left\{\begin{aligned}
&x_1+x_2-5x_4=0\\
&x_3+x_4=0\end{aligned}\right.\end{displaymath}

解: 系数矩阵为

\begin{displaymath}A=\begin{bmatrix}1&1&0&-5\\0&0&1&1\end{bmatrix}\end{displaymath}

已是简化阶梯形矩阵,故$x_1$, $x_3$为主变量,而$x_2$, $x_4$为自由未知量 (可取任意实数). 于是,取基础解系和通解分别为:

\begin{displaymath}
\vec{\eta}_1=\begin{bmatrix}-1\\1\\0\\0\end{bmatrix},\ \vec{\eta}_2= \begin{bmatrix}5\\0\\-1\\1\end{bmatrix}; \ \ \Rightarrow \ \ \vec{\eta}=k_1\vec{\eta}_1+k_2\vec{\eta}_2=\begin{bmatrix}x_1\\x_2\\x_3\\x_4
\end{bmatrix}=\begin{bmatrix}-k_1+5k_2\\k_1\\-k_2\\k_2\end{bmatrix}\end{displaymath}

其中$k_1$, $k_2\in\mathbb{R}$.
\end{eg}

\begin{eg}
求下列方程组的基础解系:

\begin{displaymath}\left\{\begin{aligned}
&x_1+3x_2-5x_3-x_4+2x_5=0\\
&2x_1+6x_2-8x_3+5x_4+3x_5=0\\
&x_1+3x_2-3x_3+6x_4+x_5=0
\end{aligned}\right.\end{displaymath}

解: 用初等行变换化系数矩阵为简化阶梯形矩阵:

\begin{displaymath}
A=\begin{bmatrix}1&3&-5&-1&2\\2&6&-8&5&3\\1&3&-3&6&1\end{bmatrix}\rightarrow
\begin{bmatrix}1&3&-5&-1&2\\0&0&2&7&-1\\0&0&2&7&-1\end{bmatrix}\rightarrow
\begin{bmatrix}1&3&0&\frac{33}{2}&\frac{-1}{2}\\0&0&1&\frac{7}{2}&\frac{-1}{2}\\
0&0&0&0&0\end{bmatrix}\end{displaymath}

$r=2$个主变量为$x_1,\ x_3,\ n-r = 5-2 = 3$个自由未知量为 $x_2,\ x_4\ , x_5$. 基础解系取为
\begin{displaymath}\vec{\eta}_1=\begin{bmatrix}-3\\1\\0\\0\\0\end{bmatrix},\ \ \vec{\eta}_2=\begin{bmatrix}-\frac{33}{2}\\0\\ \frac{7}{2}\\1\\0\end{bmatrix},\ \ \vec{\eta}_3=\begin{bmatrix}\frac{1}{2}\\7\\ \frac{1}{2}\\0\\1\end{bmatrix}\end{displaymath}

方程组的通解是:
\begin{displaymath}
\vec{X}=k_1 \vec{\eta}_1+k_2\vec{\eta}_2+k_3\vec{\eta}_3=\begin{bmatrix}
x_1\\x_2\\x_3\\x_4\\x_5\end{bmatrix}=\begin{bmatrix}-3k_1-\frac{33}{2}k_2+\frac{1}{2}
k_3\\k_1\\-\frac{7}{2}k_2+\frac{1}{2}k_3\\k_2\\k_3\end{bmatrix}\end{displaymath}

其中 $k_1,\ k_2,\ k_3$ 可取任意实数.
\end{eg}

\begin{eg}
设$A$为$m \times n$ 型实矩阵,  则 $r(A) = r(A^TA) = r(AA^T)$.
证明: 因为, 如果 $A\vec{X}=\vec{0}$ (记为方程(1)), 则 $A^T(A\vec{X})=\vec{0}$ (记为方程(2)), 所以(1)的解都是(2)的解. 反之, 若$\vec{X}$ 是 $A^TA\vec{X}=\vec{0}$ 的任一解,  则两边左乘$\vec{X}^T$, 有
\begin{displaymath}
\vec{X}^T(A^TA\vec{X})=(A\vec{X})^TA\vec{X}=\vec{X}^T\vec{0}=0.
\end{displaymath}
设列向量 $$A\vec{X}=(b_1,b_2\dots,b_m)^T,$$
则上式可化为:
$$(A\vec{X})^TA\vec{X}=b_1^2+b_2^2+\dots+b_m^2=0.$$
得$$b_1=b_2=\dots=b_m=0.$$
所以$A\vec{X}=(0,0,\dots,0)^T=\vec{0},$ 即(2)的解也是(1)的解.

综上,有 $N(A)=N(A^TA)$, 从而有$n-r(A) = n-r(A^TA)$, 故$r(A) = r(A^TA)$.
同理,考虑$A^T$, 有 $r(A^T) = r(AA^T)$. 再由$r(A)=r(A^T)$, 即得结论.
\end{eg}

\begin{eg}
若 $A$ 为 $m\times n$ 实矩阵, $r(A) = n < m$, 下式成立的是 (     ):
\begin{displaymath}
\begin{aligned}
&(A) |A^TA|=0; \ \ \ \ \ \ \ \ \ \ \ \ \ \ (B)|A^TA|\not=0;  \\
&(C) r(AA^T)=m;\ \ \ \ \ \ \ \ \ \ \ (D)r(A^TA)=n.
\end{aligned}
\end{displaymath}

答案为:D.
\end{eg}

\begin{eg}
设 $A$ 是 $m\times n$ 矩阵,  则下列结论中成立的是(    \ \     )  \\
(A) 若 $A_{m\times n}\vec{X}=\vec{0}$ 仅有零解,   则 $A_{m\times n}\vec{X}=\vec{b}$ 有唯一解;\\
(B)若 $A_{m\times n}\vec{X}=\vec{0}$ 有非零解,   则 $A_{m\times n}\vec{X}=\vec{b}$ 有无穷多解;\\
(C)若 $A_{m\times n}\vec{X}=\vec{b}$ 有无穷多解,   则 $A_{m\times n}\vec{X}=\vec{0}$ 仅由零解;\\
(D)若 $A_{m\times n}\vec{X}=\vec{b}$ 有无穷多解,   则 $A_{m\times n}\vec{X}=\vec{0}$ 有非零解;

答案为D.
\end{eg}


\begin{eg}
设 $A$ 是 $m\times n$ 矩阵,  则下列结论中成立的是( \ \ ):\\
(A) 若 $r(A) = m$,  则 $A_{m\times n}\vec{X}=\vec{0}$  有非零解.\\
(B) 若 $r(A) = m$,  则 $A_{m\times n}\vec{X}=\vec{0}$  仅有非零解.\\
(C) 若 $r(A) = m$,  则 $A_{m\times n}\vec{X}=\vec{b}$  无解.\\
(D) 若 $r(A) = m$,  则 $A_{m\times n}\vec{X}=\vec{b}$ 有解.\\
(E) 若 $r(A) = m$,  则 $A_{m\times n}\vec{X}=\vec{b}$ 有唯一解.
(F) 若 $r(A) = m$,  则 $A_{m\times n}\vec{X}=\vec{b}$ 有无穷多解.

答案为D.
\end{eg}

\begin{eg}
求下列方程组的通解:
\begin{displaymath}\left\{\begin{aligned}
&x_1-x_2-x_3+3x_5=-1\\
&2x_1-2x_2-x_3+2x_4+4x_5=-2\\
&3x_1-3x_2-x_3+4x_4+5x_5=-3\\
&x_1-x_2+x_3+x_4+8x_5=2\end{aligned}\right.\end{displaymath}

解: 用初等行变换将增广矩阵$(A, \vec{b})$化为简化阶梯形矩阵:
\begin{displaymath}
(A,\vec{b})=\begin{bmatrix}1&-1&-1&0&3&-1\\2&-2&-1&2&4&-2\\3&-3&-1&4&5&-3\\1&-1&
1&1&8&2\end{bmatrix}\rightarrow\dots\rightarrow\begin{bmatrix}1&-1&0&0&7&1\\0&0&
1&0&4&2\\0&0&0&1&-3&-1\\0&0&0&0&0&0\end{bmatrix}=C
\end{displaymath}

由于主元素位于第$1,3,4$列,不在最后一列,知$r(A) = r(A, \vec{b}) = 3$,  所以方程组有解. 且 $x_1, x_3, x_4$为主变量,而$x_2, x_5$为自由变量.

令$\begin{bmatrix}x_2\\ x_5\end{bmatrix}=\begin{bmatrix}0\\0\end{bmatrix}$,由$C$ 的最后一列, 得特解为:
\begin{displaymath}
\vec{\xi}_0=(\ ,0,\ ,\ ,0)^T.
\end{displaymath}

令自由变量$\begin{bmatrix}x_2\\ x_5\end{bmatrix}=\begin{bmatrix}1\\0\end{bmatrix}$与$\begin{bmatrix}0\\1\end{bmatrix}$ 并由C的第$2,5$列, 得导出组的一组基础解系为:
\begin{displaymath}
\vec{\eta}_1=\begin{bmatrix}1\\1\\0\\0\\0\end{bmatrix},\ \ \vec{\eta}_2=\begin{bmatrix}-7\\0\\-4\\3\\1\end{bmatrix}
\end{displaymath}

方程组的通解为:
\begin{displaymath}
\vec{\eta}=\vec{\xi}_0+k_1\vec{\eta}_1+k_2\vec{\eta}2=\begin{bmatrix}
1+k_1-7k_2\\k_1\\2-4k_2\\-1+3k_2\\k_2\end{bmatrix}.
\end{displaymath}
其中$k_1,k_2\in\mathbb{R}$.

\end{eg}

\begin{eg}
设方程组$A\vec{X}=\vec{b}$的增广系数矩阵为
\begin{displaymath}
(A,\vec{b})=\begin{bmatrix}1&1&2&-1&1\\1&-1&-2&-7&3\\0&1&a&t&t-3\\1&1&2&t-2&t+3
\end{bmatrix}\end{displaymath}
问: $a, t$取何值时,  方程组无解, 有唯一解, 无穷多解? 有无穷多解时,  求通解.

解: 用初等行变换将增广系数矩阵化为阶梯形:
\begin{displaymath}
\begin{bmatrix}1&1&2&-1&1\\0&-2&-4&-6&2\\0&1&a&t&t-3\\0&0&0&t-1&t+2\end{bmatrix}
\rightarrow \begin{bmatrix}1&1&2&-1&1\\0&1&2&3&-1\\0&0&a-2&t-3&t-2\\0&0&0&t-1&t+2\end{bmatrix}
\end{displaymath}
\\(1) 当 $a\not=2, t \not=1$ 时, $r(A,\vec{b})=r(A)=4=n$,  方程组解唯一;\\
(2)当 $t = 1$ 时,$r(A, \vec{b})=4\not=3=r(A)$,  方程组无解;\\
(3)当$a=2, t \not=1$时, 可进一步进行初等行变换
\begin{displaymath}
\begin{bmatrix}1&1&2&-1&1\\0&1&2&3&-1\\0&0&a-2&t-3&t-2\\0&0&0&t-1&t+2\end{bmatrix}
\rightarrow\begin{bmatrix} 1&1&2&-1&1\\0&1&2&3&-1\\0&0&0&-2&-4\\0&0&0&t-1&t+2\end{bmatrix}\rightarrow
\begin{bmatrix} 1&1&2&-1&1\\0&1&2&3&-1\\0&0&0&1&2\\0&0&0&0&t-4\end{bmatrix}\end{displaymath}
\\ \ \ (3-1)当$a=2$, $t\not=4$时, 方程组无解;\\
\ \ (3-2) 当$a=2$, $t=4$时, 方程组有无穷多解, 此时,
\begin{displaymath}
\begin{bmatrix} 1&1&2&-1&1\\0&1&2&3&-1\\0&0&0&1&2\\0&0&0&0&0\end{bmatrix}
\rightarrow\begin{bmatrix}1&0&0&0&10\\0&1&2&0&-7\\0&0&0&1&2\\0&0&0&0&0\end{bmatrix},\ \ \Rightarrow \vec{X}=\begin{bmatrix}10\\-7\\0\\2\end{bmatrix}+k_1\begin{bmatrix}0\\-2\\1\\0\end{bmatrix}.
\end{displaymath}
其中$k_1$为任意实数.
\end{eg}


\begin{eg}
判断下面平面的位置关系?
\begin{displaymath}
\left\{\begin{aligned}
&\pi_1:2x-y+x=2\\
&\pi_2:x+y+z=4\\
&\pi_3:2x-3y-z=2\end{aligned}\right.\end{displaymath}

解: 由于
\begin{displaymath}
\overline{A}=\begin{bmatrix}2&-1&1&2\\1&1&1&4\\2&-3&-1&2\end{bmatrix}\rightarrow
\begin{bmatrix}1&1&1&4\\0&1&1&0\\0&0&1&-3\end{bmatrix}.\end{displaymath}
于是$$r(A)=r(\overline{A})=3.$$
所以该三个平面相交于一点.
\end{eg}

\begin{eg}
判断下面平面的位置关系?
\begin{displaymath}\left\{\begin{aligned}
&\pi_1:x+2y+z=-2\\
&\pi_2:2x+3y=-1\\
&\pi_3:x-y-5z=7\end{aligned}\right.\end{displaymath}

解: 由于
\begin{displaymath}
\overline{A}=\begin{bmatrix}1&2&1&-2\\2&3&0&-1\\1&-1&-5&7\end{bmatrix}\rightarrow\begin{bmatrix}
1&0&-3&4\\0&1&2&-3\\0&0&0&0\end{bmatrix}\end{displaymath}
于是$$r(A)=r(\overline{A})=2.$$
所以该三个平面有交点, 交点构成一直线还是一平面?

又$\left\{\begin{aligned}&\vec{\beta}_1=(1,2,1,-2)\\&\vec{\beta}_2=(2,3,0,-1)\\&
\vec{\beta}_3=(1,-1,-5,7)\end{aligned}\right\}$两两线性无关, 所以交点构成一条直线.

\end{eg}

\begin{eg}
设\begin{displaymath}
A=\begin{bmatrix}0&1&2\\1&1&-1\\2&4&2\end{bmatrix},\ \ B=\begin{bmatrix}1&1&2\\2&0&-1\\-1&2&1\end{bmatrix}.\end{displaymath}
已知$AX=B$, 求$X$.

解 对分块增广矩阵$(A,B)$作初等行变换:
\begin{displaymath}\begin{aligned}
\begin{bmatrix}A&B\end{bmatrix}=&\begin{bmatrix}0&1&2&1&1&2\\1&1&-1&2&0&-1\\2&4&2&-1&2&1\end{bmatrix}\rightarrow
\begin{bmatrix}1&1&-1&2&0&-1\\0&1&2&1&1&2\\0&2&4&-5&2&3\end{bmatrix}\\ \rightarrow &\begin{bmatrix}1&0&-3&1&-1&-3\\0&1&2&1&1&2\\0&0&0&-7&0&-1\end{bmatrix}
\end{aligned}\end{displaymath}

得到$r(A)<r(A,B)$, 所以方程$AX=B$无解.
\end{eg}

\begin{eg}
设
\begin{displaymath}
A=\begin{bmatrix}1&1&-1\\2&3&-1\\1&2&0\end{bmatrix},\ \ B=\begin{bmatrix}-1&1&2\\2&0&-1\\3&-1&-3\end{bmatrix}.\end{displaymath}
已知$AX=B$, 求$X$.

解: 用高斯消元法, 得:
\begin{displaymath}\begin{aligned}
\begin{bmatrix}A&B\end{bmatrix}=&\begin{bmatrix}1&1&-1&-1&1&2\\2&3&-1&2&0&-1
\\1&2&0&3&01&3\end{bmatrix}\rightarrow\dots\rightarrow\begin{bmatrix}
1&0&-2&-5&3&7\\0&1&1&4&-2&-5\\0&0&0&0&0&0\end{bmatrix}\end{aligned}\end{displaymath}
得到$$r(A)=r(A,B)=2.$$于是方程有解,且有无穷多解.前两个为主变量,第三个为自由变量. 于是导出组的基础解系为:
$$\vec{\eta}=\begin{bmatrix}2\\-1\\1\end{bmatrix}.$$
三个(非齐次)特解分别为:
\begin{displaymath}
\vec{\xi}_1=\begin{bmatrix}-5\\4\\0\end{bmatrix},\ \ \vec{\xi}_2=\begin{bmatrix}3\\-1\\0\end{bmatrix},\ \ \vec{\xi}_3=\begin{bmatrix}7\\-5\\0\end{bmatrix}.
\end{displaymath}
故方程组$A\vec{X}_k=\vec{b_k}(1\leq k \leq 3)$的通解分别为
\begin{displaymath}\begin{aligned}
&\vec{\beta}_1=t_1\vec{\eta}+\vec{\xi}_1,\\
&\vec{\beta}_2=t_2\vec{\eta}+\vec{\xi}_2,\\
&\vec{\beta}_3=t_3\vec{\eta}+\vec{\xi}_3.\end{aligned}\end{displaymath}

其中$t_1,\ t_2,\ t_3$为任意实数, 由于矩阵方程可看做是三个独立的非齐次线性方程组, 它们的通解中的任意常数互相之间没有关联, 所以用不同的符号$t_i$表示, 所以最后得到
\begin{displaymath}
X=\begin{bmatrix}\vec{\xi}_1&\vec{\xi}_1&\vec{\xi}_1\end{bmatrix}+\vec{eta}\begin{bmatrix}
t_1&t_2&t_3\end{bmatrix}=\begin{bmatrix}2t_1-5&2t_2+3&2t_3+7\\-t_1+4&-t_2-2&-t_3-5\\t_1&t_2&t_3\end{bmatrix}
\end{displaymath}
其中$t_1,\ t_2,\ t_3\in\mathbb{R}$.
\end{eg}

%%%%%%%%%%%%%%%%%%%%%%%%%%%%%%%%%%%%%%%%%%%%%%%%%%%%%%%%%%%%%%%%%%%%%%%%%%%%%%%

\section{课后习题}

\begin{ex}\label{5.1}
求解齐次线性方程组\\
\begin{equation*}
\begin{cases}
x_1+x_2-x_3-x_4+x_5=0\\
2x_1+x_2+x_3+x_4+4x_5=0\\
4x_1+3x_2-x_3-x_4+6x_5=0\\
x_1+2x_2-4x_3-4x_4-x_5=0
\end{cases}
\end{equation*}
的通解.
\end{ex}

\begin{ex}\label{5.2}
$\lambda$何值时,齐次线性方程组
\begin{equation*}
\begin{cases}
x_1+x_2-2x_3=0\\
-x_1+\lambda x_2+5x_3=0\\
x_1+3x_2=0\\
x_1+6x_2+(\lambda+1) x_3=0
\end{cases}
\end{equation*}
有非零解?并在此时求出它的一个基础解系.
\end{ex}

\begin{ex}\label{5.3}
求解齐次线性方程组
\begin{equation*}
\begin{cases}
x_1+2x_2+ x_3+x_4+ x_5=0\\
2x_1+4x_2+3x_3+x_4+ x_5=0\\
-x_1-2x_2+ x_3+3x_4-3x_5=0\\
2x_3+5x_4-2x_5=0
\end{cases}
\end{equation*}
\end{ex}

\begin{ex}\label{5.4}
解线性方程组
\begin{equation*}
\begin{cases}
2x_1-x_2-x_3=-1\\
x_1+x_2-2x_3=1\\
4x_1-6x_2+2x_3=-6
\end{cases}
\end{equation*}
\end{ex}

\begin{ex}\label{5.5}
$t$为何值时,齐次线性方程组
\begin{equation*}
\begin{cases}
x_1+x_2+tx_3=0\\
x_1-x_2+2x_3=0\\
-x_1+tx_2+x_3=0
\end{cases}
\end{equation*}
有非零解?此时,求出其一般解.
\end{ex}

\begin{ex}\label{5.6}
证明:设$n$元齐次线性方程组$A\vec{x}=0$ 的全体解集合为$V$,则$V$是$\mathbb{R}^n$的一个子空间.
\end{ex}

\begin{ex}\label{5.7}
求下列齐次线性方程组的一个基础解系.

(1)$\left\{\begin{aligned}&2x_1-4x_2+5x_3+3x_4=0\\&3x_1-6x_2+4x_3+2x_4=0\\&4x_1-8x_2+17x_3+11x_4=0\end{aligned}
\right.$

(2)$\left\{\begin{aligned}& x_1+2x_2+3x_3+7x_4=0 \\&3x_1+2x_2+ x_3-3x_4=0 \\&     x_2+2x_3+6x_4=0 \\&5x_1+4x_2+3x_3- x_4=0     \end{aligned}\right.$
\end{ex}

\begin{ex}\label{5.8}
求下列齐次线性方程组的一个基础解系.

(1)$\left\{\begin{aligned}& x_1-2x_2+3x_3-4x_4=0 \\&       x _2- x_3+ x_4=0 \\&  x _1+3x_2      -3x_4=0 \\&x_1-4x_2+3x_3-2x_4=0     \end{aligned}\right.$

(2)$\left\{\begin{aligned}& 2x_1+3x_2- x_3+5x_4=0 \\&3x_1+ x_2+2x_3-7x_4=0 \\&4x_1+ x_2-3x_3+6x_4=0 \\& x_1-2x_2+4x_3-7x_4=0    \end{aligned}\right.$
\end{ex}

\begin{ex}\label{5.9}
求下列非齐次线性方程组的通解.

(1)$\left\{\begin{aligned}& x_1      - x_3+ x_4=2 \\& x_1-x_2+ 2x _3+ x_4=1 \\&2x_1-x_2+ x_3+2x_4=3 \\&3x_1-x_2      +3x_4=5    \end{aligned}\right.$

(2)$\left\{\begin{aligned}& x_1-2x_2+3x_3-4x_4= 4 \\&       x _2- x_3+ x_4=-3 \\& x_1+3x_2      + x_4= 1 \\&    -7x_2+3x_3+ x_4=-3 \\& x_1-3x_2+2x_3+3x_4=-5     \end{aligned}\right.$
\end{ex}

\begin{ex}\label{5.10}
求下列非齐次线性方程组的通解.

(1)$\left\{\begin{aligned}&x_1+x_2     -2x_4=-6 \\&4x_1-x_2-x_3- x_4= 1 \\&3x_1-x_2-x_3      = 0     \end{aligned}\right.$

(2)$\left\{\begin{aligned}&x_1+ x_2-2x_3+3x_4= 0 \\&2x_1+ x_2-6x_3+4x_4=-1 \\& 3x_1+2x _2-8x_3+7x_4=-1 \\& x_1- x_2-6x_3- x_4= 0     \end{aligned}\right.$
\end{ex}

\begin{ex}\label{5.11}
求解下列非齐次线性方程组.

(1)$\left\{\begin{aligned}&6x_1-2x_2+2x_3+ x_4=3 \\& x_1- x_2      + x_4=1 \\&2x_1      + x_3+3x_4=2     \end{aligned}\right.$

(2)$\left\{\begin{aligned}&x_1- x_2+2x_3+ x_4=1 \\&2x_1- x_2+ x_3+2x_4=3 \\& x_1- x_3+ x_4=2 \\&3x_1-x_2     +3x_4=5     \end{aligned}\right.$
\end{ex}

\begin{ex}\label{5.12}
求解下列非齐次线性方程组.

(1)$\left\{\begin{aligned}&x_1-3x_2+5x_3=0 \\&2x_1-x_2-3x_3=11 \\&2x_1+x_2-3x_3=5     \end{aligned}\right.$

(2)$\left\{\begin{aligned}& 3x_1+2x_2+x_3+x_4+x_5=7 \\&3x_1+2x_2+x_3+x_4-3x_5=-1 \\&5x_1+4x_2+3x_3+3x_4-x_5=9    \end{aligned}\right.$
\end{ex}

\begin{ex}\label{5.13}
 判断下列命题是否正确.

设$A\in M_{m.n}$, 对于齐次线性方程组$A\vec{x}=\vec{0}$, 若向量组$\eta_1,\eta_2,\eta_3$是它的一个基础解系,则

a. $\eta_1+\eta_2, \eta_2+\eta_3, \eta_3+\eta_1$也是$A\vec{x}=\vec{0}$的一个基础解系.

b. $\eta_1-\eta_2, \eta_2-\eta_3, \eta_3-\eta_1$也是$A\vec{x}=\vec{0}$的一个基础解系.

c. 与$\eta_1,\eta_2,\eta_3$等价的向量组$\alpha_1,\alpha_2,\alpha_3$也是$A\vec{x}=\vec{0}$的一个基础解系.

d. 与$\eta_1,\eta_2,\eta_3$等秩的向量组$\beta_1,\beta_2,\beta_3$也是$A\vec{x}=\vec{0}$的一个基础解系.
\end{ex}

\begin{ex}\label{5.14}
(单选题)已知$$Q=\begin{bmatrix}1&2&3\\2&4&t\\3&6&9\end{bmatrix},$$ $P$为非零的三阶矩阵,且$PQ=0$, 则(\ \ \ )

(A) $t=6$时, $r(P)=1$.

(B) $t=6$时, $r(P)=2$.

(C) $t\not=6$时, $r(P)=1$.

(D) $t\not=6$时, $r(P)=1$
\end{ex}

\begin{ex}\label{5.15}
设$A$是一个秩为$n-1$的$n$阶矩阵,$A$的每行元素和为$0$,求齐次线性方程组$A\vec{x}=\vec{0}$的通解.
\end{ex}

\begin{ex}\label{5.16}
已知方程组$(\romannumeral 1)$和方程组$(\romannumeral 2)$ 为
\begin{displaymath}\begin{aligned}&
(\romannumeral 1)\left\{ \begin{aligned}&x_1+x_2=0\\&x_2-x_4=0\end{aligned}\right.\\ &
(\romannumeral 2)\left\{ \begin{aligned}&x_1-x_2+x_3=0\\&x_2-x_3+x_4=0\end{aligned}\right.\end{aligned}
\end{displaymath}
求$(\romannumeral 1)$和$(\romannumeral 2)$的公共解.
\end{ex}

\begin{ex}\label{5.17}
已知线性方程组
\begin{displaymath}\left\{\begin{aligned}&x_1+ x_2+ x_3=0\\&ax_1+bx_2+cx_3=0\\&a^2 x_1+b^2 x_2+c^2 x_3=0\end{aligned}\right.\end{displaymath}
(1) $a,b,c$满足什么条件时,方程组只有零解?

(2)  $a,b,c$满足什么条件时,方程组有无数多解,此时求出基础解系.
\end{ex}

\begin{ex}\label{5.18}
当$\lambda$为何值时,非齐次线性方程组
\begin{displaymath}\left\{\begin{aligned}&
\lambda x_1+ x_2+ x_3=1\\& x_1+\lambda x_2+ x_3=\lambda \\& x_1+ x_2+\lambda x_3=\lambda ^2
\end{aligned}\right.\end{displaymath}
(1) 有唯一解

(2) 无解

(3) 有无穷多解
\end{ex}

\begin{ex}\label{5.19}
讨论$a$为何值时,线性方程组

\begin{displaymath}\left\{\begin{aligned}&
x_1+x_2-x_3=1\\&2x_1+(a+2) x_2-3x_3=3\\&-3ax_2+(a+2) x_3=-3
\end{aligned}\right.\end{displaymath}

无解,有唯一解,有无穷多解;并在有解时求解.
\end{ex}

\begin{ex}\label{5.20}
讨论$a,b$为何值时,线性方程组

\begin{displaymath}\left\{\begin{aligned}&
x_1+x_2+x_3+x_4=1\\&x_1+x_2+ax_3+x_4=1\\&x_1+ax_2+x_3+x_4=1\\&ax_1+x_2+x_3+x_4=b
\end{aligned}\right.\end{displaymath}

无解,有唯一解,有无穷多解;并在有解时求解.
\end{ex}

\begin{ex}\label{5.21}
当$c_1,c_2,c_3,c_4,c_5$满足什么条件时, 线性方程组

\begin{displaymath}\left\{\begin{aligned}&
x_1-x_2=c_1\\&x_2-x_3=c_2\\&x_3-x_4=c_3\\&x_4-x_5=c_4\\&x_5-x_1=c_5
\end{aligned}\right.\end{displaymath}

有解?此时并求出解.
\end{ex}

\begin{ex}\label{5.22}
证明: 设$\vec{\xi}_1,\vec{\xi}_2.\vec{\xi}_3$是齐次线性方程组$A\vec{x}=\vec{0}$的一个基础解系, 则$\vec{\xi}_1+\vec{\xi}_2,\vec{\xi}_2+\vec{\xi}_3,\vec{\xi}_3+\vec{\xi}_1$ 也是它的一个基础解系.
\end{ex}

\begin{ex}\label{5.23}
证明:设向量$\vec{\eta}_1,\vec{\eta}_2,\dots ,\vec{\eta}_k$是齐次线性方程组$A\vec{x}=\vec{0}$ 的一个基础解系. 则 $\vec{\xi}_1=\vec{\eta}_1,\vec{\xi}_2=\vec{\eta}_2-\vec{\eta}_1,\dots,\vec{\xi}_k=\vec{\eta}_k-\vec{\eta}_{k-1}$ 也是$A\vec{x}=\vec{0}$ 的一个基础解系.
\end{ex}

\begin{ex}\label{5.24}
设$\vec{\eta}_1,\vec{\eta}_2,\dots,\vec{\eta}_s$是非齐次线性方程组$A\vec{x}=\vec{\beta}$ 的$s$个解,$k_1,k_2,\dots,k_s$是一组常数,则$k_1 \vec{\eta}_1+k_2 \vec{\eta}_2+\dots+k_s \vec{\eta}_s$也是解的充分必要条件是$k_1+k_2+\dots +k_s=1$.
\end{ex}

\begin{ex}\label{5.25}
证明:若$\vec{x}_0$ 是非齐次线性方程组 $A\vec{x}=\vec{b}$ 的一个特解,$\vec{x}_1,\vec{x}_2,…,\vec{x}_n$是$A\vec{x}=\vec{b}$的基础解系,则$\vec{x}_0,\vec{x}_0+\vec{x}_1,\vec{x}_0+\vec{x}_2,…,\vec{x}_0+\vec{x}_n$ 线性无关,且的任何一个解可表示为
$$\vec{x}=k_0 \vec{x}_0+k_1 (\vec{x}_0+\vec{x}_1 )+k_2 (\vec{x}_0+\vec{x}_2 )+\dots+k_n (\vec{x}_0+\vec{x}_n ).$$
其中$$k_0+k_1+\dots+k_n=1.$$
\end{ex}

\begin{ex}\label{5.26}
证明: 设$A$为$m\times n$矩阵, $B$为$n\times s$矩阵,且$AB=0$,则$B$的各列都是齐次线性方程组$A\vec{x}=\vec{0}$的解.
\end{ex}

\begin{ex}\label{5.27}
证明: 设$A$为$m\times n$矩阵, $r(A)=r<n$, 则齐次线性方程组$A\vec{x}=\vec{0}$的任意$n-r$个线性无关的解都是它的一个基础解系.
\end{ex}

\begin{ex}\label{5.28}
证明:设$A$是$n$阶矩阵,$A^{*}$是$A$的伴随矩阵,则
\begin{displaymath}
r(A^{*} )=\left\{\begin{aligned}&n\ \ \ \ \ &r(A)=n\\ &1 &r(A)=n-1\\ &0 &r(A)<n-1\end{aligned}\right.\end{displaymath}
\end{ex}

\section{习题答案}
\textbf{习题 \ref{5.1} 解答:}\\
对系数矩阵施行行初等变换,将其化为行阶梯阵.
\begin{align*}
A= \begin{bmatrix}1&1&-1&-1&1\\2&1&1&1&4\\4&3&-1&-1&6\\1&2&-4&-4&-1\end{bmatrix} &
   \xrightarrow{\begin{matrix}r_2-2r_1\\r_3-4r_1\\r_4-r_1\end{matrix}}
   \begin{bmatrix}1&1&-1&-1&1\\0&-1&3&3&2\\0&-1&3&3&2\\0&1&-3&-3&-2\end{bmatrix} \\
   \xrightarrow{\begin{matrix}r_3-r_2\\r_4+r_2 \end{matrix}}
   \begin{bmatrix}1&1&-1&-1&1\\0&-1&3&3&2\\0&0&0&0&0\\0&0&0&0&0\end{bmatrix} &
   \xrightarrow{\begin{matrix}r_1+r_2\\r_2\times(-1)\end{matrix}}
   \begin{bmatrix}1&0&2&3&3\\0&1&-3&-3&-2\\0&0&0&0&0\\0&0&0&0&0\end{bmatrix}
\end{align*}
可知,系数矩阵$A$的秩为2,解空间维数为3,故基础解系含有3个解向量,有
\begin{equation*}
\vec{x}=\begin{bmatrix}x_1\\x_2\\x_3\\x_4\\x_5\end{bmatrix}=
\vec{\eta}_1\begin{bmatrix}-2\\3\\1\\0\\0\end{bmatrix}+\vec{\eta}_2\begin{bmatrix}-2\\3\\0\\1\\0\end{bmatrix}
+\vec{\eta}_3\begin{bmatrix}-3\\2\\0\\0\\1\end{bmatrix}
\end{equation*}
\textbf{习题 \ref{5.2} 解答:}\\
对系数矩阵施行行初等变换,将其化为行阶梯阵.
\begin{align*}
A= \begin{bmatrix}1&1&-2\\-1&\lambda&5\\1&3&0\\1&6&\lambda+1\end{bmatrix} &
   \xrightarrow{\begin{matrix}r_2+r_1\\r_3-r_1\\r_4-r_1\end{matrix}}
   \begin{bmatrix}1&1&-2\\0&\lambda+1&3\\0&2&2\\0&5&\lambda+3\end{bmatrix} \\
   \xrightarrow[\begin{matrix}r_3-(\lambda+1)r_2\\r_4-5r_2\end{matrix}]
   {\begin{matrix}\frac{1}{2}r_3\\r_2\leftrightarrow r_3\end{matrix}}
   %\\r_2\leftrightarrow r_3\end{array}}{\rightarrow}
 %  \stackrel{\begin{array} \frac{1}{2}r_3\\r_3-(\lambda+1)r_2\\r_4-5r_2\\r_2\leftrightarrow r_3\end{array}}{\rightarrow}
   \begin{bmatrix}1&1&-2\\0&1&1\\0&0&2-\lambda\\0&0&\lambda-2\end{bmatrix} &
   \xrightarrow{\begin{matrix}r_4+r_3\\r_1-r_2\end{matrix}}
   \begin{bmatrix}1&0&-3\\0&1&1\\0&0&2-\lambda\\0&0&0\end{bmatrix}
\end{align*}
由此,知$\lambda=2$时,系数矩阵$A$的秩为2,方程组存在非零解,基础解系为$\begin{bmatrix}3\\-1\\1\end{bmatrix}$.\\
\textbf{习题 \ref{5.3} 解答:}\\
\begin{equation*}
\begin{cases}
x_1+2x_2+ x_3+x_4+ x_5=0\\
2x_1+4x_2+3x_3+x_4+ x_5=0\\
-x_1-2x_2+ x_3+3x_4-3x_5=0\\
2x_3+5x_4-2x_5=0
\end{cases}
\end{equation*}
解:对系数矩阵$A$进行初等行变换
\begin{align*}
A= \begin{bmatrix}1&2&1&1&1\\2&4&3&1&1\\-1&-2&1&3&-3\\0&0&2&5&-2\end{bmatrix} &
   \rightarrow
   \begin{bmatrix}1&2&1&1&1\\0&0&1&-1&-1\\0&0&2&4&-2\\0&0&2&5&-2\end{bmatrix} \\
   \rightarrow
   \begin{bmatrix}1&2&1&1&1\\0&0&1&-1&-1\\0&0&0&6&0\\0&0&0&7&0\end{bmatrix} &
   \rightarrow
   \begin{bmatrix}1&2&1&1&1\\0&0&1&-1&-1\\0&0&0&1&0\\0&0&0&0&0\end{bmatrix}=U
\end{align*}
得到同解方程组的系数矩阵为$U$.
\begin{equation*}
  r(A)=r(U)=3,n-r(A)=2.
\end{equation*}
故有两个自由未知量,选主元素所在的列的未知量为主变量,即$x_1,x_3,x_4$为独立未知量,则$x_2,x_5$为自由变量,得到同解方程组
\begin{equation*}
\begin{cases}
x_1+x_3+x_4=-2x_2-x_5\\
   x_3-x_4=x_5       \\
   x_4=0
\end{cases}
\end{equation*}
取$(x_2,x_5)=(1,0)$和$(x_2,x_5)=(0,1)$ 得到基础解系
\begin{equation*}
\vec{\xi}_1=\begin{bmatrix}-2\\1\\0\\0\\0\end{bmatrix},
\vec{\xi}_2=\begin{bmatrix}-2\\0\\1\\0\\1\end{bmatrix}
\end{equation*}
于是$A\vec{x}=0$的一般解为
\begin{equation*}
\vec{x}=k_1\vec{\xi}_1+k_2\vec{\xi}_2.
\end{equation*}
即:
\begin{equation*}
\vec{x}=k_1\begin{bmatrix}-2\\1\\0\\0\\0\end{bmatrix}+
k_2\begin{bmatrix}-2\\0\\1\\0\\1\end{bmatrix} (k_1,k_2\text{任意常数})
\end{equation*}
\textbf{习题 \ref{5.4} 解答:}\\
写出增广矩阵
\begin{equation*}
\begin{bmatrix}
2&-1&-1&-1\\1&1&-2&1\\4&-6&2&-6
\end{bmatrix}
\end{equation*}
交换第一第二行,并将第三行乘以系数$\frac{1}{2}$,得
\begin{equation*}
\begin{bmatrix}
1&1&-2&1\\
2&-1&-1&-1\\
2&-3&1&-3
\end{bmatrix}
\end{equation*}
将第二个方程减去第一个方程的2 倍,第三个方程减去第一个方程的2 倍,得
\begin{equation*}
\begin{bmatrix}
1&1&-2&1\\
0&-3&3&-3\\
0&-5&5&-5
\end{bmatrix}
\end{equation*}
对第二,第三行约去公因子,得
\begin{equation*}
\begin{bmatrix}
1&1&-2&1\\
0&1&-1&1\\
0&1&-1&1
\end{bmatrix}
\end{equation*}
将第三行减去第二行,得
\begin{equation*}
\begin{bmatrix}
1&0&-1&0\\
0&1&-1&1\\
0&0&0&0
\end{bmatrix}
\end{equation*}
于是,我们有解
\begin{equation*}
\begin{bmatrix}
x_1\\x_2\\x_3
\end{bmatrix}
=\begin{bmatrix}0\\1\\0\end{bmatrix}+
\begin{bmatrix}1\\1\\1\end{bmatrix}\zeta,
\end{equation*}
即
\begin{equation*}
\begin{cases}
x_1=\zeta\\
x_2=\zeta+1\\
x_3=\zeta
\end{cases},
z\in F
\end{equation*}
\textbf{习题 \ref{5.5} 解答:}\\
对系数矩阵A进行行初等变换
\begin{equation*}
A= \begin{bmatrix}1&1&t\\1&-1&2\\-1&t&1\end{bmatrix}
   \rightarrow
   \begin{bmatrix}1&-1&2\\0&2&t-2\\0&t-1&3\end{bmatrix}
   \rightarrow
   \begin{bmatrix}1&-1&2\\0&2&t-2\\0&0&(t+1)(t-4)\end{bmatrix}
\end{equation*}
故当$t=1$或$t=-4$时,$r(A)=2<3$,方程组$A\vec{x}=0$有非零解.\\
①当$t=-1$时,
\begin{equation*}
A \rightarrow
   \begin{bmatrix}1&-1&2\\0&2&-3\\0&0&0\end{bmatrix}
\end{equation*}
此时$A\vec{x}=0$的同解方程组为
\begin{equation*}
\begin{cases}
x_1-x_2+2x_3=0\\
   2x_2-3x_3=0
\end{cases}
\end{equation*}
基础解系包括一个非零解.取$x_3$为自由变量,设$x=1$,代入得$x_2=\frac{3}{2},x_3=-\frac{1}{2}$.
得方程组的基础解系为$\vec{\xi}_1=(-\frac{1}{2},\frac{3}{2},1)^T$.
此时方程组的一般解为:
\begin{equation*}
\vec{x}=k\vec{\xi}_1=
k\begin{bmatrix}-1\\3\\2\end{bmatrix}(\text{其中}k\text{为常数}).
\end{equation*}
②当$t=4$时,
\begin{equation*}
A \rightarrow
   \begin{bmatrix}1&-1&2\\0&2&2\\0&0&0\end{bmatrix}
   \rightarrow
   \begin{bmatrix}1&-1&2\\0&1&1\\0&0&0\end{bmatrix}
\end{equation*}
之后同理,解出基础解系$\vec{\xi}_2=(-3,-1,1)^T$,
\begin{equation*}
\vec{x}=k\vec{\xi}_1=
k\begin{bmatrix}-3\\-1\\1\end{bmatrix}(\text{其中}k\text{为常数}).
\end{equation*}
\textbf{习题 \ref{5.6} 解答:}\\
若$\vec{v}_1,\vec{v}_2\in V$,即$A\vec{v}_1=0,A\vec{v}_2=0$,
则$A(\vec{v}_1+\vec{v}_2)=0$,也即$\vec{v}_1+\vec{v}_2\in V$. 若$\vec{v}\in V$,$k$ 是一个实数,
则由$A\vec{v}=0$可得$A(k\vec{v})=kA\vec{v}=k\cdot\vec{0}=\vec{0}$,也即$k\vec{v}\in V$.
故而$V$是$\mathbb{R}^n$的一个子空间.\\
\textbf{习题 \ref{5.7} 解答:}\\
(1)
\begin{displaymath}
\begin{aligned}
A=&\begin{bmatrix}2&-4&5&3\\3&-6&4&2\\4&-8&17&11\end{bmatrix}\rightarrow
\begin{bmatrix}1&-2&6&4\\3&-6&4&2\\4&-8&17&11  \end{bmatrix}\rightarrow
\begin{bmatrix}1&-2&6&4\\0&0&-14&-10\\0&0&-7&-5  \end{bmatrix}\\ \rightarrow &
\begin{bmatrix}1&-2&6&4\\0&0&1&5/7\\0&0&0&0  \end{bmatrix}\rightarrow
\begin{bmatrix}1&-2&0&-2/7\\0&0&1&5/7\\0&0&0&0  \end{bmatrix}
\end{aligned}
\end{displaymath}
故基础解系为$\begin{bmatrix}2\\1\\0\\0\end{bmatrix}\ \mbox{和}\ \begin{bmatrix}2\\0\\-5\\7\end{bmatrix} .$

(2)
\begin{displaymath}
\begin{aligned}
A=&\begin{bmatrix} 1&\ \ 2&\ \ 3&\ \ 7\\3&2&1&-3\\0&1&2&6\\5&4&3&-1   \end{bmatrix}\rightarrow
\begin{bmatrix}1&2&3&7\\0&-4&-8&-24\\0&1&2&6\\0&-6&-12&-36    \end{bmatrix}\rightarrow
\begin{bmatrix}1&\ \ 2&\ \ 3&\ \ 7\\0&1&2&6\\0&1&2&6\\0&1&2&6    \end{bmatrix}\\ \rightarrow &
\begin{bmatrix}1&\ \ 2&\ \ 3&\ \ 7\\0&\ 1&\ 2&\ 6\\0&\ 0&\ 0&\ 0\\0&\ 0&\ 0&\ 0    \end{bmatrix}\rightarrow
\begin{bmatrix}1&\ 0&\ -1&\ -5\\0&\ 1&\ 2&\ 6\\0&\ 0& \ 0&\ 0\\0&\ 0& \ 0&\ 0    \end{bmatrix}\end{aligned} \end{displaymath}
故基础解系为$\begin{bmatrix}1\\-2\\1\\0\end{bmatrix}\ \mbox{和}\ \begin{bmatrix}5\\-6\\0\\1\end{bmatrix}$.\\
\textbf{习题 \ref{5.8} 解答:}\\
(1)\begin{displaymath}
\begin{aligned}
A=&\begin{bmatrix}1&\ -2&\ \ 3&\ -4\\0&1&-1&1\\1&3&0&-3\\1&-4&3&-2   \end{bmatrix}\rightarrow
\begin{bmatrix} 1&\ -2&\ \ 3&\ -4\\0&1&-1&1\\0&5&-3&1\\0&-2&0&2  \end{bmatrix}\rightarrow
\begin{bmatrix}1&\ \ 0&\ \ 1&\ -2\\0&\  1&-1&1\\0&0&2&-4\\0&0&-1&2   \end{bmatrix}\\ \rightarrow &
\begin{bmatrix}1&\ \ 0&\ \ 1&\ -2\\0&1&-1&1\\0&0&1&-2\\0&0&0&0   \end{bmatrix}\rightarrow
\begin{bmatrix}1&\ \ 0&\ \ 0&\ \ 0\\0&1&0&-1\\0&0&1&-2\\0&0&0&0   \end{bmatrix} \end{aligned} \end{displaymath}
故基础解系为$\begin{bmatrix}0\\1\\2\\1\end{bmatrix}$.

(2)
\begin{displaymath}
\begin{aligned}
A=&\begin{bmatrix}2&\ \ 3&\ -1&\ \ 5\\3&1&2&-7\\4&1&-3&6\\1&-2&4&-7   \end{bmatrix}\rightarrow
\begin{bmatrix}1&\ -2&\ \ 4&\ -7\\0&7&-10&14\\0&9&-19&34\\0&7&-9&19    \end{bmatrix}\rightarrow
\begin{bmatrix}1&\ -2&\ \ 4&\ -7\\0&7&-10&14\\0&9&-19&34\\0&0&1&5    \end{bmatrix}\end{aligned} \end{displaymath}

至此, 其行列式按第一列, (之后的)第三行展开得
\begin{displaymath}
|A|=\left|\begin{array}{ccc}7&-10&14\\9&-19&34\\0&1&5\end{array}\right|=
-\left|\begin{array}{cc}7&14\\9&34\end{array}\right|+5\left|\begin{array}{cc}7&-10\\9&-19\end{array}\right|=-
\left|\begin{array}{cc}2&-1\\-1&-1\end{array}\right|\not=0(mod\ 5).\end{displaymath}

故$A$是可逆矩阵,基础解系仅有$\vec{0}$向量.

读者朋友,你可以尝试对一开始的系数矩阵$A$在模2下计算行列式,也可以更快得出$A$是可逆的,试一试吧.\\
\textbf{习题 \ref{5.9} 解答:}\\
(1)
\begin{displaymath}
\begin{aligned}
A=&\begin{bmatrix} 1&\ 0&-1&\ 1&\ 2\\1&-1&2&1&1\\2&-1&1&2&3\\3&-1&0&3&5  \end{bmatrix}\rightarrow
\begin{bmatrix}1&\ 0&-1&\ 1&\ 2\\0&-1&3&0&-1\\0&-1&3&0&-1\\0&-1&3&0&-1   \end{bmatrix}\rightarrow
\begin{bmatrix} 1&\ 0&-1&\ 1&\ 2\\0&1&-3&0&1\\0&0&0&0&0\\0&0&0&0&0  \end{bmatrix}\end{aligned} \end{displaymath}
可设$x_1,x_2$为主变量,通解为
\begin{displaymath}
x=\begin{bmatrix}x_1\\x_2\\x_3\\x_4\end{bmatrix}=\begin{bmatrix}2\\1\\0\\0\end{bmatrix}
+k_1\begin{bmatrix}1\\3\\1\\0\end{bmatrix}+k_2\begin{bmatrix}-1\\0\\0\\1\end{bmatrix}
\end{displaymath}
其中$k_1,k_2$为任意常数.

(2)

\begin{displaymath}
\begin{aligned}
A=&\begin{bmatrix}  1&-2&3&-4&4\\0&1&-1&1&-3\\1&3&0&1&1\\0&-7&3&1&-3\\1&-3&2&3&-5 \end{bmatrix}\rightarrow
\begin{bmatrix}1&-2&3&-4&4\\0&1&-1&1&-3\\0&5&-3&5&-3\\0&-7&3&1&-3\\0&-1&-1&7&-9   \end{bmatrix}\rightarrow
\begin{bmatrix}1&\ \ 0&\ \ 1&\ -2&\ -2\\0&1&-1&1&-3\\0&0&1&0&6\\0&0&-1&2&-6\\0&0&-1&4&-6  \end{bmatrix}\\ \rightarrow&
\begin{bmatrix}1&\ \ 0&\ \ 0&\ -2&\ -8\\0&1&0&1&3\\0&0&1&0&6\\0&0&0&2&0\\0&0&0&4&0   \end{bmatrix}\rightarrow
\begin{bmatrix} 1&\ \ 0&\ \ 0&\ \ 0&\ -8\\0&1&0&0&3\\0&0&1&0&6\\0&0&0&1&0\\0&0&0&0&0  \end{bmatrix} \end{aligned} \end{displaymath}
故方程有唯一解$x=\begin{bmatrix}-8\\3\\6\\0\end{bmatrix}$.\\
\textbf{习题 \ref{5.10} 解答:}\\

(1)
\begin{displaymath}
\begin{aligned}
A=&\begin{bmatrix} 1&\ \ 1&\ \ 0&\ -2&\ -6\\4&-1&-1&-1&1\\3&-1&-1&0&0 \end{bmatrix}\rightarrow
\begin{bmatrix} 1&\ \ 1&\ \ 0&\ -2&\ -6\\0&-5&-1&7&25\\0&-4&-1&6&18 \end{bmatrix}\rightarrow
\begin{bmatrix} 1&\ \ 1&\ \ 0&\ -2&\ -6\\0&-5&-1&7&25\\0&1&0&-1&-7 \end{bmatrix}\\ \rightarrow &
\begin{bmatrix}1&\ \ 0&\ \ 0&\ -1&\ \ 1\\0&0&-1&2&-10\\0&1&0&-1&-7  \end{bmatrix} \rightarrow
\begin{bmatrix}1&\ \ 0&\ \ 0&\ -1& \ \ 1\\0&1&0&-1&-7\\0&0&1&-2&10  \end{bmatrix}
\end{aligned} \end{displaymath}

可设$x_1,x_2,x_3$为主变量,通解为
\begin{displaymath}
x=\begin{bmatrix}x_1\\x_2\\x_3\\x_4\end{bmatrix}=\begin{bmatrix}1\\-7\\10\\0\end{bmatrix}
+k_1\begin{bmatrix}1\\1\\2\\0\end{bmatrix}
\end{displaymath}
其中$k_1$ 为任意常数.

(2)
\begin{displaymath}
\begin{aligned}
A=&\begin{bmatrix} 1&\ \ 1&\ -2&\ \ 3&\ \ 0\\2&1&-6&4&-1\\3&2&-8&7&-1\\1&-1&-6&-1&0 \end{bmatrix}\rightarrow
\begin{bmatrix}1&\ \ 1&\ -2&\ \ 3&\ \ 0\\0&-1&-2&-2&-1\\0&-1&-2&-2&-1\\0&-2&-4&-4&0  \end{bmatrix}\\ \rightarrow &
\begin{bmatrix}1&\ \ 1&\ -2&\ \ 3&\ \ 0\\0&1&2&2&1\\0&0&0&0&0\\0&1&2&2&0  \end{bmatrix}\rightarrow
\begin{bmatrix} 1&\ \ 1&\ -2&\ \ 3&\ \ 0\\0&1&2&2&1\\0&0&0&0&-1\\0&0&0&0&0 \end{bmatrix}
\end{aligned} \end{displaymath}

故原方程组无解.\\	
\textbf{习题 \ref{5.11} 解答:}\\
(1)

\begin{displaymath}
\begin{aligned}
A=&\begin{bmatrix} 6&\ -2&\ \ 2&\ \ 1&\ \ 3\\1&-1&0&1&1\\2&0&1&3&2 \end{bmatrix}\rightarrow
\begin{bmatrix}1&\ -1&\ \ 0&\ \ 1&\ \ 1\\0&4&2&-5&-3\\0&2&1&1&0  \end{bmatrix}\rightarrow
\begin{bmatrix} 1&\ -1&\ \ 0&\ \ 1&\ \ 1\\0&0&0&-7&-3\\0&2&1&1&0 \end{bmatrix}\\ \rightarrow&
\begin{bmatrix}1&\ -1&\ \ 0&\ \ 1&\ \ 1\\0&2&1&1&0\\0&0&0&1&3/7  \end{bmatrix} \rightarrow
\begin{bmatrix}1&\ -1&\ \ 0&\ \ 0&\ \ 4/7\\0&2&1&0&3/7\\0&0&0&1&3/7  \end{bmatrix}
\end{aligned} \end{displaymath}
设$x_1,x_3,x_4$为主变量,通解为
\begin{displaymath}
x=\begin{bmatrix}x_1\\x_2\\x_3\\x_4\end{bmatrix}=\frac{1}{7}\begin{bmatrix}4\\0\\3\\3\end{bmatrix}
+k_1\begin{bmatrix}1\\1\\-2\\0\end{bmatrix}
\end{displaymath}
其中$k_1$ 为任意常数.

(2)
\begin{displaymath}
\begin{aligned}
A=&\begin{bmatrix}1&\ -1&\ \ 2&\ \ 1&\ \ 1\\2&-1&1&2&3\\1&0&-1&1&2\\3&-1&0&3&5  \end{bmatrix}\rightarrow
\begin{bmatrix}1&\ -1&\ \ 2&\ \ 1&\ \ 1\\0&1&-3&0&1\\0&1&-3&0&1\\0&1&-3&0&1  \end{bmatrix}\rightarrow
\begin{bmatrix}1&\ \ 0&\ -1&\ \ 1&\ \ 2\\0&1&-3&0&1\\0&0&0&0&0\\0&0&0&0&0  \end{bmatrix}
\end{aligned} \end{displaymath}

设$x_1,x_2$为主变量,通解为
\begin{displaymath}
x=\begin{bmatrix}x_1\\x_2\\x_3\\x_4\end{bmatrix}=\begin{bmatrix}2\\1\\0\\0\end{bmatrix}
+k_1\begin{bmatrix}1\\3\\1\\0\end{bmatrix}+k_2\begin{bmatrix}-1\\0\\0\\1\end{bmatrix}
\end{displaymath}
其中$k_1,k_2$为任意常数.\\
\textbf{习题 \ref{5.12} 解答:}\\
(1)

\begin{displaymath}
\begin{aligned}
A=&\begin{bmatrix} 1&\ -3&\ \ 5&\ \ 0\\2&-1&-3&11\\2&1&-3&5 \end{bmatrix}\rightarrow
\begin{bmatrix}1&\ -3&\ \ 5&\ \ 0\\0&5&-13&11\\0&7&-13&5  \end{bmatrix}\\ \rightarrow&
\begin{bmatrix}1&\ -3&\ \ 5&\ \ 0\\0&5&-13&11\\0&1&0&-3  \end{bmatrix} \rightarrow
\begin{bmatrix} 1&\ \ 0&\ \ 0&\ \ 1\\0&1&0&-3\\0&0&1&-2 \end{bmatrix}
\end{aligned} \end{displaymath}
故原方程组只有唯一解$x=\begin{bmatrix}1\\-3\\-2\end{bmatrix}$.
(2)

\begin{displaymath}
\begin{aligned}
A=&\begin{bmatrix}3&\ \ 2&\ \ 1&\ \ 1&\ \ 1&\ \ 7\\3&2&1&1&-3&-1\\5&4&3&3&-1&9  \end{bmatrix}\rightarrow
\begin{bmatrix}3&\ \ 2&\ \ 1&\ \ 1&\ \ 1& \ \ 7\\0&0&0&0&1&2\\1&1&1&1&-1&1  \end{bmatrix}\\ \rightarrow &
\begin{bmatrix}0&\ \ 1&\ \ 2&\ \ 2& \ \ 0&\ \ 4\\0&0&0&0&1&2\\1&1&1&1&0&3  \end{bmatrix} \rightarrow
\begin{bmatrix}1&\ \ 0&\ -1&\ -1&\ \ 0&\ -1\\0&1&2&2&0&4\\0&0&0&0&1&2  \end{bmatrix}
\end{aligned} \end{displaymath}
设$x_1,x_2,x_5$为主变量,通解为
\begin{displaymath}
x=\begin{bmatrix}x_1\\x_2\\x_3\\x_4\\x_5\end{bmatrix}=\begin{bmatrix}-1\\4\\0\\0\\2\end{bmatrix}
+k_1\begin{bmatrix}1\\-2\\1\\0\\0\end{bmatrix}+k_2\begin{bmatrix}1\\-2\\0\\1\\0\end{bmatrix}
\end{displaymath}
其中$k_1,k_2$为任意常数.\\
\textbf{习题 \ref{5.13} 解答:}\\
a,c正确; b,d错误.\\
\textbf{习题 \ref{5.14} 解答:}\\
观察Q, 其第1列与第2列成比例, 故当$t=6$时, $r(Q)=1$;  当$t≠6$时, $r(Q)=2$.

由于$PQ=0$, 则有$r(P)+r(Q)≤3-r(PQ)=3$, 故仅当$r(Q)=2$时, 可唯一决定$r(P)=1$($P$ 是非零的, 故$r(P)≥1$).  选项C正确.\\
\textbf{习题 \ref{5.15} 解答:}\\
解空间维数为$dimN(A)=n-r(A)=1$, 而的行和为$0$,说明$(1,1,\dots,1)^T$是$A$ 的一组非零特解. 故通解为$(k,k,\dots,k)^T$(其中$k$为任意常数).\\
\textbf{习题 \ref{5.16} 解答:}\\
求$(\romannumeral 1)$和$(\romannumeral 2)$的公共解,就是求同时满足它们的4个方程的解,即
\begin{displaymath}\left\{\begin{aligned}&x_1+x_2         =0\\&   x_2     -x_4=0\\& x_1-x_2+x_3     =0\\&    x_2-x_3+x_4=0\end{aligned}\right.\end{displaymath}
对系数矩阵进行行初等变换,

\begin{displaymath}
\begin{aligned}
A=&\begin{bmatrix}1&\ \ 1&\ \ 0&\ \ 0\\0&1&0&-1\\1&-1&1&0\\0&1&-1&1 \end{bmatrix}\rightarrow
\begin{bmatrix}1&\ \ 1&\ \ 0&\ \ 0\\0&1&0&-1\\0&-2&1&0\\0&1&-1&1 \end{bmatrix}\rightarrow
\begin{bmatrix} 1&\ \ 0&\ \ 0&\ \ 1\\0&1&0&-1\\0&0&1&-2\\0&0&0&0\end{bmatrix}
\end{aligned} \end{displaymath}

$r=3,\ n=4,\ n-r=1$, 故基础解系有一个解向量, 为$\xi=\begin{bmatrix}-1\\1\\2\\1\end{bmatrix}$. 故$(\romannumeral 1)$和$(\romannumeral 2)$的公共解为$x=k\xi=\begin{bmatrix}-1\\1\\2\\1\end{bmatrix}$.\\
\textbf{习题 \ref{5.17} 解答:}\\

 (1) 显然,系数矩阵组成的行列式是Vandermond行列式, 因此$|A|=(a-b)(b-c)(c-a)$, 当且仅当$a,b,c$两两不同, $|A|≠0$, 此时方程组只有零解.

(2)若$a=b\not=c$,
\begin{displaymath}
\begin{aligned}
A=&\begin{bmatrix}1&\ \ \ 1&\ \ \ 1\\a&a&c\\a^2&a^2&c^2 \end{bmatrix}\rightarrow
\begin{bmatrix}1&\ \ \ 1&\ \ \ 1\\0&0&c-a\\0&0&c^2-a^2 \end{bmatrix}\rightarrow
\begin{bmatrix}1&\ \ \ 1&\ \ \ 0\\0&0&1\\0&0&0 \end{bmatrix}
\end{aligned} \end{displaymath}
此时,基础解系为$\xi_c=\begin{bmatrix}-1\\1\\0\end{bmatrix}$.

根据轮换性,当$b=c\not=a$ 时,基础解系为$\xi_a=\begin{bmatrix}0\\-1\\1\end{bmatrix}$. 当$c=a\not=b$时,基础解系为$\xi_b=\begin{bmatrix}1\\0\\-1\end{bmatrix}$.

当$a=b=c$时,
\begin{displaymath}
\begin{aligned}
A=&\begin{bmatrix}1&\ \ \ 1&\ \ \ 1\\a&a&a\\a^2&a^2&a^2 \end{bmatrix}\rightarrow
\begin{bmatrix} 1&\ \ \ 1&\ \ \ 1\\0&0&0\\0&0&0 \end{bmatrix}
\end{aligned} \end{displaymath}
以$x_1$为主变量,则基础解系为$$\xi_1=\begin{bmatrix}-1\\1\\0\end{bmatrix}\ \ \ \ \ \xi_2=\begin{bmatrix}-1\\0\\1\end{bmatrix}.$$
\textbf{习题 \ref{5.18} 解答:}\\
 $\lambda\not=1$且$\lambda\not=-2$时,方程组有唯一解$$x=\frac{1}{\lambda+2} \begin{bmatrix} -\lambda-1\\1\\ \lambda ^2+2\lambda+1\end{bmatrix}$$

$\lambda=-2$时,方程组无解.


$\lambda=1$时,方程组有无穷多解,解为$$x=\begin{bmatrix}1-a-b\\ab\end{bmatrix}$$   ($a,b$ 为任意常数).\\
\textbf{习题 \ref{5.19} 解答:}\\
\begin{displaymath}
\begin{aligned}
A=&\begin{bmatrix} 1&1&-1&1\\2&a+2&-3&3\\0&-3a&a+2&-3  \end{bmatrix}\rightarrow
\begin{bmatrix}1&1&-1&1\\0&a&-1&1\\0&-3a&a+2&-3   \end{bmatrix}\rightarrow
\begin{bmatrix} 1&1&-1&1\\0&a&-1&1\\0&0&a-1&0  \end{bmatrix}\end{aligned} \end{displaymath}
所以系数阵行列式是$a(a-1)$ .

当$a\not=0$或$a\not=1$时,方程组有唯一解$x=\begin{bmatrix}1-\frac{1}{a}\\ \frac{1}{a}\\0\end{bmatrix}.$

$a=0$时,
 \begin{displaymath}
\begin{aligned}
A=&\begin{bmatrix} 1&1&-1&1\\0&0&-1&1\\0&0&-1&0  \end{bmatrix}\rightarrow
\begin{bmatrix} 1&1&-1&1\\0&0&-1&1\\0&0&0&-1  \end{bmatrix}\end{aligned} \end{displaymath}
线性方程组无解.

$a=1$时,$$A=\begin{bmatrix}1&0&0&0\\0&1&-1&1\\0&0&0&0\end{bmatrix}.$$
线性方程组无数解, 解为
\begin{displaymath}
x=\begin{bmatrix}x_1\\x_2\\x_3\end{bmatrix}=\begin{bmatrix}0\\1\\0\end{bmatrix}+k_1
\begin{bmatrix}0\\1\\1\end{bmatrix}\end{displaymath}
其中$k_1$为任意常数.\\
\textbf{习题 \ref{5.20} 解答:}\\
 \begin{displaymath}
\begin{aligned}
A=&\begin{bmatrix} 1&1&1&1&1\\1&1&a&1&1\\1&a&1&1&1\\a&1&1&1&b \end{bmatrix}\rightarrow
\begin{bmatrix} 1&1&1&1&1\\0&0&a-1&0&0\\0&a-1&0&0&0\\a-1&0&0&0&b-1 \end{bmatrix}\end{aligned} \end{displaymath}
所以系数阵行列式是$(a-1)^3$, 当$a\not=1$时, 方程组有唯一解. 此时,

\begin{displaymath}
\begin{aligned}
A=&\begin{bmatrix} 1&\ \ 1&\ \ 1&\ \ 1&\ \ 1\\0&0&1&0&0\\0&1&0&0&0\\1&0&0&0&\frac{b-1}{a-1} \end{bmatrix}\rightarrow
\begin{bmatrix}0&\ \ 0&\ \ 0& \ \ 1&\ \ \frac{a-b}{a-1}\\0&0&1&0&0\\0&1&0&0&0\\1&0&0&0&\frac{b-1}{a-1}  \end{bmatrix} \end{aligned} \end{displaymath}

所以解为$x=\begin{bmatrix}\frac{b-1}{a-1}\\0\\0\\ \frac{a-b}{a-1}\end{bmatrix}.$

当$a\not=1$时, $$A\rightarrow \begin{bmatrix}1&1&1&1&1\\0&0&0&0&b-1\\0&0&0&0&0\\0&0&0&0&0\end{bmatrix}$$
知当$b\not=1$时, 方程组无解.

当$b=1$时,方程组等价于$x_1+x_2+x_3+x_4=1$, 解为

$$x=\begin{bmatrix}a\\b\\c\\1-a-b-c\end{bmatrix}.$$
$a,b,c$为任意常数.\\
\textbf{习题 \ref{5.21} 解答:}\\
显然有解的必要条件是$c_1+c_2+c_3+c_4+c_5=0$. 这也是充分条件.
\begin{displaymath}
\begin{aligned}
A=&\begin{bmatrix}  1&-1& & & &c_1\\ &1&-1& & &c_2\\ & &1&-1& &c_3\\ & & &1&-1&c_4\\-1& & & &1&c_5\end{bmatrix}\rightarrow
\begin{bmatrix} 1&-1& & & &c_1\\ &1&-1& & &c_2\\ & &1&-1& &c_3\\ & & &1&-1&c_4\\0&0 & & &0&0    \end{bmatrix} \end{aligned} \end{displaymath}
所以系数矩阵的秩为1, 基础解系有$5-4=1$个, 容易构造一个特解为
$x_{*}=\begin{bmatrix}0\\c_2+c_3+c_4+c_5\\c_3+c_4+c_5\\c_4+c_5\\c_5\end{bmatrix}$
基础解系为$x_1=\begin{bmatrix}1\\1\\1\\1\\1\end{bmatrix}$.故解为
$$x=\begin{bmatrix} k\\k+c_2+c_3+c_4+c_5\\k+c_3+c_4+c_5\\k+c_4+c_5\\k+c_5\end{bmatrix}$$
$k$为任意常数.\\
\textbf{习题 \ref{5.22} 解答:}\\
只要证明原基可以被新基线性表示.
\begin{displaymath}
\begin{aligned}
&\xi_1=\frac{1}{2} [(\xi_3+\xi_1 )+(\xi _1+\xi_2 )-(\xi_2+\xi_3 )]\\
&\xi_2=\frac{1}{2}[(\xi_2+\xi_3 )+(\xi_3+\xi_1 )-(\xi_1+\xi_2 )]\\
&\xi_3=\frac{1}{2} [(\xi_1+\xi_2 )+(\xi_2+\xi_3 )-(\xi_3+\xi_1 )]\end{aligned}\end{displaymath}
命题得证.\\
\textbf{习题 \ref{5.23} 解答:}\\
等价于证明两组向量可以互相线性表出,事实上只需要证明$\{\eta_i \}$可以被$\{\xi_i \}$线性表出.
$$\vec{\eta}_i=\vec{\xi}_1+\dots+\vec{\xi}_i,\ \ \ \   \forall i=1,\dots,n$$
命题得证.\\
\textbf{习题 \ref{5.24} 解答:}\\
若$k_1 \vec{\eta}_1+k_2 \vec{\eta}_2+\dots+k_s  \vec{\eta}_s$是方程组$A\vec{x}=\vec{\beta}$的解, 则
\begin{displaymath}\begin{aligned}
\vec{\beta}&=A(k_1 \vec{\eta}_1+k_2 \vec{\eta}_2+\dots+k_s \vec{\eta}_s )\\&=k_1 A\vec{\eta}_1+k_2 A\vec{\eta}_2+\dots+k_s A\vec{\eta}_s\\&=(k_1+k_2+\dots+k_s )\vec{\beta}
\end{aligned}
\end{displaymath}
又$\vec{\beta}\not=0$, 所以$$k_1+k_2+\dots+k_s=1.$$

反之,若$k_1+k_2+\dots+k_s=1.$ 则
\begin{displaymath}\begin{aligned}
A(k_1 \vec{\eta}_1+k_2 \vec{\eta}_2+\dots+k_s \vec{\eta}_s )&=k_1 A\vec{\eta}_1+k_2 A\vec{\eta}_2+\dots+k_s A\vec{\eta}_s\\&=(k_1+k_2+\dots+k_s )\vec{\beta}\\&=\vec{\beta}
\end{aligned}
\end{displaymath}
所以$k_1 \vec{\eta}_1+k_2 \vec{\eta}_2+\dots+k_s \vec{\eta}_s$是方程组$Ax=\vec{\beta}$ 的解.\\
\textbf{习题 \ref{5.25} 解答:}\\
$$c_0 x_0+c_1 (x_0+x_1 )+c_2 (x_0+x_2 )+\dots+c_n (x_0+x_n )=0$$
整理
$$(c_0+\dots+c_n) x_0+c_1 x_1+\dots+c_n x_n=0$$
我们断言必有$$c_0+\dots+c_n=0$$
否则就有形式$$x_0=d_1 x_1+\dots+d_n x_n$$
蕴含$\vec{x}_0$也是$A\vec{x}=\vec{0}$的解, 矛盾. 接着由基础解系的线性无关性知$$c_1=c_2=\dots=c_n=0.$$
故$x_0,x_0+x_1,x_0+x_2,\dots,x_0+x_n$线性无关.

设$x$满足$Ax=b$, 则$$A(x-x_0)=b-b=0$$ 故$x-x_0$可以由$Ax=0$的基础解系表出, 即$$x-x_0=c_1 x_1+\dots+c_n x_n$$
则$$x=x_0+c_1 x_1+\dots+c_n x_n$$
$$x=(1-c_1-\dots-c_n ) x_0+c_1 (x_0+x_1 )+\dots+c_n (x_0+x_n )$$
令$$k_0=1-c_1-\dots-c_n,\ \ k_i=c_i\ \  (i≥1)$$
于是命题得证.\\
\textbf{习题 \ref{5.26} 解答:}\\
设$B=(\vec{\beta}_1,\vec{\beta}_2,\dots,\vec{\beta}_s)$, 由分块矩阵知识, $$AB=(A\vec{\beta}_1,A\vec{\beta}_2,\dots,A\vec{\beta}_s )$$
由$AB=0$推出 $A\vec{\beta}_i=\vec{0}$.\\
\textbf{习题 \ref{5.27} 解答:}\\
由定义, 基础解系是解空间的一组基, 而解空间的维数是
\begin{displaymath}\dim ?N(A)=n-r(A)=n-r\end{displaymath}
故$A\vec{x}=\vec{0}$ 的任意$n-r$ 个线性无关的解构成了解空间的基.\\
\textbf{习题 \ref{5.28} 解答:}\\
 若$r(A)=n$, 则$A$是可逆矩阵, 由于$AA^{*}=|A|I$, 知$A^{*}$也是可逆矩阵, 故$r(A^{*})=n$;

 若$r(A)<n-1$, 则$A$的任意一个$n-1$阶子阵都为$0$, 由伴随矩阵定义, $A^{*}=0$, 故$r(A^{*})=0$;

 若$r(A)=n-1$, 则$|A|=0$, 且存在$A$ 的一个$n-1$ 阶子阵不为0, 则$A^{*}\not=0$. 由于$AA^{*}=|A|I=0$, 由20题知, $A^{*}$的各列都是齐次线性方程组$A\vec{x}=\vec{0}$的解. 又因为
\begin{displaymath}
\dim N(A)=n-r(A)=1.\end{displaymath}
 $A^{*}$存在一个非零列, 故$A^{*}$ 的秩等于$A^{*}$ 的列秩等于1.

%%%%%%%%%%%%%%%%%%%%%%%%%%%%%%%%%%%%%%%%%%%%%%%%%%%%%%%%%%%%%%%%%%%%%%%%%%%%%%%
%%%%%%%%%%%%%%%%%%%%%%%%%%%%%%%%%%%%%%%%%%%%%%%%%%%%%%%%%%%%%%%%%%%%%%%%%%%%%%%

\chapter{内积空间}

\section{知识点解析}
\begin{Def}
设$\vec{\alpha},\vec{\beta}\in\mathbb{R}^n$,若$\vec{\alpha}=(x_1,x_2,\cdots,x_n)^{T}$,$\vec{\beta}=(y_1,y_2,\cdots,y_n)^{T}$,则定义:
\begin{equation*}
\vec{\alpha}\cdot\vec{\beta}=x_1y_1+x_2y_2+\cdots+x_ny_n=\sum_{i=1}^{n}x_iy_i
\end{equation*}
称为向量$\vec{\alpha}$与$\vec{\beta}$的点乘,或标准内积。
\end{Def}

\begin{Def}\label{neiji}
设$\mathbb{R}^{n}$中任意两个向量$\vec{\alpha},\vec{\beta}$,均存在唯一对应的
一个实数$(\vec{\alpha},\vec{\beta})$,且满足如下的性质:
\begin{enumerate}
  \item $\vec{\alpha}\cdot\vec{\alpha}=|\vec{\alpha}|^2\geq0$且等号成立$\Leftrightarrow\vec{\alpha}=\vec{0}$;
  \item $\vec{\alpha}\cdot\vec{\beta}=\vec{\beta}\cdot\vec{\alpha}$;
  \item $(k\vec{\alpha})\cdot\vec{\beta}=\vec{\alpha}\cdot(k\vec{\beta})
        =k(\vec{\alpha}\cdot\vec{\beta})$;
  \item $(\vec{\alpha}+\vec{\beta})\cdot\vec{\gamma}=\vec{\alpha}
        \cdot\vec{\gamma}+\vec{\beta}\cdot\vec{\gamma}$.
\end{enumerate}
则称$(\vec{\alpha},\vec{\beta})$为向量$\vec{\alpha},\vec{\beta}$的内积。定义了内积的$n$维向量空间$\mathbf{R}^n$称为欧几里德空间,简称欧式空间。
\end{Def}

\begin{Def}
对$\mathbf{R}^n$中任意向量$\vec{\alpha}$,其长度(模长)$|\vec{\alpha}|$定义为:
\begin{equation*}
|\vec{\alpha}|=\sqrt{(\vec{\alpha},\vec{\alpha})}
\end{equation*}
\end{Def}

\begin{thm}
[cauchy-Schwarz不等式]$\mathbb{R}^n$中的内积满足:
\begin{equation*}
(\vec{\alpha},\vec{\beta})^2\leq(\vec{\alpha},\vec{\alpha})
(\vec{\beta},\vec{\beta})=|\vec{\alpha}|^2|\vec{\beta}|^2~~
(\forall\vec{\alpha},\vec{\beta}\in\mathbb{R}^n)
\end{equation*}
其中,等号成立当且仅当$\vec{\alpha}$与$\vec{\beta}$线性相关。
\end{thm}

\begin{Def}
对$\mathbb{R}^n$中任意两个向量$\vec{\alpha},\vec{\beta}$,它们的
夹角$\theta=\langle\vec{\alpha},\vec{\beta}\rangle$定义为:
\begin{equation*}
\cos\theta=\frac{(\vec{\alpha},\vec{\beta})}{|\vec\alpha||\vec\beta|}~~
(0\leq\theta\leq\pi)
\end{equation*}
\end{Def}

\begin{thm}
任意正交的向量组$\vec{\alpha}_1,\vec{\alpha}_2,\cdots,\vec{\alpha}_s$线性无关。
\end{thm}

\begin{cor}
在$n$维欧式空间$\mathbb{R}^n$中,任意正交向量组的向量个数不会超过$n$。
\end{cor}

\begin{Def}
在$n$维欧式空间$\mathbb{R}^n$中,由$n$个两两正交的非零向量构成的向量组称为
正交基,由单位向量组成的正交基称为标准正交基,或单位正交基。
\end{Def}

\begin{Def}
设$Q$是$n$阶方阵,满足$Q^TQ=I_n$,则称$Q$是正交矩阵,简称正交阵。
\end{Def}

\begin{thm}
正交矩阵具有下列性质
\begin{enumerate}
  \item $Q$为正交阵$\Leftrightarrow$$Q$的列(行)向量组构成$\mathbb{R}^n$的标准正交基;
  \item $Q$为正交阵,则$|Q|=1$或-1;
  \item 正交阵$Q$可逆,且$Q^{-1}=Q^{T}$仍为正交矩阵;
  \item $Q$为正交阵$\Leftrightarrow$$Q$可逆,且且$Q^{-1}=Q^{T}$;
  \item 正交阵的乘积仍是正交矩阵。
\end{enumerate}
\end{thm}

\begin{thm}
$n$维欧式空间$\mathbb{R}^n$中,任意$s\leq n$个线性无关的向量
$\vec{\alpha}_1,\vec{\alpha}_2,\cdots,\vec{\alpha}_n$均可转化为一组正交向量组
$\vec{\beta}_1,\vec{\beta}_2,\cdots,\vec{\beta}_n$,其中
\begin{align*}
\vec{\beta}_1=&\vec{\alpha}_1\\
\vec{\beta}_2=&\vec{\alpha}_2-\frac{(\vec{\alpha}_2,\vec{\beta}_1)}{(\vec{\beta}_1,\vec{\beta}_1)}\vec{\beta}_1\\
\vec{\beta}_3=&\vec{\alpha}_3-\frac{(\vec{\alpha}_3,\vec{\beta}_1)}{(\vec{\beta}_1,\vec{\beta}_1)}\vec{\beta}_1-
              \frac{(\vec{\alpha}_3,\vec{\beta}_2)}{(\vec{\beta}_2,\vec{\beta}_2)}\vec{\beta}_2\\
\ldots&\ldots\ldots \\
\vec{\beta}_n=&\vec{\alpha}_n-\frac{(\vec{\alpha}_n,\vec{\beta}_1)}{(\vec{\beta}_1,\vec{\beta}_1)}\vec{\beta}_1
              -\frac{(\vec{\alpha}_n,\vec{\beta}_2)}{(\vec{\beta}_2,\vec{\beta}_2)}\vec{\beta}_2-\ldots
                -\frac{(\vec{\alpha}_n,\vec{\beta}_{n-1})}{(\vec{\beta}_{n-1},\vec{\beta}_{n-1})}\vec{\beta}_{n-1}
\end{align*}
而且$L(\vec{\alpha}_1,\vec{\alpha}_2,\cdots,\vec{\alpha}_n)=L(\vec{\beta}_1,\vec{\beta}_2,\cdots,\vec{\beta}_n)$
进而通过把$\vec{\beta}_1,\vec{\beta}_2,\cdots,\vec{\beta}_n$单位化后可得到标准正交向量组
$\vec{\gamma}_1,\vec{\gamma}_2,\cdots,\vec{\gamma}_n$。
\end{thm}

\begin{thm}
对任意$n$阶可逆矩阵$A$,存在一个$n$阶正交矩阵$Q$及一个$n$阶主对角元素为正数的上三角阵$R$,使$A=QR$,称为可逆矩阵$A$的QR分解,并且这种分解是唯一的。
\end{thm}

\begin{thm}
设$\{\vec{\eta}_1,\vec{\eta}_2,\cdots,\vec{\eta}_t\}$是欧式空间$\mathbb{R}^n$中子空间$W$的一组正交基,对$W$中任意向量$\vec{\alpha}$,
则$\vec{\alpha}$在该正交基下的第$i$个坐标,即第$i$个线性表出系数为:
\begin{equation*}
x_i=\frac{(\vec{\alpha},\vec{\eta}_i)}{(\vec{\eta}_i,\vec{\eta}_i)},
(1\leq i\leq t)
\end{equation*}
\end{thm}

\begin{Def}
欧式空间$\mathbb{R}^n$中,给定两个向量$\vec{\alpha}$与$\vec{\beta}$,
从$\vec{\alpha}$的终点引垂线与$\vec{\beta}$共线的向量,称为$\vec{\alpha}$
在$\vec{\beta}$方向的正交投影向量,记为$(\vec{\alpha})_{\vec{\beta}}$.
\end{Def}
\begin{thm}
设$\{\vec{\eta}_1,\vec{\eta}_2,\cdots,\vec{\eta}_t\}$是欧式空间$\mathbb{R}^n$中子空间$W$的一组正交基,对$W$中任意向量$\vec{\alpha}$,
则$\vec{\alpha}$在该正交基下的第$i$个坐标$(1\leq i\leq t)$,
就是$\vec{\alpha}$向第$i$个基向量$\vec{\eta}_i$的正交投影向量$(\vec{\alpha})_{\vec{\beta}}$的长度。
\end{thm}

\begin{Def}
设$W,W_1,W_2$是欧式空间$\mathbb{R}^{n}$中的子空间,
向量$\vec{\alpha}\in\mathbb{R}^n$,则
\begin{enumerate}
  \item 若$\forall\vec{\beta}\in W$,均有$\vec{\alpha}\bot\vec{\beta}$,则称
        $\vec{\alpha}$与$W$正交,记为$\vec{\alpha}\bot W$;
  \item 若$\forall\vec{\beta}\in W_1$,$\forall\vec{\gamma}\in W_2$,均有$\vec{\beta}\bot\vec{\gamma}$,则称$W_1$与$W_2$正交,记为$W_1\bot W_2$;
  \item $W^{\bot}:=\{\vec{\beta}\in\mathbb{R}^n|\vec{\beta}\bot W\}$称为$W$在
       $\mathbb{R}^n$中的正交补。
\end{enumerate}
\end{Def}

\begin{thm}
[正交分解]设$W$是欧式空间$\mathbb{R}^n$中的一个子空间,则$\mathbb{R}^n$中每一个向量$\vec{\alpha}$可以唯一的分解为:
\begin{equation*}
\vec{\alpha}=\vec{\beta}+\vec{\gamma},s.t.\vec{\beta}\in W\text{且}\vec{\gamma}\in W^{\bot}
\end{equation*}
其中$\vec{\beta}$称为$\vec{\alpha}$在$W$上的正交投影向量,记为$(\vec{\alpha})_{W}$。更进一步,
若$\{\vec{\eta}_1,\vec{\eta}_2,\cdots,\vec{\eta}_p\}$
是$W$的一组正交基,
那么上述分解中的
\begin{equation*}
\vec{\beta}=\frac{(\vec{\alpha},\vec{\eta}_1)}{(\vec{\eta}_1),\vec{\eta}_1)}\vec{\eta}_1
+\cdots+\frac{(\vec{\alpha},\vec{\eta}_p)}{(\vec{\eta}_p),\vec{\eta}_p)}\vec{\eta}_p
\end{equation*}
\end{thm}

\begin{thm}
设$W$为$\mathbb{R}^n$中的一个子空间,则$W^{\bot}$也为$\mathbb{R}^n$中的子空间,且有
\begin{enumerate}
  \item $dimW+dimW^{\bot}=n$;
  \item $W\cap W^{\bot}={\vec{0}}$.
\end{enumerate}
\end{thm}

\begin{thm}
设$A$是$m\times n$型矩阵,则
\begin{enumerate}
  \item $A$的行空间与解空间互为正交补,即$Row(A)^{\bot}=N(A)\subseteq\mathbb{R}^n$;
  \item $A$的列空间与转置解空间互为正交补,即$Col(A)^{\bot}=N(A^T)\subseteq\mathbb{R}^m$.
\end{enumerate}
\end{thm}

\begin{thm}[最佳逼近定理]
设$W$是欧式空间$\mathbb{R}^n$中的子空间,对$\mathbb{R}^n$中任意向量
$\vec{b}$,设$(\vec{b})_{W}$为$\vec{b}$在$W$上的正交投影向量,则
\begin{equation*}
|\vec{b}-(\vec{b})_{W}|\leq|\vec{b}-\vec{w}|,(\forall\vec{w}\in W,
\text{且}\vec{w}\neq (\vec{b})_{W})
\end{equation*}
\end{thm}

\begin{thm}
对于矛盾的非齐次线性方程组$A\vec{x}=\vec{b}$,有
\begin{enumerate}
\item 可用法方程$A^TA\vec{x}=A^T\vec{b}$的解作为$A\vec{x}=\vec{b}$的最小二乘解;
\item 法方程$A^TA\vec{x}=A^T\vec{b}$必有解,且当$r(A)=n$($A$列满秩)时,$A^TA$
可逆,法方程有唯一解
\begin{equation*}
\vec{x}_0=(A^TA)^{-1}A^T\vec{b}
\end{equation*}
\end{enumerate}
\end{thm}

%%%%%%%%%%%%%%%%%%%%%%%%%%%%%%%%%%%%%%%%%%%%%%%%%%%%%%%%%%%%%%%%%%%%%%%%%%%%%%%%%%

\section{例题讲解}

\begin{eg}
在$\mathbb{R}^n$中,对于向量$\vec{\alpha}=(a_1,a_2,\cdots,a_n)^T$,
$\vec{\beta}=(b_1,b_2,\cdots,b_n)^T$,定义
\begin{equation*}
(\vec{\alpha},\vec{\beta}):=a_1b-1+2a_2b_2+\cdots+na_nb_n
\end{equation*}
不难验证上述定义满足定义\ref{neiji}中的性质$(1)\sim(4)$,故这是一个内积。
\end{eg}

\begin{eg}
在$R^4$中求与$\vec{\alpha}=(1,1,-1,1)^T,\vec{\beta}=(1,-1,-1,1)^T,
\vec{\gamma}=(2,1,1,3)^T$都正交的向量。
\end{eg}
解:设与$\vec{\alpha},\vec{\beta},\vec{\gamma}$都正交的向量
为$\vec{x}=(x_1,x_2,x_3,x_4)^T$,则
\begin{equation*}
\begin{cases}
x_1+x_2-x_3+x_4=0\\
x_1-x_2-x_3+x_4=0\\
2x_1+x_2+x_3+3x_4=0
\end{cases}
\end{equation*}
由Gauss消元法得到
\begin{equation*}
\begin{bmatrix}
1&1&-1&1\\1&-1&-1&1\\2&1&1&3
\end{bmatrix}
\rightarrow
\begin{bmatrix}
1&1&-1&1\\0&-2&0&0\\0&-1&3&1
\end{bmatrix}
\rightarrow
\begin{bmatrix}
1&0&-4&0\\0&1&0&0\\0&0&3&1
\end{bmatrix}
\end{equation*}
所以$(x_1,x_2,x_3,x_4)^T=k(4,0,1,-3)^T$,这里$k$为任意常数。

\begin{eg}
(1)$\mathbb{R}^n$的自然基$\{\vec{e}_1,\vec{e}_2,\cdots,\vec{e}_n\}$为一组
标准正交基。\\
(2)下列两组基底都是$\mathbb{R}^2$的标准正交基:
\begin{equation*}
\vec{\alpha}_1=\frac{1}{\sqrt{2}}\begin{bmatrix}1\\1\end{bmatrix},
\vec{\alpha}_2=\frac{1}{\sqrt{2}}\begin{bmatrix}1\\-1\end{bmatrix};
\vec{\beta}_1=\begin{bmatrix}\cos\theta\\ \sin\theta\end{bmatrix},
\vec{\beta}_2=\begin{bmatrix}-\sin\theta\\ \cos\theta\end{bmatrix}.
\end{equation*}
\end{eg}
\begin{eg}
设$\vec{\varepsilon}_1,\vec{\varepsilon}_2,\cdots,\vec{\varepsilon}_n$是
$\mathbb{R}^n$的一组标准正交基,$\vec{\alpha}\in\mathbb{R}^n$,求向量$\vec{\alpha}$在这组标准正交基下的坐标:$\vec{X}=(x_1,x_2,\cdots,x_n)^T$.\
\end{eg}
解:设$\vec{\alpha}=x_1\vec{\varepsilon}_1+
x_2\vec{\varepsilon}_2+\cdots+x_n\vec{\varepsilon}_n$,等式两边同时作用$\vec{\varepsilon}_j$作内积,并且利用标准正交基的充要条件$(\vec{\varepsilon}_i,\vec{\varepsilon}_j)=\delta_{ij}$就有
\begin{equation*}
(\vec{\alpha},\vec{\varepsilon}_j)=(x_1\vec{\varepsilon}_1+
x_2\vec{\varepsilon}_2+\cdots+x_n\vec{\varepsilon}_n,
\vec{\varepsilon}_j)=x_j
\end{equation*}
故$\vec{\alpha}$在这组基下的坐标向量$\vec{X}$的第$j$个分量为$x_j=(\vec{\alpha},\vec{\varepsilon}),j=1,2,\cdots,n$.
\begin{eg}
把$\mathbb{R}^3$中的基
$\vec{\alpha}_1=\begin{bmatrix}1\\1\\1\end{bmatrix}$,
$\vec{\alpha}_2=\begin{bmatrix}-1\\0\\-1\end{bmatrix}$,
$\vec{\alpha}_3=\begin{bmatrix}-1\\2\\3\end{bmatrix}$化为一组标准正交基。
\end{eg}
解:现正交化:
\begin{align*}
\vec{\beta}_1=&\vec{\alpha}_1=\begin{bmatrix}1\\1\\1\end{bmatrix}\\
\vec{\beta}_2=&\vec{\alpha}_2-\frac{(\vec{\alpha}_2,\vec{\beta}_1)}{(\vec{\beta}_1,\vec{\beta}_1)}\vec{\beta}_1
              =\frac{1}{3}\begin{bmatrix}-1\\2\\-1\end{bmatrix}\\
\vec{\beta}_3=&\vec{\alpha}_3-\frac{(\vec{\alpha}_3,\vec{\beta}_1)}{(\vec{\beta}_1,\vec{\beta}_1)}\vec{\beta}_1-
              \frac{(\vec{\alpha}_3,\vec{\beta}_2)}{(\vec{\beta}_2,\vec{\beta}_2)}\vec{\beta}_2
              =\begin{bmatrix}-2\\0\\2\end{bmatrix}
\end{align*}
再单位化,得到:
\begin{equation*}
\vec{\gamma}_1=\frac{1}{\sqrt{3}}\begin{bmatrix}1\\1\\1\end{bmatrix},
\vec{\gamma}_2=\frac{1}{\sqrt{6}}\begin{bmatrix}-1\\2\\-1\end{bmatrix},
\vec{\gamma}_3=\frac{1}{\sqrt{2}}\begin{bmatrix}-1\\0\\1\end{bmatrix}.
\end{equation*}

\begin{eg}
 给出以下线性方程组
$\begin{cases}
x+y+z=1\\x+y+z=2
\end{cases}$
的最优近似解。
\end{eg}
解:此方程组比较简单,明显是一个矛盾方程组。令
\begin{equation*}
A=\begin{bmatrix}1&1&1\\1&1&1\end{bmatrix},
\vec{b}=\begin{bmatrix}1\\2\end{bmatrix}
\end{equation*}
从而,有$A^TA=\begin{bmatrix}2&2&2\\2&2&2\\2&2&2\end{bmatrix}$,而
$A^T\vec{b}=\begin{bmatrix}3\\3\\3\end{bmatrix}$。故法方程为:
$$x+y+z=\frac{3}{2}$$
可解得法方程的通解,也即原方程的最小二乘解为
\begin{equation*}
\vec{x}_0=\begin{bmatrix}\frac{3}{2}\\0\\0\end{bmatrix}+
k_1\begin{bmatrix}-1\\1\\0\end{bmatrix}+k_2\begin{bmatrix}-1\\0\\1\end{bmatrix}
~~~(k_1,k_2\in\mathbb{R})
\end{equation*}

\begin{eg}
设有以下实验数据
\begin{table}[H]
\centering
\begin{tabular}{cccccc}
  \hline
   $x_i$ & $1$ & $2$ & $3$ & $4$ & $5$ \\
   \hline
   $y_i$ & $1.2$ & $1.5$ & $2.3$ & $2.4$ & $3.3$ \\
  \hline
\end{tabular}
\end{table}
求形如$y=ax+b$的函数,其中$a,b$是待定参数,使它与实验数据的误差平方和最小。
\end{eg}
解:假设$a,b$已经确定,则当$x=1,2,3,4,5$时,$y$应得到如下理论值,即
\begin{equation*}
\begin{cases}
a+b=y_1^*\\
a+2b=y_2^*\\
a+3b=y_3^*\\
a+4b=y_4^*\\
a+5b=y_5^*
\end{cases}
\Rightarrow
\begin{bmatrix}
1&1\\1&2\\1&3\\1&4\\1&5
\end{bmatrix}
\begin{bmatrix}
a\\b
\end{bmatrix}
=\begin{bmatrix}y_1^*\\y_2^*\\y_3^*\\y_4^*\\y_5^*\\\end{bmatrix}
\text{记为}
A\begin{bmatrix}
a\\b
\end{bmatrix}
=\vec{y}^*
\end{equation*}
它们与实验数据的误差平方和为:$\sum_{i=1}^5(y_i^*-y_i)^2$,
表示为向量内积的形式且求最小值,可表示为
\begin{equation*}
\min|\vec{y}-\vec{y}^*|^2
\end{equation*}
因此求误差平方和的最小值,就是求关于$A\begin{bmatrix}a\\b\end{bmatrix}=\vec{y}$的
最小二乘解。\\
由于$A=\begin{bmatrix}1&1\\1&2\\1&3\\1&4\\1&5\end{bmatrix}$,
$\vec{y}=\begin{bmatrix}1.2\\1.5\\2.3\\2.4\\3.3\end{bmatrix}$,
且$A$明显为列满秩矩阵,分别计算
\begin{equation*}
A^TA=\begin{bmatrix}5&15\\15&55\end{bmatrix},
A^T\vec{y}=\begin{bmatrix}10.7\\37.2\end{bmatrix},
\end{equation*}
故最小二乘解为:
\begin{equation*}
\begin{bmatrix}a_0\\b_0\end{bmatrix}=
(A^TA)^{-1}A^T\vec{y}=\begin{bmatrix}0.61\\0.51\end{bmatrix},
\end{equation*}
故满足条件的一次函数为$y=0.61+0.51x$。

\begin{eg}
设有以下实验数据
\begin{table}[H]
\centering
\begin{tabular}{cccccc}
  \hline
   $x_i$ & 1 & 2 & 3& 4 &5 \\
   \hline
   $y_i$ & 6.1 & 10.5 & 18.4& 26.5&38.2 \\
  \hline
\end{tabular}
\end{table}
求形如$y=a+bx+cx^2$的函数,其中$a,b,c$是待定参数,使它与实验数据的误差平方和最小。
\end{eg}
解:同上例分析,可说明使得误差平方和最小的二次函数系数$(a_0,b_0,c_0)^T$,就是
如下线性方程组的最小二乘解:
\begin{equation*}
\begin{cases}
a+b+c=6.1\\
a+2b+4c=10.5\\
a+3b+9c=18.4\\
a+4b+16c=26.5\\
a+5b+25c=38.2
\end{cases}
\text{记为}
A=\begin{bmatrix}
1&1&1\\1&2&4\\1&3&9\\1&4&16\\1&5&25
\end{bmatrix},
\vec{y}=\begin{bmatrix}
6.1\\10.5\\18.4\\26.5\\38.2
\end{bmatrix},
\end{equation*}
由于$A$为列满秩矩阵,分别计算
\begin{equation*}
A^TA=\begin{bmatrix}5&15&55\\15&55&225\\55&225&979\end{bmatrix},
A^T\vec{y}=\begin{bmatrix}99.8\\379.3\\1592.7\end{bmatrix},
\end{equation*}
故最小二乘解为:
\begin{equation*}
\begin{bmatrix}a_0\\b_0\end{bmatrix}=
(A^TA)^{-1}A^T\vec{y}=\begin{bmatrix}3.64\\1.35\\1.11\end{bmatrix},
\end{equation*}
故满足条件的二次函数为$y=3.64+1.35x+1.11x^2$。

%%%%%%%%%%%%%%%%%%%%%%%%%%%%%%%%%%%%%%%%%%%%%%%%%%%%%%%%%%%%%%%%%%%%%%%%%%%%%%%%%

\section{课后习题}

\begin{ex}\label{6.1}
以下定义在$\mathbb{R}^3$上的运算是否构成内积?并说明理由。其中$\vec{x}=(x_1,x_2,x_3 )^T$,
 $\vec{y}=(y_1,y_2,y_3 )^T$。\\
(1)$(\vec{x},\vec{y})=x_1y_1$;\\
(2)$(\vec{x},\vec{y})=x_1y_1+x_2^2+y_2^2$;\\
(3)$(\vec{x},\vec{y})=x_1y_1-x_2^2-y_2^2$;\\
(4)$(\vec{x},\vec{y})=x_1y_1+2x_1y_2+x_3y_3$;\\
(5)$(\vec{x},\vec{y})=x_1^2+y_1^2+y_2^2+x_3^2+y_3^2$;\\
(6)$(\vec{x},\vec{y})=x_1y_1+2x_2y_2+x_3y_3$;
\end{ex}

\begin{ex}\label{6.2}
2. 求下列$\mathbb{R}^4$中向量的模长。\\
(1)$\vec{x}=(1,2,2,0)^T$;\\
(2)$\vec{x}=(2,1,-1,-1)^T$;\\
(3)$\vec{x}=(0,1,2,-1)^T$;\\
(4)$\vec{x}=(3,0,-4,-5)^T$;
\end{ex}

\begin{ex}\label{6.3}
求下列$\mathbb{R}^4$中两个向量的内积。\\
(1)$\vec{x}=(1,1,2,0)^T$,$\vec{y}=(0,0,2,3)^T$;\\
(2)$\vec{x}=(0,1,1,2)^T$,$\vec{y}=(1,0,3,-3)^T$;\\
(3)$\vec{x}=(1,0,0,0)^T$,$\vec{y}=(1,-1,2,1)^T$;\\
(4)$\vec{x}=(0,0,0,-1)^T$,$\vec{y}=(1,0,1,-3)^T$;
\end{ex}

\begin{ex}\label{6.4}
在$\mathbb{R}^4$中,求两个向量$\vec{x}$,$\vec{y}$之间的夹角$\langle\vec{x},\vec{y}\rangle$,设\\
(1)$\vec{x}=(1,1,2,0)^T$,$\vec{y}=(0,0,0,3)^T$;\\
(2)$\vec{x}=(0,1,1,2)^T$,$\vec{y}=(0,-2,-2,-4)^T$;\\
(3)$\vec{x}=(1,0,0,0)^T$,$\vec{y}=(1,-1,0,0)^T$;\\
(4)$\vec{x}=(0,0,0,-1)^T$,$\vec{y}=(1,5,1,3)^T$;
\end{ex}

\begin{ex}\label{6.5}
已知$|\vec{x}|=4$,$|\vec{y}|=2$,$|\vec{x}-\vec{y}|=2\sqrt{6}$,求$(\vec{x},\vec{y})$。
\end{ex}

\begin{ex}\label{6.6}
设$|\vec{x}-\vec{y}|=|\vec{x}+\vec{y}|$,$\vec{x}=(1,1,2,1)^T$,$\vec{y}=(0,0,1,m)^T$,求$m$的值。
\end{ex}

\begin{ex}\label{6.7}
设$\vec{\alpha}$,$\vec{\beta}$是$R^n$ 中的两个非零向量,$k$是一个正实数,证明:\\
(1)$\langle k\vec{\alpha},\vec{\beta}\rangle=\langle\vec{\alpha},\vec{\beta}\rangle$;\\
(2)$\langle\vec{\alpha},\vec{\beta}\rangle+\langle-\vec{\alpha},\vec{\beta}\rangle=\pi$。
\end{ex}

\begin{ex}\label{6.8}
已知$\vec{\alpha}_1=(1,2,-1)^T$,$\vec{\alpha}_2=(2,3,4)^T$。\\
(1)求$\vec{\alpha}_1$与$\vec{\alpha}_2$的长度。\\
(2)求$\vec{\alpha}_1$与$\vec{\alpha}_2$的夹角。\\
(3)求与$\vec{\alpha}_1$与$\vec{\alpha}_2$ 都正交的向量。
\end{ex}

\begin{ex}\label{6.9}
通常$\vec{\alpha}$与$\vec{\beta}$的距离定义为$d(\vec{\alpha},\vec{\beta})=|\vec{\alpha}-\vec{\beta}|$,证明:
\begin{equation*}
d(\vec{\alpha},\vec{\gamma})\leq\d(\vec{\alpha},\vec{\beta})+d(\vec{\beta},\vec{\gamma}).
\end{equation*}
\end{ex}

\begin{ex}\label{6.10}
求$\mathbb{R}^4$中一单位向量与$(1,1,1,1)^T$,$(1,1,-1,-1)^T$,$(1,-1,1,-1)^T$正交。
\end{ex}

\begin{ex}\label{6.11}
在$n$维向量空间$R^{n}$中,假设$\vec{\alpha}_1,\vec{\alpha}_2,\cdots,\vec{\alpha}_{n-1}$ 线性无关,
且它们和向量$\vec{\beta}_1,\vec{\beta}_2$都正交,证明:$\vec{\beta}_1,\vec{\beta}_2$ 线性相关。\\
\end{ex}

\begin{ex}\label{6.12}
求齐次方程组
\begin{equation*}
  \begin{cases}
  2x_1+x_2+x_3+x_4=0\\
  x_1+x_2+x_3=0
  \end{cases}
\end{equation*}
的解空间(作为$\mathbb{R}^4$的子空间)的一组标准正交基。
\end{ex}

\begin{ex}\label{6.13}
求齐次方程组
\begin{equation*}
  \begin{cases}
  2x_1+x_2-x_3+x_4=0\\
  x_1+x_2-x_3=0
  \end{cases}
\end{equation*}
的解空间(作为$\mathbb{R}^4$的子空间)的一组标准正交基。
\end{ex}

\begin{ex}\label{6.14}
求齐次方程组
\begin{equation*}
  \begin{cases}
  x_1+x_2+x_3+x_4=0\\
  x_2+x_3=0
  \end{cases}
\end{equation*}
的解空间$W$的一组标准正交基以及$W^{\bot}$的一组标准正交基。
\end{ex}

\begin{ex}\label{6.15}
设$\vec{\alpha}_1=(1,1,1,1)^T$,$\vec{\alpha}_2=(3,3,1,1)^T$,$\vec{\alpha}_3=(3,1,3,1)^T$,$\vec{\alpha}_4=(3,-1,4,3)^T$。
求$\mathbb{R}^4$的一组标准正交基$\vec{\varepsilon}_1$,$\vec{\varepsilon}_2$,$\vec{\varepsilon}_3$,$\vec{\varepsilon}_4$ 使得\\
$L(\vec{\varepsilon}_1,\vec{\varepsilon}_2,\vec{\varepsilon}_3,\vec{\varepsilon}_4)=L(\vec{\alpha}_1,\vec{\alpha}_2,\vec{\alpha}_3,\vec{\alpha}_4)$。
\end{ex}

\begin{ex}\label{6.16}
假设$\vec{\varepsilon}_1$,$\vec{\varepsilon}_2$,$\vec{\varepsilon}_3$是$\mathbb{R}^3$中的一组标准正交基,证明
\begin{eqnarray*}
 \vec{\alpha}_1 &=& \frac{1}{3}(2\vec{\varepsilon}_1-\vec{\varepsilon}_2+2\vec{\varepsilon}_3) \\
 \vec{\alpha}_2 &=& \frac{1}{3}(2\vec{\varepsilon}_1+2\vec{\varepsilon}_2-\vec{\varepsilon}_3) \\
 \vec{\alpha}_3 &=& \frac{1}{3}(2\vec{\varepsilon}_2+2\vec{\varepsilon}_3-\vec{\varepsilon}_1) \\
\end{eqnarray*}
也是$\mathbb{R}^3$中的一组标准正交基。
\end{ex}

\begin{ex}\label{6.17}
设$\vec{\varepsilon}_1$,$\vec{\varepsilon}_2$,$\vec{\varepsilon}_3$,$\vec{\varepsilon}_4$,$\vec{\varepsilon}_5$是$\mathbb{R}^5$的一组标准正交基,
\begin{equation*}
  V_1=L(\vec{\alpha}_1,\vec{\alpha}_2,\vec{\alpha}_3).
\end{equation*}
其中$\vec{\alpha}_1=\vec{\varepsilon}_1+\vec{\varepsilon}_4$,
$\vec{\alpha}_2=\vec{\varepsilon}_1-\vec{\varepsilon}_2+\vec{\varepsilon}_4$,
$\vec{\alpha}_3=2\vec{\varepsilon}_1+\vec{\varepsilon}_2+\vec{\varepsilon}_3$。
求$V_1$的一组标准正交基。
\end{ex}

\begin{ex}\label{6.18}
试把$(1,0,1,0)^T$,$(0,1,0,0)^T$扩充为$\mathbb{R}^4$的一组标准正交基。
\end{ex}

\begin{ex}\label{6.19}
已知$\mathbb{R}^4$中向量$\vec{\alpha}_1=(1,0,0,0)^T$,$\vec{\alpha}_2=(0,1,1,0)^T$生成子空间
$W=L(\vec{\alpha}_1,\vec{\alpha}_2)$。 试在标准内积下,求正交补$W^{\bot}$ 的一组标准正交基。
\end{ex}

\begin{ex}\label{6.20}
写出所有三阶正交矩阵,它的元素是0或1。
\end{ex}

\begin{ex}\label{6.21}
已知$Q=\begin{bmatrix}a&-\frac{3}{7}&\frac{2}{7}\\b&c&d\\-\frac{3}{7}&\frac{2}{7}&e\end{bmatrix}$ 是正交矩阵,
求$a,b,c,d,e$的值。
\end{ex}

\begin{ex}\label{6.22}
证明:上三角形的正交矩阵一定是对角阵,且对角线上的元素为$+1$ 或者$-1$。
\end{ex}

\begin{ex}\label{6.23}
设$A$是$n$阶正交矩阵,$B$也是$n$阶正交矩阵。证明$AB$和$BA$也是正交矩阵。
\end{ex}

\begin{ex}\label{6.24}
设$A$是对称矩阵,且$A$也是正交矩阵,证明
\begin{equation*}
 A^{k}=\begin{cases}
  I,& k\geq2\text{且}k\text{是偶数}\\
  A,& k\geq2\text{且}k\text{是奇数}
  \end{cases}
\end{equation*}
\end{ex}

\begin{ex}\label{6.25}
设$A$是正交矩阵,证明$A$的伴随矩阵$A^*$ 也是正交矩阵。
\end{ex}

\begin{ex}\label{6.26}
设$\vec{\alpha}$是$\mathbb{R}^3$中的一个单位向量,$I$是3阶的单位矩阵,证明
\begin{equation*}
Q=I-2\vec{\alpha}\vec{\alpha}^T
\end{equation*}
是一个正交矩阵(称为Householder 变换),当$\vec{\alpha}=\frac{1}{\sqrt{2}}(1,0,1)^T$ 时,求出$Q$。
\end{ex}

\begin{ex}\label{6.27}
证明$A=\begin{bmatrix}\sin\theta&\cos\theta\\-\cos\theta&\sin\theta\end{bmatrix}$ 是正交矩阵。
\end{ex}

\begin{ex}\label{6.28}
设$A=\begin{bmatrix}1&1\\1&3\end{bmatrix}$,作$A$的$QR$分解。
\end{ex}

\begin{ex}\label{6.29}
设$A=\begin{bmatrix}1&0&1\\1&1&1\\1&2&3\end{bmatrix}$,作$A$的$QR$分解。
\end{ex}

\begin{ex}\label{6.30}
若一个正交矩阵中每一个元素都是$\frac{1}{2}$ 或者$-\frac{1}{2}$,那么这个矩阵是几阶的?
\end{ex}

\begin{ex}\label{6.31}
设有以下实验数据
\begin{table}[H]
\centering
\begin{tabular}{ccccc}
  \hline
   $x_i$ & 1 & 2 & 3& 4 \\
   \hline
   $y_i$ & 5.1 & 8.2 & 10& 14.8 \\
  \hline
\end{tabular}
\end{table}

(1)求形如$y=ax+b$的函数,其中$a,b$是待定参数,使它与实验数据的误差平方和最小。

(2)求形如$y=mx^2+nx+p$的函数,其中$m,n,p$是待定参数,使它与实验数据的误差平方和最小。
\end{ex}

%%%%%%%%%%%%%%%%%%%%%%%%%%%%%%%%%%%%%%%%%%%%%%%%%%%%%%%%%%%%%%%%%%%%%%%%%%%%%%%%%%

\section{习题答案}
\textbf{习题 \ref{6.1} 解答:}\\
(1)否,不满足正定性,向量$(0,1,1)^T$ 与自己的内积为0;\\
(2)否,不满足正定性,向量$(0,0,1)^T$ 与自己的内积为0;\\
(3)否,不满足正定性,向量$(1,1,;1)^T$ 与自己的内积为-1;\\
(4)否,不满足对称性,若$\vec{x}=(1,0,0)^T$, $\vec{y}=(0,1,0)^T$,那么$(\vec{x},\vec{y})=2$,$(\vec{y},\vec{x})=0$;\\
(5)否,不满足数乘,取$\vec{x}=(1,0,0)$, $\vec{y}=(0,1,0)^T$,$k=2$,那么$(k\vec{x},\vec{y})=5$,
          $k(\vec{x},\vec{y})=4$。\\
(6)是。容易验证其满足正定性、对称性、数乘以及分配律。\\
\textbf{习题 \ref{6.2} 解答:}\\
(1)$|\vec{x}|=\sqrt{1^2+2^2+2^2+0^2}=3$;\\
(2)$|\vec{x}|=\sqrt{2^2+1^2+(-1)^2+(-1)^2}=\sqrt{7}$;\\
(3)$|\vec{x}|=\sqrt{0^2+1^2+2^2+(-1)^2}=\sqrt{6}$;\\
(4)$|\vec{x}|=\sqrt{3^2+0^2+(-4)^2+(-5)^2}=5\sqrt{2}$。\\
\textbf{习题 \ref{6.3} 解答:}\\
(1)$(\vec{x},\vec{y})=1\times0+1\times0+2\times2+0\times3=4$;\\
(2)$(\vec{x},\vec{y})=-3$;\\
(3)$(\vec{x},\vec{y})=1$;\\
(4)$(\vec{x},\vec{y})=3$。\\
\textbf{习题 \ref{6.4} 解答:}\\
(1)$(\vec{x},\vec{y})=0$,故$\cos\langle\vec{x},\vec{y}\rangle=0$,因此$\langle\vec{x},\vec{y}\rangle=\frac{\pi}{2}$。\\
(2)$\pi$;\\
(3)$\frac{\pi}{4}$;\\
(4)$(\vec{x},\vec{y})=-3$,$|\vec{x}|=\sqrt{(-1)^2}=1$,$|\vec{y}|=\sqrt{1^2+5^2+1^2+3^2}=6$,\\
     因此$\langle\vec{x},\vec{y}\rangle=\cos^{-1}(-\frac{1}{2})=\frac{2\pi}{3}$。\\
\textbf{习题 \ref{6.5} 解答:}\\
$|\vec{x}-\vec{y}|^2=|\vec{x}|^2+|\vec{y}|^2-2(\vec{x},\vec{y})=24$;\\
$2(\vec{x},\vec{y})=24-|\vec{x}|^2-|\vec{y}|^2=4$;\\
因此$(\vec{x},\vec{y})=2$。\\
\textbf{习题 \ref{6.6} 解答:}\\
由于$|\vec{x}-\vec{y}|^2=|\vec{x}|^2+|\vec{y}|^2-2(\vec{x},\vec{y})$ 以及$|\vec{x}+\vec{y}|^2=|\vec{x}|^2+|\vec{y}|^2+2(\vec{x},\vec{y})$;\\
又由于$|\vec{x}-\vec{y}|=|\vec{x}+\vec{y}|$,\\
则有$(\vec{x},\vec{y})=0$,因此$2+m=0$,即$m=-2$。\\
\textbf{习题 \ref{6.7} 解答:}\\
(1)$\langle k\vec{\alpha},\vec{\beta}\rangle=\cos^{-1}\frac{(k\vec{\alpha},\vec{\beta})}{|k\vec{\alpha}||\vec{\beta}|}
   =\cos^{-1}\frac{k(\vec{\alpha},\vec{\beta})}{k|\vec{\alpha}||\vec{\beta}|}
   =\cos^{-1}\frac{(\vec{\alpha},\vec{\beta})}{|\vec{\alpha}||\vec{\beta}|}
   =\langle\vec{\alpha},\vec{\beta}\rangle$。\\
(2)因为$\langle -\vec{\alpha},\vec{\beta}\rangle=\cos^{-1}\frac{(-\vec{\alpha},\vec{\beta})}{|-\vec{\alpha}||\vec{\beta}|}
   =\cos^{-1}\frac{-(\vec{\alpha},\vec{\beta})}{|\vec{\alpha}||\vec{\beta}|}
   =\pi-\cos^{-1}\frac{(\vec{\alpha},\vec{\beta})}{|\vec{\alpha}||\vec{\beta}|}
   =\pi-\langle\vec{\alpha},\vec{\beta}\rangle$,\\
   因此$\langle\vec{\alpha},\vec{\beta}\rangle+\langle\vec{-\alpha},\vec{\beta}\rangle=\pi$。\\
\textbf{习题 \ref{6.8} 解答:}\\
(1)$|\vec{\alpha}_1|=\sqrt{1^2+2^2+(-1)^2}=\sqrt{6}$;\\
$|\vec{\alpha}_2|=\sqrt{2^2+3^2+4^2}=\sqrt{29}$;\\
(2)$(\vec{\alpha}_1,\vec{\alpha}_2)=4$,\\
$\langle\vec{\alpha}_1,\vec{\alpha}_2\rangle=\arccos\frac{4}{\sqrt{174}}$。\\
(3)设$\vec{\beta}=(x_1,x_2,x_3)^T$与$\vec{\alpha}_1$与$\vec{\alpha}_2$都正交,则
\begin{equation*}
  \begin{cases}
  x_1+2x_2-x_3=0\\
  2x_1+3x_2+4x_3=0
  \end{cases}
\end{equation*}
解得
\begin{equation*}
\vec{\beta}=k(-11,6,1)^T,\text{其中}k\text{是任意的数}.
\end{equation*}\\
\textbf{习题 \ref{6.9} 解答:}\\
\begin{align*}
(d(\vec{\alpha},\vec{\gamma}))^2=&|\vec{\alpha}-\vec{\gamma}|^2=(\vec{\alpha}-\vec{\gamma},\vec{\alpha}-\vec{\gamma})\\
=&((\vec{\alpha}-\vec{\beta})+(\vec{\beta}-\vec{\gamma}),(\vec{\alpha}-\vec{\beta})+(\vec{\beta}-\vec{\gamma}))\\
=&(\vec{\alpha}-\vec{\beta},\vec{\alpha}-\vec{\beta})+2(\vec{\alpha}-\vec{\beta},\vec{\beta}-\vec{\gamma})+
(\vec{\beta}-\vec{\gamma},\vec{\beta}-\vec{\gamma})\\
\leq&|\vec{\alpha}-\vec{\beta}|^2+2|\vec{\alpha}-\vec{\beta}||\vec{\beta}-\vec{\gamma}|+|\vec{\beta}-\vec{\gamma}|^2\\
=&((\vec{\alpha}-\vec{\beta})+(\vec{\beta}-\vec{\gamma}))^2=(d(\vec{\alpha},\vec{\beta})+d(\vec{\beta},\vec{\gamma}))^2
\end{align*}
从而$d(\vec{\alpha},\vec{\gamma})\leq d(\vec{\alpha},\vec{\beta})+d(\vec{\beta},\vec{\gamma})$。\\
\textbf{习题 \ref{6.10} 解答:}\\
设所求向量为$(x_1,x_2,x_3,x_4)^T$,则
\begin{equation*}
  \begin{cases}
  x_1+x_2+x_3+x_4=0\\
  x_1+x_2-x_3-x_4=0\\
  x_1-x_2+x_3-x_4=0\\
  x_1^2+x_2^2+x_3^2+x_4^2=1\\
  \end{cases}
\end{equation*}
求解得到:
\begin{equation*}
(x_1,x_2,x_3,x_4)^T=(\frac{1}{2},-\frac{1}{2},-\frac{1}{2},\frac{1}{2})^T.
\end{equation*}
\textbf{习题 \ref{6.11} 解答:}\\
如果$\vec{\beta}_1,\vec{\beta}_2$中含有零向量,显然二者是线性相关的,故不妨设$\vec{\beta}_1,\vec{\beta}_2$
都是非零向量。\\
因为$\vec{\alpha}_i$,$\vec{\beta}_1$正交,有:
\begin{equation*}
\vec{\alpha}_i^T\vec{\beta}_1=0,~i=1,2,\cdots,n-1,
\end{equation*}
改写为矩阵形式:
\begin{equation*}
\begin{bmatrix}\vec{\alpha}_1^T\\ \vec{\alpha}_2^T\\ \vdots\\ \vec{\alpha}_{n-1}^T\end{bmatrix}
\vec{\beta}_1=0
\end{equation*}
令矩阵
\begin{equation*}
A=\begin{bmatrix}\vec{\alpha}_1^T\\ \vec{\alpha}_2^T\\ \vdots\\ \vec{\alpha}_{n-1}^T\end{bmatrix}
\end{equation*}
因此,$\vec{\beta}_1$都是齐次方程组
\begin{equation}\label{qici}
Ax=b
\end{equation}
的非零解。由于$A$的行向量组线性无关,因此$R(A)=n-1$,故解空间的维数是1,故$\vec{\beta}_1$是它的一个
基础解系。同理,$\vec{\beta}_2$ 是齐次方程组\eqref{qici}的非零解,于是$\vec{\beta}_2$ 可由$\vec{\beta}_1$
线性表出,所以$\vec{\beta}_1,\vec{\beta}_2$ 线性相关。 \\
\textbf{习题 \ref{6.12} 解答:}\\
解方程组得到基础解系:\\
$\vec{\alpha}_1=(1,-1,0,1)^T$,$\vec{\alpha}_2=(1,0,-1,-1)^T$\\
则$W=L(\vec{\alpha}_1,\vec{\alpha}_2)$,将$\vec{\alpha}_1$,$\vec{\alpha}_2$正交化
\begin{align*}
\vec{\beta}_1=&\vec{\alpha}_1=(1,-1,0,1)^T\\
\vec{\beta}_2=&\vec{\alpha}_2-\frac{(\vec{\alpha}_2,\vec{\beta}_1)}{(\vec{\beta}_1,\vec{\beta}_1)}\vec{\beta}_1=\frac{1}{3}(1,2,-5,-1)^T
\end{align*}
再将$\vec{\beta}_1$,$\vec{\beta}_2$单位化:
\begin{align*}
\vec{\varepsilon}_1=&\frac{1}{|\vec{\beta}_1|}\vec{\beta}_1=\frac{1}{\sqrt{3}}(1,-1,0,1)^T\\
\vec{\varepsilon}_2=&\frac{1}{|\vec{\beta}_2|}\vec{\beta}_2==\frac{1}{\sqrt{31}}(1,2,-5,-1)^T
\end{align*}
因此$\varepsilon_1$,$\varepsilon_2$是解空间$W$的一组标准正交基。\\
\textbf{习题 \ref{6.13} 解答:}\\
解方程组得到基础解系:\\
$\vec{\alpha}_1=(1,0,1,-1)^T$,$\vec{\alpha}_2=(1,-1,0,-1)^T$\\
将$\vec{\alpha}_1$,$\vec{\alpha}_2$正交化
\begin{align*}
\vec{\beta}_1=&\vec{\alpha}_1=(1,0,1,-1)^T\\
\vec{\beta}_2=&\vec{\alpha}_2-\frac{(\vec{\alpha}_2,\vec{\beta}_1)}{(\vec{\beta}_1,\vec{\beta}_1)}\vec{\beta}_1=\frac{1}{3}(1,-5,-2,-1)^T\\
\end{align*}
再将$\beta_1$,$\beta_2$单位化:
\begin{align*}
\vec{\varepsilon}_1=\frac{1}{\sqrt{3}}(1,-1,0,1)^T\\
\vec{\varepsilon}_2=\frac{1}{\sqrt{31}}(1,-5,-2,-1)^T
\end{align*}
因此$\varepsilon_1$,$\varepsilon_2$是解空间一组标准正交基。\\
\textbf{习题 \ref{6.14} 解答:}\\
解方程组得到基础解系:
\begin{equation*}
\vec{\alpha}_1=(1,0,0,-1)^T,\vec{\alpha}_2=(0,1,-1,0)^T.
\end{equation*}
由于$(\vec{\alpha}_1,\vec{\alpha}_2)=0$,只须单位化:
\begin{equation*}
\vec{\alpha}_1=\frac{1}{\sqrt{2}}(1,0,0,-1)^T,
\vec{\alpha}_2=\frac{1}{\sqrt{2}}(0,1,-1,0)^T.
\end{equation*}
那么$\vec{\beta}_1,\vec{\beta}_2$是解空间$W$ 的一组标准正交基。
设$\vec{\gamma}=(y_1,y_2,y_3,y_4)$ 是$W^{\bot}$的基,那么$(\vec{\beta}_1,\vec{\gamma})=0$ 且$(\vec{\beta}_2,\vec{\gamma})=0$。
即:
\begin{equation*}
\begin{cases}
  y_1-y_4=0\\
  y_2-y_3=0
\end{cases}
\end{equation*}
解上述方程组得到基础解系:
\begin{equation*}
\vec{\gamma}_1=(1,0,0,1)^T,\vec{\gamma}_2=(0,1,1,0)^T.
\end{equation*}
由于$(\vec{\gamma}_1,\vec{\gamma}_2)=0$,只须单位化:
\begin{equation*}
\vec{\xi}_1=\frac{1}{\sqrt{2}}(1,0,0,1)^T,
\vec{\xi}_2=\frac{1}{\sqrt{2}}(0,1,1,0)^T.
\end{equation*}
那么$\vec{\xi}_1,\vec{\xi}_2$是解空间$W^{\bot}$的一组标准正交基。\\
\textbf{习题 \ref{6.15} 解答:}\\
将$\vec{\alpha}_1,\vec{\alpha}_2,\vec{\alpha}_3,\vec{\alpha}_4$ 正交化:
\begin{align*}
\vec{\beta}_1=&\vec{\alpha}_1=(1,1,1,1)^T\\
\vec{\beta}_2=&\vec{\alpha}_2-\frac{(\vec{\alpha}_2,\vec{\beta}_1)}{(\vec{\beta}_1,\vec{\beta}_1)}\vec{\beta}_1=(1,1,-1,-1)^T\\
\vec{\beta}_3=&\vec{\alpha}_3-\frac{(\vec{\alpha}_3,\vec{\beta}_1)}{(\vec{\beta}_1,\vec{\beta}_1)}\vec{\beta}_1-
              \frac{(\vec{\alpha}_3,\vec{\beta}_2)}{(\vec{\beta}_2,\vec{\beta}_2)}\vec{\beta}_2\\
             =&(3,1,3,1)^T-2(1,1,1,1)^T=(1,-1,1,-1)^T\\
\vec{\beta}_4=&\vec{\alpha}_4-\frac{(\vec{\alpha}_4,\vec{\beta}_1)}{(\vec{\beta}_1,\vec{\beta}_1)}\vec{\beta}_1-
               \frac{(\vec{\alpha}_4,\vec{\beta}_2)}{(\vec{\beta}_2,\vec{\beta}_2)}\vec{\beta}_2
                -\frac{(\vec{\alpha}_4,\vec{\beta}_3)}{(\vec{\beta}_3,\vec{\beta}_3)}\vec{\beta}_3\\
             =&(3,-1,4,3)^T-\frac{9}{2}(1,1,1,1)^T+\frac{5}{2}(1,1,-1,-1)^T+\frac{5}{2}(1,-1,1,-1)\\
             =&\frac{1}{2}(7,-11,-1,13)^T
\end{align*}
再将$\vec{\beta}_1,\vec{\beta}_2,\vec{\beta}_3,\vec{\beta}_4$ 单位化:
\begin{align*}
\vec{\varepsilon}_1=&\frac{1}{|\vec{\beta}_1|}\vec{\beta}_1=\frac{1}{2}(1,1,1,1)^T\\
\vec{\varepsilon}_2=&\frac{1}{|\vec{\beta}_2|}\vec{\beta}_2=\frac{1}{2}(1,1,-1,-1)^T\\
\vec{\varepsilon}_3=&\frac{1}{|\vec{\beta}_3|}\vec{\beta}_3=\frac{1}{2}(1,-1,1,-1)^T\\
\vec{\varepsilon}_4=&\frac{1}{|\vec{\beta}_4|}\vec{\beta}_4=\frac{1}{4\sqrt{105}}(7,-11,-1,13)^T
\end{align*}
则$\vec{\varepsilon}_1$,$\vec{\varepsilon}_2$,$\vec{\varepsilon}_3$,$\vec{\varepsilon}_4$ 是$R^4$的一组标准正交基,使得
$L(\vec{\varepsilon}_1,\vec{\varepsilon}_2,\vec{\varepsilon}_3,\vec{\varepsilon}_4)=L(\vec{\alpha}_1,\vec{\alpha}_2,\vec{\alpha}_3,\vec{\alpha}_4)$。\\
\textbf{习题 \ref{6.16} 解答:}\\
由于
\begin{eqnarray*}
   (\vec{\alpha}_1,\vec{\alpha}_2)&=&\frac{4}{9}(\vec{\varepsilon}_1,\vec{\varepsilon}_1)
   -\frac{2}{9}(\vec{\varepsilon}_2,\vec{\varepsilon}_2)-\frac{2}{9}(\vec{\varepsilon}_3,\vec{\varepsilon}_3)=0\\
   (\vec{\alpha}_1,\vec{\alpha}_3)&=&-\frac{2}{9}(\vec{\varepsilon}_1,\vec{\varepsilon}_1)
   -\frac{2}{9}(\vec{\varepsilon}_2,\vec{\varepsilon}_2)+\frac{4}{9}(\vec{\varepsilon}_3,\vec{\varepsilon}_3)=0\\
   (\vec{\alpha}_2,\vec{\alpha}_3)&=&-\frac{2}{9}(\vec{\varepsilon}_1,\vec{\varepsilon}_1)
   +\frac{4}{9}(\vec{\varepsilon}_2,\vec{\varepsilon}_2)-\frac{2}{9}(\vec{\varepsilon}_3,\vec{\varepsilon}_3)=0
\end{eqnarray*}
又由于
\begin{eqnarray*}
   (\vec{\alpha}_1,\vec{\alpha}_1)&=&1\\
   (\vec{\alpha}_2,\vec{\alpha}_2)&=&1\\
   (\vec{\alpha}_3,\vec{\alpha}_3)&=&1
\end{eqnarray*}
因此$\vec{\alpha}_1,\vec{\alpha}_2,\vec{\alpha}_3$ 也是$R^3$中的一组标准正交基。\\
\textbf{习题 \ref{6.17} 解答:}\\
首先正交化:
\begin{align*}
\vec{\beta}_1=&\vec{\alpha}_1=\vec{\varepsilon}_1+\vec{\varepsilon}_4\\
\vec{\beta}_2=&\vec{\alpha}_2-\frac{(\vec{\alpha}_2,\vec{\beta}_1)}{(\vec{\beta}_1,\vec{\beta}_1)}\vec{\beta}_1\\
             =&-\vec{\varepsilon}_2\\
\vec{\beta}_3=&\vec{\alpha}_3-\frac{(\vec{\alpha}_3,\vec{\beta}_1)}{(\vec{\beta}_1,\vec{\beta}_1)}\vec{\beta}_1-
              \frac{(\vec{\alpha}_3,\vec{\beta}_2)}{(\vec{\beta}_2,\vec{\beta}_2)}\vec{\beta}_2\\
             =&\vec{\alpha}_3-\frac{1}{2}\vec{\beta}_1+\vec{\beta}_2=\frac{3}{2}\vec{\varepsilon}_1+\vec{\varepsilon}_3-\frac{1}{2}\vec{\varepsilon}_4
\end{align*}
再将$\vec{\beta}_1,\vec{\beta}_2,\vec{\beta}_3$ 单位化:
\begin{align*}
\vec{\beta}_1=&\frac{1}{\sqrt{2}}(\vec{\varepsilon}_1+\vec{\varepsilon}_4)\\
\vec{\beta}_2=&\vec{\varepsilon}_2\\
\vec{\beta}_3=&\frac{1}{\sqrt{14}}(3\vec{\varepsilon}_1+2\vec{\varepsilon}_3-\vec{\varepsilon}_4)\\
\end{align*}
\textbf{习题 \ref{6.18} 解答:}\\
先将$(1,0,1,0)^T$,$(0,1,0,0)^T$ 扩充为$R^4$ 的一组标准正交基,令:
\begin{equation*}
\vec{\alpha}_1=(1,0,1,0)^T,\vec{\alpha}_2=(0,1,0,0)^T,\vec{\alpha}_3=(1,0,0,0)^T,\vec{\alpha}_4=(1,0,0,1)^T
\end{equation*}
显然,$\alpha_1,\alpha_2,\alpha_3,\alpha_4$ 是$R^4$的一组基底。下面将其正交化:
\begin{align*}
\vec{\beta}_1=&\vec{\alpha}_1=(1,0,1,0)^T\\
\vec{\beta}_2=&\vec{\alpha}_2-\frac{(\vec{\alpha}_2,\vec{\beta}_1)}{(\vec{\beta}_1,\vec{\beta}_1)}\vec{\beta}_1
             =\vec{\alpha}_2=(0,1,0,0)^T\\
\vec{\beta}_3=&\vec{\alpha}_3-\frac{(\vec{\alpha}_3,\vec{\beta}_1)}{(\vec{\beta}_1,\vec{\beta}_1)}\vec{\beta}_1-
              \frac{(\vec{\alpha}_3,\vec{\beta}_2)}{(\vec{\beta}_2,\vec{\beta}_2)}\vec{\beta}_2\\
             =&\begin{bmatrix}1\\0\\0\\0\end{bmatrix}-\frac{1}{2}\begin{bmatrix}1\\0\\1\\0\end{bmatrix}
             =\frac{1}{2}\begin{bmatrix}1\\0\\-1\\0\end{bmatrix}\\
\vec{\beta}_4=&\vec{\alpha}_4-\frac{(\vec{\alpha}_4,\vec{\beta}_1)}{(\vec{\beta}_1,\vec{\beta}_1)}\vec{\beta}_1-
               \frac{(\vec{\alpha}_4,\vec{\beta}_2)}{(\vec{\beta}_2,\vec{\beta}_2)}\vec{\beta}_2
                -\frac{(\vec{\alpha}_4,\vec{\beta}_3)}{(\vec{\beta}_3,\vec{\beta}_3)}\vec{\beta}_3\\
             =&\begin{bmatrix}1\\0\\0\\1\end{bmatrix}-\frac{1}{2}\begin{bmatrix}1\\0\\1\\0\end{bmatrix}-
                \frac{1}{2}\begin{bmatrix}1\\0\\-1\\0\end{bmatrix}
              =\begin{bmatrix}0\\0\\0\\1\end{bmatrix}
\end{align*}
再将$\beta_1,\beta_2,\beta_3$单位化:
\begin{align*}
\vec{\beta}_1=&\frac{1}{\sqrt{2}}(1,0,1,0)^T\\
\vec{\beta}_2=&(0,1,0,0)^T\\
\vec{\beta}_3=&\frac{1}{\sqrt{2}}(1,0,-1,0)^T\\
\vec{\beta}_4=&(0,0,0,1)^T
\end{align*}
\textbf{习题 \ref{6.19} 解答:}\\
设和$\vec{\alpha}_1$,,$\vec{\alpha}_2$都正交的向量$\vec{\beta}=(x_1,x-2,x_3,x_4)^T$,
则$(\vec{\alpha}_1,\vec{\beta})=0$,$(\vec{\alpha}_2,\vec{\beta})=0$。得到:
\begin{equation*}
  \begin{cases}
  x_1=0\\
  x_2+x_3=0
  \end{cases}
\end{equation*}
解得:
\begin{equation*}
\vec{\beta}_1=(0,0,0,1)^T,\vec{\beta}_2=(0,1,-1,0)^T
\end{equation*}
显然$(\vec{\beta}_1,\vec{\beta}_2)=0$,$\vec{\beta}_1\bot\vec{\beta}_2$。单位化:
\begin{equation*}
\vec{\gamma}_1=(0,0,0,1)^T,\vec{\gamma}_2=\frac{1}{\sqrt{2}}(0,1,-1,0)^T
\end{equation*}
则$\vec{\gamma}_1$,$\vec{\gamma}_2$是$W^{\bot}$的一组标准正交基底。\\
\textbf{习题 \ref{6.20} 解答:}\\
\begin{equation*}
 \begin{bmatrix}1&0&0\\0&1&0\\0&0&1\end{bmatrix},
 \begin{bmatrix}1&0&0\\0&0&1\\0&1&0\end{bmatrix},
 \begin{bmatrix}0&1&0\\1&0&0\\0&0&1\end{bmatrix},
\end{equation*}
\begin{equation*}
 \begin{bmatrix}0&1&0\\0&0&1\\1&0&0\end{bmatrix},
 \begin{bmatrix}0&0&1\\0&1&0\\1&0&0\end{bmatrix},
 \begin{bmatrix}0&0&1\\1&0&0\\0&1&0\end{bmatrix},
\end{equation*}
\textbf{习题 \ref{6.21} 解答:}\\
由于$Q$是正交矩阵,即:
\begin{equation*}
\begin{cases}
  a^2+b^2+(-\frac{3}{7})^2 = 1 \\
  (-\frac{3}{7})^2+c^2+(\frac{2}{7})^2 = 1 \\
  (\frac{2}{7})^2+d^2+e^2 = 1 \\
  -\frac{3}{7}a+bc-\frac{3}{7}\times\frac{2}{7} = 0 \\
  \frac{2}{7}a+bd-\frac{3}{7}e = 0 \\
  -\frac{3}{7}\times\frac{2}{7}+cd+\frac{2}{7}e = 0
\end{cases}
\end{equation*}
解得:
\begin{equation*}
\begin{cases}
  a = -\frac{6}{7}\\
  b = -\frac{2}{7} \\
  c = \frac{6}{7} \\
  d = \frac{3}{7} \\
  e = -\frac{6}{7}
\end{cases}
\text{或者}
\begin{cases}
  a = -\frac{6}{7}\\
  b = \frac{2}{7} \\
  c = -\frac{6}{7} \\
  d = -\frac{3}{7} \\
  e = -\frac{6}{7}
\end{cases}
\end{equation*}
\textbf{习题 \ref{6.22} 解答:}\\
设上三角阵$A$为$(a_{ij})_{n\times n}$,且当$i<j$时,$a_{ij}=0$。
\begin{equation*}
  AA^T=\begin{bmatrix}
  a_{11}&a_{12}&\cdots&a_{1n}\\
  0&a_{22}&\cdots&a_{2n}\\
  \vdots&\vdots&\ddots&\vdots\\
  0&\cdots&0&a_{nn}
  \end{bmatrix}
  \begin{bmatrix}
  a_{11}&0&\cdots&0\\
  a_{12}&a_{22}&\cdots&0\\
  \vdots&\vdots&\ddots&\vdots\\
  a_{1n}&a_{2n}&\cdots&a_{nn}
  \end{bmatrix}
\end{equation*}
由于$A$是正交矩阵,从而有$AA^T=I$,因此$a_{nn}a_{nn}=1$。\\
又由于$a_{1n}a_{nn}=0$,故$a_{1n}=0$;\\
从而可以得到:$a_{1j}=0,j>1$且$a_{11}^2=1$。\\
同理$a_{ij}=0,j>i$且$a_{ii}^2=1,i=1,2,\cdots,n$。
因此$A$为对角矩阵,且$A$对角元上的元素为$+1$ 或者$-1$。\\
\textbf{习题 \ref{6.23} 解答:}\\
由于$A,B$都是$n$阶正交矩阵,那么$A^TA=I$ 且$B^TB=I$。
又
\begin{equation*}
  (AB)^T(AB)=B^TA^TAB=B^TB=I
\end{equation*}
因此,$AB$是正交矩阵。同理:
\begin{equation*}
  (BA)^T(BA)=A^TB^TBA=A^TA=I
\end{equation*}
因此,$BA$也是正交矩阵。\\
\textbf{习题 \ref{6.24} 解答:}\\
由于$A$是对称矩阵,那么$A=A^T$;又由于$A$是正交矩阵,那么$A^TA=I$。故
\begin{equation*}
A^2=A^TA=I
\end{equation*}
因此,可以得到:
\begin{equation*}
 A^{k}=\begin{cases}
  I,& k\geq2\text{且}k\text{是偶数}\\
  A,& k\geq2\text{且}k\text{是奇数}
  \end{cases}
\end{equation*}
\textbf{习题 \ref{6.25} 解答:}\\
由于$AA^*=|A|I$,两边同时取转置:
\begin{equation*}
(A^*)^TA^T=|A|I
\end{equation*}
于是
\begin{equation*}
(A^*)^TA^*=(A^*)^TA^TAA^*=|A|^2I.
\end{equation*}
又由于$A$是正交矩阵,$A^TA=I$,$|A|^2=1$。所以
\begin{equation*}
(A^*)^TA^*=(A^*)^TA^TAA^*=I.
\end{equation*}
因此$A^*$是正交矩阵。
\textbf{习题 \ref{6.26} 解答:}\\
由于$\vec{\alpha}$是单位向量,即$\vec{\alpha}\vec{\alpha}^T=1$,于是
\begin{align*}
Q^TQ=&(I-2\vec{\alpha}\vec{\alpha}^T)^T(I-2\vec{\alpha}\vec{\alpha}^T)\\
    =&(I-2\vec{\alpha}\vec{\alpha}^T)(I-2\vec{\alpha}\vec{\alpha}^T)\\
    =&I-2\vec{\alpha}\vec{\alpha}^T-2\vec{\alpha}\vec{\alpha}^T+4\vec{\alpha}\vec{\alpha}^T\vec{\alpha}\vec{\alpha}^T\\
    =&I-4\vec{\alpha}\vec{\alpha}^T+4\vec{\alpha}\vec{\alpha}^T\\
    =&I
\end{align*}
因此$Q$是一个正交矩阵。
\begin{equation*}
Q=I-2\vec{\alpha}\vec{\alpha}^T
   =\begin{bmatrix}0&0&1\\0&1&0\\-1&0&0\end{bmatrix}.
\end{equation*}
\textbf{习题 \ref{6.27} 解答:}\\
由于
\begin{equation*}
 A^TA= \begin{bmatrix}\sin\theta&-\cos\theta\\ \cos\theta&\sin\theta\end{bmatrix}
 \begin{bmatrix}\sin\theta&\cos\theta\\-\cos\theta&\sin\theta\end{bmatrix}
 =\begin{bmatrix}1&0\\0&1\end{bmatrix}
\end{equation*}
因此,$A$是正交矩阵。\\
\textbf{习题 \ref{6.28} 解答:}\\
令$\vec{\alpha}_1=\begin{bmatrix}1\\1\end{bmatrix}$,$\vec{\alpha}_2=\begin{bmatrix}2\\3\end{bmatrix}$。
由Schimidt正交化,有\\
\begin{align*}
\vec{\beta}_1=&\vec{\alpha}_1=\begin{bmatrix}1\\1\end{bmatrix}\\
\vec{\beta}_2=&\vec{\alpha}_2-\frac{(\vec{\alpha}_2,\vec{\beta}_1)}{(\vec{\beta}_1,\vec{\beta}_1)}\vec{\beta}_1
             =\begin{bmatrix}1\\3\end{bmatrix}-\frac{4}{2}\begin{bmatrix}1\\1\end{bmatrix}=\begin{bmatrix}-1\\1\end{bmatrix}
\end{align*}
单位化得到:
\begin{equation*}
\vec{\gamma}_1=\frac{1}{\sqrt{2}}\begin{bmatrix}1\\1\end{bmatrix},
\vec{\gamma}_2=\frac{1}{\sqrt{2}}\begin{bmatrix}-1\\1\end{bmatrix}.
\end{equation*}
那么
\begin{equation*}
\begin{bmatrix}\vec{\alpha}_1&\vec{\alpha}_2\end{bmatrix}=
\begin{bmatrix}\vec{\beta}_1&\vec{\beta}_2\end{bmatrix}\begin{bmatrix}1&2\\0&1\end{bmatrix}
=\begin{bmatrix}\vec{\gamma}_1&\vec{\gamma}_2\end{bmatrix}\begin{bmatrix}\sqrt{2}&0\\0&\sqrt{2}\end{bmatrix}
\begin{bmatrix}1&2\\0&1\end{bmatrix}
\end{equation*}
即
\begin{equation*}
A=\begin{bmatrix}\frac{1}{\sqrt{2}}&-\frac{1}{\sqrt{2}}\\ \frac{1}{\sqrt{2}}&\frac{1}{\sqrt{2}}\end{bmatrix}
  \begin{bmatrix}\sqrt{2}&2\sqrt{2}\\0&\sqrt{2}\end{bmatrix}
\end{equation*}
\textbf{习题 \ref{6.29} 解答:}\\
令$\vec{\alpha}_1=\begin{bmatrix}1\\0\\1\end{bmatrix}$,$\vec{\alpha}_2=\begin{bmatrix}1\\1\\1\end{bmatrix}$,
$\vec{\alpha}_3=\begin{bmatrix}1\\2\\1\end{bmatrix}$,。
由Schimidt正交化,有\\
\begin{align*}
\vec{\beta}_1=&\vec{\alpha}_1=\begin{bmatrix}1\\0\\1\end{bmatrix}\\
\vec{\beta}_2=&\vec{\alpha}_2-\frac{(\vec{\alpha}_2,\vec{\beta}_1)}{(\vec{\beta}_1,\vec{\beta}_1)}\vec{\beta}_1
             =\begin{bmatrix}1\\1\\1\end{bmatrix}-\frac{2}{2}\begin{bmatrix}1\\0\\1\end{bmatrix}
             =\begin{bmatrix}0\\1\\0\end{bmatrix}\\
\vec{\beta}_3=&\vec{\alpha}_3-\frac{(\vec{\alpha}_3,\vec{\beta}_1)}{(\vec{\beta}_1,\vec{\beta}_1)}\vec{\beta}_1-
              \frac{(\vec{\alpha}_3,\vec{\beta}_2)}{(\vec{\beta}_2,\vec{\beta}_2)}\vec{\beta}_2\\
             =&\begin{bmatrix}1\\2\\3\end{bmatrix}-\frac{4}{2}\begin{bmatrix}1\\0\\1\end{bmatrix}-
               \frac{2}{1}\begin{bmatrix}0\\1\\0\end{bmatrix}\\
             =\begin{bmatrix}-1\\0\\1\end{bmatrix}
\end{align*}
单位化得到:
\begin{equation*}
\vec{\gamma}_1=\frac{1}{\sqrt{2}}\begin{bmatrix}1\\0\\1\end{bmatrix},
\vec{\gamma}_2=\begin{bmatrix}0\\1\\0\end{bmatrix},
\vec{\gamma}_3=\frac{1}{\sqrt{2}}\begin{bmatrix}-1\\0\\1\end{bmatrix}.
\end{equation*}
那么
\begin{equation*}
\begin{bmatrix}\vec{\alpha}_1&\vec{\alpha}_2&\vec{\alpha}_3\end{bmatrix}=
\begin{bmatrix}\vec{\beta}_1&\vec{\beta}_2&\vec{\beta}_3\end{bmatrix}\begin{bmatrix}1&1&2\\0&1&2\\0&0&1\end{bmatrix}
=\begin{bmatrix}\vec{\gamma}_1&\vec{\gamma}_2\end{bmatrix}\begin{bmatrix}\sqrt{2}&0&0\\0&1&0\\0&0\sqrt{2}&0\end{bmatrix}
\begin{bmatrix}1&1&2\\0&1&2\\0&0&1\end{bmatrix}
\end{equation*}
即
\begin{equation*}
A=\begin{bmatrix}1&0&1\\0&1&0\\-1&0&1\end{bmatrix}
  \begin{bmatrix}\sqrt{2}&\sqrt{2}&2\sqrt{2}\\0&1&2\\0&0&\sqrt{2}\end{bmatrix}
\end{equation*}
\textbf{习题 \ref{6.30} 解答:}~~ 4阶\\
\textbf{习题 \ref{6.31} 解答:} \\
(1)使得误差平方和最小的二次函数系数$(a_0,b_0)^T$,就是如下线性方程组的最小二乘解:
\begin{equation*}
  \begin{cases}
  a+b=5.1\\
  2a+b=8.2\\
  3a+b=10\\
  4a+b=14.8
  \end{cases}
\end{equation*}
令$A=\begin{bmatrix}1&1\\2&1\\3&1\\4&1\end{bmatrix}$ 以及$\vec{b}=\begin{bmatrix}5.1\\8.2\\10\\14.8\end{bmatrix}$,
由于$A$是列满秩的,分别计算得到:
\begin{equation*}
  A^TA=\begin{bmatrix}30&10\\10&4\end{bmatrix},A^T\vec{b}=\begin{bmatrix}110.7\\38.1\end{bmatrix}
\end{equation*}
则
\begin{equation*}
  (A^TA)^{-1}=\begin{bmatrix}\frac{1}{5}&-\frac{1}{2}\\-\frac{1}{2}&\frac{3}{2}\end{bmatrix}
\end{equation*}
于是
\begin{equation*}
 \begin{bmatrix}a_0\\b_0\end{bmatrix}=(A^TA)^{-1}A^T\vec{b}=\begin{bmatrix}3.09\\-1.8\end{bmatrix}
\end{equation*}
所以误差平方和最小的一次函数为:
\begin{equation*}
 y=3.09x-1.8
\end{equation*}
(2)同上问类似,可以求得误差平方和最小的二次函数为:
\begin{equation*}
 y=0.425x^2+0.965x+3.925.
\end{equation*}

%%%%%%%%%%%%%%%%%%%%%%%%%%%%%%%%%%%%%%%%%%%%%%%%%%%%%%%%%%%%%%%%%%%%%%%%%%%%%%%%%%%%%%
%%%%%%%%%%%%%%%%%%%%%%%%%%%%%%%%%%%%%%%%%%%%%%%%%%%%%%%%%%%%%%%%%%%%%%%%%%%%%%%%%%%%%%

\chapter{矩阵的特征值理论}

\section{知识点解析}

\begin{Def}
设$A$为$n$阶方阵,若存在数$\lambda$以及$n$维非零(列)向量$\vec{x}$,使得
$$
A\vec{x}=\lambda\vec{x}
$$
则称$\lambda$是$A$的特征值,$\vec{x}$是$A$的属于特征值$\lambda$的特征向量。
\end{Def}

\begin{Def}
下述关于$\lambda$的$n$次多项式称为$n$次方阵$A$的特征多项式
\begin{equation*}
f_A(\lambda)=|\lambda I-A|
=\begin{vmatrix}\lambda-a_{11}&-a_{12}&\cdots&-a_{n1}\\
-a_{21}&\lambda-a_{22}&\cdots&-a_{2n}\\
\cdots&\cdots&\cdots&\cdots\\
-a_{n1}&-a_{n2}&\cdots&\lambda-a_{nn}\end{vmatrix}
\end{equation*}
\end{Def}

\begin{thm}
设$A=(a_{ij})_{n\times n}$为$n$阶方阵,又设$\lambda_1,\lambda_2,\cdots,\lambda_n$为$A$的全部(复)特征值,则
\begin{enumerate}
  \item $\sum_{i=1}^n\lambda_i=\sum_{i=1}^na_{ii}=trA$;
  \item $|A|=\lambda_1\cdots\lambda_n=\Pi_{i=1}^n\lambda_i$
\end{enumerate}
\end{thm}

\begin{cor}
$n$阶方阵$A$不可逆$\Leftrightarrow$$|A|\neq 0$$\Leftrightarrow$$A$的特征值全不为零;
\end{cor}

\begin{cor}
设$\lambda_0$为$A$的特征值,$\vec{x}_0$为对应的特征向量,则
\begin{enumerate}
  \item $k\lambda_0$为$kA$的特征值,$\vec{x}_0$为对应的特征向量;
  \item $\lambda_0^k$为$A^k$的特征值,$\vec{x}_0$为对应的特征向量;
  \item $g(\lambda_0)$为$g(A)$的特征值,$\vec{x}_0$为对应的特征向量;
  \item 进一步,若$A$可逆,$\lambda^{-1}_0$为$A^{-1}$的特征值,$\vec{x}$为对应的特征向量,$\lambda_0|A|$为$A^{*}$的特征值,$\vec{x}$为对应的特征向量。
\end{enumerate}
\end{cor}

\begin{Def}
设$A$为$n$阶方阵,$\lambda$是$A$的一个特征值,则称
$$V_{\lambda}:=\{\vec{x}\in\mathbb{R}^n|A\vec{x}=\lambda\vec{x}\}$$
为$A$的关于特征值$\lambda$的特征子空间,其维数为特征值$\lambda$的几何重数。
\end{Def}

\begin{thm}
属于$A$的不同特征值的特征向量是线性无关的。
\end{thm}

\begin{Def}
设$A,B$是两个$n$阶方阵,如果存在一个$n$阶可逆矩阵$P$,使得$P^{-1}AP=B$,
则称方阵$B$相似于方阵$A$,记作$A\sim B$.
\end{Def}

\begin{cor}
相似矩阵的基本性质:
\begin{enumerate}
  \item 若$A\sim B$,$A_1\sim B_1$,则$kA\sim kB$,$A^m\sim B^m$,$A+A_1\sim B+B_1$,进而有$g(A)\sim g(B)$,对任意多项式$g(x)$成立。
  \item 相似矩阵有相同的秩,即若$A\sim B$,则$r(A)=r(B)$;进而,相似的矩阵有
  相同的可逆性,且若$A\sim B$,则$A^{-1}\sim B^{-1}$;
  \item 相似矩阵有相同的特征多项式;
  \item 相似矩阵具有相同的特征值;
  \item 相似矩阵具有相同的行列式与迹;
  \item 设$P^{-1}AP=B$,且$\lambda_0$为$A$的特征值,$\vec{x}$为相对应的特征向量,则$P^{-1}\vec{x}$为$B$的属于特征值$\lambda_0$的特征向量。
\end{enumerate}
\end{cor}

\begin{thm}
$n$阶方阵$A$的可对角化的充分必要条件是$A$有$n$个线性无关的特征向量。
\end{thm}

\begin{cor}
若$n$阶方阵$A$有$n$个互异的特征值,则$A$可对角化。
\end{cor}

\begin{cor}
若$n$阶方阵$A$可对角化,即存在可逆阵$P$与对角阵$\Lambda$,满足$P^{-1}AP=\Lambda$,则$\Lambda$对角线上的$n$个元即为$A$的$n$个特征值$\lambda_i(1\leq i\leq n)$;而可逆阵$P$的$n$个列向量$\vec{x}_i$即为属于
$\lambda_i(1\leq i\leq n)$的特征向量。
\end{cor}

\begin{Def}
设复数$z=a+bi,(a,b\in\mathbb{R}^n,i=\sqrt{-1})$,则$z$的复共轭运算记为$\bar{z}$,定义为:$\bar{z}:=a-bi$
\end{Def}

\begin{Def}
若$A=(a_{ij})_{m\times n}$,其中$a_{ij}\in\mathbb{C}$,则称$A$为$m\times n$ 型的复矩阵,复矩阵的共轭定义为:$\bar{A}=(\bar{a}_{ij})_{m\times n}$.
\end{Def}

\begin{thm}
设$A$是$n$阶实对称矩阵,则$A$的特征值都是实数。
\end{thm}

\begin{thm}
实对称矩阵$A$的属于不同特征值的特征向量是正交的。
\end{thm}

\begin{cor}
设$\lambda_i,\lambda_j$为实对称矩阵$A$的两个互异的特征值,则$V_{\lambda_i}\bot V_{\lambda_j}$。
\end{cor}

\begin{thm}
对$n$阶实对称阵$A$,总存在正交阵$Q$,使得$Q^{-1}AQ=Q^TAQ$为对角阵。
\end{thm}

\begin{cor}
$n$阶实对称阵必有$n$个线性无关的特征向量,且对$A$的全体互异的特征值
$\lambda_1,\lambda_2,\cdots,\lambda_s$,
有$\sum_{i=1}^s m_i=\sum_{i=1}^s dimV_{\lambda_i}=n$
\end{cor}

\begin{cor}
$n$阶实对称阵有$n$个相互正交的单位特征向量。
\end{cor}

%%%%%%%%%%%%%%%%%%%%%%%%%%%%%%%%%%%%%%%%%%%%%%%%%%%%%%%%%%%%%%%%%%%%%%%%%%%%%%%%%%%

\section{例题讲解}
\begin{eg}
求$A=\begin{bmatrix}1&-2\\1&4\end{bmatrix}$的所有特征值和特征向量。
\end{eg}
解:设$A$的特征值是$\lambda$,属于特征值$\lambda$的特征向量为$\vec{x}$,则$A\vec{x}=\lambda\vec{x}$,即
\begin{equation*}
\begin{bmatrix}1&-2\\1&4\end{bmatrix}\begin{bmatrix}x_1\\x_2\end{bmatrix}=
\lambda\begin{bmatrix}x_1\\x_2\end{bmatrix}
\end{equation*}
亦即
\begin{equation*}
\begin{cases}
(\lambda-1)x_1+2x_2=0\\
-x_1+(\lambda-4)x_2=0
\end{cases}
\end{equation*}
这是两个未知数两个方程的其次线性方程组,按定义,特征向量是非零向量,于是要求
上述齐次线性方程组有非零解,这等价于
\begin{equation*}
\begin{vmatrix}
\lambda-1&2\\-1&\lambda-4
\end{vmatrix}
=(\lambda-1)(\lambda-4)+2=(\lambda-2)(\lambda-3)=0
\end{equation*}
解得$\lambda_1=2,\lambda_2=3$,这就是$A$的两个特征值。\\
下面,分别求解属于特征值$\lambda_1=2$与$\lambda_2=3$的特征向量。\\
对于特征值2,代入得到齐次线性方程组:
$\begin{cases}x_1+2x_2=0\\-x_1-2x_2=0\end{cases}$
解得属于2的特征向量是$\begin{bmatrix}x_1\\x_2\end{bmatrix}=k\begin{bmatrix}2\\-1\end{bmatrix}$,
其中$k$是任意非零常数。\\
对于特征值3,代入得到齐次线性方程组:
$\begin{cases}2x_1+2x_2=0\\-x_1-x_2=0\end{cases}$
解得属于3的特征向量是$\begin{bmatrix}x_1\\x_2\end{bmatrix}=k\begin{bmatrix}1\\-1\end{bmatrix}$,
其中$k$是任意非零常数。

\begin{eg}
考虑对角阵$A=\begin{bmatrix}a_1&&&\\&a_2&&\\&&\ddots&\\&&&a_n\end{bmatrix}$,
其中$a_1,a_2,\cdots,a_n$为互不相同的实数,求$A$的特征值与特征向量。
\end{eg}
解:计算$A$的特征多项式:$|\lambda I-A|=(\lambda-a_1)(\lambda-a_2)\cdots(\lambda-a_n)$,显然得到全部特征值为
\begin{equation*}
\lambda_1=a_1,\lambda_2=a_2,\cdots,\lambda_n=a_n.
\end{equation*}
对每一个特征值$\lambda_i=a_i$,计算属于它的特征向量满足的线性方程组
$(a_iI-A)\vec{x}=\vec{0}$,即
\begin{equation*}
\begin{bmatrix}
a_i-a_1&&&&\\
&\ddots&&&\\
&&0&&\\
&&&\ddots&\\
&&&&a_i-a_n
\end{bmatrix}
\vec{x}=\vec{0}
\end{equation*}
由于$a_i\neq a_j(i\neq j)$,故上述方程组的稀疏矩阵主对角线上只有一个位置为0,
故它的基础解系为$\vec{x}_i=\vec{e}_i$
故属于特征值$\lambda=a_i$的特征向量为$k_i\vec{e}_i(k_i\neq 0)$,对所有的$i=1,2,\cdots,n$.


\begin{eg}
设$A$是三阶方阵,它的特征值为1,2,-1,$B=A^3-5A^2$,求$|B|$.
\end{eg}
解:设$A\vec{x}=\lambda\vec{x},\vec{x}\neq\vec{0}$,则$$B\vec{x}=A^3\vec{x}-5A^2\vec{x}=(\lambda^3-5\lambda^2)\vec{x}$$
取$\lambda=1,2,-1$,则$\lambda^3-5\lambda^2=-4,-12,-6$.
所以$B$的特征值为-4,-12,-6.\\
故$|B|=(-4)(-12)(-6)=-288$.

\begin{eg}
已知$n$阶方阵$A$的$n$个特征值为$\lambda_1,\cdots,\lambda_n$,求$2I-A$的特征值
以及$det(2I-A)$。
\end{eg}
解:由题设知$|\lambda I-A|=\Pi_{i=1}^n(\lambda-\lambda_i)$。下考虑$2I-A$的特征多项式
\begin{equation*}
|\mu I-(2I-A)|=|(\mu -2)I + A|=\Pi_{i=1}^n(\mu-(2-\lambda_i))
\end{equation*}
所以$2I-A$的$n$个特征值为$2-\lambda_i,i=1,2,\cdots,n$,从而
$$det(2I-A)=\Pi_{i=1}^n(2-\lambda_i)$$.

\begin{eg}
已知$A^2=A$,求矩阵$A$的特征值。
\end{eg}
解:设$\lambda$是$A$任意一个特征值,$\vec{x}$是$\lambda$所属的特征向量,则$A\vec{x}=\lambda\vec{x}$,所以$A^2\vec{x}=A(\lambda\vec{x})=\lambda A\vec{x}=\lambda^2\vec{x}$,
利用已知条件$A^2=A$与上式,有
$$\lambda\vec{x}=A\vec{x}=A^2\vec{x}=\lambda^2\vec{x}$$,
所以$(\lambda^2-\lambda)\vec{x}=\vec{0}$,因为$\vec{x}$为特征向量,所以$\vec{x}\neq\vec{0}$,故
$$\lambda^2=\lambda,\Rightarrow\lambda =0 \text{或}\lambda=1$$.

\begin{eg}
设$A=\begin{bmatrix}0.95&0.15\\0.05&0.85\end{bmatrix},
\Lambda=\begin{bmatrix}1&0\\0&0.8\end{bmatrix},
P=\begin{bmatrix}3&-1\\1&1\end{bmatrix}.$
验证$P^{-1}AP=\Lambda$,并且求$A^k$($k$为正整数)。
\end{eg}
解:由于
\begin{equation*}
AP=\begin{bmatrix}0.95&0.15\\0.05&0.85\end{bmatrix}
\begin{bmatrix}3&-1\\1&1\end{bmatrix}=
\begin{bmatrix}3&-0.8\\1&0.8\end{bmatrix}=
\begin{bmatrix}3&-1\\1&1\end{bmatrix}
\begin{bmatrix}1&0\\0&0.8\end{bmatrix}
=P\Lambda
\end{equation*}
\begin{equation*}
A^k=(P\Lambda P^{-1})^k=P\Lambda^kP^{-1}=\frac{1}{4}
\begin{bmatrix}3+0.8^k&3-3\cdot0.8^k\\1-0.8^k&1+3\cdot0.8^k\end{bmatrix}
\end{equation*}

\begin{eg}
判断下列矩阵能否对角化,如能对角化,求出可逆阵$P$与对角阵$\Lambda$,使得
$P^{-1}AP=\Lambda$.\\
(1)$A=\begin{bmatrix}2&1&1\\1&2&1\\1&1&2\end{bmatrix}$;
(2)$A=\begin{bmatrix}2&2&-1\\-1&0&1\\-1&0&2\end{bmatrix}$。
\end{eg}
解:(1)先求$A$的特征值,特征多项式为
\begin{equation*}
f_A(\lambda)=
\begin{vmatrix}\lambda-2&-1&-1\\-1&\lambda-2&-1\\-1&-1&\lambda-2\end{vmatrix}
=(\lambda-1)^2(\lambda-1)
\end{equation*}
所以得到$A$的特征值为:$\lambda-1=1$(重数为2);$\lambda-2=4$。
再求$A$的特征向量。\\
对于$\lambda_1=1$,解齐次线性方程组$(\lambda_1 I-A)\vec{x}=\vec{0}$,
得到属于特征值$\lambda_1=1$的两个线性无关的特征向量为$\vec{x}_{11}=(-1,1,0)^T,\vec{x}_{12}=(-1,0,1)^T$.\\
对于$\lambda_2=4$,解齐次线性方程组$(\lambda_2 I-A)\vec{x}=\vec{0}$,
得到属于特征值$\lambda_2=4$的一个线性无关的特征向量为$\vec{x}_{21}=(1,1,1)^T$.\\
因此,$A$存在3个线性无关的特征向量$\vec{x}_{11},\vec{x}_{12},\vec{x}_{21}$,
故$A$可对角化,且
\begin{equation*}
P=(\vec{x}_{11},\vec{x}_{12},\vec{x}_{21})=\begin{bmatrix}-1&-1&1\\1&0&1\\0&1&1
\end{bmatrix}
\end{equation*}
\begin{equation*}
\Lambda=P^{-1}AP=\begin{bmatrix}1&0&0\\0&1&0\\0&0&4\end{bmatrix}
\end{equation*}
(2)先求$A$的特征值,特征多项式为
\begin{equation*}
f_A(\lambda)=
\begin{vmatrix}\lambda-2&-2&1\\1&\lambda&-1\\1&0&\lambda-2\end{vmatrix}
=(\lambda-1)^2(\lambda-2)
\end{equation*}
所以得到$A$的特征值为:$\lambda_1=1$(重数为2);$\lambda_2=2$。
再求$A$的特征向量。\\
对于$\lambda_1=1$,解齐次线性方程组$(\lambda_1 I-A)\vec{x}=\vec{0}$,
得到属于特征值$\lambda_1=1$的一个线性无关的特征向量为$\vec{x}_{11}=(1,0,1)^T$.\\
对于$\lambda_2=2$,解齐次线性方程组$(\lambda_2 I-A)\vec{x}=\vec{0}$,
得到属于特征值$\lambda_2=2$的一个线性无关的特征向量为$\vec{x}_{21}=(0,1,2)^T$.\\
因此,$A$只存在2个线性无关的特征向量$\vec{x}_{11},\vec{x}_{21}$,因此$A$不可对角化。

\begin{eg}
(1)若$A$为$n$阶方阵,且$A$可对角化,则$A^k$($k$为正整数),$g(A)$($g$为实系数多项式)均可以对角化。\\
(2)若$A$可逆,则$A^{-1},A^{*}$也可以对角化。
\end{eg}
证明:(1)由于$A$可对角化,则存在可逆阵$P$与对角阵$\Lambda$,使得
$$P^{-1}AP=\Lambda=diag(\lambda_1,\lambda_2,\cdots,\lambda_n)$$,
故$A^k=Pdiag(\lambda_1^k,\lambda_2^k,\cdots,\lambda_n^k)P^{-1}$ $\Rightarrow$ $A^k$可对角化。
再由矩阵乘法的分配律有:
$$g(A) = P diag(g(\lambda_1),g(\lambda_2),\cdots,g(\lambda_n)) P^{-1}$$
$\Rightarrow$ $g(A)$可对角化。\\
(2)若$A$可逆,则$\lambda_i$均不为0,两边求逆可得:
$$P^{-1}A^{-1}P=diag(\lambda_1^{-1},\lambda_2^{-1},\cdots,\lambda_n^{-1})$$,
故$A^{-1}$可对角化。\\
由$A^*=|A|A^{-1}$得:
$$P^{-1}A^{*}P=|A|diag(\lambda_1^{-1},\lambda_2^{-1},\cdots,\lambda_n^{-1})$$,
故$A^*$可对角化。

\begin{eg}
设$A$为$n$阶方阵,满足$A^2=I$,证明:$A$可对角化。
\end{eg}
证明:设$\lambda$是$A$的任一特征值,$\vec{x}$是$\lambda$所属的特征向量,则
$$A^2\vec{x}=\lambda^2\vec{x},\Rightarrow(\lambda^2-1)\vec{x}=\vec{0}
(\vec{x}\neq\vec{0}),\Rightarrow\lambda=\pm 1$$
设$\lambda_1=1,\lambda_2=-1$,且$m_1,m_2$为对应的几何重数,则
$$m_1+m_2=(n-r(A-I))+(n-r(A+I))=2n-[r(I-A)+r(I+A)]$$,
由矩阵秩的相关结论知
$$n=r(2I)=r[(I-A)+(I+A)]\leq r(I-A)+r(I+A)\leq n$$
其中第二个小于等于号是因为$(I-A)(I+A)=O$。故
$$r(A-I)+r(A+I)=n,\Rightarrow m_1+m_2=n$$
从而$A$可对角化。

\begin{eg}
设$A=\begin{bmatrix}0&1&1&-1\\1&0&-1&1\\1&-1&0&1\\-1&1&1&0\end{bmatrix}$,将其对角化。
\end{eg}
解:首先求特征值与代数重数
\begin{equation*}
|\lambda I-A|=\begin{vmatrix}\lambda&-1&-1&1\\-1&\lambda&1&-1\\
-1&1&\lambda&-1\\1&-1&-1&\lambda\end{vmatrix}=(\lambda-1)^2(\lambda+3)
\end{equation*}
得到$\lambda_1=1,n_1=3,\lambda_2=-3,n_2=1$.\\
其次,对特征值$\lambda_1=1$,解齐次线性方程组$(\lambda_1 I-A)\vec{x}=\vec{0}$,
求特征向量$\vec{x}_{11}=(1,1,0,0)^T,\vec{x}_{12}=(1,0,1,0)^T,\vec{x}_{13}=(-1,0,0,1)^T$;\\
再次,对特征值$\lambda_2=-3$,解齐次线性方程组$(\lambda_2 I-A)\vec{x}=\vec{0}$,
求特征向量$\vec{x}_{21}=(1,-1,-1,1)^T$;\\
于是,令$P=\begin{bmatrix}1&1&-1&1\\1&0&0&-1\\0&1&0&-1\\0&0&1&1\end{bmatrix}$
从而$\Lambda=P^{-1}AP=
\begin{bmatrix}1&0&0&0\\0&1&0&0\\0&0&1&0\\0&0&0&-3\end{bmatrix}$

\begin{eg}
已知$A$为3阶实对称矩阵,其特征值为1,1,2,且属于2的特征向量是$(1,0,1)^T$,求$A$.
\end{eg}
解:$A$是3阶实对称矩阵,正交相似于对角阵$\Lambda=diag(1,1,2)$,属于特征值1的特征向量与属于特征值2的特征向量$(1,0,1)^T$正交,这等价于求解齐次线性方程组:
$$x_1+x_3=0$$
求解得到属于1的特征向量为:$(0,1,0)^T,(1,0,-1)^T$,且彼此正交,故得到相应的
正交矩阵为
$$Q=\begin{bmatrix}0&\frac{1}{\sqrt{2}}&\frac{1}{\sqrt{2}}\\1&0&0\\
0&-\frac{1}{\sqrt{2}}&\frac{1}{\sqrt{2}}\end{bmatrix}$$
故
$$A=Q\Lambda Q^T=\begin{bmatrix}\frac{3}{2}&0&\frac{1}{2}\\0&1&0\\
\frac{1}{2}&0&\frac{3}{2}\end{bmatrix}$$

\begin{eg}
经过统计,某地区猫头鹰和森林鼠的数量有如下的规律:每个月只有一半的猫头鹰可以存活;森林鼠的数量每个月会增加$10\%$。如果森林鼠充足(数量为$Q$),则下个月猫头鹰的数量将会增加$0.4Q$。平均每个月每只猫头鹰的捕食会导致104森林鼠死亡,试确定该系统的长期演变情况。
\end{eg}
解:设猫头鹰和森林鼠在时刻$k$的数量为$\vec{x}=\begin{bmatrix}P_k\\Q_k\end{bmatrix}$,其中$k$是以月份为单位的时间,$P_k$为研究区域中猫头鹰的数量(只),$Q_k$为研究区域中森林鼠的数量(千只),则
\begin{equation*}
\begin{cases}
P_{k+1}=0.5P_{k}+0.4Q_k\\
Q_{k+1}=-0.104P_k+1.1Q_k
\end{cases}
\Rightarrow
\begin{bmatrix}P_{k+1}\\Q_{k+1}\end{bmatrix}=
\begin{bmatrix}0.5&0.4\\-0.104&1.1\end{bmatrix}
\begin{bmatrix}P_k\\Q_k\end{bmatrix}
\end{equation*}
令$A=\begin{bmatrix}0.5&0.4\\-0.104&1.1\end{bmatrix}$,经过计算得,矩阵$A$的两个特征值为$\lambda_1=1.02,\lambda_2=0.58$,对应的特征向量分别为:
$\vec{x}_1=(10,13)^T,\vec{x}_2=(5,1)^T$,设初始向量$\vec{x}_0$可以表示为:
$\vec{x}_0=c_1\vec{x}_1+c_2\vec{x}_2$,于是对于$k\geq0$,有:
\begin{align*}
\vec{x}_k=&A^k(c_1\vec{x}_1+c_2\vec{x}_2)=c_1\lambda_1^k\vec{x}_1+c_2\lambda_2^k\vec{x}_2\\
=&c_1(1.02)^k\begin{bmatrix}10\\13\end{bmatrix}+c_2(0.58)^k\begin{bmatrix}5\\1\end{bmatrix}
\end{align*}

%%%%%%%%%%%%%%%%%%%%%%%%%%%%%%%%%%%%%%%%%%%%%%%%%%%%%%%%%%%%%%%%%%%%%%%%%%%%%%%%%%

\section{课后习题}

\begin{ex}\label{7.1}
求下列矩阵的特征值与特征向量。\\
(1)$\begin{bmatrix}1&2\\0&3\end{bmatrix}$;
(2)$\begin{bmatrix}1&3\\2&2\end{bmatrix}$;
(3)$\begin{bmatrix}1&1&1\\0&2&1\\0&0&3\end{bmatrix}$;\\
(4)$\begin{bmatrix}1&2&3\\2&1&3\\3&3&6\end{bmatrix}$;
(5)$\begin{bmatrix}1&-\sqrt{3}\\ \sqrt{3}&1\end{bmatrix}$;
\end{ex}

\begin{ex}\label{7.2}
已知$A=\begin{bmatrix}a&1&b\\0&c&0\\-4&c&1-a\end{bmatrix}$ 有一个特征值是1. 对应的特征向量是$\vec{\alpha}=(1,1,1)^T$,求$a,b,c$的值。
\end{ex}

\begin{ex}\label{7.3}
已知$A=\begin{bmatrix}1&1&2\\5&a&0\\1&b&2\end{bmatrix}$ 有一个特征向量是$\vec{\alpha}=(1,1,-1)^T$,求$a,b$的值以及$\vec{\alpha}$对应的特征值。
\end{ex}

\begin{ex}\label{7.4}
设三阶方阵$A$的特征值分别为$\lambda_1=-1,\lambda_2=1,\lambda_3=3$,对应的特征向量依次是:
\begin{equation*}
\vec{\alpha}_1=\begin{bmatrix}1\\0\\-1\end{bmatrix},
\vec{\alpha}_2=\begin{bmatrix}1\\2\\1\end{bmatrix},
\vec{\alpha}_3=\begin{bmatrix}0\\1\\-1\end{bmatrix},
\end{equation*}
求矩阵$A$。
\end{ex}

\begin{ex}\label{7.5}
求$\begin{bmatrix}2&1&1\\0&3&1\\0&0&5\end{bmatrix}^{n}$。
\end{ex}

\begin{ex}\label{7.6}
假设$\lambda_1$和$\lambda_2$是矩阵$A$ 的特征值,且$\lambda_1\neq\lambda_2$,
$\vec{\xi}_1$和$\vec{\xi}_2$分别是$\lambda_1$ 以及$\lambda_2$所对应的特征向量。\\
(1)证明$\vec{\xi}_1,\vec{\xi}_2$ 线性无关。\\
(2)证明$\vec{\xi}_1-\vec{\xi}_2$ 不是$A$ 的特征向量。
\end{ex}

\begin{ex}\label{7.7}
设$f(x)=a_0+a_1x+\cdots+a_nx^n$,$A$是$n$ 阶方阵且$f(A)=0$,若$A$没有零特征值,
证明$a_0\neq0$。
\end{ex}

\begin{ex}\label{7.8}
设$E$是$n$阶单位矩阵,如果$n$ 阶矩阵$A$ 满足$|A+E|=|A-2E|=|4E-A|=0$,求$|A^3-5A^2|$ 的值。
\end{ex}

\begin{ex}\label{7.9}
已知3阶矩阵$A$的特征值为$1,2,-1$,求$|A^*+3A|$的值。
\end{ex}

\begin{ex}\label{7.10}
已知$A$是三阶实矩阵,$tr(A)=5$且$|A|=4$,设$A$的特征值为$\lambda_1,\lambda_2,\lambda_2$,
且$\lambda_1=2$,求$\lambda_2,\lambda_2$ 的值。
\end{ex}

\begin{ex}\label{7.11}
已知$A$是$n$阶方阵,假设$\vec{\alpha}_1,\vec{\alpha}_2,\vec{\alpha}_3$ 是$A$ 的三个属于
不同特征值$\lambda_1,\lambda_2,\lambda_3$ 的特征向量,
令$\vec{\beta}=\vec{\alpha}_1+\vec{\alpha}_2+\vec{\alpha}_3$,证明$\vec{\beta},A\vec{\beta},A^2\vec{\beta}$ 线性无关。
\end{ex}

\begin{ex}\label{7.12}
已知矩阵$A=\begin{bmatrix}2&0&0\\0&1&-1\\0&-1&1\end{bmatrix}$,求$A$ 的每个特征子空间的一组基。
\end{ex}

\begin{ex}\label{7.13}
判断$A=\begin{bmatrix}1&1&0\\0&2&1\\0&0&5\end{bmatrix}$
与$B=\begin{bmatrix}1&0&0\\0&2&0\\0&0&5\end{bmatrix}$ 是否相似,并说明理由。
\end{ex}

\begin{ex}\label{7.14}
设$A\sim B$,其中
\begin{equation*}
A=\begin{bmatrix}1&-1&1\\x&4&-2\\-3&-3&5\end{bmatrix},
B=\begin{bmatrix}3&\sqrt{2}&1\\ \sqrt{2}&4&\sqrt{2}\\1&\sqrt{2}&y\end{bmatrix},
\end{equation*}
求$x$和$y$的值。
\end{ex}

\begin{ex}\label{7.15}
设$A$和$B$都是$n$阶实矩阵,且$B$可逆,证明$AB$和$BA$是相似的。
\end{ex}

\begin{ex}\label{7.16}
设
\begin{equation*}
  A=\begin{bmatrix}1&2&-2\\0&0&0\\0&0&0\end{bmatrix}
\end{equation*}
试问$A$能否对角化?如果可以对角化,求出使$A$ 相似于对角矩阵的相似变换矩阵$P$,同时写出这个对角阵。
\end{ex}

\begin{ex}\label{7.17}
设$\vec{\alpha}=(a_1,a_2,\cdots,a_n)^T$,$\vec{\beta}=(b_1,b_2,\cdots,b_n)^T$且$\vec{\alpha}^T\vec{\beta}=1$,
$E$是$n$阶单位矩阵,令$A=E+\vec{\alpha}\vec{\beta}^T$。\\
(1)求$A$的特征值和特征向量。\\
(2)讨论$A$是否可以对角化;若可以,求可逆阵$P$ 以及对角阵$\Lambda$,使$P^{-1}AP=\Lambda$。若不行,说明理由。
\end{ex}

\begin{ex}\label{7.18}
设$A=\begin{bmatrix}3&-1&1\\0&x&y\\0&z&w\end{bmatrix}$,已知$A$有3线性无关的特征向量,
$\lambda=1$是$A$的二重特征值。\\
(1)求$x,y,z,w$的值。\\
(2)将$A$对角化,求可逆阵$P$以及对角阵$\Lambda$,使$P^{-1}AP=\Lambda$。
\end{ex}

\begin{ex}\label{7.19}
已知二阶实矩阵$A=\begin{bmatrix}a&c\\b&d\end{bmatrix}$,若$bc>0$,证明$A$可以对角化。
\end{ex}

\begin{ex}\label{7.20}
判断$A=\begin{bmatrix}1&-3&3\\3&-5&3\\6&-6&4\end{bmatrix}$ 与
是否可以对角化,若可以,请求出可逆阵$P$ 以及对角阵$\Lambda$,若不可以,请说明理由。
\end{ex}

\begin{ex}\label{7.21}
已知$A$是实对称矩阵,满足$A^2+3A=O$ 且$r(A)=2$,其中$O$表示零矩阵,求$A$的全部特征值。
\end{ex}

\begin{ex}\label{7.22}
设$A$是3阶实对称矩阵,其特征值为$\lambda_1=1,\lambda_2=2,\lambda_3=-3$,$\vec{\alpha}_1=(1,1,1)^T$是
$A$属于$\lambda_2$的一个特征向量,记$B=A^2+2A$,求$A$的所有特征值以及特征向量。
\end{ex}

\begin{ex}\label{7.23}
已知实对称矩阵$A=\begin{bmatrix}1&-2&0\\-2&2&-2\\0&-2&3\end{bmatrix}$,将其正交对角化。
\end{ex}

\begin{ex}\label{7.24}
已知实对称矩阵$A=\begin{bmatrix}1&-2&2\\-2&-2&4\\2&4&-2\end{bmatrix}$,将其正交对角化。
\end{ex}

\begin{ex}\label{7.25}
设$A$是3阶实矩阵,且$A$有3个正交的特征向量,证明$A$是对称矩阵。
\end{ex}

\begin{ex}\label{7.26}
设$A$是$n$阶实对称矩阵,$\lambda_1,\lambda_2,\cdots,\lambda_n$ 是$A$ 的特征值,
相应的标准正交特征向量为$\vec{\alpha}_1,\vec{\alpha}_2,\cdots,\vec{\alpha}_n$,证明
\begin{equation*}
  A = \lambda_1\vec{\alpha}_1\vec{\alpha}_1^T+\cdots+\lambda_n\vec{\alpha}_n\vec{\alpha}_n^T
\end{equation*}
\end{ex}

\begin{ex}\label{7.27}
设$A$是$n$阶实对称矩阵,且存在正整数$k$,使得$A^{k}=O$,其中$O$是$n$ 阶零矩阵,则$A=O$。
\end{ex}

\begin{ex}\label{7.28}
在某省,每年有比例$p$的农村居民移居城镇,有比例为$q$的城镇居民移居到农村,假设该省的人口总数不变,
且上述人口迁移的规律不变,将$n$ 年后农村人口和城镇人口占总人口的比例分别记为$x_n$ 和$y_n$ ($x_n+y_n=1$)。\\
(1)求关系式$\begin{bmatrix}x_{n+1}\\y_{n+1}\end{bmatrix}=A\begin{bmatrix}x_n\\y_n\end{bmatrix}$ 中的矩阵$A$;\\
(2)设目前农村人口和城镇人口相等,即$\begin{bmatrix}x_0\\y_0\end{bmatrix}=\begin{bmatrix}0.5\\0.5\end{bmatrix}$,
求$\begin{bmatrix}x_{n}\\y_{n}\end{bmatrix}$。
\end{ex}

%%%%%%%%%%%%%%%%%%%%%%%%%%%%%%%%%%%%%%%%%%%%%%%%%%%%%%%%%%%%%%%%%%%%%%%%%%%%%%%%%%%%%

\section{习题答案}

\textbf{习题 \ref{7.1} 解答:}\\
(1)考虑$A$的特征多项式:
\begin{align*}
  |A-\lambda E|&=\begin{vmatrix}1-\lambda&2\\0&3-\lambda\end{vmatrix}\\
   & =(1-\lambda)(3-\lambda)
\end{align*}
因此$A$的特征值为$\lambda_1=1,\lambda_2=3$。 对于特征值$\lambda_1=1$,解方程组$(A-E)\vec{x}_1=\vec{0}$,
得到特征向量$\vec{x}_1=k(1,0)^T$。同理,对于特征值$\lambda_2=3$,解方程组$(A-3E)\vec{x}_2=\vec{0}$,
得到特征向量$\vec{x}_2=k(1,-1)^T$。\\
(2)特征值为:$\lambda_1=-1,\lambda_2=4$;\\
对应的特征向量为:$\vec{x}_1=k(-\frac{3}{2},1)^T,\vec{x}_2=k(1,1)^T$。\\
(3)特征值为:$\lambda_1=1,\lambda_2=2,\lambda_3=3$;\\
对应的特征向量为:$\vec{x}_1=k(1,0,0)^T,\vec{x}_2=k(1,1,0)^T,\vec{x}_3=k(1,1,1)^T$。\\
(4)特征值为:$\lambda_1=-1,\lambda_2=0,\lambda_3=9$;\\
对应的特征向量为:$\vec{x}_1=k(-1,1,0)^T,\vec{x}_2=k(-1,-1,0)^T,\vec{x}_3=k(1,1,2)^T$。\\
(5)特征值为:$\lambda_1=1+\sqrt{3}i,\lambda_2=1-\sqrt{3}i$;\\
对应的特征向量为:$\vec{x}_1=k(i,1)^T,\vec{x}_2=k(-i,1)^T$。\\
\textbf{习题 \ref{7.2} 解答:}\\
根据题目,有
\begin{equation*}
  \begin{bmatrix}a&1&b\\0&c&0\\-4&c&1-a\end{bmatrix}
  \begin{bmatrix}1\\1\\1\end{bmatrix}=
  \begin{bmatrix}1\\1\\1\end{bmatrix}
\end{equation*}
解得:
\begin{equation*}
  \begin{cases}
  a = -3\\
  b = 4\\
  c = 1
  \end{cases}
\end{equation*}
\textbf{习题 \ref{7.3} 解答:}\\
设$\vec{\alpha}$所对应的特征值为$\lambda$,则$A\vec{\alpha}=\lambda\vec{\alpha}$。
得到:
\begin{equation*}
  \begin{cases}
  0 = \lambda\\
  5+a = \lambda\\
  1+b-2= -\lambda
  \end{cases}
\end{equation*}
解得
\begin{equation*}
  \begin{cases}
  \lambda=0\\
  a = -5\\
  b = 1
  \end{cases}
\end{equation*}
\textbf{习题 \ref{7.4} 解答:}\\
由于$A\vec{\alpha}_1=\lambda_1\vec{\alpha}_1$,$A\vec{\alpha}_2=\lambda_2\vec{\alpha}_2$,
$A\vec{\alpha}_3=\lambda_3\vec{\alpha}_3$ 可得
\begin{equation*}
A(\vec{\alpha}_1,\vec{\alpha}_2,\vec{\alpha}_3)
=(A\vec{\alpha}_1,A\vec{\alpha}_2,A\vec{\alpha}_3)
=(\lambda_1\vec{\alpha}_1,\lambda_2\vec{\alpha}_2,\lambda_3\vec{\alpha}_3)
\end{equation*}
令$P=(\vec{\alpha}_1,\vec{\alpha}_2,\vec{\alpha}_3)$,则有
\begin{equation*}
AP=P\begin{bmatrix}\lambda_1&0&0\\0&\lambda_2&0\\0&0&\lambda_3\end{bmatrix}
\end{equation*}
由于$\vec{\alpha}_1,\vec{\alpha}_2,\vec{\alpha}_3$ 线性无关,$P$可逆,所以
\begin{align*}
A=&P\begin{bmatrix}\lambda_1&0&0\\0&\lambda_2&0\\0&0&\lambda_3\end{bmatrix}P^{-1}\\
 =&\begin{bmatrix}1&1&0\\0&2&1\\-1&1&-1\end{bmatrix}
   \begin{bmatrix}-1&0&0\\0&1&0\\0&0&3\end{bmatrix}
   \begin{bmatrix}1&1&0\\0&2&1\\-1&1&-1\end{bmatrix}^{-1}\\
  =&\begin{bmatrix}-\frac{1}{2}&\frac{1}{2}&\frac{1}{2}\\
     -1&2&-1\\ \frac{5}{2}&-\frac{3}{2}&\frac{3}{2}\end{bmatrix}
\end{align*}
\textbf{习题 \ref{7.5} 解答:}\\
设$A=\begin{bmatrix}2&-1&2\\0&3&1\\0&0&5\end{bmatrix}$,
则$A$的特征值为$2$,$3$,$5$,对应的特征向量为$k(1,0,0)^T$,
$k(-1,1,0)^T$,$k(1,1,2)^T$,
那么令$P=\begin{bmatrix}1&-1&1\\0&1&1\\0&0&2\end{bmatrix}$,则有
\begin{equation*}
A=P\begin{bmatrix}2&0&0\\0&3&0\\0&0&5\end{bmatrix}P^{-1}
\end{equation*}
因此
\begin{equation*}
A^{n}=P\begin{bmatrix}2^{n}&0&0\\0&3^{n}&0\\0&0&5^{n}\end{bmatrix}P^{-1}
=\begin{bmatrix}2^{n}&2^n-3^n&\frac{1}{2}(3^n+5^n-2^{n+1})\\0&3^n&\frac{1}{2}(5^n-3^n)\\0&0&5^n\end{bmatrix}
\end{equation*}
\textbf{习题 \ref{7.6} 解答:}\\
(1)假设$\xi_1$和$\xi_2$是线性相关的,即存在不为零的常数$k$,使得$\xi_1=k\xi_2$。
由于$A\xi_2=\lambda_2\xi$,那么
\begin{equation*}
  A\xi_1=A(k\xi_2)=kA\xi_2=k\lambda_2\xi_2
\end{equation*}
又由于$A\xi_1=\lambda_1\xi_1$,那么$k\lambda_2\xi_2=\lambda_1\xi_1=k\lambda_1\xi_2$,
又$k\neq0$,故$\lambda_2=\lambda_1$。 这与$\lambda_1\neq\lambda_2$矛盾,
故$\xi_1$和$\xi_2$线性无关。\\
(2)假设$\xi_1-\xi_2$是$A$的特征值,即存在$\lambda_0$,使得:
\begin{equation*}
  A(\xi_1-\xi_2)=\lambda_0(\xi_1-\xi_2)
\end{equation*}
由于$A\xi_1=\lambda_1\xi_1$以及$A\xi_2=\lambda_2\xi$,因此:
\begin{equation*}
  \lambda_1\xi_1-\lambda_2\xi_2=\lambda_0(\xi_1-\xi_2)
\end{equation*}
即$(\lambda_1-\lambda_0)\xi_1-(\lambda_2-\lambda_0)\xi_2=0$。 由(1)可知,
$\xi_1$和$\xi_2$是线性无关的,故
$\lambda_1-\lambda_0=\lambda_2-\lambda_0=0$
这与$\lambda_1\neq\lambda_2$矛盾。
故$\xi_1-\xi_2$不是$A$的特征值。\\
\textbf{习题 \ref{7.7} 解答:}\\
设$\lambda$是$A$的特征值,由于$f(A)=0$,那么$f(\lambda)=0$,即$\lambda$ 是$f(x)=0$ 的根,
 又由于$\lambda\neq0$,那么$a_0\neq0$。\\
\textbf{习题 \ref{7.8} 解答:}\\
通过题目可知,$A$的特征值为$-1,2,4$;因此
$A^3-5A^2$的特征值为$-6,-12,-16$,故$|A^3-5A^2|=-1152$。\\
\textbf{习题 \ref{7.9} 解答:}\\
由于$|A|=1\times2\times2=-2$,因此$A$ 可逆。故$|A^*|=|A|A^{-1}=-2A^{-1}$,即
\begin{equation*}
  A^*+3A=-2A^{-1}+3A
\end{equation*}
令$\varphi(x)=-\frac{2}{x}+3x$,则$\varphi(1)=1$,$\varphi(2)=5$,$\varphi(-1)=-1$。故
\begin{equation*}
  |A^*+3A|=|-2A^{-1}+3A|=\varphi(1)\times\varphi(2)\times\varphi(-1)=-5
\end{equation*}
\textbf{习题 \ref{7.10} 解答:}\\
由题意可知:
\begin{equation*}
  \begin{cases}
  \lambda_2+\lambda_3=5-\lambda_1=3\\
  \lambda_2\lambda_3=\frac{4}{\lambda_1}=2\\
  \end{cases}
\end{equation*}
解得
\begin{equation*}
  \begin{cases}
  \lambda_2=1\\
  \lambda_3=2
  \end{cases},
  \text{或者}
    \begin{cases}
  \lambda_2=2\\
  \lambda_3=1
  \end{cases},
\end{equation*}
\textbf{习题 \ref{7.11} 解答:}\\
假设存在常数$k_1,k_2,k_3$,使得
\begin{equation*}
  k_1\vec{\beta}+k_2A\vec{\beta}+k_3A^2\vec{\beta}=0
\end{equation*}
由于$\vec{\beta}=\vec{\alpha}_1+\vec{\alpha}_2+\vec{\alpha}_3$,即
\begin{equation*}
  k_1(\vec{\alpha}_1+\vec{\alpha}_2+\vec{\alpha}_3)+
  k_2A(\vec{\alpha}_1+\vec{\alpha}_2+\vec{\alpha}_3)+
  k_3A^2(\vec{\alpha}_1+\vec{\alpha}_2+\vec{\alpha}_3)=0
\end{equation*}
又由于$A\vec{\alpha}_1=\lambda_1\vec{\alpha}_1,A\vec{\alpha}_2=\lambda_2\vec{\alpha}_2,A\vec{\alpha}_3=\lambda_3\vec{\alpha}_3$,化简得到
\begin{equation*}
  (k_1+k_2\lambda_1+k_3\lambda_1^2)\vec{\alpha}_1+
  (k_1+k_2\lambda_2+k_3\lambda_2^2)\vec{\alpha}_2+
  (k_1+k_2\lambda_3+k_3\lambda_3^2)\vec{\alpha}_3)=0
\end{equation*}
由于$\vec{\alpha}_1,\vec{\alpha}_2,\vec{\alpha}_3$ 线性无关,故
\begin{equation}\label{aa}
\begin{cases}
  k_1+k_2\lambda_1+k_3\lambda_1^2=0\\
  k_1+k_2\lambda_2+k_3\lambda_2^2=0\\
  k_1+k_2\lambda_3+k_3\lambda_3^2=0
\end{cases}
\end{equation}
将\eqref{aa}可以看做是关于$k_1,k_2,k_3$ 的方程组,则系数矩阵的行列式是一个三阶的范德蒙行列式,
且因为$\lambda_1,\lambda_2,\lambda_3$ 是互不相等,从而
\begin{equation*}
\begin{vmatrix}
  1&\lambda_1&\lambda_1^2\\
  1&\lambda_2&\lambda_2^2\\
  1&\lambda_3&\lambda_3^2
\end{vmatrix}=(\lambda_1-\lambda_2)(\lambda_2-\lambda_3)(\lambda_3-\lambda_1)\neq0
\end{equation*}
因此$k_1=k_2=k_3=0$,故$\vec{\beta},A\vec{\beta},A^2\vec{\beta}$ 线性无关。\\
\textbf{习题 \ref{7.12} 解答:}\\
\begin{align*}
  |A-\lambda E|&=\begin{vmatrix}2-\lambda&0&0\\0&1-\lambda&-1\\0&-1&1-\lambda\end{vmatrix}\\
   & =(2-\lambda)[(1-\lambda)^2-1]=(2-\lambda)^2\lambda
\end{align*}
因此$A$的特征值为$\lambda_1=0,\lambda_2=2(\text{二重})$。 首先,考虑$A\vec{x}=\vec{0}$,
得到$\lambda_1$的特征子空间的基为$(0,1,1)^T$;考虑$(A-2E)\vec{x}=\vec{0}$,得到$\lambda_2$ 的
特征子空间的基为$(1,0,0)^T,(0,1,-1)^T$。\\
\textbf{习题 \ref{7.13} 解答:}\\
因为$|A-\lambda E|=(1-\lambda)(2-\lambda)(5-\lambda)$,故$A$ 有3个不同的特征值,从而有3 个不同的特征向量,故$A$
可以对角化,即存在可逆矩阵$P$,使得:
\begin{equation*}
P^{-1}AP=\begin{bmatrix}
  1&0&0\\
  0&2&0\\
  0&0&5
\end{bmatrix}=B
\end{equation*}
因此$A\sim B$。\\
\textbf{习题 \ref{7.14} 解答:}\\
由于$A\sim B$,故$tr(A)=tr(B)$,即$1+4+5=3+4+y$,因此$y=3$。
于是
\begin{equation*}
|B-\lambda E|=\begin{vmatrix}3-\lambda&\sqrt{2}&1\\ \sqrt{2}&4-\lambda&\sqrt{2}\\1&\sqrt{2}&3-\lambda\end{vmatrix}
=(6-\lambda)(2-\lambda)^2.
\end{equation*}
因此,$B$的特征值为$\lambda_1=2(\text{二重})$ 以及$\lambda_2=6$。\\
由于$A\sim B$,故$A$的特征值为$\lambda_1=2(\text{二重})$以及$\lambda_2=6$。
\begin{equation*}
  |A-6E|=\begin{vmatrix}-5&-1&1\\x&-2&-2\\-3&-3&-1\end{vmatrix}=8-4x
\end{equation*}
故$x=2$。\\
\textbf{习题 \ref{7.15} 解答:}\\
取$P=B$,则$P^{-1}BAP=AB$,因此$AB$与$BA$是相似的。\\
\textbf{习题 \ref{7.16} 解答:}\\
由$|A-\lambda E|=0$解得特征值$\lambda_1=\lambda_2=0$,$\lambda_3=1$。下面求对应于特征值的特征向量:
当$\lambda_1=\lambda_2=0$时,由方程组$A-\lambda E)\vec{x}=\vec{0}$得到对应的两个线性无关的特征向量:
\begin{equation}
  \vec{\alpha}_1=(0,1,1)^T;\vec{\alpha}_2=(-2,1,0)^T.
\end{equation}
当$\lambda_3=1$时,由方程组$A-\lambda E)\vec{x}=\vec{0}$得到对应的两个线性无关的特征向量:
\begin{equation}
  \vec{\alpha}_3=(1,0,0)^T.
\end{equation}
显然$\vec{\alpha}_1,\vec{\alpha}_2,\vec{\alpha}_3$ 线性无关,故$A$可对角化。令矩阵:
\begin{equation*}
 P=(\vec{\alpha}_1,\vec{\alpha}_2,\vec{\alpha}_3)=\begin{bmatrix}0&-2&1\\1&1&0\\1&0&0\end{bmatrix}
\end{equation*}
则$P$为所求相似变换矩阵,且
\begin{equation*}
 P^{-1}AP=\begin{bmatrix}0&0&0\\0&0&0\\0&0&1\end{bmatrix}
\end{equation*}
\textbf{习题 \ref{7.17} 解答:}\\
设$\lambda$是$A$的特征值,$\vec{\xi}\neq0$ 是相应的特征向量,则
\begin{equation*}
A\xi=(E+\vec{\alpha}\vec{\beta}^T)\vec{\xi}=\lambda\vec{\xi}
\end{equation*}
左乘以$\vec{\beta}^T$,得
\begin{equation*}
(1+\vec{\beta}^T\vec{\alpha})\vec{\beta}^T\vec{\xi}=\lambda\vec{\beta}^T\vec{\xi}
\end{equation*}
若$\vec{\beta}^T\vec{\xi}\neq0$,则$\lambda=1+\vec{\beta}^T\vec{\alpha}=2$;
若$\vec{\beta}^T\vec{\xi}=0$,则$A\vec{\xi}=\lambda\vec{\xi}$,故$\lambda=1$;
即$A$的特征值为$\lambda_1=1$,$\lambda_2=3$。
对于$\lambda_1=1$,即$\vec{\beta}^T\vec{\xi}=0$,故可取
\begin{eqnarray*}
  \vec{\xi}_1&=& (-b_2,b_1,0,\cdots,0)^T \\
  \vec{\xi}_2&=& (-b_3,0,b_1,\cdots,0)^T\\
  \cdots \\
  \vec{\xi}_{n-1} &=& (-b_n,0,0,\cdots,b_1)^T
\end{eqnarray*}
对于$\lambda_2=2$,可取$\vec{\xi}_n=\vec{\alpha}$。\\
(2)由于$A$有$n$个线性无关的特征向量,故可对角化,且
\begin{equation*}
P=(\vec{\xi}_1,\vec{\xi}_2,\cdots,\vec{\xi}_n),\Lambda=diag{1,1,\cdots,1,3}.
\end{equation*}
\textbf{习题 \ref{7.18} 解答:}\\
由于$A$有3个线性无关的特征向量且$\lambda=1$ 是$A$的2重特征值,故$A$的属于
特征值$\lambda=1$的线性无关的特征向量有2 个,从而$r(A-E)=3-2=1$。由于
\begin{equation*}
  A-E=\begin{bmatrix}2&-1&1\\0&x-1&y\\0&z&w-1\end{bmatrix}
\end{equation*}
故为使$r(A-E)=1$,则
\begin{equation*}
 \begin{cases}
 x-1=0\\
 y=0\\
 z=0\\
 w-1=0
 \end{cases}
\end{equation*}
解得
\begin{equation*}
 \begin{cases}
 x=1\\
 y=0\\
 z=0\\
 w=1
 \end{cases}
\end{equation*}
因此,$A=\begin{bmatrix}3&-1&1\\0&1&0\\0&0&1\end{bmatrix}$。\\
(2)从而确定$A$的另一个特征值为$\lambda_1=3$,其对应的特征向量$(1,0,0)^T$;
对于$\lambda_2=\lambda_3=1$,其对应的线性无关特征向量为$(1,2,0)^T,(1,0,-2)^T$。
因此,令$P=\begin{bmatrix}1&1&1\\0&2&0\\0&0&-2\end{bmatrix}$,则:
\begin{equation*}
 P^{-1}AP=\begin{bmatrix}2&0&0\\0&1&0\\0&0&1\end{bmatrix}
\end{equation*}
\textbf{习题 \ref{7.19} 解答:}\\
\begin{equation*}
|A-\lambda E|=\begin{vmatrix}a-\lambda&c\\b&d-\lambda\end{vmatrix}=\lambda^2-(a+d)\lambda+ad-bc
\end{equation*}
又由于$bc>0$,则$(a+d)^2-4(ad-bc)=(a-d)^2+4bc>0$,因此$A$有两个不同的特征值,即$A$ 可以对角化。\\
\textbf{习题 \ref{7.20} 解答:}\\
可以对角化。
\begin{equation*}
  P=\begin{bmatrix}1&1&1\\1&0&-1\\0&-1&2\end{bmatrix},
  P^{-1}AP=\begin{bmatrix}-2&0&0\\0&-2&0\\0&0&4\end{bmatrix}
\end{equation*}
\textbf{习题 \ref{7.21} 解答:}\\
设$\lambda$是$A$的特征值,由题意知,$\lambda^2+3\lambda=0$,
因此$A$的特征值只可能是$-3$和$0$ 是$A$ 的特征值。因为$A$是对称矩阵,所以$A$可以经过正交变换化为
对角矩阵$\Lambda$,由于$r(A)=2$,故$r(\Lambda)=2$,由此可知$A$的特征值为:
\begin{equation*}
  \lambda_1=\lambda_2=-3,\lambda_3=0
\end{equation*}
\textbf{习题 \ref{7.22} 解答:}\\
易知$A^n\vec{\alpha}_1=\lambda_1^n\vec{\alpha}_1(n=1,2,...)$,于是
\begin{equation*}
  B\vec{\alpha}_1=A^2\vec{\alpha}_1+2A\vec{\alpha}_1=(\lambda_1^2+2\lambda)\vec{\alpha}_1=3\vec{\alpha}_1
\end{equation*}
因此$\vec{\alpha}_1$是$B$的特征向量,$B$ 的特征值可以由$A$的特征值以及$A$ 和$B$ 的关系得到,
即$B$的全部特征值为$3,8,3$。由于$A$是实对称的矩阵,那么$B$也是实对称的,设$B$ 属于$\lambda_2=3$的特征值对应
的特征向量为$(x_1,x_2,x_3)^T$,则有$x_1+x_2+x_3=0$,求得$B$属于$\lambda_2=3$的特征值对应的特征向量为:
\begin{equation*}
  \vec{\alpha}_2=(1,0,-1)^T,\vec{\alpha}_3=(1,-1,0)^T
\end{equation*}
\textbf{习题 \ref{7.23} 解答:}\\
\begin{equation*}
 |A-\lambda E| = \begin{vmatrix}1-\lambda&-2&0\\-2&2-\lambda&-2\\0&-2&3-\lambda\end{vmatrix}
               =(\lambda+1)(\lambda-2)(\lambda-5)
\end{equation*}
因此$A$的特征值为$\lambda_1=-1,\lambda_2=2,\lambda_3=5$。
对于$\lambda_1$,考虑方程组$(A+E)\vec{x}=\vec{0}$,
解得其对应的单位特征向量为$\vec{\alpha}_1=(\frac{2}{3},\frac{2}{3},\frac{1}{3})^T$;\\
对于$\lambda_2$,考虑方程组$(A+E)\vec{x}=\vec{0}$,
解得其对应的单位特征向量为$\vec{\alpha}_2=(-\frac{2}{3},\frac{1}{3},\frac{2}{3})^T$;\\
对于$\lambda_3$,考虑方程组$(A+E)\vec{x}=\vec{0}$,
解得其对应的单位特征向量为$\vec{\alpha}_3=(\frac{1}{3},-\frac{2}{3},\frac{2}{3})^T$。
故令$Q=\begin{bmatrix}\frac{2}{3}&\frac{2}{3}&\frac{1}{3}\\-\frac{2}{3}&\frac{1}{3}&\frac{2}{3}\\
\frac{1}{3}&-\frac{2}{3}&\frac{2}{3}\end{bmatrix}$,那么
\begin{equation*}
  Q^TAQ=\begin{bmatrix}-1&0&0\\0&2&0\\0&0&5\end{bmatrix}
\end{equation*}
\textbf{习题 \ref{7.24} 解答:}\\
\begin{equation*}
  Q=\begin{bmatrix}\frac{1}{3}&\frac{2}{\sqrt{5}}&\frac{2}{\sqrt{5}}
       \\ \frac{2}{3}&-\frac{1}{\sqrt{5}}&0\\-\frac{2}{3}&0&\frac{1}{\sqrt{5}}\end{bmatrix},
  Q^TAQ=\begin{bmatrix}-7&0&0\\0&2&0\\0&0&2\end{bmatrix}
\end{equation*}
\textbf{习题 \ref{7.25} 解答:}\\
设$A$的3个正交特征向量是$\vec{\xi}_1,\vec{\xi}_2,\vec{\xi}_3$,
单位化后是$\vec{\eta}_1,\vec{\eta}_2,\vec{\eta}_3$ 仍是$A$的特征向量,
令$Q=(\vec{\eta}_1,\vec{\eta}_2,\vec{\eta}_3)$,则$Q$是正交矩阵,$Q^{-1}=Q^T$,
由于$A$是3阶实矩阵且有3个正交的特征向量,从而$A$可以对角化,即
\begin{equation*}
  Q^{-1}AQ=\Lambda
\end{equation*}
其中$\Lambda$为对角阵,因此$A=Q\Lambda Q^T$,故
\begin{equation*}
  A^T=Q\Lambda^TQ^T=Q\Lambda Q^T=A
\end{equation*}
\textbf{习题 \ref{7.26} 解答:}\\
令$Q=(\vec{\alpha}_1,\vec{\alpha}_2,\cdots,\vec{\alpha}_n)$,则
\begin{equation*}
  Q^{-1}AQ=\Lambda=diag\{\lambda_1,\lambda_2,\cdots,\lambda_n\}.
\end{equation*}
于是$A=Q\Lambda Q^T=\lambda_1\vec{\alpha}_1\vec{\alpha}_1^T+\cdots+\lambda_n\vec{\alpha}_n\vec{\alpha}_n^T$。\\
\textbf{习题 \ref{7.27} 解答:}\\
由于$A$是$n$阶实对称矩阵,故存在正交阵$Q$以及对角矩阵$\Lambda$,使得
\begin{equation*}
  A=Q\Lambda Q^T
\end{equation*}
故$A^{k}=Q\Lambda^kQ^T$,即$\Lambda^k=O$,所以$\Lambda=O$。 从而$A=O$。\\
\textbf{习题 \ref{7.28} 解答:}\\
由题意知:
\begin{eqnarray*}
  x_{n+1} &=& x_n+qy_n-px_n=(1-p)x_n+qy_n \\
  y_{n+1} &=& y_n+px_n-qy_n=px_n+(1-q)y_n
\end{eqnarray*}
可用矩阵表示为:
\begin{equation*}
\begin{bmatrix}x_{n+1}\\y_{n+1}\end{bmatrix}=
\begin{bmatrix}1-p&q\\p&1-q\end{bmatrix}
\begin{bmatrix}x_n\\y_n\end{bmatrix}
\end{equation*}
即
\begin{equation*}
A=\begin{bmatrix}1-p&q\\p&1-q\end{bmatrix}
\end{equation*}
(2)由于$\begin{bmatrix}x_{n+1}\\y_{n+1}\end{bmatrix}=A\begin{bmatrix}x_n\\y_n\end{bmatrix}$,
可知$\begin{bmatrix}x_{n}\\y_{n}\end{bmatrix}=A^n\begin{bmatrix}x_0\\y_0\end{bmatrix}$。
由
\begin{equation*}
  |A-\lambda E|=\begin{vmatrix}1-p-\lambda&q\\p&1-q-\lambda\end{vmatrix}=(\lambda-1)(\lambda-1+p+q)
\end{equation*}
得到$A$的特征值为$\lambda_1=1,\lambda_2=r$,其中$r=1-p-q$。
得到相应的特征向量为$\vec{\alpha}_1=(q,p)^T,\vec{\alpha}_2=(-1,1)^T$。
令$P=\begin{bmatrix}q&-1\\p&1\end{bmatrix}$,$\Lambda=\begin{bmatrix}1&0\\0&r\end{bmatrix}$,故
\begin{equation*}
  A^n=P\Lambda^nP^{-1}=\frac{1}{p+q}\begin{bmatrix}q+pr^n&q-qr^n\\p-pr^n&p+qr^n\end{bmatrix}
\end{equation*}
故
\begin{equation*}
 \begin{bmatrix}x_{n}\\y_{n}\end{bmatrix}=\frac{1}{2(p+q)}\begin{bmatrix}2q+(p-q)r^n\\2p+(q-p)r^n\end{bmatrix}
\end{equation*}

%%%%%%%%%%%%%%%%%%%%%%%%%%%%%%%%%%%%%%%%%%%%%%%%%%%%%%%%%%%%%%%%%%%%%%%%%%%%%%%%%%%%
%%%%%%%%%%%%%%%%%%%%%%%%%%%%%%%%%%%%%%%%%%%%%%%%%%%%%%%%%%%%%%%%%%%%%%%%%%%%%%%%%%%%

\chapter{矩阵与变换}

\section{知识点解析}
\begin{Def}
给定矩阵$A\in M_{m\times n}$, 则如下定义由$A$确定的矩阵映射,
\begin{displaymath}\begin{aligned}
\varphi_A: & \mathbb{R}^n\ \rightarrow\ \mathbb{R}^m\\
& \vec{\alpha}\ \mapsto\ A\vec{\alpha}.\end{aligned}\end{displaymath}
特别地,当$m=n$时, $\varphi_A$称为矩阵变换.
\end{Def}

\begin{thm}
任意$\mathbb{R}^n$到$ \mathbb{R}^m$的线性映射$\sigma$均为矩阵映射, 且其表示矩阵的第 $j$ 列就是第 $j$ 个自然基$\vec{e}$在$\sigma$作用下的像.

\end{thm}

\begin{thm}
在$\mathbb{R}^3$中, 以单位向量$\vec{l} =(a,b,c)^T$为旋转轴, 旋转角度为$\theta$ 的旋转变换的变换矩阵为:
\begin{displaymath}
T_{\vec{l},\theta}=\begin{bmatrix} a^2+(1-a^2)\cos\theta & ab(1-\cos\theta)-c\sin\theta & ac(1-\cos\theta)+b\sin\theta\\
ab(1-\cos\theta)+c\sin\theta & b^2+(1-b^2)\cos\theta & bc(1-\cos\theta)-a\sin\theta  \\
ac(1-\cos\theta)-b\sin\theta& bc(1-\cos\theta)+a\sin\theta & c^2+(1-c^2)\cos\theta\end{bmatrix}\end{displaymath}
\end{thm}


\begin{Def}
设$\varphi_B:\mathbb{R}^l \rightarrow \mathbb{R}^n, \varphi_A:\mathbb{R}^n \rightarrow \mathbb{R}^m$为矩阵映射, 则$\varphi_A$与$\varphi_B$的复合映射$\varphi_A\circ \varphi_B$如下定义:
\begin{displaymath}
 \varphi_A\circ \varphi_B(\vec{x}):=\varphi_A( \varphi_B(\vec{x}))\ \ \ (\forall \vec{x}\in \mathbb{R}^l).\end{displaymath}
\end{Def}

\begin{Def}
设$A$为$n$阶方阵, 若有子空间$W\in\mathbb{R}^n$, 满足$\forall \vec{x}\in W$, 均有$A\vec{x}\in W$ , 则称$W$ 是矩阵变换$A$的不变子空间.
\end{Def}

\begin{Def}
在$\mathbb{R}^n$, 设$(\vec{\alpha},\vec{\beta})$表示标准内积, $\varphi_A$ 为一个矩阵变换, 若$\forall \vec{\alpha},\vec{\beta}$, 均有
$$(\varphi_A(\vec{\alpha}),\varphi_A(\vec{\beta}))=(A\vec{\alpha}  ,A\vec{\beta})=(\alpha,\beta).$$
则称$\varphi_A$为正交变换.
\end{Def}

\begin{thm}
在$\mathbb{R}^n$中, $\varphi_A$为矩阵变换, 则下述条件等价
\begin{enumerate}
\item $\varphi_A$为正交变换(保内积);
\item $|A\vec{\alpha}|=|\vec{\alpha}|(\forall \vec{\alpha}\in \mathbb{R}^n)$(保长度);
\item $A$为正交矩阵.
\end{enumerate}
\end{thm}

\begin{thm}
在$\mathbb{R}^2$与$\mathbb{R}^3$中, 第一类正交变换必为旋转变换. 第二类正交变换必为反射变换, 或某个反射变换与某个旋转变换的复合.
\end{thm}

%%%%%%%%%%%%%%%%%%%%%%%%%%%%%%%%%%%%%%%%%%%%%%%%%%%%%%%%%%%%%%%%%%%%%%%%%%%%%%%%%%%%

\section{例题讲解}

\begin{eg}
令$A=\begin{bmatrix}0&1\\1&0\end{bmatrix}$, 求下列列向量在矩阵变换$A$作用下的像,
\begin{displaymath}
\vec{e}_1=\begin{bmatrix}1\\0\end{bmatrix},\ \vec{e}_2=\begin{bmatrix}0\\1\end{bmatrix},\ \vec{\alpha}=\begin{bmatrix}-1\\-2\end{bmatrix}.\end{displaymath}
\end{eg}
解: \begin{displaymath}\begin{aligned}
&A\vec{e}_1=\begin{bmatrix}0&\ 1\\1&\ 0\end{bmatrix}\ \begin{bmatrix}1\\0\end{bmatrix}
=\begin{bmatrix}0\\1\end{bmatrix};\\
&A\vec{e}_2=\begin{bmatrix}0&\ 1\\1&0\end{bmatrix}\ \begin{bmatrix}0\\1\end{bmatrix}
=\begin{bmatrix}1\\0\end{bmatrix};\\
&A\vec{\alpha}=\begin{bmatrix}0&\ 1\\1&0\end{bmatrix}\ \begin{bmatrix}-1\\-2\end{bmatrix}
=\begin{bmatrix}-2\\-1\end{bmatrix}.
\end{aligned}\end{displaymath}

\begin{eg}
(平移变换)设变换$\varphi_:  \mathbb{R}^2\ \rightarrow\ \mathbb{R}^2$ 将所有的点向右移动$a_1$个单位, 同时向上移动$a_2$个单位, 那么可以用向量表示为: \begin{displaymath}
\varphi(\vec{x})=\varphi\begin{bmatrix}x_1\\x_2\end{bmatrix}=\begin{bmatrix} x_1+a_1\\x_2+a_2\end{bmatrix}=\begin{bmatrix}x_1\\x_2\end{bmatrix}+\begin{bmatrix}
a_1\\a_2\end{bmatrix}=\vec{x}+\vec{\alpha},\end{displaymath}
其中, $\vec{\alpha}=\begin{bmatrix}a_1\\a_2\end{bmatrix}$. 由矩阵变换的定义, 可以看出平移变换一般并不是矩阵变换. 而这一点, 从下一小节矩阵映射(变换)的性质也可看出.
\end{eg}

\begin{eg}
在$\mathbb{R}^3$中, 矩阵变换把线段变成什么图形?
\end{eg}
解: 设线段的端点分别为$X$和$Y$, 则以线段上点为终点的向量可表示为
\begin{displaymath}
\lambda\vec{OX}+(1-\lambda)\vec{OY}\ \ \ (\forall \lambda\in[0,1])
\end{displaymath}
如果记$\vec{OX}=\vec{\alpha}, \vec{OY}=\vec{\beta}$, 则在矩阵变换$A$的作用下, 可得
\begin{displaymath}
A(\lambda\vec{\alpha}+(1-\lambda)\vec{\beta})=\lambda A\vec{\alpha}+(1-\lambda)A \vec{\beta}\ \ \ (\forall  \lambda\in[0,1])
\end{displaymath}
这依然是一条线段, 两个端点分别为向量$A\vec{\alpha}$与$A \vec{\beta}$的终点.

因此矩阵变换将线段变成线段, 或者变成点 (线段的退化情形).

\begin{eg}
设$A=\begin{bmatrix}1&2\\2&1\end{bmatrix}$, 在平面$\mathbb{R}^2$中三点:
$$E=\begin{bmatrix}1\\1\end{bmatrix}, \ \ F=\begin{bmatrix}-1\\0\end{bmatrix}, \ \
G=\begin{bmatrix}1\\-1\end{bmatrix}.$$
构成的$\triangle EFG$, 在$A$作用下变换成什么图形?
\end{eg}
解: 由于矩阵变换将线段变为线段,故$A$作用将$\triangle EFG$映为$\triangle E^{'}F^{'}G^{'}$
$$E^{'}=A\begin{bmatrix}1\\1\end{bmatrix}=\begin{bmatrix}3\\3\end{bmatrix},\ \ F^{'}=A\begin{bmatrix}-1\\0\end{bmatrix}=\begin{bmatrix}-1\\-2\end{bmatrix},\ \ G^{'}=A\begin{bmatrix}1\\-1\end{bmatrix}=\begin{bmatrix}-1\\1\end{bmatrix}.$$


\begin{eg}
对$\mathbb{R}^3$中的线投影变换$P_{\vec{l}}=\vec{l}\vec{l}^T$, 其中$\vec{l}$为单位向量.
\begin{displaymath}P_{\vec{l}}\vec{l}=(\vec{l}\vec{l}^T)\vec{l}
=\vec{l}(\vec{l}^T\vec{l})=\vec{l}\ \Rightarrow\ \lambda=1.\end{displaymath}

特征子空间$L(\vec{l})$维数为1.

若$\vec{v}$与$\vec{l}$正交, 有

$$P_{\vec{l}}\vec{v}=(\vec{l}\vec{l}^T)\vec{v}
=\vec{l}(\vec{l}^T\vec{v})=\vec{l}\dot 0=\vec{0}\ \Rightarrow \ \lambda=0.$$

特征子空间为$L(\vec{l})^{\bot}$, 维数为$n-1$.

\end{eg}

\begin{eg}
对$\mathbb{R}^3$中的面投影变换$P_{\pi}=I-\vec{n}\vec{n}^T$, 其中$\vec{n}$为投影平面$\pi$的单位法向量.
$$P_{\pi}\vec{n}=\vec{n}-\vec{n}(\vec{n}^T\vec{n})=\vec{0}\ \Rightarrow \ \lambda_1=0.$$

特征子空间为$L(\vec{n})$, 1维.

若$\vec{v}$与$\vec{n}$正交, 则$$P_{\pi}\vec{v}=\vec{v}-\vec{n}(\vec{n}^T\vec{v})=\vec{v}-\vec{0}=\vec{v}. \Rightarrow \ \lambda_2=1.$$

特征子空间为$L(\vec{n})^{\bot}$, 2维.

\end{eg}

\begin{eg}
$\mathbb{R}^3$中的线反射变换: $R_{\vec{l}}=-I+2\vec{l}\vec{l}^T$. 其中$\vec{l}$为反射直线的单位向量.
$$R_{\vec{l}}\vec{l}=-\vec{l}+2\vec{l}\vec{l}^T\vec{l}=\vec{l}\ \Rightarrow \  \lambda_1=1$$

特征子空间为$L(\vec{l})$, 1维.

若$\vec{v}$与$\vec{l}$正交, 有$R_{\vec{l}}\vec{v}=-\vec{v}$. 所以$\lambda_2=-1$, 特征子空间为$L(\vec{l})^{\bot}$, 2维.
\end{eg}

\begin{eg}
对$\mathbb{R}^3$中的面反射变换$R_{\pi}=I-2\vec{n}\vec{n}^T$, 其中$\vec{n}$为反射平面$\pi$的单位法向量.

$$R_{\pi}\vec{n}=\vec{n}-2\vec{n}\vec{n}^T\vec{n}=-\vec{n}\ \Rightarrow \ \lambda_1=-1.$$

特征子空间为$L(\vec{n})$, 1维.

若$\vec{v}$与$\vec{n}$正交, 则$$R_{\pi}\vec{v}=\vec{v}-\vec{0}=\vec{v}\ \Rightarrow \ \lambda_2=1$$

特征子空间为$L(\vec{n})^{\bot}$, 2维.

\end{eg}

\begin{eg}
对$\mathbb{R}^2$中的旋转变换$T_{\theta}=\begin{bmatrix} \cos\theta&-\sin\theta\\ \sin\theta&\cos\theta\end{bmatrix}.$

特征多项式为
$$|\lambda I-T_{\theta}|=\lambda^2-(2\cos\theta)\lambda +1.$$
特征值$\lambda_{1,2}=\cos\theta\pm\sqrt{1-\cos^2\theta}i.$

当$\theta\not=0,\pi$时, $T_{\theta}$无实特征值, 从而没有实特征向量. 这说明, 此时$T_{\theta}$在$\mathbb{R}^2$中, 除$\vec{0}$之外, 没有向量是保持方向不变或方向相反的.
\end{eg}

\begin{eg}
用坐标系替换的方式,重新推导$\mathbb{R}^2$内线投影变换的矩阵公式.

设$\vec{l}$为投影直线的单位向量, 选取$\{\vec{l},\vec{l}^{\bot}\}$为新的坐标系, 其中
$$\vec{l}=\begin{bmatrix}l_1\\l_2\end{bmatrix},\ \vec{l}=\begin{bmatrix}-l_2\\l_1\end{bmatrix}$$
故过渡矩阵为
$$P=(\vec{l},\vec{l}^{\bot})=\begin{bmatrix} l_1&-l_2\\l_2&l_1\end{bmatrix}$$
设该变换在旧基$\{\vec{e}_1,\vec{e}_2\}$与新基$\{\vec{l},\vec{l}^{\bot}\}$下的矩阵表示分别为$A$和$B$. 于是
有$B=P^{-1}AP$. 易知$B=\begin{bmatrix}1&0\\0&0\end{bmatrix}$, 从而有
$$A=PBP^{-1}=\begin{bmatrix}l_1&-l_2\\l_2&l_1\end{bmatrix}\begin{bmatrix}1&0\\
0&0\end{bmatrix}\begin{bmatrix}l_1&l_2\\-l_2&l_1\end{bmatrix}=\begin{bmatrix}
l_1^2&l_1l_2\\l_1l_2&l_2^2\end{bmatrix}=\vec{l}\vec{l}^T.$$
\end{eg}

\begin{eg}
设平行光束的方向为$\vec{\beta}=\begin{bmatrix}b_1\\b_2\end{bmatrix}$, 投影直线的方向向量为$\vec{\alpha}=\begin{bmatrix}a_1\\a_2\end{bmatrix}$. 设$\vec{\alpha}$与$\vec{\alpha}$不共线(也不一定正交). 那么, 沿$\vec{\beta}$方向在$\vec{\alpha}$上的投影变换所对应的矩阵是什么?

用坐标系替换的方法来解决. 以$\{\vec{\alpha},\vec{\beta}\}$为新坐标系, 于是过渡矩阵为
$$P=(\vec{\alpha},\vec{\beta})=\begin{bmatrix}a_1&b_1\\a_2&b_2\end{bmatrix}.$$
易知在新坐标系下投影变换的矩阵为$\begin{bmatrix}1&0\\0&0\end{bmatrix}$, 故在原基(自然基)下的矩阵为
$$A=P\begin{bmatrix}1&0\\0&0\end{bmatrix}P^{-1}=(a_1b_2-a_2b_1)^{-1}\begin{bmatrix}
a_1b_1&-a_1b_1\\a_2b_2&-a_2b_1\end{bmatrix}.$$
\end{eg}

\begin{eg}
对$\mathbb{R}^n$中的旋转变换,
$$R_{\theta}=\begin{bmatrix}\cos\theta&-\sin\theta\\ \sin\theta &\cos\theta\end{bmatrix},$$
易知, $\forall \theta\in [0, 2\pi), \vec{\alpha}_1=\begin{bmatrix} \cos\theta\\ \sin\theta\end{bmatrix}, \vec{\alpha}_2=\begin{bmatrix} -\sin\theta\\ \cos\theta\end{bmatrix}.$ 总满足
$$(\vec{\alpha}_i, \vec{\alpha}_j)=\delta_{ij}(1\leq i, j\leq 2).$$
故$R_{\theta}$为正交阵, 且$\det R_{\theta}=1$, 所以旋转变换为正交变换.
\end{eg}

\begin{eg}
对$\mathbb{R}^n$中的线反射变换$R_{\vec{l}}=-I+2\vec{l}\vec{l}^T$(简记为$R$), 其中$\vec{l}$为反射轴方向的单位向量. 由于
\begin{displaymath}\begin{aligned}
RR^T&=(-I+2\vec{l}\vec{l}^T)(-I+2\vec{l}\vec{l}^T)^T\\
&=I^2-4\vec{l}\vec{l}^T+4(\vec{l}\vec{l}^T)(\vec{l}\vec{l}^T)\\
&=I-4\vec{l}\vec{l}^T+4\vec{l}(\vec{l}^T\vec{l})\vec{l}^T\\
&=I-4\vec{l}\vec{l}^T+4\vec{l}\vec{l}^T=I\end{aligned}\end{displaymath}
其中$\vec{l}^T\vec{l}=(\vec{l},\vec{l})=1$. 故$R_{\vec{l}}$为正交阵. 线反射变换为正交变换.
\end{eg}

\begin{eg}
对$\mathbb{R}^n$中的面反射变换, $R_{\pi}=I-2\vec{n}\vec{n}^T$, 其中$\vec{n}$为反射平面$\pi$的单位法向量. 由于$R_{\pi}=-R_{\vec{n}}$. 由上例可知, $R_{\pi}$也为正交变换, 且将$\vec{n}$扩充为$\mathbb{R}^3$的一组标准正交基$\{\vec{n}=\vec{n}_1,\vec{n}_2,\vec{n}_3\}$后, $R_{\pi}$可正交对角化. 即
$$Q^TR_{\pi}Q=\begin{bmatrix} -1&&\\&1&\\&&1\end{bmatrix}.$$
其中$Q=(\vec{n}_1,\vec{n}_2,\vec{n}_3)$为正交阵. 从而, 面反射变换$R_{\pi}$ 为正交变换, 且$\det R_{\pi}=-1$.
\end{eg}

%%%%%%%%%%%%%%%%%%%%%%%%%%%%%%%%%%%%%%%%%%%%%%%%%%%%%%%%%%%%%%%%%%%%%%%%%%%%%%%%%%%

\section{课后习题}
\begin{ex}\label{8.1}
1. 以下哪些变换不是矩阵变换?\\
A. 关于点$(1,1)$的旋转变换.\\		
B. 关于直线$y=kx$的对称变换.\\
C. 关于向量$(3,4)$的平移变换.\\		
D. 在$y$轴上的投影变换.
\end{ex}

\begin{ex}\label{8.2}
恒等变换是(    ).\\
A. 保持所有向量不变的矩阵变换.\\
B. 保持所有向量方向不变的矩阵变换.\\
C. 保持所有向量长度不变的矩阵变换.\\
D. 保持所有向量夹角不变的矩阵变换.
\end{ex}

\begin{ex}\label{8.3}
矩阵变换$\begin{bmatrix}1&0\\0&-2\end{bmatrix}$ 的作用是在$x_1$方向上(~~~~),(~~~~).
\end{ex}

\begin{ex}\label{8.4}
设$\vec{\alpha}_0=\vec{OO^{'}}$, 则向量$\vec{x}$作关于$O^{'}$的中心对称是(~~~~).(用$\vec{\alpha}_0,\vec{x}$填写)
\end{ex}

\begin{ex}\label{8.5}
在$\mathbb{R}^2$中,求解满足如下矩阵变换的矩阵.\\
(1) 关于$y$轴轴对称变换.\\
(2) 关于原点逆时针旋转90度.\\
(3) 为关于直线$x_1=-x_2$的对称变换.\\
(4) 在$x$轴上的投影.\\
(5) 关于原点顺时针旋转90度.\\
(6) 关于原点顺时针旋转45度.
\end{ex}

\begin{ex}\label{8.6}
设过原点的直线L的单位方向向量为$\vec{l}_0$ (列向量),则向量$\vec{x}$ 作关于$L$ 的轴对称后变为(     ).\\
A. $(I-2\vec{l}_0 \vec{l}_0^T )\vec{x}$  \\
B. $(I+2\vec{l}_0 \vec{l}_0^T )\vec{x}$\\
C. $(-I-2\vec{l}_0 \vec{l}_0^T )\vec{x}$  \\
D. $(-I+2\vec{l}_0 \vec{l}_0^T )\vec{x}$
\end{ex}

\begin{ex}\label{8.7}
在$\mathbb{R}^2$中,给定列向量$\vec{l}$ ,设$L$为过原点,方向为$\vec{l}$的直线。求$L$的线反射变换矩阵$R_{\vec{l}}$.\\
(1) $\vec{l}=\begin{bmatrix}0\\1\end{bmatrix}$\\
(2) $\vec{l}=\begin{bmatrix}1\\0\end{bmatrix}$\\
(3) $\vec{l}=\begin{bmatrix}1\\1\end{bmatrix}$\\
(4) $\vec{l}=\begin{bmatrix}-3\\4\end{bmatrix}$\\
(5) $\vec{l}=\begin{bmatrix}12\\-5\end{bmatrix}$
\end{ex}

\begin{ex}\label{8.8}
(1) 设过原点的平面$\pi$的单位法向量为$\vec{n}_0$, 则向量$\vec{x}$作关于$\pi$ 的镜面对称后变为(     ).\\
A. $(-I-2\vec{n}_0\vec{n}_0^T )\vec{x}$\\
B. $(-I+2\vec{n}_0\vec{n}_0^T )\vec{x}$\\
C. $(I-2\vec{n}_0\vec{n}_0^T )\vec{x}$\\
D. $(I+2\vec{n}_0\vec{n}_0^T )\vec{x}$\\
(2). 设$\vec{l}_0$为单位向量, 则向$\vec{l}_0$ 方向投影对应的线投影矩阵为(     ).\\
A. $-\vec{l}_0 \vec{l}_0^T$\\
B. $\vec{l}_0 \vec{l}_0^T$\\
C. $I-\vec{l}_0 \vec{l}_0^T$\\
D. $I-\vec{l}_0 \vec{l}_0^T$\\
(3). 设过原点的平面$\pi$的单位法向量为$\vec{n}_0$, 则关于$\pi$的投影对应的面投影矩阵为(      ).\\
A. $I-\vec{n}_0 \vec{n}_0^T$\\
B. $I+\vec{n}_0 \vec{n}_0^T$\\
C. $-\vec{n}_0 \vec{n}_0^T$\\
D. $-\vec{n}_0 \vec{n}_0^T$
\end{ex}

\begin{ex}\label{8.9}
平面上绕原点逆时针旋转$\theta$角对应的旋转矩阵为(      ).\\
A.  $\begin{bmatrix}-\cos\theta&\sin\theta \\ \sin\theta&\cos\theta\end{bmatrix}$\\
B.  $\begin{bmatrix}\cos\theta&-\sin\theta \\ \sin\theta&\cos\theta\end{bmatrix}$\\
C. $\begin{bmatrix}\cos\theta&\sin\theta \\ -\sin\theta&\cos\theta\end{bmatrix}$\\
D. $\begin{bmatrix}\cos\theta&\sin\theta \\ \sin\theta&-\cos\theta\end{bmatrix}$
\end{ex}

\begin{ex}\label{8.10}
$\phi_A,\phi_B$分别为对应于2 阶矩阵$A$,$B$ 的映射,记$\phi_C=\phi_A\circ\phi_B$. 任取向量$\vec{x}=(x_1,x_2 )^T$,
$\phi_B\begin{bmatrix}x_1\\x_2\end{bmatrix}=\begin{bmatrix}x_1+2x_2\\2x_1+3x_2\end{bmatrix}$,
而$\phi_C\begin{bmatrix}x_1\\x_2\end{bmatrix}=\begin{bmatrix}2x_1+3x_2\\5x_1+8x_2\end{bmatrix}$,则矩阵$A$为(     ).\\
A.  $\begin{bmatrix}0&1\\-1&2\end{bmatrix}$\\
B.  $\begin{bmatrix}0&1\\1&2\end{bmatrix}$\\
C.  $\begin{bmatrix}12&7\\31&18\end{bmatrix}$\\
D.  $\begin{bmatrix}4&7\\-1&-2\end{bmatrix}$
\end{ex}

\begin{ex}\label{8.11}
11. (1) 计算$A^{10}\cdot\vec{b}$ 其中$A=\begin{bmatrix}0&1\\-2&3\end{bmatrix},\vec{b}=\begin{bmatrix}3\\4\end{bmatrix}$.\\
(2) 计算$A^4\cdot\vec{b}$,其中$A=\begin{bmatrix}1&2\\2&1\end{bmatrix},\vec{b}=\begin{bmatrix}6\\-4\end{bmatrix}$.
\end{ex}

\begin{ex}\label{8.12}
设$\vec{l}=\begin{bmatrix}1\\1\end{bmatrix}$,$P_{\vec{l}}$是$\mathbb{R}^2$ 中对$\vec{l}$ 的线投影变换矩阵,求$P_{\vec{l}}^{2016}\begin{bmatrix}4\\2\end{bmatrix}$.
\end{ex}

\begin{ex}\label{8.13}
设$\vec{l}=\begin{bmatrix}3\\4\\5\end{bmatrix}$,$P_{\vec{l}}$是$\mathbb{R}^3$ 中对$\vec{l}$的线投影变换矩阵,求$P_{\vec{l}}^{314}\begin{bmatrix}50\\50\\50\end{bmatrix}$.
\end{ex}

\begin{ex}\label{8.14}
设$\vec{n}=\begin{bmatrix}1\\2\\2\end{bmatrix}$,$R_{\pi}$是$\mathbb{R}^3$ 中对$\pi$的面反射变换矩阵,其中$\vec{n}$是$\pi$ 的法向量.
求$R_{\pi}^{233}\begin{bmatrix}-9\\-9\\-9\end{bmatrix}$.
\end{ex}

\begin{ex}\label{8.15}
设$\vec{n}=\begin{bmatrix}2*\\*2\\1**\end{bmatrix}$,$R_\pi$是$\mathbb{R}^3$ 中对$\pi$的面反射变换矩阵,
其中$\vec{n}$是$\pi$的法向量. 但是法向量$\vec{n}$已经模糊辨认不清. 求$R_{\pi}^{666}\begin{bmatrix}1\\2\\3\end{bmatrix}$.
\end{ex}

\begin{ex}\label{8.16}
判断下列命题真假.\\
(1)从一组基到另一组基的过渡矩阵可能是奇异阵.\\
(2)从一组基到任意一组基有且仅有一个过渡矩阵.\\
(3)同一个线性变换在不同基下的矩阵相似.
\end{ex}

\begin{ex}\label{8.17}
判断下列命题真假.\\
(1)刚体变换(在变换前后能保持任意两点距离,两直线夹角不变的变换)一定是正交变换.\\
(2)正交变换的逆变换也为正交变换.\\
(3)两个正交变换的加法也为正交变换.\\
(4)两个正交变换的乘法也为正交变换.
\end{ex}

%%%%%%%%%%%%%%%%%%%%%%%%%%%%%%%%%%%%%%%%%%%%%%%%%%%%%%%%%%%%%%%%%%%%%%%%%%%%%%%%%%%%%

\section{习题答案}

\textbf{习题 \ref{8.1} 解答:}\\
A, C.(矩阵变换必将零向量映为零向量,只要观察变换对零向量的作用即可)\\
\textbf{习题 \ref{8.2} 解答:}\\
A. (显然,由定义所得.)\\
\textbf{习题 \ref{8.3} 解答:}\\
保持不变, 在$x_2$反方向上伸长为原来的2 倍.\\
\textbf{习题 \ref{8.4} 解答:}\\
$2\vec{\alpha}_0-\vec{x}$. (由中点公式,$\vec{x}+\vec{y}(\text{设为答案})=2\vec{\alpha}_0$,得证.)\\
\textbf{习题 \ref{8.5} 解答:}\\
(1)设该变换为$\eta$,将$(x,y)^T$ 变为$(-x,y)^T$. 可以看做
\begin{eqnarray*}
    \eta(x) &=& -1\cdot x+0\cdot y \\
    \eta(y) &=& 0\cdot x+1\cdot y
\end{eqnarray*}
写成矩阵形式为
\begin{equation*}
\eta\begin{bmatrix}x\\y\end{bmatrix}=
\begin{bmatrix}-1&0\\0&1)\end{bmatrix}
\begin{bmatrix}x\\y\end{bmatrix}
\end{equation*}
所以η的变换矩阵为$\begin{bmatrix}-1&0\\0&1\end{bmatrix}$.\\
(2) 我们利用复数工具,将$(x,y)^T$视为$x+yi$,
则平面上的点绕原点逆时针旋转90度即等效于对应的复数乘以复数$i$,结果为$-y+xi$.
由(1)的做法,该矩阵变换为$\begin{bmatrix}0&-1\\1&0\end{bmatrix}$.\\
(3) 方法同上,根据
\begin{eqnarray*}
    \eta(x) &=& - y \\
    \eta(y) &=& - x
\end{eqnarray*}
得变换矩阵为$\begin{bmatrix}0&-1\\-1&0\end{bmatrix}$.\\
(4) 方法同上,根据
\begin{eqnarray*}
    \eta(x) &=& x \\
    \eta(y) &=& 0
\end{eqnarray*}
得变换矩阵为$\begin{bmatrix}1&0\\0&0\end{bmatrix}$.\\
(5)  方法同上,根据$(x+yi)(-i)=y-xi$
\begin{eqnarray*}
    \eta(x) &=& y \\
    \eta(y) &=& -x
\end{eqnarray*}
得变换矩阵为$\begin{bmatrix}0&1\\-1&0\end{bmatrix}$.\\
(6)方法同上,
根据$(x+yi)(\cos{\frac{-\pi}{4}}?+i\sin{\frac{-\pi}{4}})=\frac{\sqrt{2}}{2}(x+y)+\frac{\sqrt{2}}{2}(y-x)i$,得
\begin{eqnarray*}
    \eta(x) &=& \frac{\sqrt{2}}{2}x+\frac{\sqrt{2}}{2}y\\
    \eta(y) &=& -\frac{\sqrt{2}}{2}x+\frac{\sqrt{2}}{2}y
\end{eqnarray*}
得变换矩阵为$\begin{bmatrix}\frac{\sqrt{2}}{2}&\frac{\sqrt{2}}{2}\\
-\frac{\sqrt{2}}{2}&\frac{\sqrt{2}}{2}\end{bmatrix}$.\\
\textbf{习题 \ref{8.6} 解答:}\\
分析:设直线$L$的单位法方向向量为$\vec{n}_0$,
注意有$(\vec{l}_0,\vec{l}_0)=\vec{l}_0^T \vec{l}_0=1,(\vec{l}_0,\vec{n}_0)=(\vec{l}_0^T \vec{n}_0=\vec{n}_0^T\vec{l}_0=0$.
这里,$(\cdot,\cdot)$ 表示向量的内积,即数量积.
注意到选项里的四个变换可统一写为$aI+b\vec{l}_0 \vec{l}_0^T$,
我们只需要在一组线基上观察作用,即可以获得其完整信息.
显然关于$L$的轴对称变换$\eta$,应有$\eta\vec{l}_0=\vec{l}_0$,$\eta\vec{n}_0=-\vec{n}_0$,
则$\vec{l}_0=\eta\vec{l}_0=(aI+b\vec{l}_0 \vec{l}_0^T ) \vec{l}_0=aI\vec{l}_0+b\vec{l}_0 (\vec{l}_0^T \vec{l}_0 )=a\vec{l}_0+b\vec{l}_0\Rightarrow a+b=1$;
$-\vec{n}_0=\eta\vec{n}_0=(aI+b\vec{l}_0 \vec{l}_0^T ) \vec{n}_0=a\vec{n}_0 \Rightarrow a=-1,b=-2$,故选D.\\
\textbf{习题 \ref{8.7} 解答:}\\
根据公式$R_{\vec{l}}=-I+2\vec{l}_0 \vec{l}_0^T$.\\
(1)$R_{\vec{l}} =-I_2+2\begin{bmatrix}0\\1\end{bmatrix}\begin{bmatrix}0&1\end{bmatrix}
     =-I_2+2\begin{bmatrix}0&0\\0&1\end{bmatrix}=\begin{bmatrix}-1&0\\0&1\end{bmatrix}$\\
(2)$R_{\vec{l}} =-I_2+2\begin{bmatrix}1\\0\end{bmatrix}\begin{bmatrix}1&0\end{bmatrix}=-I_2+2\begin{bmatrix}1&0\\0&0\end{bmatrix}=
      \begin{bmatrix}1&0\\0&-1\end{bmatrix}.$\\
(3) 要将$\vec{l}$单位化,\\
      $R_{\vec{l}} =-I_2+2\begin{bmatrix}\frac{1}{\sqrt{2}}\\ \frac{1}{\sqrt{2}}\end{bmatrix}
      \begin{bmatrix}\frac{1}{\sqrt{2}}&\frac{1}{\sqrt{2}}\end{bmatrix}=-I_2+2\begin{bmatrix}1&1\\1&1\end{bmatrix}=
      \begin{bmatrix}0&1\\1&0\end{bmatrix}.$\\
(4) 要将$\vec{l}$单位化,\\
      $R_{\vec{l}} =-I_2+2\begin{bmatrix}\frac{-3}{5}\\ \frac{4}{5}\end{bmatrix}
      \begin{bmatrix}\frac{-3}{5}&\frac{4}{5}\end{bmatrix}=-I_2+\frac{2}{25}\begin{bmatrix}9&-12\\-12&16\end{bmatrix}=
      \begin{bmatrix}\frac{-7}{25}&\frac{-24}{25}\\ \frac{-24}{25}&\frac{7}{25}\end{bmatrix}.$\\
(5) 要将$\vec{l}$单位化,\\
      $R_{\vec{l}} =-I_2+2\begin{bmatrix}\frac{12}{13}\\ \frac{-5}{13}\end{bmatrix}
      \begin{bmatrix}\frac{12}{13}&\frac{-5}{13}\end{bmatrix}=-I_2+\frac{2}{169}\begin{bmatrix}144&-60\\-60&25\end{bmatrix}=
      \begin{bmatrix}\frac{119}{169}&\frac{-120}{169}\\ \frac{-120}{169}&\frac{-119}{169}\end{bmatrix}.$\\
\textbf{习题 \ref{8.8} 解答:}\\
(1) 分析:设平面$\pi$上的一个单位方向向量为$\vec{l}_0$,
则有$(\vec{n}_0,\vec{n}_0)=\vec{n}_0^T\vec{n}_0=1$,$(\vec{n}_0,\vec{l}_0)=\vec{n}_0^T\vec{l}_0=0$.
设关于$\pi$的镜面对称变换为$\eta=aI+b\vec{n}_0\vec{n}_0^T$,应有$\eta\vec{l}_0=\vec{l}_0,\eta\vec{n}_0=-\vec{n}_0$,
则$-\vec{n}_0=\eta\vec{n}_0=(aI+b\vec{n}_0\vec{n}_0^T)\vec{n}_0=aI\vec{n}_0+b\vec{n}_0 (\vec{n}_0^T\vec{n}_0)=a\vec{n}_0
+b\vec{n}_0a+b=-1$; $\vec{l}_0=\eta\vec{l}_0=(aI+b\vec{n}_0\vec{n}_0^T )\vec{l}_0=a\vec{l}_0 \Rightarrow a=1$,故选C.\\
(2)  方法同上,设线投影变换$\eta=aI+b\vec{l}_0\vec{l}_0^T$ 根据$\eta\vec{l}_0=\vec{l}_0$,$\eta\vec{n}_0=0$,得
\begin{equation*}\begin{cases}a+b=1\\a=0\end{cases}\end{equation*}
故选B.\\
(3)  方法同上,设面投影变换$\eta=aI+b\vec{n}_0 \vec{n}_0^T$ 根据$\vec{l}_0=\vec{l}_0$,$\eta\vec{n}_0=0$,
\begin{equation*}\begin{cases}a=1\\a+b=0\end{cases}\end{equation*}
故选A.\\
\textbf{习题 \ref{8.9} 解答:}\\
B.  (视为乘以复数$\cos\theta+i\sin\theta?$)\\
\textbf{习题 \ref{8.10} 解答:}\\
B.  (可以得到矩阵$B=\begin{bmatrix}1&2\\2&3\end{bmatrix}$,而$C=\begin{bmatrix}2&3\\5&8\end{bmatrix}$,$AB=C$)\\
\textbf{习题 \ref{8.11} 解答:}\\
(1) 由特征值理论,$A$的特征多项式为$\lambda^2-3\lambda+2$,两个特征值为1,2,
其中属于1的特征向量取$\vec{x}_1=\begin{bmatrix}1\\1\end{bmatrix}$,
属于2的特征向量取$\vec{x}_1=\begin{bmatrix}1\\2\end{bmatrix}$,
将$\vec{b}$在$A$取定的特征向量上进行分解,有$\vec{b}=2\vec{x}_1 +\vec{x}_2$.
则由性质$A^{10}\cdot\vec{b}=A^{10}(2\vec{x}_1+\vec{x}_2)=2A^{10}\cdot\vec{x}_1+A^{10}\cdot\vec{x}_2=2\vec{x}_1+2^{10}\vec{x}_2=
\begin{bmatrix}2\\2\end{bmatrix}+ \begin{bmatrix}1024\\2048\end{bmatrix}=\begin{bmatrix}1026\\2050\end{bmatrix}$\\
(2)  同上,由特征值理论,$A$的特征多项式为$\lambda^2-2\lambda-3$,
两个特征值为-1,3,其中属于-1的特征向量取$\vec{x}_1=\begin{bmatrix}1\\-1\end{bmatrix}$,
属于3的特征向量取$\vec{x}_2=\begin{bmatrix}1\\1\end{bmatrix}$,
将$\vec{b}$在$A$取定的特征向量上进行分解,有$\vec{b}=5\vec{x}_1 +\vec{x}_2$.
则由性质$A^4\cdot\vec{b}=A^4(5\vec{x}_1+\vec{x}_2)=5A^4\cdot\vec{x}_1+A^4\cdot\vec{x}_2=5(-1)^4\vec{x}_1+3^4\vec{x}_2=
\begin{bmatrix}5\\-5\end{bmatrix}+ \begin{bmatrix}81\\81\end{bmatrix}=\begin{bmatrix}86\\76\end{bmatrix}$\\
\textbf{习题 \ref{8.12} 解答:}\\
$(3,3)^T$.(由于投影变换是幂等变换,即$\eta^2=\eta$,故$P_{\vec{l}}^{2016}=P_{\vec{l}}$)\\
\textbf{习题 \ref{8.13} 解答:}\\
$(72,96,120)^T$. (由于投影变换是幂等变换,故$P_{\vec{l}}^{314}=P_{\vec{l}}$ ,
由公式$P_{\vec{l}}=\vec{l}_0 \vec{l}_0^T$,得
\begin{equation*}
P_{\vec{l}}=\begin{bmatrix}\frac{3}{5}\\ \frac{4}{5}\\1\end{bmatrix}
\begin{bmatrix}\frac{3}{5}&\frac{4}{5}&1\end{bmatrix}=\frac{1}{25}\begin{bmatrix}9&12&15\\12&16&20\\15&20&25\end{bmatrix}.
\end{equation*}
\begin{equation*}
P_{\vec{l}}^{314}\begin{bmatrix}50\\50\\50\end{bmatrix}=
\begin{bmatrix}9&12&15\\12&16&20\\15&20&25\end{bmatrix}
\begin{bmatrix}2\\2\\2\end{bmatrix}=
\begin{bmatrix}72\\96\\120\end{bmatrix}.
\end{equation*}
\textbf{习题 \ref{8.14} 解答:}\\
$(1,11,11)^T$.  (由于反射变换是对合变换$\eta^2=1$,故$R_{\pi}^{233}=R_{\pi}$)\\
\textbf{习题 \ref{8.15} 解答:}\\
$(1,2,3)^T$.  ($R_{\pi}^{666}=1$,即$\mathbb{R}^3$上的恒等变换,与$\vec{n}$ 无关)\\
\textbf{习题 \ref{8.16} 解答:}\\
(1)假.  (一定是可逆阵。由于两组基可以互相表示,所以过渡矩阵是可逆的.)\\
(2)真. (因为一组新的基(作为向量看)在原来的基上的线性表示是唯一的.)\\
(3)真. (由过渡矩阵的复合知.)\\
\textbf{习题 \ref{8.17} 解答:}\\
(1)假.(平移变换是刚体变换,但不是正交变换,因为正交变换必将零向量映为零向量. )\\
(2)真. (若$AA^T=A^TA=I$,则$A^(-1) (A^T )^(-1)=(A^T A)^(-1)=I$,$(A^T )^(-1) A^(-1)=(AA^T )^(-1)=I$)\\
(3)假. (反例:$A=I$,$B=-I$)\\
(4)真. (若$AA^T=A^T A=I$,$BB^T=B^T B=I$,则$AB(AB)^T=ABB^T A^T=AA^T=I$, $(AB)^T AB=B^T A^T AB=B^T B=I$).

%--------------------------------------------------------------------
%在本地编译把下面这行注释去掉
%\end{CJK}
%--------------------------------------------------------------------
\end{document}
-